\chapter{关于中国的前言}

% \section{百年未有之大变局}

% 中国近几十年的发展有目共睹,我们的衣食住行得到了飞跃式的提高,坚硬的城乡二元
% 壁垒被打破,阶级流动前所未有,物质生活日益丰富,诸如此类。联合国也在多次报告
% 中提及,中国是为全球减贫作出最大贡献的国家。但同时我们也面临民族国家与全球化
% 等困境下的巨大挑战,面对凯恩斯毒药的暴雷……人民不容易,国家也确实不容易。

% 资本主义危机避无可避、无法解决,只能转移来转移去。这一准则在民族国家与全球资
% 本化语境里的应用使世界政治竞技场愈加残酷。特别是在2008年世界经济危机之后,虽
% 然中国因强有力的国家管制未受它国那样损失——当然也存后遗症,但它国已从全球化
% 一端向民族国家一端加速倾斜,使中国的世界市场地位受到更大挑战。特朗普上台后,
% 更加速让美国倾向于经济民族国家立场,并用此来盘剥全球,中国这一世界市场、全工
% 业体系大国自然成为其最大对手。

% 笔者认为,
% 习近平总书记在2017年作出重要论断“当今世界正处在百年未有之大变局”,高屋建瓴
% 说明了当前世界的复杂局势,将历史、政治、经济、文学等科学融会贯通,涉及一战和
% 二战之间、1929年美国经济大萧条、民族国家与全球化、地缘政治、战争等方面,给出
% 了世界格局将在普遍危机中愈发苍茫惨烈的激辩重大警示。

% 以阿根廷为例,当它面临秃鹫基金的绞杀时数次突围不得解,最终在以美国为首的部分
% 发达工业国家和世界金融集团压力下被迫高额偿债时,华尔街为阿根廷掉滴上一滴“善
% 良的”眼泪后立即弹冠相庆,华尔街鼓吹“从表面上看,似乎秃鹫基金是最大赢家。但
% 实际上,这起官司今天走到终点,对于阿根廷来说是个双赢的结果——\textbf{该国将最终得
%   以重返国际资本市场},重新开展国际融资,以帮助经济走出困境。”\footnote{可参
%   考\url{https://wallstreetcn.com/articles/230968}。}多么好的双赢啊!阿根廷发
% 行百年8\%的国债,比索崩溃,休克疗法等。阿根廷不再哭泣,它已经没有眼泪……

% 我们国家的一些金融从业者对华尔街的一系列操作热血沸腾,倾倒崇拜,欲让中国彻底
% 新自由主义化、全球化以有利于本集团利益。\section{本篇立意}

笔者已过不惑,越发感觉一切都是渺小的。仍具相当动物性和原始性的人给了世界历史
社会太多限制,世界历史社会给了国家太多限制,国家给了政府和领导人太多限
制……而凡人平民无疑是最为受限和苦难的。在这些丰富张力之下,若有一只上帝之眼
去观看任何一人的一生,那都是怎样辉煌又深刻丰富的悲喜剧呢!

% 不过任何层面总有点光明可以逃逸出重重限制,总有希望,总有多样可能。

人类常常一直缺乏直面社会现实的勇气,常将幻梦蒙在真实和悲苦之上,这层面纱或遮
掩或美化或扭曲。中国在数千年沧桑历史中形成的一种绝望尤使太多中国人爱做幻梦,
常常自我麻醉、视而不见现实悲剧那并不让人开心的力量。

李大钊先生早在百年前就说:
\begin{quotation}
  \textbf{中国自古昔圣哲,即习为托古之说,以自矜重}:孔孟之徒,言必称尧舜;老
  庄之徒,言必称黄帝;墨翟之徒,言必称大禹;许行\footnote{先秦农家学派代表人物}之徒,
  言必称神农。此风既倡,后世逸民高歌,诗人梦想,大抵慨念黄、农、虞、夏、无怀、
  葛天的黄金时代,以重寄其怀古的幽情,\textbf{而退落的历史观,遂以隐中于人心。其或
    征诛誓诰,则称帝命;衰乱行吟,则呼昊天;生逢不辰,遭时多故,则思王者,思
    英雄。}而王者英雄之拯世救民,又皆为应运而生、天亶天纵的聪明圣智,而中国哲
  学家的历史观,遂全为\textbf{循环的、神权的、伟人的历史观}所结晶。一部整个的中国史,
  迄兹以前,遂全为是等史观所支配,以潜入于人心,深固而不可扰除。
\end{quotation}

今日又改变了多少呢?

在笔者的日常生活中,偶有贩夫走卒直指一些问题深刻本质,历史经验和个人的现实经
验给予了他们一些超过“转椅上的专家”的真知灼见。很多人其实不是不知道,不是不
能知道,只是不愿面对。

痴痴幻梦可以让人在世界悲凉残酷时感受到并不存在的温暖。过去我认为这是国人的虚
弱,现在我更接受这是理性,是沧桑历史赋予我们的、用以应对现实的理性。面对了又
能怎样呢?一切又有何用呢?太阳照常从东方升起,哪有什么迷雾外的彼岸伊甸园?六
朝何事,只成门户私计,人民的自愿幻梦,怎不是看透了世事的大智慧?!

但我还是想在还能写的时候,把历史写下来。民为重,社稷为轻、君次之。我想写的是
民的历史,而非人斗人或某些具体个人的历史。不管是社稷还是权力者都应当成为民的
历史的注解,而非相反。悠悠世间,浊浪苍生,民总是这样翻来覆去起伏着,悲喜着,
树欲静而风不止……但当我尽可能去避免描述政治人物,尤其是政治斗争,为了更好书
写中国人的历史的时候,才发现政治主线是不可避免的,因此关于政治的描写也不得已
追加不少。

有一点需要声明,笔者仅为高中文凭,一介匹夫,本身所知就甚少,接触到的资料又极
少和片面,另外很多内在动因其实只有局内人才可知悉,这一切均使本部分内容可能出
现较多幼稚谬误,让人贻笑大方,不可轻信。

即使如此,为什么敢写敢说,因为笔者仍有些天真相信真实和睁眼看世界这二者的力量,
真诚求索的前提下,未达山峰只是在路上并不是那么难堪,。

笔者所述中国史仅为一些片段,并非全貌。一是因为前文所述笔者能力有限,二是因为
历史往往相通,乃至重复,无需面面俱到便可䌷绎得出一些更为本质的内在逻辑。笔者
希望借这些历史片段去纪念在时代沉浮中漂泊的人民,如能激发起个别读者公平正义之
心思自是最好,不能的话,也足可聊以自慰。

或许一切都是徒劳,每朝每代都有人在旷野中呼唤;或许一切是最好,人性的缺点所保
证的是人类整体的繁衍生息,代代基因的传递;或许这只是不识庐山真面目的愚者所作所
为……

但我还是想写下来,总有人会去做这样的事,我愿成为其中的一员。写完了,向我那总
做傻事一直逃避的青春致敬完毕,也就可以大方去做我的小家幻梦,继续另一段逃避历
程了。


本篇多有一些批评,着重对事不对人,并且这也只是求真求实的副作用,还请理智探讨。

% 很多国人常常呼唤一位
% 个人英雄,常常建立关于某个个人的偶像崇拜。在这种意识形态之下,我们常常使“自
% 己”缺场,,隐患颇多。

% 为应对悲苦人生,尼采所说日神阿波罗的幻梦便长期占据主要地位,现代性理性对其它理性
% 的压制又使之愈演愈烈。

% 只有敢于直面悲苦,尊重生命本身,尊重人之所以为人,在纵情忘我的对生命原始冲动、创
% 造的追求中才能去求得真实满足和慰藉。\footnote{这其中也暗含一种悲观的可能——人是否
%   能大量摆脱动物性和原始性……}


%   笔者只是人间一介草民。按理说,笔者各方面素养完全不足以较好完成对中国方方
%   面面的论述。那么为什么还要完成本篇?

% 相较知识、政治、经济界人,我的草民身份和个人特质使我更为敢言、直接,更想寻求客观
% 公正描述,更为注重凡人在时代波涛中的跌宕起伏。

% 国人常将国家作为坚不可摧,或可凭高层领导的英明步步走高,这是不科学的。国家面
% 临的是一个错综复杂实践中的现实社会,有些选择必须要面对,但这些选择的选项在一
% 些时候可能都是灰色的,无一较为完善,如薄一波在《若干重大决策与事件的回顾(上
% 下)》形容统购统销时所说“两种`炸药'中的选择”。\cite[259]{boyibo}

% 中国不能再激荡,中国不能乱!有问题和分歧要探讨和争辩,可以声音大些但要理智,
% 尤其要综合全局。人之艰难、国之不易,上层下层均要互相考量和体谅。拥护党和国家
% 的领导,采用合适渠道、方法是前提,过度激进要不得。


% 在这种民的历史之下,我们将可以得到一个更为清晰醒目的认识,再用这种认识去造福于民,
% 寻得社会的真正进步。我的所作,将有诸多错误和不足。我之所以自不量力去从事历史描述,
% 因为我相信我的作品最重要的是激发大众和他人的思考,激发他人对我所作的批判,从而实
% 现独立之个人,自由之思想,光明之未来。即使拙作无人问津,毫无影响力,也已在这条道
% 路上投下一颗石子。


% “关于中国”这部分着重于通过理清中国历史脉络以及笔者对当今中国问题的个人不成熟见
% 解来激发人的思考,从而减弱幻梦、尊重真实、勇敢面对、正视错误、科学发展、强盛国家、
% 保障人民以及为了人类的美好明天。这一目的并非有笔者亲自实现,实际上这系列文章应该
% 是几乎无人问津外加毫无影响力,但笔者可以成为立志于从事这方面的一系列推动者中的一
% 员。

% \section{本篇文章结构}

% 第x章到第x章为总结当代中国从1949建国年至今的经济重大事件,第x章为总结说明和笔者对
% 当今中国的思考建议。本系列着重参考了薄一波《若干重大决策与事件的回顾(上下)》,
% 邓书杰,李梅,吴晓莉和苏继红所著《中国历史大事详解丛书(套装共9册
% )》\cite{zhengshujie}\footnote{本套书为电子书,无法注明具体页码},blablablabla以
% 及对这些参考资料的追根溯源。



%%% Local Variables:
%%% mode: latex
%%% TeX-master: "../main"
%%% End:
