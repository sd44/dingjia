\chapter[早期苏俄科社实践]{政治经济学视角下的\\早期苏俄科社实践}
\label{chap:russiachina}

\section{序言}

苏联问题,无论过去还是现在,东方还是西方都是一个敏感话题。对其研究常常带有各
种不同意识形态、政治、经济利益集团的功利目的,在不同时期和环境为不同利益背书;
也常被个人主观倾向所利用;在民间个人,则常常被自身迷梦所影响,简单片面的赞美
或者抨击。

笔者认为,如果刨除政治经济学,只用其他视角去考察苏联列宁和斯大林统治时期历史
的话,无法形成一个完整链条。历史与政治经济学的考察相结合,将更为完整展现苏联
早期内在的满满张力——起因、经过、结果、成就、矛盾和不足;认识马克思本人科学
社会主义思想和早先社会主义国家的问题所在;弄清楚有关社会主义本身的一些问题,
比如社会主义与资本主义的关系及其局限和不足;加深历史唯物认识并破除对于救世主
或个人的迷信与崇拜……总之,这样的考察结果将会在多方面提供更多借鉴和意义。

本章关于苏联的资料,在政治经济学方面的论据主要取材自批判色彩浓厚的M. C. 霍华
德 与 J. E. King所著《马克思主义经济学史 1883-1929》、《马克思主义经济学
史 1929-1990》,历史资料多取材于沈志华主编《一个大国的崛起与崩溃——苏联历史
专题研究 1917-1991》,闫永飞《苏联社会主义过渡时期的新经济政策》,黄立茀主编
《新经济政策时期的苏联社会》。这无疑犯了取材样本过少的错误,容易偏颇。但笔者
实在无能为力再进行过多展开了。望得到大家建议、批判。

笔者对所有参考文献并不全然赞同,对个别文献更是批判大于赞同,为此将以上材料进
行了\textbf{批判性整合},融入自己观点。

\section{马克思“科学社会主义”的缺陷}
\label{sec:marxkexue}

自马克思中学毕业前夕作文《青年在选择职业时的考虑》至其生命终结时未完成的一些
作品,我们都可以鲜明看到贯穿马克思思想的一条主线——\textbf{社会伦理}。马克思的社会
伦理不直接讲求个人伦理,它讲求人的“\textbf{社会性}”。因为“\textbf{人的本质不是单个人所
  固有的抽象物,在其现实性上,它是一切社会关系的总和}”。人之恶主要是由综合历
史社会关系影响,真正解放的人本质只有在相当和谐的社会才能实现。

大部分人将马克思学说分为三个部份,哲学、政治经济学、科学社会主义。其中科学社
会主义部份描述了一个美丽前景:一个自由人的联合体、自由王国、共产主义社会。但
是马克思、恩格斯均没有描述具体实践方式。

为什么马克思反对资本主义,马克思在《1857-1858年经济学手稿》中说过一句话
说,“\textbf{这种一切(本)人反对一切人的战争所造成的结果, 不是普遍的肯定, 而是
  普遍的否定。}”\footnote{括号中的文字为笔者所加。这句话的前半句起源于英国哲学家托
  马斯·霍布斯\cite[531]{karlvol46a},发展自对马克思具有重大影响的德国哲学家黑格
  尔“市民社会是个人私利的战场,是一切人反对一切人的战场,同样,市民社会也是
  私人利益跟特殊公共事务冲突的舞台,并且是它们二者共同跟国家的最高观点和制度
  冲突的舞台。”\cite[309]{hegelyuanli}。}在这种普遍的否定中,人与人的关系被物与
物的关系所遮盖,也就是马克思所说的“\textbf{资本拜物教}”。

如何实现从资本主义社会到共产主义社会的转变,马克思在《哥达纲领批判》中给出了
一个易懂简洁的说明:首先,是一个“\textbf{无产阶级的革命专政}”下的过渡期
\begin{quotation}
  (共产主义社会)不是在它自身基础上已经发展了的,恰好相反,是刚刚从资本主义
  社会中产生出来的,因此它在各方面,在经济、道德和精神方面都还带着它脱胎出来
  的那个旧社会的痕迹。

  ……

  所以,在这里平等的权利按照原则仍然是\textbf{资产阶级权利},虽然原则和实践在这里已
  不再互相矛盾,而在商品交换中,等价物的交换只是平均来说才存在,不是存在于每
  个个别场合。\cite[434]{maenwen3}
\end{quotation}

这就是“\textbf{共产主义社会的第一阶段}”,或者称为共产主义的初级阶段,也一般称
为\textbf{社会主义社会},是资本主义到共产主义的过渡阶段,\textbf{它仍带有(平均来说)等价
  交换的资产阶级法权}。

那么,为什么初级阶段可以发展自资本主义社会呢?

% 笔者认为简单地说明可以借助于《德意志意识形态》和《1857-1858年经济学手稿》中的
% 几句话:
% \begin{quotation}
%   这种\textbf{普遍交换},他们的互相联系,表现为对他们本身来说是\textbf{异己的、无关的}东
%   西,表现为一种物。在交换价值上,人的社会关系转化为物的社会关系;人的能力转
%   化为物的能力。\cite[103-104]{karlvol46a}

%   这种“\textbf{异化}”(用哲学家易懂的话来说)当然只有在具备了两个实际前提之后才会
%   消灭。要使这种异化成为一种“\textbf{不堪忍受的}”力量,即成为革命所要反对的力量,
%   就必须让它把人类的大多数变成完全“\textbf{没有财产的}”人,同时这些人又\textbf{同现存的
%     有钱有教养的世界相对立},而这两个条件都是\textbf{以生产力的巨大增长和高度发展为
%     前提}的……其次,只有随着\textbf{生产力的这种普遍发展},人们的\textbf{普遍交往}才能建
%   立起来。\cite[165-166]{maenxuanji1}

%   在世界市场上,单个人与一切人发生联系,但同时这种联系又不以单个人为转移,这
%   种情况甚至发展到这样的高度,以致这种联系的形成已经同时包含着突破它自身的条
%   件。

%   要使这种个性(全面发展的个人,即自由人)成为可能,能力的发展就要达到一定
%   的程度和全面性,这正是以建立在交换价值基础上的生产为前提的, 这种生产才在
%   产生出个人同自己和同别人的\textbf{普遍异化}的同时,也产生出个人关系和个人
%   能力的\textbf{普遍性和全面性}。\cite[108-109]{karlvol46a}
% \end{quotation}

“生产力的巨大增长和高度发展”\cite[538]{maenwen1},等价交换商品成为现实,个人也
与一切人普遍异化的同时又与一切人发生关系;同时导致创造最多价值的人反而沦
为“没有财产的”、“普遍异化”的无产阶级,与资产阶级对立。马克思政治经济学基
于历史唯物对资本主义的破坏性分析鞭辟入里,但是他求得建设性的共产主义自身却并
不是历史唯物的,而是伦理价值目的论,并且是彻底偏离了“科学”二字的伦理价值目
的论。

% 马克思认为
% \begin{quotation}
%   共产主义对我们来说不是应当确立的状况,不是现实应当与之相适应的理想。我们所
%   称为共产主义的是那种消灭现存状况的现实的运动。这个运动的条件是由\textbf{现有的前
%     提}产生的。\cite[539]{maenwen1}
% \end{quotation}

他的不科学成分笔者总结如下:
\begin{enumerate}
\item 人是动物的一种,马恩学说夸大了人相对于动物的优越性。他最终要实现的是人超越
  动物性和原始性,使人类与自然中的天空大地等一体。笔者记得马恩也都曾在文章或书
  信中说过这是个永远无法到达但可无限趋近的最终阶段,并非是要人性完全超越动物
  性\footnote{笔者已不记得马恩类似话语出现在哪几篇文献里,印象里马克思在论述康德时有
    提到过,恩格斯一封书信中比较直接提出,如有读者知晓还望告知。}。但这点其实
  并不是最重要,我们接着看。

\item 《资本论》第一卷是马克思对资本主义一般规律的现实考察,主要基于曼彻斯特大工
  厂生产模式。在《资本论》第三卷中加入了金融资本、土地、平均利润率的考量,天
  才地、先驱地批判了金融资本。但最终还是小看了资本主义的“革命性”和“进步性”,就如后现
  代主义代表人物之一利奥塔所认为的:\textbf{就当前而言,资本主义可以消融一切对立和
    异己的东西,能够吞噬一切。}

  其实马克思自己也在《<政治经济学批判>序言》中说过:“\textbf{无论哪一个社会形态,
    在它所能容纳的全部生产力发挥出来以前,是决不会灭亡的;而新的更高的生产关
    系,在它的物质存在条件在旧社会的胎胞里成熟以前,是决不会出现的。}”结合历
  史经验和当今现实,资本主义仍在发展,仍未发挥其全部生产力,那么在此情况
  下“新的更高的生产关系”是否会产生,统治阶级是否是无产阶级呢?答案是不言自
  明的。

\item 即使按照马克思对于历史唯物的考察,或者按照列宁、斯大林颇有争议的“五段
  论”去考察人类历史,我们不无悲伤地发现,从奴隶社会至资本主义社会,每种巨大
  的社会变迁中,每一个新兴的、替代了之前上层阶级的阶级,无一不是代表了比旧上
  层阶级更为先进的生产力、更为强劲的经济增长。但这种新上层阶级的兴起是较
  为\textbf{历史自然形成},与旧上层阶级的\textbf{融合、过渡}一般也较为和谐,有时甚至是直
  接脱胎于旧上层阶级。而“无产阶级专政”首先是由理论指导,并非像马克思所
  说“现实的运动”;其次为了实践它所需要的暴力实在是太暴力了。
\end{enumerate}

综上,马克思的共产主义确实是一种基于马克思强烈“改造世界”、解放全人类愿望、
偏离了科学的价值目的论。

笔者认为,马克思这一仍带有“空想社会主义”色彩的伦理目的论与希腊哲学家苏格拉
底有着同样的结构,而不是柏拉图。
\begin{quotation}
  (苏格拉底关于善的知识)包括两个有重大意义的先决条件:一个是心理
  学上的,即明显的唯理智论;另一个是伦理学上的,即明显的幸福论……

  就此苏格拉底把世俗说教的一般道德提高到科学的水平。\cite[113]{wendeerban}
\end{quotation}

苏格拉底认为美德即智慧,恶源于无知,试图将德性等同于客观规律,借此让人类世界
成为一个幸福的伦理世界,并为贯彻这一信念而殉葬。马克思很可能也是如此,最终导
致不管他怎样为“科学”一词辩护,这都使其难以摆脱“目的论”的范畴。

马克思晚年关注过俄国革命的可能,与俄国民粹派有过交流,提出过资本主义不发达国
家跨越资本主义“卡夫丁峡谷”的猜想,但这一猜想并非是有集体土地基础和意识的俄
国可以独立革命成功;而是俄国革命的最终成功仍要依托于“\textbf{西方发达国家无产阶级
  革命的呼应}”。\cite{mamincui}帕尔乌斯和托洛茨基所提“\textbf{不断革命论}”的革命动
力学也正是因此建筑于西方革命成功,这是\textbf{必要条件}。

需要注意的是,这里批判的是马克思科学社会主义中的“科学”,不是彻底否定历史唯
物主义,更不是全盘否定马克思。没有任何一人的社科理论可以健全,我们要辩证看待,
取其精华,去其糟粕,辩证发展。马克思对于资本主义的考量至今仍有巨大意义,甚至
直接应验;科学社会主义中也提供了些理念、理想和具体方法,只是我们将更加负重前
行。

\section{杜冈-巴拉诺夫斯基对苏俄的影响}

杜冈-巴拉诺夫斯基被霍华德和King高度称赞,他早期是一个“合法马克思主义者”,后
来转变为寻求“社会合作”的资产阶级自由派,因而被列宁批判为“自由派教授”。因
他的这种转变、列宁对他的批评以及俄语受众小等原因,他也常常被国人忽视。

霍华德和King认为杜冈-巴拉诺夫斯基在\textbf{十九世纪末二十世纪初}的观点,直接或间接
影响了托洛茨基、列宁、布哈林、普列奥布拉任斯基、斯大林等人。笔者认为,如不了解杜冈-巴
拉诺夫斯基理论,便缺失了苏联、特别是斯大林时期痴迷于重工业发展的\textbf{重要思想历
  史源头}。

杜冈-巴拉诺夫斯基通过对马克思《资本论》第二卷“再生产图式”的批判和推演,得出:
在\textbf{减少奢侈品消费,限制社会消费}的情况下,将节约出来的这部分资本\textbf{投资于生产
  部类},可以扩大社会再生产。加强机器等促使\textbf{资本有机构成提高}的应用,可以使
扩大再生产\textbf{对劳动力的需求降低},而不是像马克思所说一样随之增长,如此\textbf{消费部类}的
消费需求也可以继续降低。“也就是说,\textbf{社会财富增加,社会收入(人的工资)减少,
  也是可能的}”,相比于消费需求,“生产资料需求更大程度上决定了社会生
产”。\cite{lijingdugang}霍华德和King将其结论进一步总结为“\textbf{剩余价值的实现越来
  越独立于消费需求}”。

杜冈-巴拉诺夫斯基提出“\textbf{不平衡与综合发展}”。各国发展条件、经济制度不一,但
资本主义对于无限生产的欲望已经使当前形成一个\textbf{世界市场},各国也均要通过这个世
界市场彼此发生联系。

俄国是世界市场中一个落后于发达国家的“\textbf{后发国家}”。
\begin{quotation}
  发端于彼得大帝时期的俄国,\textbf{现代化是生存的必然选择}。…… 新型生
  产必须在\textbf{大规模工厂}中进行,以便有效地利用进口技术。仅\textbf{商业资本家}拥有资
  源和能力来筹办新工厂。由于提供自由劳动力的国内市场并不存在,因此,\textbf{国家}有
  必要进行\textbf{直接干预},以提供\textbf{强制劳动}。

  杜冈-巴拉诺夫斯基认识到,“\textbf{后发者}”发展模式明显与\textbf{英国}模式不同,……后
  发国家的工业化缺乏自发性,国家行为发挥了支配性的作用,非资本主义制度与资本
  主义关系\textbf{结合}在一起,公开的\textbf{强制补充了市场训诫},技术引进发挥了重要作用。
  他强调,\textbf{俄国工业资本主义的发展意味着整个俄国经济日益融入世界市
    场}。\cite[174-175]{mazhengshi}
\end{quotation}

杜冈强调国家规训下的现代化工业是“后发”俄国唯一出路的论调,奠定了列宁、托洛
茨基、普列奥布拉任斯基、“粮食危机后”斯大林的\textbf{抑农重工}思想。后来布哈林反对
普列奥布拉任斯基、斯大林等人应用“\textbf{杜冈主义}”,希望\textbf{工农均衡发展}。

杜冈的思想中,已经有了民族国家与全球化的张力。布哈林对此理论的延伸加强堪称经
典,并且远远超越苏联时代。直至1985年广场协议之后几年学界才开始重建民族国家与
全球化理论,但似乎并未脱离布哈林的理论框架。

\section{二十世纪初至十月革命的俄国}

\begin{quotation}
  (十九世纪末、二十世纪初)绝大多数东方国家\textbf{不具有}被马克思称为资本主义曙光
  的种种有利条件(即地理环境、金矿、奴隶贸易、殖民地等),相反却普遍受到先进
  资本主义国家的压迫、排挤和剥削(它们是作为被剥夺者而纳入现代资本主义经济体
  系的)这一客观原因外,更主要的则在于东方社会自身的历史条
  件。\cite[2-3]{bigrussia}
\end{quotation}

俄国是一个半亚洲式的、落后沙皇专制下的小农国家,他的农业国性质以及知识分子阶
层的崛起,使其受马克思影响较深,对马克思所说“\textbf{资本主义制度所带来的一切极端
  不幸的灾难}”,尤其是资本主义原始积累对农民的极强压迫剥削,带有较深恐
惧。\cite[137]{mazhengshi}。

\begin{quotation}
  1917年之前,作为社会主义\textbf{必要的前提条件,一个充分发展的资本主义}是无论如何
  也不可避免的。\textbf{所有的非农民政党都赞同资产阶级民主革命},认为它是落后的沙皇
  俄国\textbf{唯一可能的革命形式}。\cite[86]{mazhengshi}
\end{quotation}

% 晚年马克思将重心放在了政治实践上,尤其关注俄国革命和亚细亚生产方式\footnote{霍华德
%   和King认为马克思的亚细亚生产方式分析比较片面。}。

早期俄国民粹主义者,希望借助“农村公社”集体化与西方相异的特质,实行\textbf{土地国
  有化},试图不经过资本主义发达阶段直接跨入社会主义社会,并与晚年马克思、恩格
斯有过直接的交流。晚年马克思和恩格斯在否定后,又考虑了跨越“卡夫卡峡谷”的可
能,但这一可能建立在俄国无产阶级革命与西方无产阶级革命的彼此呼
应。\cite{mamincui}

“俄国马克思主义之父”普列汉诺夫从历史唯物主义出发批判民粹主义者,认为\textbf{俄国社
会主义革命的必要历史前提为资产阶级民主革命},针对如何尽快过渡到社会主义,提
出了资产阶级民主革命和社会主义革命“\textbf{两阶段论}”。其中,资产阶级民主革命阶段,
无产阶级成立独立政治组织保护自己权益,并与资产阶级结成同盟。如果在这一进程中
资产阶级抵制无产阶级,已经壮大的无产阶级则进行夺权斗争。

笔者认为可以将《1883-1929》的评价可以总结如下,两阶段论的缺陷在于它\textbf{匮乏阶级
  动力学},过于重视政治而忽视了经济,忽视了俄国落后资产阶级的革命意向并没有这
么强烈;甚至因为落后而依附于旧有沙皇专制统治阶级。此种情景下,无产阶级如何接
受并超越第一阶段中的实际从属地位呢?

普列汉诺夫将实现社会主义为目标的所有理念称为“\textbf{代数学}”,将这些人中所产生的
派系、理论差别称为“\textbf{算术}”,从而将社会民主党、孟什维克、布尔什维克和社会革
命党(民粹主义政党)于1917年前长期求同存异地联系在一起。

继承普列汉诺夫衣钵的孟什维克,在演化过程中于1905年走向了支持资产阶级领导权,
削弱无产阶级,社会主义政党负责对政府施加压力的道路。
\begin{quotation}
  早在1905年革命之后,孟什维克就逐渐形成了关于俄国未来革命的理论。他们认为,
  在俄国这样一个经济落后的国家里,革命的任务是为资本主义的充分发展开辟道路。
  而既然革命是资产阶级民主性质的,那么就\textbf{应该由资产阶级来领导},来掌握政权。
  社会主义政党将实行对资产阶级政府\textbf{施加压力}的政策,以争取实现工人阶级和劳动
  人民的经济要求和政治要求,为向社会主义过渡创造条件。二月革命中,孟什维克实
  践了这一理论,认为立宪民主党是“最有资格执政的民主派”。\textbf{政权应该集中在由
    自由主义政党的代表组织的政府手中。}至于社会革命党,1905年革命的结局使它得
  出了与孟什维克类似的结论。在二月革命中,社会革命党人起先是尽量避免掌握权力,
  继而又同孟什维克一起,与立宪民主党实行合作。\cite[38]{bigrussia}
\end{quotation}

列宁在不同时期吸收了不同派别学说,对于社会现实变化极为敏感,并在政治表现上极
为果决。针对右倾资本主义思想,列宁提出激进的“\textbf{工农民主专政}”。托洛茨基提出
了更为激进的“\textbf{不断革命论}”,倡导“\textbf{资产阶级革命嵌入社会主义革命}”,在阶
级动力学和一系列先知色彩的预测上表现突出,在1917年与列宁走到了一起。两者均提
出联合农民的具体纲领,列宁在1914年起就采取了“\textbf{革命失败主义}”——“这次战争
是帝国主义之间因分赃不均而爆发的,战争并不符合工人的利益,无产阶级只是在给资
产阶级白白充当炮灰而已,因此,应变帝国主义战争为国内战争,乘机推翻本国资产阶
级政权。”\cite{shibaizhuyi}

更直接的描述:与主张“\textbf{护国战争}”的绝对多数社会革命党、立宪自由派和孟什维克
相反,\textbf{列宁主张俄国投降,目的在于将俄国战争失败的压力转变为国内工农无产阶级
  革命的动能,其和平政策赢得了相当民心和军心。}

1917年的俄国,形势复杂、风云变幻、莫名吊诡,难以用常识和理智去理解。二月革命
突然爆发,“伊里奇的梦醒了”,当时本已对社会主义掌权绝望的列宁立刻展开一系列操作,
在党内外多次以小博大、以少博多,审时度势、勇猛坚定,或许还要加上运气,数次力
挽狂澜。这段历史建议直接观看《一个大国的崛起与崩溃》。

1918年1月5日下午4点布尔什维克十月底在彼得格勒武装夺权后成立政权。但即使如此,
布尔什维克在11月12日选举出的立宪会议代表名单中,仍然只占据 $\mathbf{24\%}$ 的
份额。这使布尔什维克更加强烈意识到自己无法在立宪会议中占据主导地位,于是继续采取
一系列悍然措施,甚至逮捕与会代表。\textbf{1月5日下午4点立宪会议召开,1月6日凌晨4点
  与会代表被驱离,当天宣布解散……}1月10日,全俄苏维埃第三次代表大会开幕,并
取代了立宪会议的职能,通过了《被剥削劳动人民权利宣言》。

没有充分理由相信列宁是一个投机机会主义者,或是窃国犯。相反,列宁始终是一个坚
定的马克思主义者。站在当时列宁视角来看,如果不采取断然措施,立刻夺取政权,那
么俄国革命的胜利果实必然是直接送入资产阶级政权之手,所谓起监督权的社会主义成
分只能越来越稀薄和边缘化,如此绝不会出现普列汉诺夫所说第二阶段——社会主义革
命阶段。另外,作为一个政治家,待时而果决行动是优点而非缺点。

\section{十月革命之后到新经济政策}

\begin{quotation}
  面对 1917 年以后的困难,布尔什维克完全可能\textbf{没有任何社会主义式的解决方法},
  这也许是我们在理解共产主义经济学具有的理论上的不稳定和冲突的特征时,需要考
  虑的最重要的因素。\textbf{革命政权继承了一个濒临崩溃的经济}。 1914 年后成年男性人
  口中的\textbf{三分之一},被动员起来参与到战争中,落后的俄国经济已经极其脆弱,很难
  承受一场全面的战争。革命和内战更是毁灭性的。假设 1913 年工业产出指数
  为 100, 1917 年下降到 75, 1921 年为 31,而农业生产在 1917 年下降为 90,四
  年后下降到 60。在俄国内战期间,在\textbf{西方资本主义国家的封锁下},俄国的对外贸
  易事实上几乎完全中止。随后的\textbf{复苏十分迅速},工业和农业指数在 1928 年分别上
  升到 133 和 125。然而,将 1913至1928 年作为一个整体来看,\textbf{俄国仍然远远落后
    于西方}。与 1870 年至 1913 年 2.5\% 的年增长率相比,这一时期的年产出增长
  率只有极低的 0.8\%,前一个时期人口的年增长率为 0.9\%,但在 1913 年后下降
  到 0.3\%。\cite[286-287]{mazhengshi}
\end{quotation}

《1881-1929》将苏联在1917年革命成功后到1929年之间的苏联经济史分为了三个不同的
过渡阶段,笔者沿用这种划分,并将其与其他参考文献有关内容批判整合如下:
\begin{enumerate}
\item 1917年10月到1918年6月:农民夺取了土地,但却是\textbf{以传统公社原则进行了重新分
    配,使新政府颁布的土地国有化法令成为多余,并降低了生产率}。实行的少数工业
  国有化大多是\textbf{地方行为},并实行了“\textbf{工人控制}”,私人资本家受\textbf{工厂委员会
    和当地布尔什维克官员}监督。列宁将其描述为和公社国家结合的“\textbf{国家资本主
    义}”。

  列宁在《\textbf{论“左派”幼稚病和小资产阶级性}》中,对要求更为激进实行全面社会主义
  的“左派”进行了批判,其中列宁观点主要是:
  \begin{enumerate}
  \item \textbf{从投降派转向护国派。}俄帝国主义时代,采取投降主义,是因为这是非正义的、帝
    国主义间的战争。如果此时支持战争则会有利于帝国主义,损害社会主义发
    展。“1917年10月25日以后我们是护国派……必须保卫社会主义祖国”。

  \item \textbf{进一步完全打倒资产阶级和彻底消除怠工的政策是幼稚的。之前}采取这样措施
    是为在政治局面中占据主导地位,\textbf{掌权后}所面临的则是综合现实状况,身为执政党
    要贴合实际,采取过渡策略。

  \item 列宁在多方面论述了\textbf{国家资本主义的必要性},苏联的“\textbf{计划}”特征也初具雏形。
    \begin{quotation}
      \textbf{国家资本主义}较之我们苏维埃共和国目前的情况,将是一个\textbf{进步}。如
      果\textbf{国家资本主义}在半年左右能在我国建立起来,那将是一个\textbf{很大的胜利},
      那将极其可靠地保证\textbf{社会主义}一年以后在我国最终地\textbf{巩固起来而立于不败之
        地}。
    \end{quotation}
    如果不采取\textbf{国家资本主义},那么小农国家内占优势的\textbf{小资产阶级投机商(尤其是
    投机粮商)}将既与国家资本主义作斗争,也将与社会主义作斗争。

    在工农领导和监督下的国家资本主义,通过\textbf{正确计算和分配}将为一种保障。应仿
    效\textbf{德国国家资本主义}。
    \begin{quotation}
      如果德国无产阶级革命胜利,那么就能轻而易举地一下子击破任何帝国主义的蛋
      壳(笔者按:意即世界社会主义的胜利将不需面对大的困难)……如果德国革命
      迟迟不“诞生”,我们的任务就是要学习德国人的国家资本主义,全力仿效这
      种\textbf{国家资本主义},要不惜采用\textbf{独裁}的方法加紧仿效,甚于当年的彼得”。
    \end{quotation}
  \end{enumerate}

\item 1918年6月到1921年初:\textbf{国有化和紧缩经济}。为应对协约国和国内白军武装压力,
  实行\textbf{战时共产主义} : \textbf{试图征收农民全部剩余价值,优先恢复工业},导致1921年
  苏联爆发大饥荒,爆发\textbf{50多起大规模农民起义,工业也处于瘫痪状态};\textbf{取消}公
  用事业、住房、铁路交通和基本食物配给的\textbf{收费};工业品\textbf{不通过货币}而是进行
  直接配置,工资以\textbf{实物}发放;对城市劳动力实行\textbf{军事纪律};\textbf{工人阶级的自治
    从属于等级制的控制};对反革命分子实施“\textbf{红色恐怖}”。

  战时共产主义政策可化为两阶段:前段时具有政治、战时经济方面的\textbf{客观合理性};后
  段的\textbf{军事管制}则是向共产主义过渡的一次\textbf{失败的主观尝试}。列宁自述:
  \begin{quotation}
    由于我们\textbf{企图过渡到共产主义}。到1921年春天我们就在经济战线上遭受了\textbf{严重
      的失败},这次失败比高尔察克、邓尼金或皮尔苏茨基使我们遭受的任何失败都
    要\textbf{严重得多,危险得多}。这次失败表现在:我们上层制定的经济政策同下层脱节,
    它没有促成生产力的提高, 而\textbf{提高生产力本是我们党纲规定的紧迫的基本任
    务}。\cite[184]{lenin42}
  \end{quotation}

\item 1921年初:列宁得出结论,“\textbf{要么是经济政策的根本改变,要么是他的政府被暴力
    推翻}”。开始实行\textbf{新经济政策},列宁将这一阶段视为“\textbf{过渡性的混合体制}”,
  并认为“现在我们处于必须\textbf{再后退}一些的境地,不仅要退到\textbf{国家资本主义}上去,
  而且要退到\textbf{由国家调节商业和货币流通}。”\cite[283]{leninlunshe} 。


  \begin{quotation}
    新经济政策并不是一开始就是完善的政策体系,而是通过\textbf{不断的摸索、实践}逐步
    完善起来的。这里最重要的是对\textbf{市场机制}的认识,也就是说,\textbf{在社会主义经济
      建设中是否允许引进市场机制。}\cite[144-145]{bigrussia}
  \end{quotation}

  新经济政策的主要内容有:恢复了农民对于农业\textbf{剩余交易的权力},将征收余粮改
  为\textbf{粮食税},商业、农业可以\textbf{雇佣劳工};对中小手工业企业实行\textbf{非国有
    化},\textbf{大力发展合作社};把企业交付\textbf{租赁},鼓励\textbf{合资企业}和敦促共产主义
  者“\textbf{学会贸易}”,\textbf{保留国有工业尤其是大工业的控制}而不松手,涌现一批辛迪
  加、托拉斯;恢复\textbf{商品、货币、市场、价值、金融机制},允许\textbf{私人资本}。

  联共(布)第十三次代表大会中,列宁说“正如所预料的那样,私人资本并没有投入
  生产(因为全部工业的 $\sfrac{9}{10}$ 已掌握在国家手中),而投入了商业。”中
  后期开始逐步\textbf{加强对私人资本和富农的打压},逐步加强\textbf{国家计划职能和集
    中化}。

  一个新资产阶层“\textbf{耐普曼}\footnote{耐普曼:俄文 \texttt{Нэпман} 的音译,字面意思是新经
    济政策的狂热爱好者,实际上是指新经济政策后城市中新兴起来的工商业私营业主。
    列宁用“耐普曼”一词表达对他们的嘲讽。}”产生了,列宁采取了\textbf{利用且限
    制}的态度,至30年代左右斯大林消除了耐普曼。

  1923年,因国家对工、农产品价格的调控过度,爆发“\textbf{剪刀差危机}”,工业品价格
  相对于农产品来说过高,富农、投机商利用这一政策作为工农中间商赚取差价,\textbf{农
    民不愿在市场出售粮食}。1927年末到1928年春又发生了“\textbf{粮食危机}”“\textbf{粮
    食罢工}”,农民藏起粮食,不愿把粮食交给国家,农产品供应不足。
\end{enumerate}


\section{新经济政策时期的争论}

苏联新经济政策前后历经了列宁和斯大林两位统治者,源于杜冈思想之\textbf{苏联必须大力
  发展大机器工业是党内绝大多数共识}。但围绕如何调节工农之间的强烈矛盾,避免工
农撕裂,对农业部门的经济支持力度多大,苏联发生多次激烈争论,形成派别之争。

% 对于苏联的研究常常注重列宁与斯大林新经济政策的不同之处,而轻视了他们之间
% 的\textbf{政策连续性}。了解这段时期内苏联政府的争论情况,有助于我们理解斯大
% 林中后期政策,也有助于对整个苏联社会主义产生深刻认识。

以下只介绍主要流派的主要思想以及以斯大林为首的联共(布)中央的应对:
\begin{enumerate}
\item 资产阶级知识分子和党派要求支持新经济政策,进而和平演变至资本主义。特别
  是“\textbf{路标转换派}”号召放弃与苏维埃政权的对抗状态,转而加入此时苏维埃社会主
  义建设中去。列宁采取了欢迎并警惕的态度\cite[2.1节]{yanyongfeinep},
  斯大林权力稳固后持谴责态度,并于1934年开始对其逮捕、清洗。\cite{wenyilubiao}

\item 1923年秋,针对新经济政策,\textbf{托洛茨基反对派}形成。

  托洛茨基提出“\textbf{工业专政}”,奥布拉任斯基提出“社会主义原始积累”等“\textbf{超工
    业化}”观点\cite[2.2节]{yanyongfeinep}。两种观点都基于错综复杂矛盾下\textbf{一国不
    可能建成社会主义,且具急迫性}。

  托派提出,工农业价格剪刀差问题,根本上乃是党的错误政策和错误领导之下的工农
  业产品价格比例失调,富商、耐普曼等私人资本从中获取绝大部分利润是次要原因。

  托派着重于从\textbf{工业分配领域}中获取利润,具体内容:进一步大力加强国家计划对国
  有化工业的扶持,进一步加速工业发展,工业地位要高于金融、农业,提升工人无产
  阶级政治地位及待遇、福利,排挤削弱私人资本,对富农、私人资本征收重税,贫农
  免税,用出口粮食的外汇来购买国外原材料和先进工业机器。压低农产品收购价格、
  抬高工业品价格。

  \textbf{随着工业的发展,农民转变为无产者、农业工人,国民经济发展将具有统一基础。结
  果是全面国有化和集中制},私人资本和前资本主义成分被消除。我们可以从中明显看
  到杜冈-巴拉诺夫斯基的影响。若非如此,则俄国国有工业发展无法获得足够剩余,俄
  国农民的阶级分化将加剧,私人资本会扩大市场影响,俄国将走上资本主义道路。

 \item  联共(布)中央多数和斯大林、布哈林对托洛茨基反对派提出批评。他们认为危机的
  原因是国家扶持下的工业恢复速度优先于农业恢复速度,合作社和国营商业未能有效
  起到工农联结作用,应该\textbf{优先帮助和支持农民经济的恢复}。新经济政策中本应以恢
  复工业品价格为目标的\textbf{托拉斯、辛迪加走向反面,错误利用垄断优势抬高流通价格,
    获得利润。}工业利润不应从之前的\textbf{分配}领域,而应从\textbf{生产}领域获得,如提高
  劳动生产率、降低杂费开支、缩减流通环节等等。认为托派意见低估了农民经济,\textbf{将
    农民经济看作被剥削的“殖民地”},这会有害于国家政治和经济。

  笔者认为,联共(布)中央此时论断更为正确性。托洛茨基反对派的意见正是人为应
  用的造成大量失地农民的资本原始积累,毫无疑问会进一步加大农村矛盾,破坏工农
  联盟,理论上过于简单莽撞和一厢情愿。但是斯大林在“粮食危机”后学习贯彻的正
  是这种“应用杜冈主义”的托洛茨基反对派意见。

  1925年4月联共(布)第十四次代表会议根据莫洛托夫报告决
  议\cite[538-550]{jueyi2}, 认为应当进一步\textbf{提高和恢复整个农民经济}。对歉收地区
  农户、贫困农户继续加强国家扶持。提高农业商品化,消除农村中的战时共产主义残
  余,停止以行政手段对付私营商业和富农等。
  \begin{quotation}
    需要采取法律的(特别是经济的)措施,向那些在农村中放高利贷和对贫农进行奴
    役性剥削的富农进行斗争。……党和苏维埃国家的基本的实际任务,就是通过发展
    农村合作化(机器协作社、供耕社、牲畜管理联合社等等)的办法全力促进劳动农
    户的联合……尽力支持和巩固大规模的国营农业——国营农场。国营农场应当成为
    经营大规模的先进经济的榜样,并给予周围农民以经济和文化上的帮助……继续支
    持集体农庄运动(农业协作社,农业劳动组合,农业公社等等),因为在联合起来
    的农户和雇农完全自愿参加的条件下,这一运动正在有效的发展着。”
  \end{quotation}
  具体措施有减税赋、增加农业贷款、降低农业贷款利率,降低工业品价格,将一些国
  家土地转交给农民使用等等。

\item 会议之后,以\textbf{季诺维也夫和加米涅夫为首,以列宁格勒为阵地的新反对派}形成。

  季诺维也夫和加米涅夫为首的新反对派不同意决议,主张加强与资本主义的斗争。他
  们认为1925年以来的新经济政策已经退却太多,国家资本主义走强,社会主义成分正
  在成长但远未成长,在整体向资本主义让步。富农力量依托政策变得过于强大,应当
  在农村挑起阶级斗争,依靠贫农消灭农村中的剥削阶级——富农分
  子。\cite[2.4.2节]{yanyongfeinep}

  1926年春夏,新反对派与托洛茨基反对派结盟,组成\textbf{反对派联盟}。

  反对派联盟观点主要还是托洛茨基式的,要求\textbf{对富裕农村地区加征税收,进一步加
    快工业化的发展速度,扩大日用品生产,提高日用工业品价格和工人工资等,“工
    人专政”、“社会主义原始积累”成为联盟共同认知。}

  可能因组织压力越来越巨大,反对派联盟为争取地位进行了一系列\textbf{跨越党组织的
    宣传和集会}。1927年9月份秘密印刷散布《反对派政纲》;将反对政府的意见扩大
  至国际社会;11月7日十月革命十周年纪念日,在莫斯科、列宁格勒组织游行示威,被
  政府驱散;11月14日,中央委员会和中央监察委员会将托洛茨基和季诺维也夫等人开
  除出党;11月16日,托派越飞自杀,临终遗言埋怨托洛茨基不如列宁果决、不敢为天
  下先。托洛茨基、季诺维也夫、加米涅夫在越飞葬礼上发表讲话,号召队列士兵对前
  军事委员会主席托洛茨基致敬“乌拉”,士兵无动于衷,这基本上宣告反对派联盟已
  无力翻盘。

  托洛茨基于1929年1月被驱逐出苏联,1940年8月在墨西哥寓所被暗杀。季诺维也夫和
  加米涅夫于1936年8月被枪毙。

\item 布哈林集团。

  布哈林具有早期联共(布)领导人所缺乏的温柔浪漫色彩。他认为当前社会主义还是
  落后的,整合资本主义的过程应该\textbf{慢一点},一步步来。

  1925年4月,他号召农民“\textbf{发财吧,积累吧}”,借助于富农经济的提升,国家从富
  农中取得资金来发展中农、贫农。这进一步退却的“新经济政策”思想立足点,在
  于“\textbf{我们还没有力量什么都自己干}”。\cite[368-371]{buhalinwen1}

  1926年2月,布哈林提出“总的说来,在我们这里,阶级斗争会趋于消失,但这并不意
  味着在一定时间内它不会尖锐起来……而相反地,会趋于尖锐化,因而在这个一定阶
  段,我们的任务就是要在这种尖锐的阶级斗争中以相应的方式行
  事。”\cite[18-19]{buhalinwen2} 这一观点被中央评价为“\textbf{阶级斗争熄灭论}”。

  布哈林总体上希望\textbf{工农平衡},农业是全部经济的基础。应当提高粮食收购价格,发
  展小农经济。相比于\textbf{集体农庄},\textbf{合作社}才是当前小农经济现实状况下,农民走向社会
  主义的\textbf{主要道路}。合作社的发展将为农民提供更多个人利益,表明社会主义体制的优
  越性,使农民自愿进入这一体系,富农合作社也将长入到这一体系,从而在未来将农
  民引入集体耕作。这一观点被中央评价为“\textbf{富农和平长入社会主义}”。

  1928年的粮食危机,使布哈林集团形成并明朗化。1928年1月,中央采取旨在打击富农
  和投机商的“非常措施”。1928年4月13日起,布哈林集团和斯大林矛盾公开化、尖锐
  化,双方贯彻和加强了之前的观点。斯大林坚持\textbf{工农业剪刀差,快速工业化,大力
    发展重工业,农业集体化,农业“贡税”论}。笔者赞同布哈林观点,此时斯大林就
  是“\textbf{托洛茨基主义}”。斯大林提出了坚持\textbf{农业“贡税”}的必要性:
  \begin{quotation}
    农民不仅向国家缴纳一般的税,即直接税和间接税,而且他们在购买工业品时还因
    为价格较高而多付一些钱,这是第一;而在出卖农产品时多少要少得一些钱,这是
    第二。这是为了发展为全国(包括农民在内)服务的工业而向农民征收的一种额外
    税。这是一种类似“\textbf{贡税}”的东西,是一种类似超额税的东西;为了保持并加快
    工业发展的现有速度,保证工业满足全国的需要,继续提高农村物质生活水平,然
    后完全取消这种额外税,消除城乡间的“\textbf{剪刀差}”,我们\textbf{不得不暂时}征收这
    种税。\cite[139-140]{stalin11}

    也许为了更加“慎重”起见,应当延缓重工业的发展,把主要是供应农民市场的轻
    工业变成我国工业的基础吧?无论如何不应当!这样做就是\textbf{自杀},就是破坏我国
    全部工业,连轻工业在内。这样做就是离开我国工业化的口号,\textbf{把我国变成世界
      资本主义经济体系的附庸}。
  \end{quotation}

  1929年11月,布哈林被解除联共(布)中央政治局委员和《真理报》主编职
  务。1937年布哈林以“人民公敌”的罪名被捕入狱并被开除出苏联共产
  党。1938年3月15日,布哈林与集团内李可夫等人被枪决。
\end{enumerate}

斯大林通过在政治对手之间的合纵连横,逐个扫清了他们。

笔者认为,除重要的个人权力欲望以外,以上所述几位对马克思颇有研究的苏联领导人
其实有一条共通的认识——他们都明晓苏俄历史缺少了重要的一环——\textbf{发达资本主
  义}!

布哈林与斯大林决裂的根本原因,也不是表面上斯大林、托洛茨基和联共(布)中央指
责他的“阶级斗争熄灭论”和“富农和平长入社会主义”,而是因为他们对于\textbf{苏联经
  济发展的急迫性}认识不同。

列宁、托洛茨基、中后期斯大林等人认为在苏联强烈内忧外患、时刻面临政权颠覆危险的紧急
情况下,必须抓紧改革,必须强力发展生产力!哪怕是以工农剪刀差为代价。而布哈林
希望的则是社会普适的、“渐渐地”发展。对于当前急迫性和发展快慢的认识不同,使
他们走向了最终的决裂。

\section{斯大林模式}

随着斯大林领袖地位的进一步稳固,斯大林模式开始逐渐成型。

\begin{enumerate}

\item “\textbf{围攻富农}”的政策受到一些富农激烈反抗——放火、煽动、阻碍粮食收购、甚
  至暗杀,“粮食收购危机”被认为是富农对苏维埃的反抗。1929年,政策转为\textbf{全盘
    集体化}和\textbf{消灭作为阶级的富农};全乡全村土地纳入集体农庄;全国3\%--5\%
  的农民被划为富农,不允许富农加入集体农庄,并将其土地、机器、财产收归集体农庄;
  将富农、反抗粮食收购和集体农庄的家庭大规模迁徙至偏远、恶劣地区。在实际执行
  过程中,富农家庭的划分被严重扩大。从1930至1932年全盘集体化结束,被迁徙富农
  家庭约为\textbf{40万户},人数约为\textbf{180万人}。\cite{xulongbinsu}

  1932-1933年,苏联爆发大饥荒,当前学界大多数认可乌克兰“大饥荒”死亡人数
  在\textbf{300万}左右,约占苏联饥荒中死亡总数的\textbf{一半}。\cite{wukelanjihuang}

  苏联体制下的官僚刻板追求数字政绩是富农迁徙扩大化以及大饥荒的主要原因之一。
  为什么在苏联这种极权体制下,容易产生无视大量人民生命安危和伦理道德的官僚化,
  这需要很大重视和反思。笔者认为,简单将其评断为意识形态宣传的群体化错误,而
  不从\textbf{个人、人性}上找原因,远不能达到全面和深刻的水平。两方面都要深刻反思。

  合作社系统所属机器拖拉机站和维修空间移交给全苏机器拖拉机站总管理局;扩
  大\textbf{预购合同制},1933年预购合同制转为\textbf{义务交售制},定额上交农产品,剩余产
  品归集体农庄和个人所有,使集体农庄和个人获得了更多利润。

  斯大林实现了苏联式的资本原始积累。

\item 斯大林模式也取得一些飞速发展,以下内容摘自多个文献。

  \begin{quotation}
    1931年苏联购买的机器设备约占世界机器设备出口总量的30\%,到1932年这一比重
    上升为50\%。1929-1933年,苏联用于购买机器设备的外汇开支达60.1亿卢
    布。\cite[277]{bigrussia}

    正当大萧条席卷西方资本主义经济时,苏联经济的增长率急剧提
    高。 1928 至 1937年间,工业生产增长 \textbf{3倍},从不足国民生产的1/3 发展到接
    近1/2。1937 至 1953 年间,工业再次增长\textbf{两倍}多,到斯大林去世时接近于总产
    出的 60\%。这一切只有通过\textbf{大规模投资}才会成为可能。每年平均有 20\% 以上
    的产出用于积累,\textbf{工人与农民的消费显著下降};工人的实际工资直
    到 20 世纪 50 年代初才再度达到 1928 年的水平,而农民的生活水平下降得更多,
    需要更长的时间恢复到原有水平。\cite[31]{mazhengshi2}

    (1933年1月10日,联席会议决议在对第一个五年计划总结中提到,)\textbf{苏联已从旧
      俄时代的落后的小农国家变为技术和经济最发达的先进国家之一}……\textbf{苏联已由
      农业国变为工业国},从而巩固了\textbf{国家在经济上的独立},因为苏联已经能够在
    本国的企业内生产大部分必须的设备。……\textbf{苏联已由小农国家变成了拥有规模最
      大的农业的国家}……\textbf{失业现象已经消失}……(因欧美经济大萧条影响)在美
    国,根据官方统计,但是加工工业中的在业人数,就由1928年的850万人减少
    到1932年的550万人。而根据美国劳工联合会的统计,美国整个工业中的失业人数,
    在1932年底已经达到\textbf{1200万}人……在英国,根据官方统计,失业人数
    由1928年的129万人增加到1932年的\textbf{280万}人。在德国,根据官方统计,失业人数
    由1928年的1376000人增加到1932年的\textbf{550万}人。\cite[321-329]{jueyi4}

    到1934年,\textbf{社会主义成分的比重},在国民收入中已占99.1\%,在工业总产值中已
    占99.8\%,在农业总产值中已占98.5\%,在商业企业零售商品流转额中已
    占100\%。\cite[5.4.2节]{huanglifunep}
  \end{quotation}

  这时候其实走的正是应用杜冈“\textbf{社会财富增加,社会收入(消费)减少}”的道路。

\item 1929年,除住房建设合作社所有的住房以外,一切住房都转变为\textbf{国家所有制},并
  逐渐形成党对住房事业的高度集中管理。并且国家对\textbf{住房建设合作社}的扶持逐年减
  少。

  1937年,第二个五年计划完成时,住房租赁合作社和住房建设合作社及其联社被撤销,
  由地方苏维埃和国家机关及工业企业直接管理国有住房。\textbf{百分之百国有化}。党高度
  集中的管理最终形成,成为高度集中的指令性计划经济的一个组成部
  分。\cite[第7章末]{huanglifunep} 。

  但是\textbf{工人的等级化以及官僚的等级化}在住房方面也有明显体现,其中官僚整体待遇
  比工人待遇要好得多。但斯大林时期官僚主义并未形成自成自治体系,斯大林倾向于
  瓦解分裂官僚体系,勃列日涅夫时期才是官僚主义巅峰。

\item 在考察受斯大林执政期间迫害致死和间接致死的具体人数上,常掺杂过多集团的功利
  目的或个人主观随意。有彻底妖魔化,也有为其辩护甚至美化的倾向,可谓光怪陆离。
  其实不管这个数字是800万还是2000万,\textbf{无论如何也无法否定斯大林时期存在暴力极
    权、反人道和邪恶的一面}。

  关于大清洗断代和具体迫害致死人数,笔者倾向于吴恩远的观点。吴恩远《苏联“大
  清洗”问题争辩的症结及意义》\cite{wuenyuanzhengbian}可谓是一篇雄文,推荐大家观
  看。1937-1938年确实是大清洗运动高潮,这点是共识。吴也在《苏联“三十年代大清
  洗”人数考》一文中列举斯大林其他时期政治镇压的数据。
  \begin{quotation}
    1937-1938年的大清洗运动虽然只有一年多时间,但仅以死刑犯为例,被
    判\textbf{死刑}的政治犯人数约 \textbf{68.2万},占1921-1940年判死刑的政治犯总人
    数\textbf{749421}人的91\%,占1930-1950年代被判死刑786098人的87\%,也就是约
    占“整个斯大林时代”被判死刑总人数的90\%,更占马文认定的大清洗年代被判死
    刑总人数的99\%以上(1935年仅有1229人判死
    刑,1936年为1118人,1939年为2552人)。
  \end{quotation}

  另外,1937-1938年,按吴的数据,以政治镇压罪名被逮捕的政治犯人数
  为\textbf{130-150万}。整个斯大林时期,被判刑的政治犯人数为\textbf{380万}左
  右。

  斯大林发起的大清洗及一系列扩大化运动,波及党政军及文化、科学、艺术行业,使
  苏联政治学、经济学发展从此迟滞,斯大林本人的极权统治得以建立。其积极性的一
  面就是通过这些恐怖统治手段,工业劳动纪律得以增强,农村反抗难以再兴风浪,官
  僚的反对意见日渐消失、高度围绕斯大林极权中心。“大清洗”的政策宣传使一些平
  民、贫民反倒对其充满热情和希望。苏联进入斯大林个人崇拜时期。

  据赫鲁晓夫和一些文章,斯大林临终前曾想\textbf{再次发动一场针对官僚的“大清
    洗”}……
\end{enumerate}

\section{早中期苏联科社实践总结}
\label{sec:sushijian}

笔者认为,列宁、托洛茨基、奥布拉任斯基、1928年后的斯大林其实具有着一贯性和连
续性。这种一贯性基于杜冈-巴拉诺夫斯基的理论和马克思历史唯物主义以及科社理论的
认识。

\textbf{马克思历史唯物要求社会主义的建立要以“生产力的巨大增长和高度发展为前提”,
  就是建立在资本主义发达条件下},这一点也被俄国十九世纪末、二十世纪初大部分党
派和知识分子认同。苏联领导人一方面具有对历史唯物主义的深刻认识,另一方面在十
月革命时便有着对历史唯物主义的背离。当然,这种背离在马克思论述科社时便已存
在。

\subsection{苏联领导人对马克思思想的正反应用}
\label{subsec:shenkerenshi}


十月革命胜利后,葛兰西立刻写了一篇热情洋溢的文章——《反<资本论>的革命》。文
章中观点可以总结为,\textbf{落后、物质条件不足的俄国所取得的革命胜利,战胜了教条的
  马克思历史唯物主义。}葛兰西错了,\textbf{苏俄正因违背了历史唯物主义才在建国之初就
  深深埋下这一祸根。}

在苏俄领导人仓促上台后,面对着极为恶劣的环境,没有丰富的资本主义物质条件,苏
俄无法成为一个真正的社会主义国家,在国内外的夹攻中甚至会在很短的时间内覆灭。
在资本主义发达、世界经济市场已经形成的情况下,一国也无法建成社会主义,对后发
达国家来说这更是艰难。列宁、托洛茨基、奥布拉任斯基、中后期斯大林均持有这种基
于历史唯物主义的共同认识。\textbf{因此他们更为急迫的想要获得资本发达,为此不惜代价。}


在农村问题上,二十世纪早期列宁真正想要的是沙皇俄国推动激进的“\textbf{美国式道
  路}”\cite{chenxintianamerica}。直接来说,就是列宁希望\textbf{土地国有化,更为彻底的
  消除传统地主、农奴主、土地贵族等落后生产方式,获取大量失地农民支持,然后国
  家将土地无偿或低价转给资本主义生产方式的农场主,实现资本发达化、大量农民无
  产化的发达资本主义生产方式}。如果再直接些表述的话,列宁希望沙俄采用马克思批
判的鲜血淋漓的\textbf{资本主义原始积累}。为此,俄国不仅可以实现资本发达的条件,也可
以大幅增进社会主义夺权的动力。

列宁认为,保守改良的“\textbf{普鲁士道路}”——人人皆有田可种,对于农民占
比 $\sfrac{4}{5}$ ,且贫农为绝大多数的小农经济苏俄来说,只能是生产方式的倒退,
不管这田地是归农民个人私有还是全部归国家。

如前所述,列宁除战时共产主义阶段的失败尝试以外,一直希望实现的是愈加强大的国
家资本主义,这也是马恩学说中的“\textbf{物质基础}”。

至于托洛茨基,他其实对工农矛盾有着深刻认识,但寄希望于“不断革命论”:
\begin{quotation}
  落后俄国的工人阶级有可能先于工业发达国家的无产阶级获取政权。然而,在托洛茨
  基看来,在孤立的状况下,是无法维持这一政权的。\textbf{最终,必然和农民发生冲突},
  因为\textbf{农民对无产阶级专政的支持只限于完成土地革命。维护无产阶级统治所要求的
    集体主义措施,将导致与农民的分道扬镳},无产阶级掌握政权措施的结果,将会同
  时削弱这种统治的非无产阶级基础。在此意义上,\textbf{土地问题既是俄国社会主义革命的
  最大帮手,也是它的一个主要挑战者。}

  不断革命论结果陷入矛盾之中,只有革命超越了民族的界限,并且\textbf{成为世界范围内
    持续的“不断的”革命时,这个矛盾才可能解
    决。}\cite[224-225]{mazhengshi}
\end{quotation}

托洛茨基试图通过“\textbf{不断革命论}”:将阶级斗争推入扩大至西方,西方无产阶级力量
强大后,世界无产阶级联盟,反过来哺育落后俄国,来消解社会主义革命中的剪刀差问
题。但托洛茨基对于西方无产阶级胜利的论断相当不成熟,一厢情愿。

前文已经论述过杜冈-巴拉诺夫斯基限制消费扩大生产的观点,托洛茨基“超工业化”的
观点,之后斯大林也在相当程度上吸取了托派观点。笔者在这里作进一步说明。

\textbf{世界市场资本主义价值规律仍在制约苏俄,斯大林多次承认这点,克里夫和沃勒斯坦
  也是因此论述苏联的国家性质是国家资本主义。}

苏联作为一个贫农、中农占绝大部分人口比的落后小农经济国家,其意识形态面临世界
政治、经济封锁、军事侵犯以及威胁,如果苏联(资本)经济不发达,则无法免除沦
为“\textbf{资本主义国家附庸}”的命运,甚至“\textbf{等同于自杀}”。而国内层出不穷的农业
危机、农民叛乱、官僚离心也常使苏维埃政权濒临毁灭的危险,从而使苏联领导人在内
忧外患之下高度精神紧张和敏感。

另外对于苏联这一后发国家来说,已不具备早先英美等资本主义发达国家类似的优势条
件。发展轻工业是西方资本主义发达国家的早期路径。轻工业企业使用劳动力相对少,
并且西方轻工企业数量多,发达程度高。对于苏联这类后发国家来说,几乎已经无法再
从轻工业中获取世界市场的价值优势。

苏联重工业本身也不占价值优势,但是通过压低农业品价格、实现农业集体化、从农民
中游离出农业工人和工业工人,减少农民和工人收入,从古拉格和大迁徙中获取廉价劳
动力及集约化土地等一系列措施可以获取大量超额利润。再将这部分利润附加上大量国
家资本,重点投入到资本密集、人力密集的重工业,辅以计划和联合,将\textbf{使苏联重工
  业取得世界市场的价值优势}。从而重工业可以反哺农业集体化以及苏联经济、工人农
民,苏联将成为强于资本主义国家的社会主义国家。

不管用怎样的理论去维护“社会主义原始积累”、“贡税”,它所遵照和向往的仍绝对
属于获取“\textbf{资本主义原始积累}”,只是这种积累前面冠以了“\textbf{国家}”,由苏维埃
社会主义国家掌握一切生产资料,对资本予以调节,对工农有一定的回馈。苏联其实不
可避免的成为恩格斯五六十年前所说的“\textbf{总资本家}”。

斯大林的政策,即使在政权稳固、全盘集体化、重工业得以迅速发展,成为欧洲第一强
国、世界第二强国后,仍然忽略了工农收入、特别是农民的等比例发展。在他生命的尽
头,仍想再次激进加速重工业发展……

\subsection{苏联的国家性质}

如果具有政治经济学基础和客观态度,便知苏联确实是一种\textbf{国家资本主义与社会主义
  的混合经济,其中国家资本主义占据绝对主导地位},并且斯大林采取了暴力集权统治。全
盘否定他的小社会主义性质或大资本主义性质,都是显著错误。如果将其视
为17、18、19世纪的生产方式,则更是愚蠢无知,或者包藏祸心。如前所述,其实苏联
在相当多层面尊重资本价值规律、尊重历史唯物主义,才走向这样道路。

整个过程像是一个笑话。本为反对资本主义鲜血淋漓“资本积累”的国家,因强烈的革
命理想,居然走上在某些方面比资本主义还要资本主义的“原始积累”道路。虽然无法
否认斯大林全世界无产阶级胜利的革命理想……


斯大林模式中社会主义指令性经济这只“\textbf{看得见的手}”总有一天要面对资本主义市场
这只“\textbf{看不见的手}”的挑战。计划经济中的官僚、技术专家制定全国计划的各方面欠
缺都将强烈暴露出来。软预算约束\footnote{软预算约束:指当一个经济组织遇到财务上的困境
  时,借助外部组织的救助得以继续生存这一经济现象。}、一些部门为完成硬性计划指
标牺牲质量、各产业部门优先级的人力调节、官僚的贪腐等……其实在马克思、恩格斯
的科学社会主义中,也并不是由一个鲜明的上层国家机器来行使“计划”职能……回首
恩格斯在《反杜林论》中的两段话,读者们,你们怎么想呢?
\begin{quotation}
  \textbf{资本主义生产方式}起初排挤工人,现在却在\textbf{排挤资本家}了,完全像对待工人那
  样把他们赶到过剩人口中去,虽然暂时还没有把他们赶到产业后备军中去。

  但是,\textbf{无论向股份公司的转变,还是向国家财产的转变,都没有消除生产力的资本属
  性。}在股份公司的场合,这一点是十分明显的。而现代国家也只是资产阶级社会为了
  维护资本主义生产方式的一般外部条件使之不受工人和个别资本家的侵犯而建立的组
  织。

  \textbf{现代国家,不管它的形式如何,本质上都是资本主义的机器,资本家的国家,理想
  的总资本家。它越是把更多的生产力据为己有,就越是成为真正的总资本家,越是剥
  夺更多的公民。工人仍然是雇佣劳动者、无产者。资本关系并没有被消灭,反而被推
  到了顶点。}
\end{quotation}

而在我们去考察历史上各个社会主义国家时,甚至可能不得不去思考这些国家,是否
是“\textbf{党取代阶级}”、或“\textbf{党的机关取代了党}”、或\textbf{知识分子和农民联盟而不是
  无产阶级}取得胜利、或\textbf{其他强力社会主义国家军事和政治压力下的本国改制}。马
克思的科学社会主义理论可能有个巨大错误——\textbf{工人阶级并不代表更为先进生产力和
  生产关系}。

其实列宁自己也否定工人阶级至上主义,所以要由先锋队引导。
\begin{quotation}
  列裴伏尔同意列宁的观点,工人阶级是革命行动的基础,但自身有局限性,革命依靠工人
  阶级和其他社会阶级和阶层的联盟,而且 “工人阶级只有在各种力量处于一种特定的
  平衡并且有一种富有首创精神的\textbf{政治思想引导}时,它才发挥革命性作用”。\cite{zhangxiaoyi}

  (列裴伏尔就此认为,)因此革命只能随形势而动,也就是在某些阶级关系中完成,
  这时农民与知识分子都进入这些关系的整体中来。\textbf{工人阶级本身并不是革命的,也
    不造就革命,也不为自己革命。个人阶级的革命性本质或革命的自然属性并不存
    在。}\cite[127]{xingcun}
\end{quotation}

列宁在沙俄时期就因其激进的社会主义理论受到党内外的质疑,其中亚·萨·马尔丁诺
夫曾引用恩格斯《德国农民战争》中的一段历史唯物主义文本来反对列宁观点,现在看
来,恩格斯这段话仍是振聋发聩,有效的切中了问题所在,并具普世意义,有必要全部
引用。

\begin{quotation}
  对于\textbf{激进派的领袖}来说,最糟糕的事情莫过于在运动还没有达到成熟的地步,还没
  有使他所代表的阶级具备进行统治的条件,而且也不可能去实行为维持这个阶级的统
  治所必须贯彻的各项措施的时候,就被迫出来掌握政权。

  他\textbf{所能}做的事,并不取决于他的意志,而取决于不同阶级之间对立的发展程度,取
  决于历来决定阶级对立发展程度的物质生活条件、生产关系和交换关系的发展程度。

  他\textbf{所应}做的事,他那一派要求他做的事,也并不取决于他,而且也不取决于阶级斗
  争及其条件的发展程度;他不得不恪守自己一向鼓吹的理论和要求,而这些理论和要
  求又并不是产生于当时社会各阶级相互对立的态势以及当时生产关系和交换关系的或
  多或少是偶然的状况,而是产生于他对于社会运动和政治运动的一般结果所持的或深
  或浅的认识。

  于是他就不可避免地陷入一种无法摆脱的进退维谷的境地:他\textbf{所能}做的事,同他迄
  今为止的全部行动,同他的原则以及他那一派的直接利益是互相矛盾的;而他\textbf{所应}做
  的事,则是无法办到的。总而言之,他被迫不代表自己那一派,不代表自己的阶
  级,\textbf{而去代表在当时运动中已经具备成熟的统治条件的那个阶级}。他不得不为运动
  本身的利益而维护一个\textbf{异己阶级}的利益,不得不以\textbf{空话和诺言}来对自己的阶级进行
  搪塞,声称那个\textbf{异己阶级}的利益就是本阶级的利益。谁要是陷入这种窘境,那就无
  可挽回地要遭到失败。\cite[303-304]{maenwen3}
\end{quotation}

按照马恩唯物主义来说,从没有什么\textbf{救世主},社会的形态和发展使个人只可有限改进,
违背历史唯物,则将面临失败命运。对于\textbf{救世主}的幻想,往往来源于人类自欺的迷梦。
每个人生活在这风雨雷电变幻莫测的人世间,都会有这样那样不切实际的迷梦,这迷梦
来自于个人试图超脱现实困境的希望,是种自我保护机制。但若要真正的认清和改良现
实,恐怕就要求人自身尽量将这迷梦的成分缩微,直面惨淡人生,正视淋漓鲜血,为客
观真实腾出更多的位置。



% \subsection{关于“耐普曼”的中国认识}

% \unsure[inline]{本章节可能作为本章附录。因笔者是业余初级民科,学术方面可能错误% 多多,遭人耻笑。欢迎提供批评指导建议}

% 与霍华德和King对“耐普曼”的强烈批判态度不同,在知网搜索国内“耐普曼”学术论文,% 结果不多,\textbf{思想高度统一},且与\textbf{苏联八九十年代论文}高度统% 一,\textbf{普遍以为苏联新经济政策对“耐普曼”的限制,违反了资本规律,所以导致新% 经济政策失败,并对苏联社会主义进程产生负面影响。}尤以吴恩远1987年论文《论耐普曼% 的组成、性质及作用》\cite{wuenyuan}为重,但是吴恩远在其论文中\textbf{直接将所有小% 商、小业主、小资本家、大资本家全部划为“耐普曼”阶级},这种粗暴的划分笔者认为不% 妥。然后以郭春生2008、2009年发表的两篇几乎一致内容的论文为重,本文选用2009年发表% 且内容更为详实些的《新经济政策时期“耐普曼”的发展问题研究》。在此文中,郭春% 生\textbf{只提私营业主发展},并且认为国营企业进展远远不如私营业发展,直接与沈志华% 书中历史数据相悖。吴恩远、郭春生均没有过多批判耐普曼中在工农两头赚取差价,造成不% 良影响的投机资本成分。笔者认为两者论文均不完整之处,有待商榷。/unsure[inline]{斯% 大林时期取得的成就(带有较多缺陷)是否也能说明这种论断是不足够全面的呢?}
% \begin{quotation} % \textbf{在商业发展中国营成分起决定性的作用},其1925/26年上半年的商品流转量% 比1921/22年下半年\textbf{增长24.6倍},而私人商业仅\textbf{1.8倍}。商业的社会% 结构也发生相应变化。1922年下半年私商(主要是原“背口袋的人”(沈志华此处背口% 袋的人即是耐普曼))的成分几乎占商品流转的73\%,国营商业约占18\%,合作社% 占9.5\%。三年半以后,商业中私人成分的比重降为25\%,国营上升为57\%,合作社% 为18\%。私商主要把资本用于零售商业,国营企业主要是批发,而合作社则两者兼有。

% 在批发商业中,国营商业和合作社占据优势地位。托拉斯联合起来的国营工业是市场工% 业品的主要供应者。1922年为组织工业品的销售和生产企业的原料供应,开始建立由各% 托拉斯联合起来的辛迪加,以协调各托拉斯的商业和供应工作。股份公司在批发商业中% 占有一定的地位。建立股份公司的最初目的是吸收本国和外国私人资本。同时也建立纯% 粹的国营股份公司。1924年10月1日已经有82个股份公司在活动,27个公司在筹备。开展% 业务的股份公司有固定资本1.18亿卢布,其中国营组织占86.2\%,合作社和社会组织% 占1.9\%,私人资本占11.9\%。
% \end{quotation}

% 郭春生同样参与了《新经济政策时期的苏联社会》的编纂工作,负责第六章《耐普曼的兴起% 于消亡》。在此章节中,郭春生对于苏联政策的批判力度远小于2009年论文。

% 是学术能力不足?亦或是被组织规诫?亦或是沈志华数据错误?亦或是笔者太异想天开,% 毕竟笔者是初级民科?笔者认为,\textbf{个人社会学}正在于社会原子个人,尤其是自发% 的、无组织、甚至是中底层个人的无畏可以突破一些规诫,所以会有它的作用。个人社会% 学的难题在于要求个人要有较强科学理性态度,或许我不具备所以才会质疑两位教授。



% TIPS:% 还需要后退,不仅要退到国家资本主义上去,而且要退到由国家调节商业和货币流通,我们% 的任务就是经商做买卖。

% 布尔什维克党经常说,在国内战争中同农民建立了政治联盟,而缺乏经济联盟。然而,缺乏% 经济联盟作基础的政治联盟是不牢靠的。

% 十月革命后,布尔什维克党把全部希望寄托在世界革命上,指望西方先进国家立即爆发无产% 阶级革命,建立无产阶级政权,支持落后俄国的革命和建设,


% 之所以单独提起这五种成分,是因为西方一些莫名其妙、富有明显政治功利色彩的所谓% 教授常常脱离世界实际情况,脱离“封建”“资本”语境去刻意抹黑苏联,而不是站在% 客观科学理论基础上去批判。就当时情况来看,真正成熟完全的资本主义国家只有美国,但他% 们只将苏联体制归于

% 1917年的俄国,形势复杂、风云变幻、历史吊诡,甚至难以用常识和理智去理解。以下内容% 整理自《一个大国的崛起与崩溃》。\cite{bigrussia} \unsure[inline]{这段内容是不是要% 全部删除呢?一是因为超大量引自沈志华,是否不妥?二是与中心思想似乎偏离,望批评建议。}
% \begin{quotation}
% \begin{enumerate}
% \item 1917年2月23日到3月2日(俄历),沙皇制度在8天之内迅速土崩瓦解。一切都如此突% 然,如此出人意料,以至于到现在仍被称为“二月革命之谜”。

% \item 2月27日\textbf{杜马临时委员会}成立……革命初期,临时政府把地方自治机关作% 为地方政权的唯一基础。在二月革命从首都向外省发展时,地方自治机关成为临时政府% 的权力在地方上的支柱。地方自治机关成为临时政府的权力在地方上的支柱。

% \item 但这个决定也引起了许多地方群众的强烈不满。他们认为,任命那些名声不好的地% 方自治会议主席担任政治委员是对革命的嘲弄……(3、4月份,民众由抗议升级至自发% 逮捕地方政治委员及解除其职务。)要求一切政府官员均由选举产生的呼声越来越高。

% \item 社会执行委员会出现于二月革命期间,并很快遍布全国,在省、县、区各级积极活% 动。同地方自治机关相比,\textbf{社会执行委员会是群众自发的组织},具有更广泛的% 社会基础,政治立场也更为激进。3月中旬,临时政府决定省和县的政治委员由选举产生,% 力图争取社会执行委员会的支持。

% \item (二月革命后的)布尔什维克在组织上是比较弱的,总共只有2.3万名党员,在彼得% 格勒只有2000人左右,许多地方组织尚未恢复。……当时,在彼得格勒,布尔什维克有% 两个领导中心:一个是中央俄罗斯局,另一个是彼得堡委员会。它们在如何对待临时政% 府和仍在进行的战争问题上持不同立场。

% \item 现在,国内布尔什维克党内虽然在具体问题上还存在分歧,但占主导地位的观点是:% 俄国目前的革命是\textbf{资产阶级民主革命},它的目标\textbf{不可能是社会主义共% 和国},因为对于落后的俄国来说,资本主义是一个\textbf{不可避免的阶段}。正如% 《真理报》复刊后第一期上的文章所说,革命的“根本任务是实行民主共和制”。

% \item 在很快确认了(二月革命)这个事实后,列宁受到极大震撼,他的思想发生了急剧% 的转变。……他还以《\textbf{远方来信}》的形式为《真理报》写文章,表达他对时局% 和党的策略的看法,但列宁的观点显然是那些从流放地回来的布尔什维克领导人所不能% 接受的。在列宁回国前,《\textbf{真理报}》只发表了他的一篇文章,而且\textbf{把% 其中尖锐批评临时政府、孟什维克和社会革命党的内容删去了}。……4月3日回到彼得% 格勒。列宁一到彼得格勒就批评前去迎接他的加米涅夫……第二天,列宁在布尔什维克% 党的工作者会议和出席全俄苏维埃会议的布尔什维克代表与孟什维克代表联席会议上作% 了关于战争和革命问题的报告,这个报告的提纲后来即以《四月提纲》而著名。……在% 列宁发表演说的会议上,\textbf{公开表态支持他的只有柯仑泰},但她的发言“引起的% 只是讽刺、嘲笑和喧闹”。有一些老布尔什维克甚至因此产生不满而转入了孟什维克的% 队伍。……在彼得堡委员会4月8目的会议上,列宁的提纲被交付表决,结% 果\textbf{以13票反对、2票赞成、1票弃权被否决}。

% \item 4月8日开始在党组织中就列宁的提纲进行辩论。列宁的提纲首先得到了\textbf{普% 通工人党员的支持},他们比较容易地接受了关于实现革命转变、建立无产阶级专政的% 思想。\textbf{在彼得格勒和莫斯科的党委这一层反对}列宁的提纲,但\textbf{多数区% 级组织和所有基层组织都拥护}。4月中旬开始,彼得格勒、莫斯科等地方党组织相继% 召开代表会议,通过了支持《四月提纲》的决议。但在\textbf{布尔什维克党的领层% 导}中,仍然存在强烈的\textbf{反对}意见。

% \item 根据马克思主义的一般原理对俄国社会进行分析,显然只能得出\textbf{在当时的% 俄国实现社会主义的物质条件尚未成熟的结论。}

% \item 布尔什维克接受了列宁《四月提纲》的思想后,开始为把民主革命转变为社会主义革命而斗争。

% \item 在历经四月危机、六月危机后,1917年7月初,俄国发生了又一场严重的政治危% 机。3~4日,彼得格勒的工人和士兵开始了反对临时政府、要求“一切政权归苏维% 埃”的运动,人数最多时达到50万左右。4日晚上至5日凌晨,支持苏维埃和临时政府的% 军队平息了运动。事后官方正式的侦查结果把\textbf{七月事件}定性为布尔什维克受德% 国指使挑起的暴动,其目的是破坏俄国的军事努力以有利于德国及其盟国。而布尔什维% 克则断然否认这一指控,

% \item 近年来不少俄罗斯史学论著通常把七月事件视为布尔什维克组织的\textbf{一次不% 成功的夺权活动}。

% 实际情况是,\textbf{七月事件既具有一定的自发色彩,又明显受到布尔什维克的影响% 以至于具体领导};它既是群众反对临时政府、要求苏维埃掌握政权的运动,也是布尔% 什维克夺取政权的尝试。之所以如此复杂,是因为在七月事件的过程中,布尔什维克党% 内存在着明显的\textbf{意见分歧和步调不一},即便是党的中央委员会也一再犹豫,多% 次改变立场。

% \item 在\textbf{7月3日一天里},布尔什维克中央对运动的态度随着局势的发展% 而\textbf{三次变化},从\textbf{试图制止}(据托洛茨基说,“没人料到此事,也没% 人希望发生此事”。),到\textbf{领导运动}(晚上7时,在塔夫利达宫召开苏维埃工% 人部会议,加米涅夫表示支持运动。彼得格勒委员会成员、全市代表会议以及各部队和% 工厂的代表,共同通过一个决议:不再试图阻止群众,而要领导运动……),再% 到\textbf{取消支持}(孟什维克和社会革命党控制的中央执行委员会的态度是明确而又% 坚决的,拒绝要求,问题应交由中央执行委员会讨论,夜深,布尔什维克领导层取消几% 个小时前发出的“和平示威的号召”)。

% \item 7月4日,运动出人意料地重新发展起来,有了更大的规模,而且很快失去了控% 制。……上午11时左右开始,陆续有一些部队走上街头。参加游行示威的人数达到% 了50万左右。……武装的克朗施塔得水兵、士兵和工人约一万人……列宁在4日一早乘坐% 早间列车去彼得格勒,……\textbf{克朗施塔得水兵包围了大楼,并要求与列宁见面。% 列宁起先拒绝出面},但后来还是不得不向坚持己见的克朗施塔得布尔什维克让步,同% 意与水兵见面……(列宁)表示相信“全部政权归苏维埃”的口号终将实现,最后又呼% 吁克朗施塔得人克制、坚强和遵守纪律。

% \item 7月4日晚上,政府情报部门邀请中立的卫戍部队各团代表到总参谋部,竭力向他们% 证明布尔什维克拿了\textbf{德国人}的钱、彼得格勒街头的武装暴动受到\textbf{德国% 人}策动……7月5日凌晨1时,原先保持中立的一些卫戍部队团队到苏维埃所在地塔夫% 利达宫表达对苏维埃领导层和临时政府的支持。早晨6时左右,布尔什维克的《真理报》% 编辑部被捣毁。彼得格勒的街道很快恢复了常态。到7月6日早上,工人们复工了。

% \item 5日凌晨2~3时,布尔什维克党中央委员会通过决议,\textbf{呼吁工人和士兵停止% 示威}。……七月事件使布尔什维克的力量受到沉重打击,一些布尔什维克活动家被控% 通敌遭到逮捕;列宁因受通缉被迫转入地下;从4月以来党员人数的增长也停止了。

% \item 七月事件过后,李沃夫辞去总理职务。\textbf{新组成的以克伦斯基为首的第二届% 联合内阁}中有8人来自社会革命党、人民社会主义党和孟什维克,7人来自立宪民主党% 等自由主义政党。

% \item 7月9日,\textbf{工兵代表苏维埃中央执行委员会与农民代表苏维埃执行委员会联% 席会议}根据策列铁里的建议通过了一项决议,\textbf{宣布克伦斯基政府是“拯救革% 命的政府”并拥有“无限权力”}。

% \item 8月中旬,临时政府在莫斯科召开了规模盛大的国务会议,希望动员全社会力量再造% 俄国。布尔什维克持抵制立场,未派代表参加。

% \item 8月20日,立宪民主党中央以多数票通过了支持建立军事专政的决定。米留可夫认% 为,“生活将迫使社会和人民接受关于外科手术不可避免的思想。”他断言,克伦斯基% 将同\textbf{科尔尼洛夫}妥协,因为他“别无选择”。

% \item 8月24日,\textbf{科尔尼洛夫}把克雷莫夫指挥的第三骑兵军从前线调往彼得格% 勒,\textbf{试图控制首都,建立军事独裁政权}。

% \item 于是,克伦斯基通电全国,宣布科尔尼洛夫为反叛,并解除了他的总司令职务。来% 自苏维埃、工会和包括布尔什维克党在内的\textbf{各社会主义政党的代表成立了“人% 民同反革命斗争委员会”},携手平息叛乱。

% \item 9月1日,\textbf{科尔尼洛夫在大本营被逮捕},叛乱遭到彻底失败。当% 天,\textbf{克伦斯基}组成了一个\textbf{没有立宪民主党人参加的五人执政内阁并亲% 任总司令},同时\textbf{宣告俄罗斯为民主共和国}。次日,\textbf{全俄中央执行% 委员会拒绝了布尔什维克提出的把政权交给苏维埃的决议案},宣布支持克伦斯基的执% 政内阁。

% \item 由于八月叛乱的失败,国内政治格局发生了重大变动,在总体上出现了\textbf{左倾化}。\textbf{极右翼力量}因组织和参与叛乱而受到\textbf{毁灭性打击},事实上不可能再参与政治角逐。\textbf{主要的自由主义政党立宪民主党因与军事叛乱有牵连而名声扫地}。

% \item 8月底,布尔什维克提出的由苏维埃接管权力的决议案被彼得格勒苏维埃和莫斯科苏% 维埃所接受。9月9日,彼得格勒苏维埃以压倒多数\textbf{通过了对主席团的不信任案},% 齐赫泽、策列铁里等被迫辞职。稍后,\textbf{托洛茨基当选为彼得格勒苏维埃主% 席}。……被称为“\textbf{苏维埃的布尔什维克化}”……

% \item 布尔什维克经历了七月事件打击之后仅仅过了两个月时间,就恢复甚至加强了力量,% 到9月初控制了彼得格勒、莫斯科和其他一些城市的苏维埃,迅速地接近了政权。

% \item 9月3日,全俄工兵代表苏维埃中央执委会和全俄农民代表苏维埃执委会联合作出决% 定,“召集一切民主组织和地方自治民主机关的代表大会,以解决政权组织问题,这一% 政权应能把国家引到\textbf{立宪会议}”。9月14日,全俄民主会议召开。全俄民主会% 议\textbf{排除了资产阶级分子},\textbf{1000余名代表}均来自苏维埃、合作社、自% 治机关、工会、土地委员会等民主组织,几乎都分别属于某个社会主义政党或派别、团% 体。\textbf{布尔什维克只在少数大城市、芬兰和波罗的海舰队有较大影响,因此只% 有89名代表参加。}

% \item 在10月初召开的\textbf{预备议会}上,托洛茨基言辞激烈地要求把政权交给苏维埃,布尔什% 维克代表退出了会议。列宁在9月末或10月初回到了彼得格勒,把发动武装起义夺取政权% 的问题提上了日程。10月10日中央作出关于武装起义决议的依据,

% \item 10月20日,彼得格勒苏维埃以德军已经逼近首都为理由,成立了\textbf{革命军事委员会},作为指挥起义的机关。

% \item 克伦斯基在10月24日的预备议会会议上曾要求获得特别授权以对付布尔什维克暴动,% 但\textbf{未获同意}。深夜,预备议会以微弱多数通过了由\textbf{孟什维克领导人唐% 恩}起草的决议案,其中要求政府立即向盟国建议举行和平谈判、立即把土地交给土地% 委员会、尽快召开\textbf{立宪会议}。唐恩认为,这是在最后一分钟向政府指明的可能% 得救的道路,因为这些措施的实行将使布尔什维克失去立足点。唐恩和郭茨立即赶到冬% 宫,要求\textbf{克伦斯基}马上公布这一决定,但遭到\textbf{拒绝}。

% \item 由于掌握了彼得格勒绝大部分武装力量,布尔什维克领导的武装起义进展顺利,几% 乎没有遇到真正的抵抗。整个起义过程中,一共\textbf{死6人,% 伤50人}。\textbf{25日晚上开幕的苏维埃第二次代表大会}面临政权更迭的既成事实。% 在代表大会上,布尔什维克和它的盟友左派社会革命党人及其支持者居于多数地位。作% 为对布尔什维克武力夺权的抗议,51名社会革命党、孟什维克和崩得的代表在暴风雨般% 的喧嚣中退出了大会。

% \item 在只剩下布尔什维克及其支持者的大会上,通过了《告工人、士兵和农民书》,宣% 布代表大会已经掌握政权,规定各地全部政权一律归当地工兵农代表苏维埃。大会通过% 的和平法令和土地法令在很大程度上是对既成局面的承认,因为军队不愿打仗,事实上% 已经瓦解;农民早已在自行夺取并分配地主土地。8个月来,临时政府就是因为在这些最% 迫切的问题上拖延不决而丧失了大多数群众的支持,而布尔什维克则从和平和土地的口% 号中获得了力量。

% \item 苏维埃二大决定,在立宪会议召开之前,成立苏维埃政府即人民委员会来管理国家。因左派社会革命党领导人没有接受加入政府邀请,成立了由清一色的布尔什维克组成的政府。

% \item 从1917年3月2日到1918年1月6日,\textbf{立宪会议问题}是影响群众的情绪、政党% 的活动、政府的政策以至俄国革命进程和俄罗斯国家发展方向的重大问题之一。

% \item 1917年10月27日,人民委员会通过决议,明确立宪会议选举应在预定日期11月12日% 进行。从逻辑上讲这个日期对\textbf{布尔什维克是很有利的,因为它已掌握了政权}。% 立宪会议选举共选出了715名代表,其中社会革命党370名,\textbf{布尔什维% 克175名,}左派社会革命党40名,孟什维克15名,立宪民主党(人民自由党)17名,% 另有一些其他党派和民族组织的代表。按照党派提出的名单进行的选举,\textbf{基本% 上反映了俄国社会政治力量的对比关系。}

% \item 11月8日,布尔什维克就讨论了驱散立宪会议的可能性问题,并且确认采取这样的行% 动不会引起\textbf{左派社会革命党人}的反对。在获悉最后选举结果后,列宁立即表% 示:“一切权力归立宪会议”是\textbf{反革命口号},“立宪会议如果同苏维埃政权背% 道而驰,那就必然注定要在政治上死亡”。苏维埃政权随即采取了针对立宪会议的密集% 措施。

% \item 11月23日,根据人民委员会的命令,逮捕了全俄立宪会议筹备委员会中的立宪民主% 党和社会革命党成员。11月26日,在临时政府确定的立宪会议召开日期11月28日的前两% 天,人民委员会决定,立宪会议第一次会议召开的条件是:根据全俄选举委员会政治委% 员乌里茨基的邀请到达彼得格勒的全俄立宪会议代表多于400人,会议只能由人民委员会% 授权的人士宣布开幕。

% \item 晚上,人民委员会通过“关于逮捕反革命内战祸首的法令”,宣布立宪民主党为人% 民敌人的党,其领导人必须逮捕并送交革命法庭审判,责成地方苏维埃对该党进行特别% 管制。11月29日公布的“关于立宪民主党领导的资产阶级反革命暴动的政府公告”强调,% 无论要付出多大的代价,资产阶级的叛乱都将被镇压。

% \item 12月1日,逮捕数十名立宪民主党活动家,其中包括当选的立宪会议代表,立宪民主% 党事实上被取缔。当天,人民委员会决定罢免全俄立宪会议选举事务委员会主席阿维诺% 夫和20余名委员,由乌里茨基负责管理全俄立宪会议选举事务委员会的一切事务。人民% 委员会还宣布,立宪会议代表必须在领导全俄立宪会议选举事务委员会的政治委员那里% 登记并取得塔夫利达宫办公室发放的临时证件。12月20日,人民委员会颁布法令,确% 定1918年1月5日在代表不少于400人的情况下召开立宪会议。12月23日,人民委员会宣布% 在彼得格勒实行战时状态,忠于布尔什维克的部队被调入首都。


% \end{enumerate}
% \end{quotation}




% 布尔什维克的意思是“多数派”,孟什维克的意思是“少数派”。但这里的多数和少数指的% 是《火星报》编辑部领导层的构成,在俄国国内革命力量来说,孟什维克占绝对多数。

% 二月革命推翻沙皇后,布尔什维克、社会民主党、孟什维克% 1917年2月23日到3月2日
% \begin{quotation} % 1917年2月23日到3月2日(俄历),沙皇制度在8天之内迅速土崩瓦解。一切都如此突然,如此出人意料,以至于到现在仍被称为“二月革命之谜”。\cite[26]{bigrussia}

% 二月革命后的俄国处于历史的十字路口。布尔什维克与俄国其他政治力量一样,都面临历史性的抉择。布尔什维克国内组织依据建党初期就已明确的革命理论,准备走上资产阶级民主共和国框架内的\textbf{合法反对派}之路。而列宁回国后从根本上扭转了这一趋势,把党领上了夺取政权、建立无产阶级专政的道路。\cite[42]{bigrussia}

% 1917年二月革命推翻沙皇制度之后,俄国主要政治力量达成协议,由立宪会议来决定国家治理形式并解决和平、土地、民族等重要问题;在立宪会议召开之前,成立临时政府管理国家。从1917年3月2日到1918年1月6日,立宪会议问题是影响群众的情绪、政党的活动、政府的政策以至俄国革命进程和俄罗斯国家发展方向的重大问题之一。\cite[75]{bigrussia}


% 在布尔什维克党内,主张公开同政权对抗、发动武装起义的声音越来越高。在10月初召开的预备议会上,托洛茨基言辞激烈地要求把政权交给苏维埃,布尔什维克代表退出了会议。列宁在9月末或10月初回到了彼得格勒,把发动武装起义夺取政权的问题提上了日程。10月10日中央作出关于武装起义决议的依据,除了国际形势、军事形势以及国内政治形势等客观因素之外,主要就是“无产阶级政党在苏维埃中获得了多数……人民转而信任我们党”。\cite[73]{bigrussia}

% 10月20日,彼得格勒苏维埃以德军已经逼近首都为理由,成立了革命军事委员会,作为指挥起义的机关。

% 由于掌握了彼得格勒绝大部分武装力量,布尔什维克领导的武装起义进展顺利,几乎没有遇到真正的抵抗。整个起义过程中,一共死6人,伤50人。25日晚上开幕的苏维埃第二次代表大会面临政权更迭的既成事实。在代表大会上,布尔什维克和它的盟友左派社会革命党人及其支持者居于多数地位。作为对布尔什维克武力夺权的抗议,51名社会革命党、孟什维克和崩得的代表在暴风雨般的喧嚣中退出了大会。

% 在只剩下布尔什维克及其支持者的大会上,通过了《告工人、士兵和农民书》,宣布代表% 大会已经掌握政权,规定各地全部政权一律归当地工兵农代表苏维埃。大会通过的和平法% 令和土地法令在很大程度上是对既成局面的承认,因为军队不愿打仗,事实上已经瓦解;% 农民早已在自行夺取并分配地主土地。8个月来,临时政府就是因为在这些最迫切的问题上% 拖延不决而丧失了大多数群众的支持,而布尔什维克则从和平和土地的口号中获得了力量。% 苏维埃二大决定,在\textbf{立宪会议}召开之前,成立\textbf{苏维埃政府即人民委员% 会}来管理国家。因左派社会革命党领导人\textbf{没有接受加入政府邀请},成立了由% 清一色的布尔什维克组成的政府。\cite[74-75]{bigrussia}

% 1917年10月27日,人民委员会通过决议,明确立宪会议选举应在预定日期11月12日进行。从逻辑上讲这个日期对布尔什维克是很有利的,因为它已掌握了政权。立宪会议选举共选出了715名代表,其中社会革命党370名,布尔什维克175名,左派社会革命党40名,孟什维克15名,立宪民主党(人民自由党)17名,另有一些其他党派和民族组织的代表。按照党派提出的名单进行的选举,基本上反映了俄国社会政治力量的对比关系。

% 1917年11~12月间围绕立宪会议问题发生的一系列事件,其关键是政权问题。苏维埃领导层分析了立宪会议选举中占优势的社会革命党可能在立宪会议上采取的立场,认为它会利用自己的多数地位拒绝接受被剥削劳动人民权利宣言,并宣称自己是“俄罗斯土地的主人”。基于这种判断,全俄中执委在立宪会议开幕前两天,1月3日,通过了又一个重要决定:“\textbf{在俄罗斯共和国全部权力归苏维埃和苏维埃机关。}因此,无论是什么人、什么机构赋予自己国家政权的职能,都将被认为是反革命行为。苏维埃政权将以其拥有的一切手段直至使用武力来镇压任何这类企图。”

% 1918年1月5日,是立宪会议开幕的日子。这天,彼得格勒和莫斯科发生了支持立宪会议的和平示威。游行遭到武力镇压,有人员伤亡。……大会多数决定不将布尔什维克党团提出的《被剥削劳动人民权利宣言》提交讨论,即拒绝按照人民委员会的要求把权力交给苏维埃并自行宣布解散立宪会议。于是布尔什维克和左派社会革命党人和部分穆斯林党团代表退出了会议。……1月10日,全俄苏维埃第三次代表大会开幕,并取代了立宪会议的职能,通过了《被剥削劳动人民权利宣言》。

% 布尔什维克在十月革命前后对立宪会议采取了截然不同的立场,被卢森堡称为“\textbf{令人迷惑不解的转变}”。布尔什维克领导人在夺取政权过程中对于召开立宪会议的支持和肯定,是争取群众支持、扩大自己社会基础的需要,就如同布尔什维克在土地、和平等其他迫切问题上所做的那样。但真实的选举结果对于这个立志利用世界大战创造的前所未有的机会夺取政权并实现自己纲领的党来说是\textbf{不能接受}的,驱散立宪会议对于布尔什维克党来说是合乎逻辑的选择,夺取政权的目的不是为了把它交出去。

% 1917年12月2日选举产生的领导立宪会议布尔什维克党团的临时局主要成员有季诺维也夫、加米涅夫、拉林、李可夫等人,他们认为召开立宪会议是俄国革命的结束阶段,主张人民委员会停止对立宪会议的召开和活动进行控制。加米涅夫的支持者表示,由俄国社会民主工党(布)中央委员会领导党团是不合适的。他们把立宪会议视为保住民主力量统一的唯一机会,准备与其他社会主义者实现联合。[164]列宁认为这些意见是不顾阶级斗争和国内战争现实条件的资产阶级民主观点,为此他坚持在12月11日重新选举了布尔什维克党团临时局,并在会上通过了他提出的《关于立宪会议的提纲》,其中强调“苏维埃共和国是比有立宪会议的普通的资产阶级共和国更高的民主形式”。

% 俄国立宪会议的命运与俄国的社会发展水平也是有关系的。俄国还没有形成立宪会议能够依靠的比较成熟的社会阶层,还缺乏足够强大和牢固的支持立宪会议的社会基础。当时的俄国还是一个农民的国家,

% \end{quotation}






%%% Local Variables:
%%% mode: latex
%%% TeX-master: "../main"
%%% End:
