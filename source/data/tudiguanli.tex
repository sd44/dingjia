《当代央地关系》之耕地红线与土地金融。

本节几乎照摘周飞舟和谭飞智所著《当代中国的中央地方关系》。笔者对原文个别不同
看法,用脚注加以说明。

自改革开放以来,我国工业化、城镇化推进速度越来越快,大量耕地农用地被占用是土
地财政\footnote{“土地财政”四字为笔者所加。}发展的必然要求。尤其是进入21世纪以来,
各地土地财政带来大拆大建,形成规模不小的失地农民群体,频发的群体性事件成为危
害社会稳定的重要因素。自中央层面来看,基于保障\textbf{国家粮食战略安全}\footnote{原文是
“国家粮食安全”,笔者加入“战略”二字,突出粮食安全地位。}以及维持改革以来
确立的家庭联产承包责任制的基本经营组织方式的要求,实行严格的耕地保护制度,以
占补平衡为代表的系列政策应运而生,\textbf{18亿亩耕地红线}作为一项政治任务贯穿而下,
实行一把手问责制。

\subsection{耕地占补平衡}

1997年4月,中共中央国务院发布11号文《关于进一步加强土地管理、切实保护耕地的通
知》,并明确提出省(区市)必须保持耕地总量动态平衡的要求,同时确定了实行占用
耕地与开发复垦\footnote{土地复垦是指对生产建设活动和自然灾害损毁的土地,采取整治措施,
  使其达到可供利用状态的活动。}挂钩的政策,首次明确提出“耕地占补平衡”的概念。

随后在1998年8月,《中华人民共和国土地管理法》再次修订,明确提出“实行占用耕地
补偿制度”,要求占用耕地与开发复垦耕地相平衡。

1999年2月4日,国土资源部下发了《关于切实做好耕地占补平衡工作的通知》要求要确
保建设占地“\textbf{占一补一}”,逐步实现耕地占用的\textbf{先补后占、占优补优、不补不占}。自此,
耕地占补平衡政策开始在全国各地大规模实施。

2006年以前,占补平衡考核采取的是“算大帐”的方法——以区域为单位,考核区域内
的总占总补平衡。这种方法存在的漏洞是,很多建设用地项目并没有实现法律所规定的
占补平衡,建设用地占用耕地项目单位的补充耕地与土地开发整理脱钩。同时,由于区
域内的占补平衡考核仅仅关注于数量,一些建设项目\textbf{占优补劣}的现象比较突出。

2006年6月8日,国土资源部第3次部务会议通过了《耕地占补平衡考核办法》,于当
年8月1日起施行。“耕地占补平衡考核,\textbf{以建设用地项目为单位进行}”“耕地占补平
衡,实行占用耕地的\textbf{建设用地项目与补充耕地的土地开发整理项目挂钩}制度。”不再
采取大锅饭式的算大账。这一管理思路,为后来的增减挂钩所延续,即采用“\textbf{封闭运
  行}”的项目制运作模式,

随着工业化、城镇化的大势所趋,\textbf{“保耕地红线”成为地方政府沉重的政治负担和资金负
担。}耕地占补平衡政策自出台以来,在各地具体实施过程中主要存在:耕地的“\textbf{实占虚
补}”;补充耕地的“\textbf{实优虚劣}”以及\textbf{农地非农化和非粮化}的风险。耕地占补平衡制度实
行以来,各地实际工作中建设占用耕地长期以“先占后补”和“边占边补”方式为主,
加上对补充耕地的监督力度不够,导致建设占用耕地占而不补、占多补少的问题经常发
生。国土资源部因此颁布《关于进一步加强土地整理复垦开发工作的通知》,规定
从2009年开始,除国家重大工程可以暂缓外,非农占用耕地全面实行“\textbf{先补后占}”。

% 从地方政府角度出发,其更多的是从如何提高土地生产效益的角度出发的,因此如果单
% 纯地维持原有以粮食为主的种植结构难以达到提高效益的目的,转变生产结构成为必然
% 的选择,农地非农化、非粮化在所难免。所谓粮食安全的担忧也并非地方所考虑的问题。
% 在这一点上,中央与地方之间的矛盾凸显。

由于耕地的开垦整理需要一定的工程周期,因而由“先占后补”到“先补后占”的转变,
开启了\textbf{耕地占补平衡指标化}的进程,各地纷纷建立\textbf{占补平衡指标储备库}。提前储
备补充耕地,需新增建设用地时再从库中支取“指标”。

% 中国耕地红线粮食战略安全和土地金融的交织摩擦,使地区\textbf{狂热开发}建设用地的意
% 图受限\footnote{此句为笔者所加。以中国为一整体的角度来考虑,各地重复建设、大干快上,
% 工业用地扭曲的拿地或租赁价格、房地产的癫狂实属狂热无疑。},尤其是经济发达地
% 区与产粮大省更加受限于补充耕地资源较少。

% 从另一各方面来说,耕地红线也使可转化为新增建设用地的农地更加稀缺,加重了土地
% 金融的严重程度。\footnote{此句为笔者所加。}

\subsection{土地置换与指标折抵}

1999年,国土资发358号文《国土资源部关于土地开发整理工作有关问题的通知》。

韩长赋: 中国农村土地制度改革

耕地退化、污染严重,一些地方占好地、补坏地,占水地、补旱地,2016 年全国优高
等耕地面积仅占 29. 5%。


http://cn.chinagate.cn/news/2020-06/05/content_76131126.htm

2018年《跨省域补充耕地国家统筹管理办法》的出台,进一步明确了跨省域补充耕地资
金要全部用于巩固脱贫攻坚成果和支持实施乡村振兴战略,从而充分发挥经济发达地区
和资源丰富地区资金、资源互补优势,推动区域协调发展。
