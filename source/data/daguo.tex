\chapter{中国的下一步要怎么走?}

\begin{quotation}
  当今经济和社会混乱背后存在着结构性紧张和矛盾,社会科学几乎无能为力帮助解决
  这一问题。然而,它所能做的是揭示它们,并指出其历史连续性,以便充分理解当前
  危机。它还可以——而且必须——指出民主国家正在戏剧性沦为全球投资者寡头集团
  的债务催收机构……今天,经济权力似乎比以往任何时候都更成为政治权力\footnote{笔者注:
    新自由主义所要求的是经济自由优于政治自由。笔者认为其实可以说,经济权力大
    于政治权力,政治从属于经济。},而公民似乎几乎被完全剥夺了捍卫自身民主的力
  量,以及他们向政治经济施加影响的能力;公民的利益诉求与资本所有者的利益诉求
  是无法比拟的。事实上,回顾1970年代以来的一系列资本主义危机,似乎确实有可能
  在发达资本主义中解决社会冲突,这一次完全有利于现在牢牢扎根于其政治上无懈可
  击的据点——国际金融资产阶级。
\end{quotation}

笔者认为马克思足够深刻地揭示了现代世界的破坏性,但在建设性上乏善可陈,宏观、
大师叙事40余年来也欠缺实质性的发展,所以笔者无力去探讨中国未来发展路径应是怎
样,但自认可发现个别“忧国忧民”“专家”政策提案所描述美妙画卷背后隐藏的真实破坏
性——一些可以,让党和国家陷入大崩溃、让人民生活日益困苦的破坏性。笔者确信这
“个别专家”要么是天真幼稚的蠢,要么是为背后利益集团服务的刻意为恶。



是否“拉美化”或者陷入中等收入陷阱只是个结果

政府超大的转移支付!!!

应该从GDP总量增长的考核转变为人均GDP的考核。

笔者是经济学的未入门者,而这位XX教授却像是婴儿般“天真无邪”

应该从GDP总量增长的考核转变为人均GDP的考核。特别是对那些人口流出地来说,追求人均GDP、人均收入和生活质量才是长久之计,

“现在沿海的工作,要么是待遇太差,要么是做不了。”

一般来说,经济增长模式取决于资源禀赋,包括土地、劳动力和资本。

已经相对较低的就业弹性还在下降,这似乎表明,中国经济正在滑向“资本深化过度”的境地。35

企业要的是利润,而地方政府要的则是税收。与劳动密集型企业相比,资本密集型的企业通常能够贡献更多的企业所得税。在其他条件相同的情况下,与劳动密集型企业相比,资本密集型企业的计税工资相对较少,而大规模的固定资产投入则仅按照直线法计算折旧,并在企业收入总额中扣除,因此,资本密集型企业有更高的所得税应税额。在中国的财政制度安排下,地方政府能够从资本密集型企业收得更多的所得税,也就相应地能从上级政府获得更多的所得税税收返还。


规模经济

最优的路径就是国家内部的自由移民。

劳动力自由流动,{\bf 最终}实现地区间的劳动生产率和收入均等,是大国发展唯一可行的战略选择。
区域经济学的精髓就是一句话:“在集聚中走向平衡。”

{\bf 从长期来看,}

在欧盟内部不能做到劳动力完全自由流动。


但是,城市化和地区之间的{\bf 自由}移民是一个几十年甚至上百年的过程,在这个过程中,如果我们只看特定时段的局部地区,就可能会觉得,劳动力流动并没有带来地区间收入水平差距的缩小。

1、美国的“地多人少”是通过屠戮原住民获得的大片土地,每一块土地下面都有印第安人的尸骨。而我们本身就是原住民,脚下的土地是由我们共同的祖先开拓的土地,再大的利益也不能将屠刀挥向同胞。

2、美国的“农民”也不是我们常识中的农民,而是农场主,或者说“地主”

要让其致富,关键的措施就是减少人口,给钱还是次要的。

城市发展的问题核心是解决“3M”问题,即分别用Time、Grime和Crime三个词表示的拥挤、污染和犯罪问题……这被称为“拥挤效应”,俗称“城市病”

为城市问题、区域问题和国家问题。报告用三个D——Density(密度)(规模经济体现
为三大效应,第一是分享sharing——分享固定投入。第二个效应是劳动力市场的专业
匹配matching,第三个是学习效应learning——劳动力专业化以后就越做越好,这叫自
己跟自己学,积累经验就是这个意思)。、Distance(距离,空间和时间压缩到大都市圈)和Division(分割)——
来构建自身的整个分析框架。其中,密度带来规模经济,而城市是规模经济的集中体现。
距离产生地区间的贸易成本,而基础设施的建设可以降低贸易成本,就好像拉近了地区
之间的距离。相应的,区域发展政策的重点就是通过基础设施建设缩小地区间的距离,
降低贸易成本。

规模经济是与空间集聚相伴随的。

集聚意味着生产力的高度发达,需要的人工越来越少

随着农民不断减少,剩下的农民从农民变成规模化经营的农场主,或者农场里的“农业工人”,收入也将提高。制约农业发展的瓶颈要素是土地,土地是不可能无限增加的,所以一旦土地决定了农产品的增长极限的时候,要提高农民的人均收入只有减少农业人口,这在卓玛与松茸的故事里也讲过。而工业和服务业的发展却可以不断地进行资本积累,不断地创造就业。

城市部门是不是能够持续地提高生产率,并为农村进城的移民源源不断地创造就业,

经济学里面缺资本和不缺资本的标准是谁的投资回报高,而不是谁的资本数量少。资本往美国流,因为美国投资回报最高,所以美国最缺资本。同样的道理,中国哪里缺资本?总体上仍然是资本回报最高的东部。

第一,国家的统一。

第二,经济效率的提高。

第三,区域之间的平衡发展。
大国发展的“不可能三角”,而如果要破解这个“不可能三角”,就必须改变“平衡”的定义,将经济资源和人口均匀分布意义上的“平衡”,转变为人均GDP、人均实际收入和生活质量意义上的“平衡”,而这种人均意义上的平衡恰恰是可以在人口自由流动的过程中实现的,不需要大规模地采取行政控制的手段来对人口流动进行限制。“人均”的荒谬


如果在一国内部,移民不自由,就会出现本书第一章里说到的问题,即国家内部的地区间收入差距问题。而不管是通过财政转移支付的方式,还是通过帮欠发达地区还债的方式,发达地区都需要负起相应的责任。读者可能会问,发达地区为什么要负起这个责任?道理并不复杂,因为这是统一国家的必需,而且,发达地区恰恰是因为处于一个统一国家和统一货币区的内部,享受了统一市场的好处,获得了来自欠发达地区不断流入的劳动力资源,并且恰恰因为自身是这个国家统一货币区的一部分而成了金融中心。只想要统一的好处,不想承担统一的义务,这是任何国家的政治都不会允许的。


中国已经到了呼吁每一个省、每一个市、每一个县、每一个人放弃本地思维,顾全公共
利益的时候了。(中央的更加集权,行政性配置,是否可能?地方不均衡如何解决?落后地区如何照
顾?)

集聚也的确会带来坏处,比如说拥挤、污染和犯罪。当集聚带来的好处不够高,而坏处体现出来之后,集聚的水平就相应地稳定下来。而在这个过程中,非常重要的调节变量就是生产要素的价格,集中体现在地价、房价和劳动力工资上。


内部的全球化流动 或者 中国地区化

如果生产要素能够充分自由地跨地区再配置,那么,不同城市将能够有效地形成差异化的分工体系,生产要素的地区间配置效率能够进一步提高。更重要的是,在资源跨地区再配置的过程中,一些大城市及其周边地区将形成经济集聚的趋势,并能够进一步发挥规模经济效应,提高劳动生产率。

一方面,对于市场上的债务违约,中央当然希望打破“刚性兑付”的预期,以免将什么责任都揽在中央,让人们形成地方债务没有风险的预期;另一方面,面对事实上已经难以偿付的地方债务,最终还是会被认为将由中央政府来兜底,从长期来看,这也恰恰可能造成地方政府不计后果地借债的局面,这就是经济学里典型的“道德风险”问题


世界主要经济体基尼系数基本已达70以上,无论是国家人均还是大地区级的人均GDP这两
个指标已难以说明其中人民的经济水平,它们更能说明的只是富人资本。

https://m.thepaper.cn/newsDetail_forward_22666274

吸纳不了,黄奇帆 美国 金融华尔街30万人

大国大城聚焦的是重复建设,到处产业升级(高大上投资项目)的土地金融弊端,

过度资本密集化的产业发展路径导致工业化的速度远远快于服务业发展的速度,工业化
的进程也远远快于人口的城市化进程。这在收入分配上的体现,则是这样的产业发展模
式造成劳动收入占国民收入之比下降。 (它居然提倡服务业?HOW)

地方政府却没有积极性加大教育和培训的投入,

展消费型服务业

给定一个国家劳动力的教育水平,反而使大城市吸引大量的低技能劳动力前去工作。

高密度、马路多而窄的模式反而可以引导服务业多样性、生活的便利性和出行需求的减少

餐饮、家政

发达国家才不会以天下大同为己任呢!所以,发达国家的政策一定是只要高技能人才,低技能劳动力他不要。

高收入、好的公共服务,伴随着相对严重的“城市病”和高房价,其实恰恰是地区间生
活质量平衡的表现。不过,我还要强调一下,这里说的“城市病”在一个国家内部成为
生活质量平衡的机制,是指横向的比较。在本书的下篇中,我们还会说到,从历史的维
度来看,城市人口规模的扩张并不一定伴随更为严重的“城市病”,“城市病”可以通
过{\bf 技术和管理的手段}来解决。

在生产效率更高的地方以更快的速度来推进城市化

地价和房价本身就成为低效率企业和劳动力进入大城市的障碍,这就是市场的力量。

沿海到中西部买这个土地指标…… 随着建设用地指标的跨地区再配置(可看下央地关系重庆模式的问题再行批判)

农民在放弃土地的时候要满足{\bf 自愿、有就业、有社会保障、土地(或建设用地指标)能够市场定价这几个基本的前提},让农民避免被剥夺。而以禁止市场交易的方式来保障农民,这是最为荒谬的逻辑,其结果恰恰是给不尊重土地产权的行政性剥夺找到了借口。(农业税?羊吃人?农业工业化的挤出)

分别有67.2\%和63.2\%的新生代农民工认为“{\bf 收入太低}”和“{\bf 住房问题}”
          是制约在城市定居的重要困难和障碍。所以,{\bf 政府可以做的是为有定
          居意愿的农民工创造条件,而不是以农民没有进城意愿作为借口放缓城市化}。

在“蛋糕做大”的情况下,中央财政就更有能力来进行{\bf 区域间和城乡间的财政转移},但
未来的财政转移应该更多地用于{\bf 城乡和区域间公共服务的均等化},比如说提高欠发达地
区中小学教师和医护人员的待遇,相应地减少欠发达地区(特别是农村)政府的财政负
担,以促进区域和城乡间在生活质量上的平衡,实现“动人”和“动钱”的良性互动。

服务业岗位,从而吸纳更多的农村人口进城,使得城市化率不断提高,直至75%,甚至80%以上的水平。


如果一个城市可以通过基础设施和人力资本的投入,从而放大正外部性,会引导城市进一步向有效的更大规模城市发展;如果一个城市可以通过技术进步和政府的管理措施减少负外部性,也可以使这个城市更加有效地运转。(“资本家不愿做的事,政府也不愿做”,为什么不让资本家去做呢?因为外部性?可实际上在历史经验及现实情况下,前者是政府的持续赤字,后者是太过乐观——政府的无能为力(包括美国))

我并不是反对为欠发达地区提供财政补贴,我要讨论的是,在政府进行财政转移的时候,应当考虑转移的数量、地点和结构,如果违反市场经济的规律,盲目追求人口和经济资源的均匀分布,那么,一系列低效率的后果就必然会出现。

西部和边陲的发展,因为这是政治和国防的问题,但政治和国防与我们这里讨论的问题没什么关系。

2014--2020新型城镇化规划只是北京、上海超大城市限制发展。

若人口不从欠发达的地区流出,就很难提高这些地方的劳动生产率。中国总有一天会达到75%以上的城市化率。如果制造业和服务业能够不断地加强国际竞争力,不断地创造就业,那么中国的城市化率达到80%以上也并非不可能。在这个过程当中,随着农村人口的减少,农业的劳动生产率必然不断提高。不用过于担心农村人口的减少会危害农业,恰恰相反,人口的流出是农村地区提高规模经营程度和劳动生产率的前提条件。随着空心村现象不断发展,一部分的农村社区将逐渐消失,原有的宅基地不断空出,复耕为农业用地,现有的一家一户的农业经营模式就将逐渐转变为大农场的模式,这样,农民的收入水平才可能不断提高。只有规模经营才能让农业成为能够致富的产业,才可能让一部分人愿意留在农村当农民。不同的是,未来的农民和现在的农民含义就不一样了,他们应该受教育水平更高、更年轻,才能适应经营大规模农业的需要。

通过中央向地方的财政转移支付,可以为人口流出地的养老提供更多资源。更重要的是,全国的养老保障体系将逐步走向一体化,这时,即使人口流出地面临更为严重的老龄化问题,也不需要担心了,这才是解决问题的根本出路。

退休年龄太低、导致养老金入不敷出的问题,已经在通过全国范围内推迟退休年龄来缓解了。而对于相对问题更为严重的人口流出地来说,要进一步缓解养老危机,那就只能是在全国范围内逐步推行养老体系的一体化,在全国水平上寻求养老金的收支平衡。

在转型期,由于公共服务的享受权仍然与户籍身份挂钩,同时,在公共服务供给不可能短期内快速增长的情况下,改革不能在一夜之间取消户籍与公共服务的挂钩,因此,户籍的转换或者特大城市实施的积分落户制度就必须设置门槛,形成对于一部分人的事实上的公共服务歧视。那么,这样过渡时期的“歧视”应该遵循什么原则?这就要本着理性原则,尽量将仅仅为公共服务而流动的人口识别出来。

从理论上来说,劳动力迁移主要基于两种因素。一种是纯粹的经济动机,即为了更高的收入和更好的就业机会,而另一种则主要是为了城市的公共服务和其他自然禀赋,比如环境和气候。

因此,政府可以通过一定的办法来识别出仅仅为了公共服务而迁移的人群。其中,最好的办法就是认为有就业记录和社会保障记录的人群是以就业为主要目的的迁移人群。虽然这一标准并不完美,却是现行条件下能找到的最好指标。

相反,政府不能主观臆断,根据劳动力的职业、行业以及经营规模来 决定是否城市需要某些人群。


正如我反复说的那样,当遇到人口流动带来的公共服务需求与其供给的矛盾的时候,一方面要通过增加公共服务的供给总量来缓解矛盾,而不应该通过限制人口的流入来回避问题;另一方面,要不断地减少户籍和公共服务之间的挂钩程度。

无知之幕,天堂。己所不欲,勿施于人,


尊重的规律是什么?是有权势者对无权势的尽情盘剥。


社会保障健全则储蓄的动机将减弱。

经济学家的研究发现,在欧洲,收入不平等减少快乐,而在美国,这种效应却不强。这实际上就和美国社会不同收入阶层之间的流动性更强有关,其实,“美国梦”的道理就是说每个人都平等地拥有致富的机会。

中国城市的土地是国有的,这在根本上可以允许中国城市政府作出更好的规划,以及向低收入居住区提供基本而必要的公共服务,来防止大面积贫民窟的出现。

避免城市出现贫民窟问题的关键是要通过城市发展源源不断地为进城农民创造就业机会,并且为其提供适度的公共服务和社会保障。

对于部分国家出现的贫民窟现象,要作有针对性的分析。城市经济是否可以持续增长,从而为农村移民创造就业是非常重要的问题,一些拉美国家经济增长乏力,制约了低收入阶层提高收入的机会。而在印度这样经济增长较快的国家,城市经济高度偏向现代服务业和信息技术产业,为低技能者创造的就业机会有限。

中国的地方政府最大化本地利益,缺乏彼此之间的协调,国家利益被忽视,这极大地影响了中国发展的道路和可持续性。

经济发展对于空间集聚的要求越来越高,而地方政府却追求本地经济规模和税收规模的最大化。

,释放劳动生产率

生产要素的充分流动增进社会和谐,将进一步减少经济和社会资源的无谓消耗。

%%% Local Variables:
%%% mode: latex
%%% TeX-master: "../main"
%%% End:
