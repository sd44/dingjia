是否“拉美化”或者陷入中等收入陷阱只是个结果

规模经济

最优的路径就是国家内部的自由移民。

劳动力自由流动,{\bf 最终}实现地区间的劳动生产率和收入均等,是大国发展唯一可行的战略选择。
区域经济学的精髓就是一句话:“在集聚中走向平衡。”

{\bf 从长期来看,}

在欧盟内部不能做到劳动力完全自由流动。


但是,城市化和地区之间的{\bf 自由}移民是一个几十年甚至上百年的过程,在这个过程中,如果我们只看特定时段的局部地区,就可能会觉得,劳动力流动并没有带来地区间收入水平差距的缩小。

1、美国的“地多人少”是通过屠戮原住民获得的大片土地,每一块土地下面都有印第安人的尸骨。而我们本身就是原住民,脚下的土地是由我们共同的祖先开拓的土地,再大的利益也不能将屠刀挥向同胞。

2、美国的“农民”也不是我们常识中的农民,而是农场主,或者说“地主”

要让其致富,关键的措施就是减少人口,给钱还是次要的。

城市发展的问题核心是解决“3M”问题,即分别用Time、Grime和Crime三个词表示的拥挤、污染和犯罪问题。

为城市问题、区域问题和国家问题。报告用三个D——Density(密度)(规模经济体现
为三大效应,第一是分享sharing——分享固定投入。第二个效应是劳动力市场的专业
匹配matching,第三个是学习效应learning——劳动力专业化以后就越做越好,这叫自
己跟自己学,积累经验就是这个意思)。、Distance(距离,空间和时间压缩到大都市圈)和Division(分割)——
来构建自身的整个分析框架。其中,密度带来规模经济,而城市是规模经济的集中体现。
距离产生地区间的贸易成本,而基础设施的建设可以降低贸易成本,就好像拉近了地区
之间的距离。相应的,区域发展政策的重点就是通过基础设施建设缩小地区间的距离,
降低贸易成本。

规模经济是与空间集聚相伴随的。

集聚意味着生产力的高度发达,需要的人工越来越少

随着农民不断减少,剩下的农民从农民变成规模化经营的农场主,或者农场里的“农业工人”,收入也将提高。制约农业发展的瓶颈要素是土地,土地是不可能无限增加的,所以一旦土地决定了农产品的增长极限的时候,要提高农民的人均收入只有减少农业人口,这在卓玛与松茸的故事里也讲过。而工业和服务业的发展却可以不断地进行资本积累,不断地创造就业。

城市部门是不是能够持续地提高生产率,并为农村进城的移民源源不断地创造就业,


中国已经到了呼吁每一个省、每一个市、每一个县、每一个人放弃本地思维,顾全公共
利益的时候了。(中央的更加集权,行政性配置,是否可能?地方不均衡如何解决?落后地区如何照
顾?)

集聚也的确会带来坏处,比如说拥挤、污染和犯罪。当集聚带来的好处不够高,而坏处体现出来之后,集聚的水平就相应地稳定下来。而在这个过程中,非常重要的调节变量就是生产要素的价格,集中体现在地价、房价和劳动力工资上。


内部的全球化流动 或者 中国地区化

如果生产要素能够充分自由地跨地区再配置,那么,不同城市将能够有效地形成差异化的分工体系,生产要素的地区间配置效率能够进一步提高。更重要的是,在资源跨地区再配置的过程中,一些大城市及其周边地区将形成经济集聚的趋势,并能够进一步发挥规模经济效应,提高劳动生产率。

一方面,对于市场上的债务违约,中央当然希望打破“刚性兑付”的预期,以免将什么责任都揽在中央,让人们形成地方债务没有风险的预期;另一方面,面对事实上已经难以偿付的地方债务,最终还是会被认为将由中央政府来兜底,从长期来看,这也恰恰可能造成地方政府不计后果地借债的局面,这就是经济学里典型的“道德风险”问题


世界主要经济体基尼系数基本已达70以上,无论是国家人均还是大地区级的人均GDP这两
个指标已难以说明其中人民的经济水平,它们更能说明的只是富人资本。

https://m.thepaper.cn/newsDetail_forward_22666274

吸纳不了,黄奇帆 美国 金融华尔街30万人

大国大城聚焦的是重复建设,到处产业升级(高大上投资项目)的土地金融弊端,

发达国家才不会以天下大同为己任呢!所以,发达国家的政策一定是只要高技能人才,低技能劳动力他不要。
