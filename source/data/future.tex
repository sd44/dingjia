自大萧条以来,政策制定者很少(如果有的话)面临像今天这样多的不确定性。其中一
个例子是,市场不仅期望财政整顿,而且同时期望未来经济增长的合理前景。两者如何
结合在一起尚不清楚。

  这里用两句话来说明间接税和直接税:

   1、间接税是穷人税,每个人税率一样,比如普通商品的增值税。

   2、直接税是富人税,根据累进算法,富人交更多的税,比如个税和尚未开征的遗产税。

美国直接税的大头是中产阶级,而不是富人阶级。

批判高屋建瓴有余,推销的却是从属利益集团的利益,分裂。

政商旋转门

常驻的匪帮,依靠垄断,全民收税……赵燕菁核心

大一统和合法性,转移支付,全民的,自建国后开始的

精神和物质文化需要,已不再是……人为打压后果……

财富分化加剧已是共识,就分配尽量公平几乎都有论述,那其应当说明如何可以归于公众?需仔细论述,魑魅魍魉往往在此不得藏身。

种种天方夜谭式的救市方案。

警惕一切需要政府、企业(现在预警下=持续负债)大力兜底的政策。如这些政策制造出来的贫民窟、发达地区向落后地区的转移支付(如大国大城的叙述矛盾,一方面反对转移支付,一方面又提倡向没有享受公共投资好处的地方进行适度的财政转移。。当然,前者是全面的,后者是个别方向的,如“落后缺失”,但是这两种转移支付真的有本质区别么?“问题是,什么叫支持当地的发展?是把公共财政投向欠发达地区直接建开发区办厂,还是把钱投向欠发达地区建设基础设施和公共服务?直接把钱投向生产性的项目就必须最终接受市场竞争的考验,如果经济欠发达地区恰恰是在地形比较复杂、远离大的城市中心和交通干线的地方,那么很有可能生产成本和运输成本过高,从而大规模的生产性投资缺乏市场竞争力。对此,看看全国遍地开花的开发区,看看政府砸了多少钱,换来多少产出就知道了。偏离当地比较优势的投资不仅不会有效地帮助欠发达地区提高经济发展水平,而且可能使这些地方背上沉重的财政负担,对此,前文已强调多次。同时,如前所述,在缺乏发展工业比较优势的地方,硬要搞招商引资,地方政府官员也苦不堪言,结果招来的往往还是东部不要的污染和高能耗产业。” PS:也要考虑其中的正确性,如当地比较优势的投资。道路和机场的建设更为重要。 ,政府的财政转移支付主要是用于支持欠发达地区的公共事业,特别是教育和医疗事业,从而帮助欠发达地区的居民提高生活质量。发达地区的教育和医疗事业正是个烂账,看美国 图)。

“公众基金”也是解决贫富差距问题的社会新实践。通常的做法主要有两种:一种就是在生产循环的前端,通过生产资料公有制(也就是“国有企业”)来实现均富的目标。另一种就是在生产循环的后端,通过高税收和转移支付,实现均富的目标。

后者根据不同标准每年按照面积和效果评价获得非建设用地补偿金。

改革后,村集体的“田底”可以是混合的产权组合,并可在资本市场上转让。不同集体的产权可以合并,也可以到市场上“招商”,寻找能够提供优质公共服务的运营商。


为什么重资产要由政府提供?为什么企业不愿投入重资产?不盈利!

一国债务与GDP之比在五年内涨幅超过30%的话,大概率会在随后的五年内爆发经济危机,美国次贷危机后的十年,可以说连续已有两个五年债务都增长了30%以上。所以,按现行美国举债速度和赤字增加速度如果不变的话,要不了几年,美国经济将被债务大山压倒。

税收制度日益累退的特征

黄奇帆

美国会采取什么经济措施来缓和平衡美国政府债务率过高的问题呢?从过去几十年历史经验看,大概率会采取三种措施。

第一种是美元贬值、通货膨胀。为了维护美元地位,维持债务融资来源,美国采取直接违约的可能性极小,但却不能排除美国政府以间接方式违约。这些方法包括美元贬值和通货膨胀。有三个历史性案例:一是1933年,美国因美元贬值,废除国债的黄金条款,国债购买者不能按原契约换取相应黄金;二是第二次世界大战后,美国采取通胀办法,每年通胀6%,五年总债务占GDP比例能减少20%左右,十年能降低40%左右;三是1971年美国单方面停止美元兑换黄金,致使布雷顿森林体系崩溃,继而确定了牙买加体系。总之,采用美元贬值和通货膨胀变相违约,早已是美国减债减赤的惯用手法。

第二种是通过加息缩表剪羊毛,以邻为壑转嫁危机。美国每一次加息周期往往会演化出某一领域或某一地区的经济危机、金融危机。20世纪80年代以来,美国有4轮加息周期,80年代加息的尽头是拉美债务危机;90年代加息的尽头是亚洲金融危机;2003年开始的加息周期尽头是全球金融危机。目前这轮加息周期从2015年12月开始,2015年、2016年、2017年各加息一次,2018年已经3次,预期全年加息4次。那么,这次加息的尽头是在什么地方、什么领域出现大级别的危机呢?由于加息,美元走强,近几个月继巴西里拉之后,南非兰特、印度卢比、印尼盾、俄罗斯卢布、阿根廷比索都在大幅贬值。因此,现在大家有种预感,近期的新兴市场货币贬值是否表示新一轮金融危机将表现在新兴市场?

第三种是以全球经济老大的实力改变游戏规则,大打贸易战意图获取超额利益弥补、化解债务困境。美国经济结构有很大问题,其GDP中85%来源于以金融为中心的服务业,制造业只占11%。美国巨额的贸易赤字根本就是自己的经济结构造成的,而不是别国造成的。怪罪于别国,完全是一种得了便宜还卖乖的行为。金融业属于精英产业,对劳动力吸纳能力非常差,比如美国金融中心华尔街总共才吸纳30万人就业。

总之,解决危机最不能容忍的办法有三种:一是不能为了掩盖矛盾、缓和矛盾而把现在的危机推向未来,导致未来更大的危机;二是不能用一个倾向掩盖另一个倾向,走极端,采取一种措施解决一个危机而引发另一个更为严重的危机;三是不能以邻为壑地把自己的问题转嫁给别人,利用自己的强国地位、货币信用为所欲为。
