\chapter{生命}

\begin{quotation}
  生命权是最基本,最重要的人权,如果无法充分保障人的生命权,那么一切其它权利
  都是空中楼阁。无端剥夺人的生命,或者肆意对人施加恐吓、虐待和折磨,就是用一
  种非人权的待人方式。任由这种情况发生,个人权利就无从谈起。所以一般各国的刑
  法都将侵害他人生命权的罪行量刑最重。

  生命权是一个人之所以被当作人类伙伴所必须享有的权利。\cite{renquanwiki}
\end{quotation}

\section{关于无国界医生的一场争议}

\begin{quotation}
  无国界医生是一个独立的国际医疗人道救援组织,致力为受武装冲突、疫病和天灾影
  响,以及遭排拒于医疗体系以外的人群提供紧急医疗援助。无国界医生只会基于人们
  的需要提供援助,不受种族、宗教、性别或政治因素左右。
\end{quotation}

传媒、文学作品等往往将MSF成员描绘为在枪林弹雨里冒着极大生命危险抢救难民的英雄。
这是过于煽情偏颇的。MSF医生的生命同样是生命,MSF组织并不会让医生为救人
而面临随时死亡的危险,只会辟出安全区域设立医院为难民实行救治。

知乎网有关于无国界医生蒋励的一张帖子热度较高,文字回复条数过千。题目是,《如
何看待北京医生辞职去阿富汗参加无国界医生?》。\cite{kandaijiangli}

蒋励在北京大学医学部完成八年本硕博连读教育后,顺利入职北京大学人民医院。受师
姐和领导屠铮影响,“2012年参加了无国界医生。2013年3月至6月在无国界医生位于阿
富汗霍斯特的妇产医院工作(同时前往阿富汗的另一MSF中国内地成员——麻醉医生赵一
凡去了昆都士省的外科创伤医院),2014年1月再次去往无国界医生在巴基斯坦蒂默加拉
的医院工作。”,\cite{jiangli}。阿富汗霍斯特医院情况总结如下,床位\textbf{60}张,每
月\textbf{1200多例}分娩,相关医护人员有\textbf{2位妇产科医生,4位国际助产士,2位麻醉医生}组
成,也就是说,这8个人分工协作,24小时不间断的进行每天平均40余例的分娩“流水线
作业”。挽救大量的新生儿和妊娠母亲的生命。

但有关这个帖子的一些回复真是让人完全意想不到,瞠目结舌。其中点赞1700多次、点
赞次数排行第四的匿名帖子反对无国界医生对阿富汗的援助,提倡\textbf{绝育论},“已经生
育三个及其以上孩子的妇女向援助医院请求接生必须以切除子宫或者上节育环作为交换
条件,然后由国际组织建立隔离带,优先为已经绝育的妇女及其子女提供庇护,食物,
医疗和基础教育,把儿童从中剥离出来接受现代教育。”,有帖子发表类似观点“没有
条件接受教育的人就没有出生和生存的权利。”,除此之外,还有“愚善”,“学医救
不了阿富汗”等脑沟回清奇的言论。

在网络上,不止此贴所涉及国家,我们还可以看到,关于罗姆人(吉普赛人)、社会底
层人士的优生学绝育论。这种绝育论宣扬缺乏教育和社会资源的群体,或者相比自己群
体拥有更高不和谐的群体(这种水平判定其实只是个人不负责的主观判断),应当尽量
抑制他们彼此间繁殖新生生命,对于正在或已经出生的新生生命,“\textbf{社会上流人
  士}”有义务、有必要剥夺新生儿父母的监护权、教育权,让社会特殊的学校或机构行
使监护权、教育权,使他们脱离这个不文明的群体……

\section{“科学”的优生学绝育论}

优生学常是举着科学的旗号,被一些别有用心或者精神脆弱之人利用,行反伦理、反人
类之实,让我们看看优生学在历史上曾被误用滥用的历史吧,以下内容节选自邱仁宗所
著《一本医学家、遗传学家、决策者和立法者必读的书——《从“安乐死”到最终解
决》》\cite{yousheng}:

\begin{quotation}
  1881年Francis Galton提出“优生学”,当时被定义为“通过优化生育改良人种的科
  学” 。于是在北美和欧洲兴起了一场将弱智、残疾、在竞争中处于劣势的人绝育、禁
  止他们第一章探讨种族灭绝的意识形态背景, 即固守“人类不平等”入境的优生运
  动。1907年美国印第安那州颁布了第一部将精神病人、性罪错者、智力低下者、道德
  堕落者和癫痫病人绝育的法律,到了30年代中期已有半数以上的州通过了类似的法律。

  对日耳曼人或德意志人的优良品质深信不疑的德国医生和科学家提出了“种族卫
  生”(Rassenshy giene)概念。

  1920年德国律师Carl Binding和医生Alfred Hoche出版了第一本题为《授权毁灭不值
  得生存的生命》的书。

  正是纳粹政权使种族卫生计划成为现实,它决心保持德意志血统的纯洁性,清理德意志
  的基因库,将“人类不平等”这一思想制度化。1933年7月颁布《 防止具有遗传性疾病
  后代法》,即绝育法,对患有各类精神和肉体疾病的病人实行强制绝育。1933年11月颁
  布《反危险惯犯法》和《安全和改革措施法》, 授权将反社会者关进国营医院,对性犯
  罪实行阉割手术。1935年9月颁布《 帝国公民法》和《德意志血统和尊严保护法》,二
  者统称纽伦堡种族法,正式在法律上排斥犹太人、吉卜赛人、黑人。

  1939年10月,希特勒签署了一份文件,文件称:“一些根据人道的判断被确认为不可治愈
  的病人在确诊后准许被实施慈悲死亡。”

  后来将残疾人安乐死的计划进一步在德国占领区扩大实施。接着大规模屠杀吉卜赛人
  和犹太人,进行所谓“最终解决”,被杀害的人数达600万人。
\end{quotation}

在数年前,这种极端蔑视生命权的言论在中国没有一丁点市场,几乎见不到有人去支持。
如今我们不需要在网络上,就算是在现实中的社交场合,都能听到这些优生学绝育论或
灭绝论。我们忘了中国人历史上被称为“东亚病夫”“黄皮猪”的屈辱历史了么?那在
当时我们中国人是不是应该被理所当然的消灭呢?

\section{被漠然视之的生命权}

一些老司机告诉我们“把人撞成重伤,不如直接撞死人”的言论,并有撞到陌生人后多
次碾压致受害者死亡的多个现实案例;笔者也在现实生活中听闻,包工头因第一时间
将重伤建筑工人送至医院,被乙方负责人怒斥,因为这增加了花销;某地的同胞们学习
电影《盲井》,整村组团外出靠矿井下杀人,勒索矿主获利,并组成整个犯罪链条;部
分军事爱好者们狂想爆发战争,中国重锤他国或地区,这是他们的民族自信啊;我们还
可以看到一些所谓的左派们希望重回激进,工运、革命;一些所谓的右派们想着带路西
方列强的“民主”和“自由”……在他们看来,大量生命面对困境或者死亡这种残酷性竟
是“\textbf{实现伟大梦想}”所必不可少、不可或缺、不能避免的。

宣扬这些反生命文化的不是未受教化之人,他们中反而有不少人接受过高等教育,有些
人甚至是博士。笔者认为,这反映出我国在关于生命权,关于正义和权利的人文社科教
育方面是严重失位的,存在着可怕的匮乏与空洞。

上下以经济建设为中心、大众向钱看的同时,将人文社科教育远远抛在了后面,留下的
只有应试,这些极端反生命权的言论也因此得以抬头,这真是对“中华上下五千年传统
文明”的莫大讽刺。长此以往,即使经济保持快速发展,即使我们自然科学知识取得了
世界领先地位,我们仍将成为人类社会文明的荒漠。荒漠中将只有个别人聊以自慰的小
绿洲,而这小绿洲于国于家并无多大用处,聊可自慰而已。这最终仍会反过来导致经济、
科学的大幅倒退,乃至民族认同感的消逝。

\section{20世纪的战争}

我们迅速的将中国20世纪所经受的各种生命惨剧忘却脑后,而\textbf{中国是20世纪因战争和
  暴行受害最大、死亡人数最多的国家},其实离我们最近的一场战争——1979年中越战
争,从爆发之日算起距今算起还不到40年。马修·怀特(Matthew
White)\cite{mattwhite}致力于研究统计战争及暴行导致的死亡人数,他参考多种资料,
立场比较中立,数据统计相对详实可信。虽然如此,因西方世界中战争及暴行相关参考
资料大多带有反共意识形态影响,这无疑也会使马修所使用参考文献的数据出现偏差,
建议大家批判性阅读。

\begin{table}[h]
  \centering
  \begin{tabular}{@{}llllll@{}}
    \toprule & 军事 & 军事附带 & 大屠杀 & 饥荒 & 合计 \\ \midrule
    战争时期 & 3700万 & 2700万 & 4100万 & 1800万 & 12300万 \\
    和平时期 & 0 & 0 & 4000万 & 4000万 & 8000万 \\
    合计 & 3700万 & 2700万 & 8100万 & 5800万 & 20300万 \\ \bottomrule
  \end{tabular}
  \caption{20世纪死于战争、屠杀和压迫的人数统计表}
  \capsource{来源:Matthew White, Historical Atlas of the Twentieth Century, 2010}
  \label{tab:20stdied}
\end{table}

据马修,第一次世界大战(1914-1918),全球死亡人数约1500万。第二次世界大战
(1939-1945),全球死亡人数约6600万。20世纪全球全部死亡人数大约为55亿,其中死
于20世纪战争、人类暴行的人数约2.03亿,也就是说,在20世纪,全球死亡人口中平均
每27人中就有1人的死因是战争或人类暴行。

中国方面,军阀战争(1917-1928)期间各方军队战死约20万,因屠杀和饥荒死去的民众
约60万,共计死亡约80万。第一次国共内战(1928-1937)期间,死于战争、屠杀和饥荒
的军民约500万。抗日战争(1937-1945)期间,国共两党军事死亡人数约180万,平民死
亡人数约800万,伪军死亡人数约20万左右,合计死亡约1000万人。第二次国共内战
(1945-1949)期间军民死亡人数约250万。中国平民和士兵的死亡人数比在世界上位列
前茅。

\section{人相食}

在中国史书记载上,“人相食”屡次出现,已经成为一种历史固有表
述\cite{renxiangshi}。以 $ \mbox{年份} \div \mbox{次数} $ 简单统计,《资治通鉴》中大约平
均30多年出现一次人相食,《二十四史》中因记录详细,跨越朝代久远,大约20多年出
现一次人相食。

在《中国古代的食人》一书中,郑麒来(美籍韩裔)将食人分为两类,“\textbf{求生性食人}意
味着人们为自己的生物性生存而互食,与求生性食人密切联系的食物匮乏往往由战争、
内乱等人祸或干旱、饥馑、虫灾等天灾引起。\textbf{习得性食人}……更多受制于文化因素,
诸如爱与恨。”\cite[152]{9787500415480}。“在研究中国历史文献的过程中,
我们发现153例与战争(直接)有关的食人事例,均由战时或战后的饥饿和饥荒引
起。……间接相关的有74例,两个数字加起来,即全部与战争有关的食人事例便共
有227例”,\textbf{几乎每朝每代均发生这种求生性的群体人相食。而以上统计数据并不包括地
方方志所记载的“人相食”数据。}

综观人类历史,就民计民生方面来说,人类并不是常说的总体向上进步的螺旋上升,而
是因种种权力欲望带来的生发死灭这一悲喜交加的交替循环。且在这权欲的循环中,向
上的、喜剧的经历苦短而观众甚少,常常是你方唱罢他登场;向下的、悲剧的经历恨长
而观众众多,常常是意悬悬半世心,枉费了卿卿性命。难道福柯是对的吗:
\begin{quotation}
  \textbf{人性并不会在持续不断的斗争中逐渐进步,直到最后达到普遍的互惠,最终以法律
    准则取代战争;相反,人类将其每一种暴戾都深深地潜藏于法律体系之中,因而所
    谓人性的进步只不过是从一种统治形式过渡到了另一种统治形式而已。}
\end{quotation}


或许是笔者悲观,对于人类,尤其对于中国人来说,我们很可能会将“岁大饥”、“人
相食”在未来的某一天继续下去,再次书写食人的历史。至于那在哪一天,是否一百年、
二百年后,那就不知了。当前来说,幸好前人与我们现在的人何干,而我们本身也善于
遗忘。幸哉乐哉!\bigskip


朋友们,我们下次战争再见吧,祝你平安。
