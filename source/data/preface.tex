\chapter{序言}
\label{chap:preface}

\section*{写作背景}

我出生于山东省一个财政困难县,幼时幸得父母、长辈宠爱有加,家中虽不富裕却不缺
衣短穿,也因此不谙世事,不解人间种种,总活在自我的小天地中。等到初中二年级时
便遭恶果反噬,世间一切对我来说太过未知和魔幻。尼采所述人类社会那怒吼着“你应
该”的巨龙形象,让我只敢蒙着头瑟瑟发抖于被窝之中无所适从。这世界究竟是什么样
子?为什么会这样?我该怎么做?一切茫然无措。高中跌跌撞撞上完就赶快滚回家里蹲
起逃避一切,3年后才参加工作,真正步入这魔幻人世间……

2014年,我所在单位和宿舍均搬至济南市化纤厂路,步行不足200米便是丁家庄菜市场。
约莫过了两年,我才注意到菜市场后面的一个中大型城中村——丁家庄城中村。我在家
乡时听说过、见过贫困的农户,却未曾想到大城市城中村的人文环境竟如此奇特、与众
不同,便在此租了一个单间,以求近距离了解。一两个月租期到期后,也常漫步于丁家
庄。

人已经在场,可无知如我怎样去深入了解这样一大风土人情呢?深圳的邱文建议我采用
社会学的视角去观察丁家庄,并多次对我鼓励和指导。我接受了他的建议,学习社会学
知识,继而延伸到历史、政治经济学后,本电子书便立项了,此书也可理解为我个人三
观的阶段性总结% ,我希望借此能更加理解和融入这个世界
。

然而,并没有什么彼岸或者终点……

\section*{贫困的社会科学和历史实践}

我对人文社科的探索,只是发现遍野荒芜杂乱。

致力于人类共同体发展的大师、宏观叙事一片寂静。马克思的学说可分为三部分:经典
哲学部分,因经典哲学只满足于认识世界、而对改造世界有先天缺陷,早被马克思自己
扬弃。“一切社会变迁和政治变革的终极原因……不应当到有关时代的哲学中去寻找,
而应当到有关时代的经济中去寻找”。马克思由此走向政治经济学,足够深刻地揭示了
资本主义的破坏性,但在建设性上只有对“自由人的联合王国”等希翼,乏善可陈,实
则充满绝望。他的科学社会主义部分,试图通过“科学”论述指向彼岸,但只有纲领而
缺少具体实践措施。

最满足社会主义资本发展条件、已形成国有垄断和强大党派的德国没有走向社会主义,
最不满足经济发达条件的俄国却建立起了苏联,由此开始一系列“社会主义改造”。

20世纪50年代之初,早期苏联、世界实践和理论上的困境,使激进左翼思想在法国等国
由强盛急转为衰颓,深受尼采和马克思影响的后现代主义理论又发扬起来。他们认识到
了现代性(资本理性)“可以吞噬一切”。后现代主义反对总体论、本质论和霸权,倡
导微观的欲望、解辖域化、少数族群权利及多元化,于悲观和批判中求彼岸。但不过几
年便被虚无思想所占据,后来转出一批虚无主义者。这时其实已经不能叫做后现代主义
了,它已没有了真正的批判精神。其实疲弱的当代人哪有什么勇气去信奉虚无主义呢!万
物皆虚,那唯有自己的切实体验为真。所谓当代虚无主义不过是个人功利主义的画皮而
已,且丑陋。

1968年法国、日本、美国、西德、意大利、墨西哥、巴西、泰国等国“未曾被预见,也
不可预知”地突然爆发主要以学生青年为主力的左翼运动,“夺权的最后一次机会”,
然而这些运动紧接着就被更强大的右翼浪潮压倒。

70年代左右新自由主义出现,80年代英国撒切尔夫人和美国里根总统将之大规模应用,
全球化影响至今。联合国外债与人权问题独立专家也多次提交报告批判货币基金组织、
世界银行、发达国家借助新自由主义对他国压榨剥削。

1985年9月22日,美国为干预外汇市场,使美元对日元及德国马克等主要货币有秩序性地
贬值,解决巨额贸易赤字,与日本、德国、英国和法国在美国纽约广场饭店签订广场协
议。加之日本之后的系列操作,开启了日本“失去的三十年”的序幕。由此,民族国家
和全球化的巨大张力成为理论研究热点和世界发展方向。如今的中美贸易战也在这一范
畴之中。其实在苏联时期布哈林早已提供了完整的民族国家和全球化理论框架,一直未
被重视。

美国自上世纪70年代以来,一直以超发货币向世界各国(包括美国自己)民众收取铸币
暗税,每次超发引起的经济危机又以更大规模的货币超发、大水漫灌来应对,终使美国
政府和人民均陷入债务和货币缩水泥潭之中,唯有最富有的人越来越富。各主要经济体
也纷纷均向美元看齐,开启了滥发货币的狂欢。这不过是寅吃卯粮、越积越大泡沫炸弹
终将爆炸。各国危机中又都野蛮希望能将危机转嫁至他国……

当代哲学不必说改造世界,就连对认识\textbf{当前世界}似也已提不起兴趣。

人的原始性和动物性真是没多少变化呵。

\section*{贫困的个人}

本书框架草稿于2019年完成时,绝望情绪不减反增:在宏观社会认识上,我没有找到任
何建设性的方法,仍只能意识到马恩所说的破坏性;在微观个人实践上,世界仍是时时
让我吃惊于他的魔幻与不可思议,只不过个人能以更好心态去应对绝望。

并且我发现,我只不过鹦鹉学舌而已。同样的理论认识,也有不少人理解,我身边的草
根们也可直透本质。只是理解无用,他们对此选择了麻木或者忽略,各自的生活已然如
此艰难,哪有闲暇他顾……

纵然我有大能,可获知不那么贫困的理论,想必对我个人生活也无多大用处。抽象、一
般的人文社科理论与个人具体的现实实践之间有太多太多的不同,指导不了多少东西。
何况我所知只是贫困的呢,写作之路就此终止。

恍惚间又过五年,来到了2024,我已过四十岁,身心机能均开始走下坡路,也无意再去
搜寻什么大道理,已经开始“认命”,接受自己的各种不足,一切均是自己种下的果,
果再酸却也不必强求什么。我还是要继续生活的啊,就算不再去面对那人类社会谜题,
可如何自得,如何获取一种踏实感呢?

一些日本作品似乎提供了一个角度——礼赞万物为生而做的一切努力和挣扎,包括恶行。
慕强而轻善恶,也就在相当程度上将人性与兽性等同,强力大于善恶,我想这也是日本
至今仍在不断滋生万恶军国主义思想的原因,不可不警惕。

令人意外的是,我忽然发现儒释道这类心性学问倒是挺适合求得心灵踏实,真是未曾设
想的道路……些微欣慰之余,又怅然若失。我知道过去的我即将死去,新的自己将
要“重生”,同时死去的还会有这本书。

除去真诚,此书似是无用且丑陋。可我确实很喜欢这本书啊。它终归有着我的存在,如
果今时我不去修订完成,那这本书就再也完成不了啦。好吧,随意吧,那就写完吧。起
码在写作时,我能体会到久违的踏实感。其实这种“踏实”也是逃避,当我走向丁家庄
之后开始写这样一本费时费力又各方面不讨好的电子书时,何尝不是我惯熟二十余年的
懦弱手段——逃避,逃避虽可耻却有用。我一直知道,也一直践行着自己这贫困之道……

我安慰自己,此书无用、估计也不会有二三十个读者,却也是了不起的一本书。他了不
起在仍敢发声直言,仍希望向上走。总要、也总会有人去做这些“无用之事”,我愿成
其中一粒沙尘。


尼采曾盛赞一位夫人,因其对自己的孩子说“亲爱的,你总是做傻事,做傻事让你特别
快乐”。我想我也是这样一个傻小孩吧,哈哈……

笔者孙滨文责自负。

\section*{许可协议}

本书源代码及PDF成品文件放置在 \url{https://github.com/sd44/dingjia},采
用 Creative Commons “署名-非商业性使用-相同方式共享 4.0 国际”许可协议,可在
遵守协议的前提下自由散播、拷贝、修改等。

我在立项之初便想实现社会化编辑,但一直无人参加,只好明知不可为而为之,硬着头
皮写下很多超乎我能力的文字。加之笔者又爱大放厥词,肯定错漏颇多;在此再次征求
修订校对或合作者,一切皆可改。可通过github或邮箱联系。

项目之初错用了小众且有难度的\LaTeX 排版PDF,使他人难以方便参与协作。如果预计合作
者增删改内容可达本书2\%以上,我就将\LaTeX 源代码转换为更适合公众编辑
的mediawiki。

但恐怕还是难以如愿——不会有人深度参与。那本书永远停留在当前征求意见稿阶段就
好,我也欣慰于它的使命已可告终结。

{\raggedleft

孙滨 \qquad\qquad \par

2024年6月 \quad}

\chapter{鸣谢}

感谢丁家庄愿意信任我并接受社会调查的人们。感谢深圳的邱文,使我踏上对社会的
研究和学习道路,并多次给予鼓励和指导;成都的saintjoe,也与我就摄影和社会展开
探讨。

非常感谢知乎和微信上的“到芬兰车站”与“东方拖拉机厂打工人读书小组,特别是行
止、澎啊湃、Exphilonous、杜若致远、白鹤飞、寒霜雪蝶等人,大家的洞见交流和激烈
争论使我获益良多。


最为特别的感谢,献给我的大儿子——子墨。在我不想继续写作或者担心直言后果时,
却是你这个小小少年在鼓励我,只是要爱自己、保护自己多一些啊。

最大最大的感谢给予我的妻子康利。结婚数年来,我屡屡不务正业,不事生产,毫无建树,总
做些傻事惹人耻笑,妻十余年来默默承担起照顾家人和孩子的工作,受苦受累颇多,却
无怨言,静静包容我的一切胡闹。

谨以此书献给我的两个孩子,子墨和子韩。希望书中有些内容可以使你们在将来少走些
弯路,多些对社会的认识和理解。




% 但韩康利虽不% 支持但也不反对,事实上纵容我傻乐,并默默承担起家中老人与两个孩子的繁重照顾工作。% 此生我最大的幸运是娶到了韩康利,因这一幸运我不再有任何一点立场指责上天待我凉薄。焦虑、质疑、批判是人类进步的必要要素。不能认可批判的建设性价值,将面对停滞、倒退或向更为危险的境地发展。批判要建立在对真实和事实的求证基础上,它不止有批评,同样包括对好方面的肯定。我们生活在一个资本现代性的理性愈加强大的时代,它强大和贪心到试图去规训,并能够在相当程度上规训其它一切理性,妄图称王使所有理性臣服。这更需要我们具有批判精神。

% 这本电子书是个四不像但又什么都有的怪物,涉及政治、社科、经济、人文等诸多方面,
% 贯穿始终的核心是笔者自身应用社会学对于社会和人\improve{以后加入资本?}的思考。
% 笔者并非学富五车的专业人士,自身缺陷与恶习比比皆是,这均使本书内容存在各种各
% 样缺陷,惹人耻笑。它也几乎不会产生任何影响力或激起什么波澜。但我想它最重要和
% 最强大的意义,是展现出一种个人真诚、直接和积极的社会学历程(详情请
% 见\cref{chap:gerenshehuixue}),一些人也正自觉或不自觉地走在这样的道路上。笔
% 者愿作这条道路上的一块铺路石,以使同行者不那么孤单、寂寞。这条道路的发展将使
% 人类和社会受益。

% 在国际政治的角斗场中,如果中国因为社会剧痛剧变从而变得贫弱多病,就必然被许多国家% 暂时搁置他们彼此之间的争斗,转头立刻联合起来对中国群起攻之,分而食之,进而敲骨吸% 髓,让中国万劫不复。看不到这一点的人,在政治上是幼稚无知或者是别有用心的。之所以% 会造成这种局势,并非是因我们中国和外国的政治体制或意识形态不同,实质上我认为我们% 与他国的共同点——即使在政体和意识形态上——远比不同点多的多,甚至呈多倍比例。只是因% 为中国物产丰富,疆域辽阔,市场第一,发展态势迅猛及潜力巨大,且在历史上我们同欧美% 各国的交流,没有他们彼此之间的互通交流频繁。我要声明,我坚决拥护党和国家的领导,% 对所谓西方自由民主等陷阱和中国重改良等坚决排斥。但是我坚持轻改良,所谓正能量,不% 是不说难听的话、不指出事物缺陷,而是使事物向更好方向发展的力量,改良是需要的。

% 在这过程中,在项目进行过程中,喜闻有两位姑娘,可能是大学生也在进行丁家庄的社会学% 问卷调查工作,同意接受调查的村民或租户可以获得50元奖励。我感觉可能是国立大学调查,% 很是欣喜,希望中国能在社科方面快速发展。

% 如果有可能的话,我想做的下一个项目是不像丁家庄项目这么宏大难控的,它的表达方式更% 加具体和容易引起社会影响,同时也更加注重感官而不是理性,可以更多借助摄影等方式,% 那就是关于ADHD(小儿多动症)的批判。小儿多动症是一个症候群,其中一些表现可以归为% 个人特质或非精神性病变等,国外已有多部相关著作,深度理论构建应当可以从反精神病学% 与精神病批判上汲取养分(即使没有深度理论,项目应当也可获得成功)。学习社科这一年% 多来,我已经有爱上社科了,但就现实情况来看,丁家庄项目是我进行的第一个社科项目,% 也很可能是最后一个。我的年龄、基础、精力、家庭、经济能力等似乎很难允许我再做类似% 的工作。如果有读者能进行ADHD这个项目就太好了。

% 特别感谢泰安的李玉刚,与我进行多次最接地气的激烈讨论并直言不讳。特别% 感谢丁家庄愿意信任我并接受调查的人们。% 感谢微信上广州HiFi_Tam所建摄影群,其中北京刘烜超,广州Hifi_Tam,广州邱邱,南通兽% 无不摄,上海Keith,厦门Resean均对本项目不成气,但我也无力去更改的摄影部份提出宝% 贵意见(以上排名均安拼音顺序,不分先后)。


% 谨以此书献给我的妻子韩康利。

% 2014年,笔者的工作和住宿地点均变更为济南市化纤厂路,距离丁家庄菜市场不% 足500米。2016年7月份,笔者为练习街头摄影常在街头漫无目的地游荡,有次穿过菜市场忽% 然发现了一片新景象——丁家新村,济南人俗称丁家庄,这是一个城中村。笔者虽已在济南工% 作9年,在丁家庄附近生活2年,也去过济南市诸多地方,但从不知道有丁家庄城中村这样一% 个存在。

% 虽然笔者就出生和成长在一个贫困县,也去过几次农村,但初入丁家庄仍因它表面的破败和% 杂乱而产生一种恐惧感,总感觉可能有治安危险。它既不同于城市也不同于农村,一些地方% 甚至比农村还要显得困窘残破。

% 如何不流于表面地展示这里?如何避免消费苦难,去真正深入地表现这个地方?笔者二三十% 次进入丁家庄城中村,仍未找到答案。生活在深圳的邱文向我提了一个建议,用人类学的眼% 光去拍摄、组织照片,直接展示他们的衣食住行、教育娱乐等方面,使其呈现出丁家庄城中% 村的整体生活面貌,并可作为一个样本进行留存。

% 笔者在邱文的建议的基础上,最终选择了社会学方向来做丁家庄的全面考察。“空间生% 产”这一块比较知名的学者有列斐伏尔、大卫·哈维等人,这些学者基本都是批判实践社会学% 方向。批判社会学的奠基人一般被认为是马克思,列斐伏尔、大卫·哈维等人也深受马克思主% 义影响,要较好理解他们的著作,难以跳过马克思。因此我又学习了《资本论》等马克思、% 恩格斯著作,这使本书带有马克思色彩。

% 社会学学习的进展,和笔者摄影水平的粗浅,笔者渐感有必要% 随着学习的深入,也因笔者摄影技术的粗浅,笔者渐感摄影的无力,% 当代世界,不管是左派还是右派,大多数人对马克思的理解常常是肤浅甚至错误,“马克% 思”往往成为一种标签和符号,一些人看到这个标签和符号就产生强烈的好恶感和价值判定。% 希望大家能够从理论内容本身来判定,而非这种先入为主。

%%% Local Variables:
%%% mode: latex
%%% TeX-master: "../main"
%%% End:
