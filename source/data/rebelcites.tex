\textbf{贯穿于整个资本主义历史,城市化从来都是吸收剩余资本和剩余劳动力的关键手段,城
镇化凭借不断变更空间和场所的使用功能,实现空间垄断及垄断地租,进一步推动资本
积累。}(p3)

庞氏骗局、击鼓传花、只要我跑得比你快  金融游戏

2023年三季度末,房地产开发贷款余额13.17万亿元,同比增长4%,增速比上年末高0.3个百分点。个人住房贷款余额38.42万亿
\url{https://www.gov.cn/lianbo/bumen/202311/content_6913312.htm}

城镇权即是对城镇化过程的某种控制权,对建设城市和改造城市方式的某种控制权,从
根本上和以激进的方式来实现对城市的控制权。

在大卫看来,城市生活质量也已经成为一种有钱人的商品。

在平等的权利之间,力量起决定作用。

资本主义永远都在生产城市化所要求的剩余产品。反之,资本主义也需要城市化来吸收
无止境生产出来的剩余产品(惊险地一跃)。因此,资本主义发展与城市化之间呈现出一种内在的联系。

如果\textbf{劳动力短缺},劳动力工资太高,那么必须重新培训现存的劳动力(以技术手段引起
失业,或打击有组织的工人阶级力量——如撒切尔和里根在20世纪80年代所采用的——
是两种基本方法),或者找到新的劳动力。从一般意义上讲,必须要找到新的生产方式;
从特定意义上来讲,必须要找到新的自然资源。

\textbf{城乡 资本的原始积累}

奥斯曼清楚地认识到,他的使命是通过城市化来帮助解决剩余资本和失业问题。……当
建筑师雅克·伊尼亚斯·希托夫向奥斯曼展示他关于巴黎新林荫大道的设计方案时,奥
斯曼驳回了这个方案,他说:“不够宽……你的设计宽度是40米,我要的是120米。”
他吞并了郊区,重新设计了整个街区,如Les Halles。要做到这一点,奥斯曼需要\textbf{
新的金融机构和债务工具},即建立在圣西门基础上的流动信贷和不动产信贷。实际上,
他建立了一个 \textbf{原始凯恩斯体系},利用债务融资,改善城市基础设施,从而解决剩余资本的出路问题。

1848年的经济危机是一个明显的剩余资本和剩余劳动力限制而无法利用的危机。……拿
破仑公布了一个宏大的海内外基础设施投资项目。……最重要的是重建巴黎的城市基础
设施。……奥斯曼使用了类似\textbf{凯恩斯}\footnote{不是经济学发明经济模式,而是经济学总结发现已有
  的经济模式}的体制,利用债务融资,改善城市基础设施,从而解决剩余资本的出路问题。

摩西改变了有关城市发展的思维尺度,通过(债务融资建设的)高速公路系统和基础设
施改造,通过郊区化,通过对城市和整个都市区域的重新建设来吸收剩余产品,进而解
决剩余资本的吸收问题。

掏空为代价……隔离在新的繁荣之外……负面影响……反抗。

美国的郊区化……1973年全球房地产泡沫的破灭……

正如威廉·塔布指出,\textbf{国家权力和金融机构间不稳定的联盟}让纽约度过了1975年的金融
危机,率先对这个问题作出了新自由主义的回答;以牺牲个人阶级的生活水准为代价来
保护资本的权力。同时放松对市场的管制,让其相对自由发展。(p11)

% 一辆区域危机和崩溃的过山车(1997-1998年东亚和东南亚、1998年的俄罗斯、2001年
% 的阿根廷等)。



直到2008年,美国人一直认为,住宅市场是美国经济重要的稳定器,尤其是在20世纪90
年代末高技术崩盘之后。通过新的建设,房地产市场直接吸收了大量剩余资本,而在历
史性的低利率条件下,抵押贷款再融资的浪潮导致了住宅资产价格急剧飞涨,从而推动
了美国国内消费品市场和服务市场。美国用平均日借款20亿美元来维系其不可满足的消
费,同时以借贷融资开展阿富汗和伊拉克战争。

\textbf{中国的城市化发展对全球经济以及吸收剩余资本具有巨大的影响。}

与以往所有的城市化进程一样,……\textbf{一直依赖于建立新的金融体制和措施,以便组织
  起维持城市化的信贷需要。开始于20世纪80年代的金融创新,特别是销售给世界范围
  投资者的地方抵押贷款的证券化和打包,建立持有债务抵押债券的新型金融机构,都
  发挥了关键的作用。……整体利率下降……}(p13) 结合下上海城投债收购公司,并
且说明\textbf{局内人}比专家更容易了解,而笔者这样的门外汉难以触及核心。这类金融创新
的收益是多方面的:他分散了风险,允许剩余储备更容易地介入过剩的住房需求。通过
金融机构间的协调,这类金融创新导致整体利润率下降(给那些创造奇迹的金融中介机
构带来了巨大的财富)。但是,分散风险并不等于消除风险。而且,由于风险可以转嫁
到其他地方,这样广泛的风险分散甚至鼓励了\textbf{更加风险的地方行为}。由于\textbf{没有适当
的风险评估管理},抵押贷款市场已经失控……财政亏空,都再次呈现在2008年的次贷和
住宅资产危机之中。

(1997-1998年东亚和东南亚、1998年的俄罗斯、2001年
% 的阿根廷等)

\textbf{几乎在世界上的每一个城市都可以看到因有钱人而产生的建设高潮——而有钱人通常具
有类似的令人沮丧的性格。与之相伴的是农民在农业工业化和商业化中变得一无所有,
洪水般地涌向城市,沦落为贫困的城市移民。}可以结合下小城市分房住民伴随 \textbf{城市
空心化}的阶级掉落。

\textbf{在过去的几十年里,新自由主义让富裕精英的阶级力量得以复原。}中国的亿万富翁数量
相比于世界……

吸收剩余已通过“\textbf{建设性摧毁}”引起了反反复复的城市重建。由于穷人、弱势群体和在
政治权利上被边缘化的那些人总是首当其冲且受到最严重的影响,所以城市重建基本上
总是具有阶级性的。新的城市是在旧城市的残骸上建立起来,因而需要暴力。

http://www.marxistjuris.com/show.asp?id=601

https://zhuanlan.zhihu.com/p/35519542

\begin{quotation}
  我所说的“欧斯曼计划”,是指把工人区,特别是把我国大城市中心的工人区从中豁
  开的那种已经普遍实行起来的办法,而不论这是为了公共卫生或美化,还是由于市中
  心需要大商场,或是由于敷设铁路、修建街道等交通的需要。不论起因如何不同,结
  果到处总是一样:最不成样子的小街小巷没有了,资产阶级就因为这种巨大成功而大
  肆自我吹嘘,但是,这种小街小巷立刻又在别处,并且往往就在紧邻的地方出现。\pagescite[][243]{maenwen3}

  住房短缺也是这样。现代大城市的扩展,使城内某些地区特别是市中心的地皮价值人
  为地、往往是大幅度地提高起来。原先建筑在这些地皮上的房屋,不但没有这样提高
  价值,反而降低了价值,\textbf{因为这种房屋同改变了的环境已经不相称;它们被拆除,改
  建成别的房屋。}市中心的工人住房首先就遇到这种情形,因为这些住房的房租,甚至
  在住户挤得极满的时候,也决不能超出或者最多也只能极缓慢地超出一定的最高额。
  这些住房被拆除,在原地兴建商店、货栈或公共建筑物。波拿巴政权曾通过欧斯曼在
  巴黎利用这种趋势来大肆敲诈勒索,大发横财。但是欧斯曼的幽灵也曾漫步伦敦、曼
  彻斯特和利物浦,而且在柏林和维也纳似乎也感到亲切如家乡。结果工人从市中心被
  排挤到市郊;工人住房以及一般较小的住房都变得又少又贵,而且往往根本找不到,
  因为在这种情形下,建造昂贵住房为建筑业提供了更有利得多的投机场所,而建造工
  人住房只是一种例外。\pagescite[][193]{maenwen3}

\end{quotation}

\begin{quotation}
  在当今的时代条件下,我们发现恩格斯对住宅问题的分析也具有某种空间权利及空间
  正义的考量。他对资本统治下城市空间的分化隔离、无产阶级和小资产阶级生存空间
  被剥夺的现象进行了科学分析,把城市空间的矛盾归结为资本主义生产方式本身,实
  乃深刻的思想洞察。\pagescite[][74]{zhuzhaiwenti}

  我们可以从以下两个层面来理解资本逻辑的特征:其一,\textbf{资本逻辑是以资本的增殖为
    终极目的而把一切变成为实现这一目的之手段的一种强制关系。}在这一关系中,资
  本增殖是一切行动所围绕旋转的轴心,而作为创造资本的主体--劳动者却被降低为被
  剥削的对象和役使工具,其创造的剩余价值被资本家无偿占有。其二,资本逻辑的展
  开过程就是资本逻辑内在矛盾形成和发展的过程。资本逻辑的过度膨胀造成了无产阶
  级的极端贫困,激化了社会矛盾,产生了诸多社会问题以至于威胁到资本的统治,从
  而迫使它寻求维系剥削与总体利润最大化的长远机制,其外在表现就是资产阶级解决
  社会问题的种种尝试,这在一定程度上反映出资本主义制度自我修复、自我进化的能
  力。当然,这只是资本逻辑的自我循环。

  资本拜物教的信徒对于社会问题的认识必然基于对资本逻辑的认同,他们从不怀疑这
  一逻辑本身……这也就决定了资产阶级在看待社会问题上的局限
  性。\pagescite[][100]{zhuzhaiwenti}

  房租的背后是资本的生产链条以及经济规律的调节作用。其中主要的一环就是地租,
  这是由于城市的迅速发展使得现实的经济条件发生了重大变化,因为工业和商业资本
  的有机构成高于农业,所以为了\textbf{加快资本周转和创造剩余价值},就必须让渡给土地所
  有者更多的地租,以便于城市土地顺畅地参与市场资源的配置。

  恩格斯:我把社会问题的充分解决当作采取赎买出租屋办法的前提。
\end{quotation}

\begin{quotation}
  现代大城市的扩展,使城内某些地区特别是市中心的地皮价值人为地、往往是大幅度地
  提高起来。原先建筑在这些地皮上的房屋,不但没有这样提高价值,反而降低了价值,因
  为这种房屋同改变了的环境已经不相称;它们被拆除,改建成别的房屋。市中心的工人
  住房首先就遇到这种情形,因为这些住房的房租,甚至在住户挤得极满的时候,也决不能
  超出或者最多也只能极缓慢地超出一定的最高额。这些住房被拆除,在原地兴建商店、
  货校或公共建筑物。波拿巴政权曾通过欧斯曼在巴黎利用这种趋势来大肆敲诈勒索,大
  发横财。但是欧斯曼的幽灵也曾漫步伦敦、曼彻斯特和利物浦,而且在柏林和维也纳似
  乎也感到亲切如家乡。结果工人从市中心被排挤到市郊;工人住房以及一般较小的住
  房都变得又少又贵,而且往往根本找不到,因为在这种情形下,建造昂贵住房为建筑业提
  供了更有利得多的投机场所,而建造工人住房只是一种例外。\pagescite[][252]{mawen3}
\end{quotation}

向开发商出售土地为填补地方政府金库提供了一棵利润丰厚的摇钱树……“地方政府债
务的急剧上升和对投资公司借贷控制不力”(多由政府主导)现在被认为是中国经济
的主要风险,不仅给中国也给全世界的增长前景蒙上了深深阴影。截至 2011 年,中国
政府估计市政债务约为 2.2 万亿美元,相当于“全国国内生产总值的近三分之一”,其
中 80\% 的债务可能由账外投资公司持有,这些公司由市政府主导,但严格来说不是市
政府的一部分。这些组织正在以极快的速度建设新的基础设施和标志性建筑,使中国城
市如此壮观。但市政当局的累积债务是巨大的。一波违约浪潮“可能成为中央政府的巨
大负债,中央政府背负着约2万亿美元的债务:”崩溃之后是长期“日本式停滞”的可能
性是非常真实的。2011年中国经济增长机器的放缓已经导致进口减少,而这反过来又将
反弹到世界上所有那些在中国原材料市场的支持下蓬勃发展的地区。英文版 60页


% Please add the following required packages to your document preamble:
% \usepackage{booktabs}
% \usepackage{multirow}
% \usepackage{graphicx}
% \usepackage[table,xcdraw]{xcolor}
% Beamer presentation requires \usepackage{colortbl} instead of \usepackage[table,xcdraw]{xcolor}
\begin{table}[htbp!]
  \centering
  \StartDefiningTabulars
  \resizebox{\textwidth}{!}{%
    \begin{tabular}{@{}lcccccccccccccll@{}}
      \toprule
      & \multicolumn{6}{c}{单位:万亿元} & \multicolumn{1}{l}{} & \multicolumn{6}{c}{与GDP的百分比} &  &  \\
      \mrowcell & 2018 & 2019 & 2020 & 2021 & 2022 & 2023 &  & 2018 & 2019 & 2020 & 2021 & 2022 & 2023 &  &  \\ \midrule
      中央政府 & 15 & 17 & 21 & 23 & 26 & 29 &  & 16 & 17 & 20 & 20 & 22 & 23 & \cellcolor[HTML]{FFCCC9} & \cellcolor[HTML]{FD6864} \\
      地方政府 & 18 & 21 & 26 & 30 & 35 & 40 &  & 20 & 22 & 25 & 27 & 30 & 32 & \multirow{-2}{*}{\cellcolor[HTML]{FFCCC9}\begin{tabular}[c]{@{}l@{}}官方政\\ 府债务\end{tabular}} & \cellcolor[HTML]{FD6864} \\
      地方政府融资平台(可能) & 35 & 40 & 45 & 50 & 57 & 66 &  & 38 & 40 & 44 & 44 & 48 & 53 &  & \cellcolor[HTML]{FD6864} \\
      政府基金\footnotemark[1] & 6 & 7 & 9 & 12 & 14 & 16 &  & 6 & 7 & 9 & 10 & 12 & 13 &  & \multirow{-4}{*}{\cellcolor[HTML]{FD6864}\begin{tabular}[c]{@{}l@{}}IMF\\ 增扩\\ 政府\\ 债务\end{tabular}} \\
      家庭 & 48 & 55 & 63 & 71 & 73 & 75 &  & 52 & 56 & 62 & 62 & 61 & 61 &  & \cellcolor[HTML]{34CDF9} \\
      企业(不包括地方政府融资平台) & 105 & 111 & 121 & 128 & 142 & 153 &  & 115 & 112 & 118 & 113 & 119 & 123 &  & \multirow{-2}{*}{\cellcolor[HTML]{34CDF9}\begin{tabular}[c]{@{}l@{}}私人部\\ 门债务\end{tabular}} \\
      &  &  &  &  &  &  &  &  &  &  &  &  &  &  &  \\
      \textbf{备忘项} &  &  &  &  &  &  &  &  &  &  &  &  &  &  &  \\
      合计 & 227 & 252 & 285 & 315 & 348 & 380 &  & 248 & 254 & 278 & 277 & 291 & 306 &  &  \\
      广义政府债务 & 33 & 38 & 47 & 54 & 62 & 69 &  & 36 & 38 & 45 & 47 & 51 & 56 &  &  \\
      IMF增扩政府债务 & 74 & 85 & 101 & 115 & 132 & 151 &  & 80 & 86 & 98 & 101 & 110 & 122 &  &  \\
      IMF增扩政府债务(仅地方) & 53 & 61 & 71 & 80 & 92 & 106 &  & 58 & 62 & 69 & 70 & 77 & 85 &  &  \\
      &  &  &  &  &  &  &  &  &  &  &  &  &  &  &  \\
      名义GDP & 92 & 99 & 103 & 114 & 120 & 124 &  &  &  &  &  &  &  &  &  \\ \bottomrule
    \end{tabular}%
  }
  \StopDefiningTabulars

  \raggedright
  \footnotemark[1]{\tiny 政府指导基金和专项建设基金(仅包括社会资本部分)。}

  \centering
  \caption{中国非金融部门债务}
  \label{tab:chinadebt}
  \capsource{来源:CEIC数据有限公司;Capital IQ;中国财政部;以及IMF工作人员的估计。}
\end{table}


\begin{table}[hbtp!]
  \centering
  \StartDefiningTabulars

  \resizebox{\textwidth}{!}{%
    \begin{tabular}{@{}llllllllllllllll@{}}
      \toprule
      &  & 1950s & 1960s & 1968 & 1970s & 1980s & 1986 & 1990s & 2000s & 2004 & 2010s & 2019 & 2020 & 2021 & 2022 \\ \midrule
      \multicolumn{2}{l}{\textbf{世界}} & \textbf{96.8} & \textbf{101.5} & \textbf{106.4} & \textbf{115} & \textbf{144.3} & \textbf{156.8} & \textbf{180.1} & \textbf{196.8} & \textbf{198.6} & \textbf{219.4} & \textbf{228.9} & \textbf{258} & \textbf{248.1} & \textbf{238.1} \\
      \multicolumn{2}{l}{\textbf{发达经济体}} & \textbf{110} & \textbf{115.4} & \textbf{118.9} & \textbf{133.6} & \textbf{165.4} & \textbf{177.1} & \textbf{202} & \textbf{229.5} & \textbf{225.9} & \textbf{267.4} & \textbf{268.8} & \textbf{301.3} & \textbf{290.1} & \textbf{277.9} \\
      & 欧洲地区 & 53.5 & 68.1 & 70.7 & 118.5 & 144.1 & 147.3 & 176.5 & 214.4 & 208.6 & 255.1 & 246.7 & 273.6 & 266.8 & 254.4 \\
      & 日本 & 13.1 & 79.7 & 127.9 & 157.5 & 229.5 & 242.2 & 296.1 & 336.7 & 333.5 & 386.9 & 400.9 & 442.2 & 439.8 & 447.4 \\
      & 英国 & 134 & 120.5 & 129 & 116.7 & 118.7 & 125.7 & 153.1 & 205.2 & 197.3 & 248.5 & 240.8 & 279.1 & 269.3 & 252.1 \\
      & 美国 & 133.5 & 140.4 & 138.7 & 139.8 & 163.2 & 176.8 & 189.1 & 218.3 & 217.7 & 256 & 260.3 & 297.4 & 283.5 & 273.9 \\
      \multicolumn{2}{l}{\textbf{新兴市场经济体}} & \textbf{28.2} & \textbf{33.1} & \textbf{37.8} & \textbf{39.5} & \textbf{65.5} & \textbf{72.6} & \textbf{87.2} & \textbf{101.6} & \textbf{102.7} & \textbf{148.5} & \textbf{176.8} & \textbf{201.2} & \textbf{195.3} & \textbf{191.2} \\
      & 中国 &  &  &  &  & 71.3 & 73.4 & 97.2 & 137.9 & 142.4 & 211.5 & 246.8 & 268.8 & 264.9 & 272.1 \\
      & 其他 & 28.2 & 33.1 & 37.8 & 39.5 & 65.7 & 72.5 & 85.4 & 91 & 91.6 & 110.6 & 123.1 & 141.1 & 131.8 & 124.2 \\
      \multicolumn{2}{l}{\textbf{低收入发展中国家}} & \textbf{} & \textbf{} & \textbf{} & \textbf{20.2} & \textbf{43.6} & \textbf{51.5} & \textbf{73.7} & \textbf{61.9} & \textbf{65.3} & \textbf{61.9} & \textbf{77.3} & \textbf{85.3} & \textbf{87.4} & \textbf{87.8} \\ \bottomrule
    \end{tabular}%
  }
  \caption{全球债务总额(占GDP百分比,加权平均数)}
  \label{tab:totaldebt}


  \resizebox{\textwidth}{!}{%
    \begin{tabular}{@{}llllllllllllllll@{}}
      \toprule
      &  & 1950s & 1960s & 1968 & 1970s & 1980s & 1986 & 1990s & 2000s & 2004 & 2010s & 2019 & 2020 & 2021 & 2022 \\ \midrule
      \multicolumn{2}{l}{\textbf{世界}} & \textbf{56.2} & \textbf{39.8} & \textbf{36.3} & \textbf{33.1} & \textbf{47.6} & \textbf{54.3} & \textbf{62} & \textbf{66.5} & \textbf{69.8} & \textbf{81} & \textbf{84.9} & \textbf{100.4} & \textbf{96} & \textbf{92} \\
      \multicolumn{2}{l}{\textbf{发达经济体}} & \textbf{64} & \textbf{44.3} & \textbf{39.5} & \textbf{36.1} & \textbf{50.7} & \textbf{57.5} & \textbf{66.4} & \textbf{75.3} & \textbf{76.8} & \textbf{104.6} & \textbf{105.4} & \textbf{124.4} & \textbf{118.7} & \textbf{113.5} \\
      & 欧洲地区 & 32.5 & 24.7 & 25 & 27.7 & 47 & 52.1 & 67 & 69.9 & 69.7 & 90.9 & 85.9 & 99.2 & 97.3 & 93.2 \\
      & 日本 & 13.1 & 10.1 & 11.8 & 23.4 & 64.3 & 74 & 89 & 166.6 & 169.5 & 227.5 & 236.4 & 258.7 & 255.4 & 261.3 \\
      & 英国 & 134 & 81.5 & 70.9 & 57.5 & 40.3 & 41 & 38.1 & 42.6 & 39.8 & 84.7 & 85.5 & 105.6 & 105.9 & 101.4 \\
      & 美国 & 69.7 & 54.4 & 48.7 & 43.6 & 51.6 & 57.7 & 66.3 & 64.1 & 66.1 & 104.1 & 108.7 & 133.5 & 126.4 & 121.4 \\
      \multicolumn{2}{l}{\textbf{新兴市场经济体}} & \textbf{15.5} & \textbf{18.8} & \textbf{20.6} & \textbf{21.4} & \textbf{35.6} & \textbf{40.2} & \textbf{41.5} & \textbf{40.9} & \textbf{44.1} & \textbf{44.3} & \textbf{55.7} & \textbf{65.8} & \textbf{64.8} & \textbf{65.2} \\
      & 中国 &  &  &  &  &  &  & 21.2 & 26.9 & 26.4 & 44.3 & 60.4 & 70.1 & 71.8 & 77.1 \\
      & 其他 & 15.5 & 18.8 & 20.6 & 21.4 & 38.6 & 45.8 & 46.3 & 44.7 & 49 & 44 & 52 & 61.9 & 58.4 & 55.3 \\
      \multicolumn{2}{l}{\textbf{低收入发展中国家}} & \textbf{} & \textbf{} & \textbf{} & \textbf{15.6} & \textbf{36.2} & \textbf{43.2} & \textbf{64.8} & \textbf{45.8} & \textbf{51.2} & \textbf{34.8} & \textbf{42.9} & \textbf{48.5} & \textbf{48.5} & \textbf{48.4} \\ \bottomrule
    \end{tabular}%
  }
  \caption{全球公共债务(占GDP百分比,加权平均数)}
  \label{tab:publicdebt}

  \resizebox{\textwidth}{!}{%
    \begin{tabular}{@{}llllllllllllllll@{}}
      \toprule
      &  & 1950s & 1960s & 1968 & 1970s & 1980s & 1986 & 1990s & 2000s & 2004 & 2010s & 2019 & 2020 & 2021 & 2022 \\ \midrule
      \multicolumn{2}{l}{\textbf{世界}} & \textbf{40.6} & \textbf{61.7} & \textbf{70} & \textbf{81.9} & \textbf{96.7} & \textbf{102.5} & \textbf{118.1} & \textbf{130.3} & \textbf{128.8} & \textbf{138.3} & \textbf{144} & \textbf{157.6} & \textbf{152.1} & \textbf{145.7} \\
      \multicolumn{2}{l}{\textbf{发达经济体}} & \textbf{46} & \textbf{71.1} & \textbf{79.4} & \textbf{97.5} & \textbf{114.6} & \textbf{119.6} & \textbf{135.6} & \textbf{154.1} & \textbf{149.1} & \textbf{162.7} & \textbf{163.4} & \textbf{177} & \textbf{171.4} & \textbf{164.4} \\
      & 欧洲地区 & 21 & 43.3 & 45.7 & 90.7 & 97 & 95.2 & 109.5 & 144.5 & 138.9 & 164.2 & 160.7 & 174.4 & 169.4 & 161.1 \\
      & 日本 &  & 116.1 & 116.1 & 134.1 & 165.2 & 168.2 & 207.1 & 170.1 & 164 & 159.4 & 164.5 & 183.5 & 184.4 & 186.1 \\
      & 英国 &  & 55.6 & 58.1 & 59.1 & 78.3 & 84.6 & 115 & 162.5 & 157.5 & 163.8 & 155.3 & 173.5 & 163.4 & 150.8 \\
      & 美国 & 63.8 & 86 & 90 & 96.1 & 111.6 & 119.2 & 122.8 & 154.2 & 151.6 & 151.9 & 151.5 & 163.9 & 157 & 152.5 \\
      \multicolumn{2}{l}{\textbf{新兴市场经济体}} & \textbf{12.6} & \textbf{14.3} & \textbf{17.2} & \textbf{18.1} & \textbf{29.9} & \textbf{32.4} & \textbf{45.7} & \textbf{60.7} & \textbf{58.6} & \textbf{104.2} & \textbf{121.2} & \textbf{135.4} & \textbf{130.5} & \textbf{126} \\
      & 中国 &  &  &  &  & 71.3 & 73.4 & 86.6 & 110.9 & 116.1 & 167.1 & 186.4 & 198.7 & 193 & 195 \\
      & 其他 & 12.6 & 14.3 & 17.2 & 18.1 & 27.1 & 26.7 & 39.1 & 46.4 & 42.6 & 66.6 & 71 & 79.2 & 73.4 & 68.9 \\
      \multicolumn{2}{l}{\textbf{低收入发展中国家}} & \textbf{} & \textbf{4.1} & \textbf{4.6} & \textbf{4.7} & \textbf{7.4} & \textbf{8.3} & \textbf{8.8} & \textbf{16.1} & \textbf{14} & \textbf{27} & \textbf{34.5} & \textbf{36.9} & \textbf{38.9} & \textbf{39.3} \\ \bottomrule
    \end{tabular}%
  }
  \StopDefiningTabulars
  \caption{全球私人债务(占GDP百分比,加权平均数)}
  \capsource{数据来源:国际货币基金组织全球债务监测报告, 2023 \\
    \url{https://www.imf.org/-/media/Files/Conferences/2023/2023-09-2023-global-debt-monitor.ashx}\par
    注意:表头年份列加`s'标志的是测算的十年平均值,例如1950s表示1950--1959年
    的平
    均债务水平。\\
    1/3的中国私人债务数据为地方政府债务—— \textbf{地方政府融资平台
      债务}和其他\textbf{预算外政府基金债务}。}
  \label{tab:privatedebt}
\end{table}


近几十年来,中国一直是推动全球债务的重要力量。……中国的\textbf{总债务
  与GDP之比}从1980年代中期的70\%左右(接近当时的新兴市场平均水平)增长
到2022年的\textbf{272\%},接近美国,增长了近四倍。不过,以美元计算,中国的总债务
(47.5万亿美元)仍明显低于美国(接近70万亿美元)。中国债务占GDP之比的上升是其
他大型经济体所无法比拟的。从2009年开始,增幅明显加快,尤其是非金融企业债
务……2008-2022年期间,全球债务与GDP之比增长的一半以上可归因于中国债务与GDP之
比的快速上
升。
\footnote{\url{https://www.imf.org/zh/Blogs/Articles/2023/09/13/global-debt-is-returning-to-its-rising-trend}}

贯穿整个资本主义历史,国家通过税收拿走一部分剩余资本。在其社会民主阶段,这一
比例显著上升,国家控制了大部分剩余。最近30年新自由主义一直向剩余资本私有化发
展……其主要成就是,阻止了国家按照20世纪60年代的方式来增加税收。更进一步的举
措是建立新的治理体系,将国家和企业利益相结合,并且在城市改造中应用货币权力,
确保国家机构的剩余资本支出利好于企业资本和上层阶级。\pagescite[][23]{harvey2012rebel}



%%% Local Variables:
%%% mode: latex
%%% TeX-master: "../main"
%%% End:
