\chapter{马克思主义视角的空间生产摘抄}

\section{恩格斯 《论住宅问题》}

臧峰宇《恩格斯〈论住宅问题〉研究读本》:
\begin{quotation}
  我所说的“欧斯曼计划”,是指把工人区,特别是把我国大城市中心的工人区从中豁
  开的那种已经普遍实行起来的办法,而不论这是为了公共卫生或美化,还是由于市中
  心需要大商场,或是由于敷设铁路、修建街道等交通的需要。不论起因如何不同,结
  果到处总是一样:\textbf{最不成样子的小街小巷没有了,资产阶级就因为这种巨大成功而
    大肆自我吹嘘,但是,这种小街小巷立刻又在别处,并且往往就在紧邻的地方出
    现。}\pagescite[][243]{maenwen3}

  住房短缺也是这样。现代大城市的扩展,使城内某些地区特别是市中心的地皮价值人
  为地、往往是大幅度地提高起来。原先建筑在这些地皮上的房屋,不但没有这样提高
  价值,反而降低了价值,\textbf{因为这种房屋同改变了的环境已经不相称;它们被拆除,改
  建成别的房屋。}市中心的工人住房首先就遇到这种情形,因为这些住房的房租,甚至
  在住户挤得极满的时候,也决不能超出或者最多也只能极缓慢地超出一定的最高额。
  这些住房被拆除,在原地兴建商店、货栈或公共建筑物。波拿巴政权曾通过欧斯曼在
  巴黎利用这种趋势来大肆敲诈勒索,大发横财。但是欧斯曼的幽灵也曾漫步伦敦、曼
  彻斯特和利物浦,而且在柏林和维也纳似乎也感到亲切如家乡。结果工人从市中心被
  排挤到市郊;工人住房以及一般较小的住房都变得又少又贵,而且往往根本找不到,
  因为在这种情形下,建造昂贵住房为建筑业提供了更有利得多的投机场所,而建造工
  人住房只是一种例外。\pagescite[][193]{maenwen3}

  在当今的时代条件下,我们发现恩格斯对住宅问题的分析也具有某种空间权利及空间
  正义的考量。他对资本统治下城市空间的分化隔离、无产阶级和小资产阶级生存空间
  被剥夺的现象进行了科学分析,把城市空间的矛盾归结为资本主义生产方式本身,实
  乃深刻的思想洞察。\pagescite[][74]{zhuzhaiwenti}

  我们可以从以下两个层面来理解资本逻辑的特征:\textbf{其一,资本逻辑是以资本的增殖
    为终极目的而把一切变成为实现这一目的之手段的一种强制关系。}在这一关系中,
  资本增殖是一切行动所围绕旋转的轴心,而作为创造资本的主体--劳动者却被降低为
  被剥削的对象和役使工具,其创造的剩余价值被资本家无偿占有。\textbf{其二,资本逻辑
    的展开过程就是资本逻辑内在矛盾形成和发展的过程。}资本逻辑的过度膨胀造成了
  无产阶级的极端贫困,激化了社会矛盾,产生了诸多社会问题以至于威胁到资本的统
  治,从而迫使它寻求维系剥削与总体利润最大化的长远机制,其外在表现就是资产阶
  级解决社会问题的种种尝试,这在一定程度上反映出资本主义制度自我修复、自我进
  化的能力。当然,这只是资本逻辑的自我循环。

  资本拜物教的信徒对于社会问题的认识必然基于对资本逻辑的认同,他们从不怀疑这
  一逻辑本身……这也就决定了资产阶级在看待社会问题上的局限
  性。\pagescite[][100]{zhuzhaiwenti}

  房租的背后是资本的生产链条以及经济规律的调节作用。其中主要的一环就是地租,
  这是由于城市的迅速发展使得现实的经济条件发生了重大变化,\textbf{因为工业和商业资本
  的有机构成高于农业,所以为了加快资本周转和创造剩余价值,就必须让渡给土地所
  有者更多的地租,以便于城市土地顺畅地参与市场资源的配置。}
\end{quotation}

\section{列斐伏尔}

张笑夷《列斐伏尔空间批判理论研究》:
\begin{quotation}
  \textbf{空间的生产不是资本主义生产方式自然而然的再生产,而是被构想的和深思熟虑的
    结果。空间具有政治性,它是政治统治的工具,空间的生产是现代主义国家的政治
    战略。}“空间与政治国家的关联比曾经的领土与民族国家的关联更牢固。它不仅被
  生产力、生产关系和所有权生产;而且它是一种\textbf{政治产品},具有行政和残暴统治性
  的产品、由政治国家上层统治关系和战略决定的产品。并且,这不是在某一政治国家
  范围内,而是在国际和全球范围内,在全球国家体系范围内的生产。”

  从列斐伏尔的观点来看,资本主义生产方式生产了它自己的空间,在空间的生产的进
  程中资本主义的生产方式改变为 “\textbf{国家生产方式}”。同时,空间不仅成了生产力
  要素、生产关系和所有权关系要素,还完全成为政治性的,政治性的空间在当今资本
  主义社会中成为主导性的。空间的政治性主要表现为,首先,空间是意识形态的,它
  是社会技术专家治国制的表象。其次,空间是实践的,是政治权力的工具和手段。最
  后,空间是战略性的,它从属于政治目标,被纳入了剩余价值的生产。

  在过去资本主义的长期发展中,土地曾被视为封建地主阶级的残余而被忽视,建筑业
  的重要性曾远远不及钢铁生产、制糖工业等。而现代资本主义生产实践则相反,\textbf{土
    地进入了生产关系再生产的范畴,在新资本主义的结构性生产关系中处于中心地
    位。}显而易见,政府的住房规划就促进了这种\textbf{以 “不动产” 的动产化为特
    征}的空间的生产。住房建筑与土地不可分割,土地构成住房价值的一部分,于是,
  被分割的一块一块土地成了空间性的产品。“因而,资本投资在房地产部门中找到了
  一个避难所,一个补充性和互补性的剥削领域。......在一些国家中,比如西班牙和
  希腊,房地产部门已经成为由相当熟悉的政府干预形式所构成的经济的一个必不可少
  的组成部分。在其他国家,比如日本,求助于房地产部门来弥补通常的生产--消费循
  环带来的困境并增加利润,这已是稀松平常之事:甚至对房地产部门进行事先预测和
  规划。” 基于作为整体的空间的生产的新资本主义的增长战略为了实现空间的生产而
  进行的空间动员开始于土地,然后,这种动员延伸到地下空间和地上空间,从地下的
  能源、原材料资源、地面的土地资源到地上的被建筑或各种需要分隔出来的空间容量
  都被赋予了交换价值,作为整体的空间成了一个更庞大的 “商品世界”。过去我们买
  卖或租赁的是土地,而今是房屋、楼层、公寓、停车场、游泳池等各种各样的可交换
  可计量的碎片化的空间。 “\textbf{空间成为商品,把空间中的商品特征发展到了极致。}”

  列斐伏尔说: “如果我没记错的话,在 20 世纪 60 年代初,有一个关于空间战略的
  高层决策;不是欧洲的,不是欧洲的空间战略,而是一个法国的空间战略。换句话说,
  它描绘着中心化,巴黎的中心化。巴黎必须变成像鲁尔或英国的巨大都市一样的财富
  和权力的都市核心。这是关于空间政策的政治决定。”因此,\textbf{中心化需要更高的政
    治理性,也就是需要国家或者叫作都市理性以更有效的方式,也就是在全球范围和
    整体上生产空间,通过这种空间秩序的中心性驱逐边缘要素,强有力地集中财富、
    行为手段、知识、信息和文化。同时,因为中心化是一种政治决定,它还需要技术
    和知识的代理人,也就是规划者为中心化提供不证自明的合理性。}

  工业化带来的破坏——史无前例的大规模地重建,即\textbf{在整个社会的规模上进行重
    建。} 这一过程的推进,伴随着许多越来越深刻的矛盾。现存的生产关系被推广、
  扩张了;在同时把农业和都市的存在整合起来的过程中,它们又带来了一些新的矛盾:一
  方面,\textbf{拥有某些未知的权力的决策中心}已经形成,因为这些中心集中了财富、压迫
  性的权力和信息;另一方面,对过去的城邑的破坏,使得各种形式的隔离成为可能,
  各种社会力量无情地将人们\textbf{在空间中分隔开来}。由此,一种广泛意义上的社会关系
  解体了,而与之相伴随的,则是和所有制关系密切相关的那些关系,被集中化
  了。”总体来说,资本主义国家和政治权力主持整合历史城市和农业,把地下、地面
  和地上的空间以及世界范围的空间作为整体进行规划,为寻求日益稀缺的能源、水、
  光等资源而被重组。\textbf{在这个过程中形成的都市空间既是统一的又是分离的,都市空
    间被分割和分隔成彼此分离又相互叠加的异常复杂的空间碎片,国家和政治权力保
    证碎片化的都市空间相互联系,同时,国家和政治权力正是通过历史城市的碎片化
    和中心化建构来保证都市空间的统一性。}

  当国家和政治权力占据它所生产的空间时,日常成了政治建筑屹立其上的土壤。权力
  处心积虑地联合技术和实证知识,小心翼翼地维持着日常生活的连续性,把都市社会
  伪装成具有虚假透明性的 “抽象空间”,结果是,这层神秘的面纱把都市社会的日常
  生活笼罩在恐怖主义之中, “\textbf{对社会成员来讲,到处弥漫着恐怖,暗藏着暴力,压力
    来自四面八方,只能通过超人的努力来避免或转移这种压迫;每个成员都是恐怖分子,
    因为他们都想掌权;因此,无须有一个独裁者;每个成员都自我背叛和自我惩罚;恐怖不
    能被定位,因为它来自四面八方,来自每一件事; ‘系统’ (如果能被称为 ‘系统’
    的话)掌控着每个单独的成员,并使每个成员服从整体,也就是,服从一个战略,一个
    隐藏的结局,这些目标除了掌权者外无人知晓,也无人质疑。}”

  在法国就存在着一个过于庞大的中心,这就是法国的首都巴黎。巴黎作为决策和舆论
  中心统治、剥削着分布在巴黎周围的从属性和被等级化的空间,从而在法国内部建立
  起了一种\textbf{新殖民主义},形成了\textbf{“超发达、超工业化、超都市化的地区” 与欠发达和
    贫困状况日益加剧地区的不平衡发展的矛盾}。同时,他也指出,在现代世界
  里,\textbf{“边缘” 具有多重含义}。首先,边缘在广义上包括资本主义生产方式下被剥夺生
  产工具的\textbf{世界无产阶级}。狭义上来讲包括世界范围内的不发达国家特别是\textbf{前殖民地
  国家}。其次,在资本主义国家内部,边缘指那些\textbf{远离中心的区域}。比如,法国的布
  列塔尼,大不列颠的爱尔兰、威尔士和苏格兰,意大利的西西里岛和南部地区等。再
  次,边缘指\textbf{城市的边缘地区、城郊的居民等}。最后,边缘还指那些\textbf{社会和政治的
    边缘群体},特别是青年和妇女、同性恋者、绝望的人、“精神错乱” 的人、吸毒
  者等。中心和边缘的矛盾不仅仅表现为单方面的中心对边缘的控制和剥削,以及中心
  和边缘发展不平衡的加剧。

  同化和同质化的过程必然伴随着激烈的反抗,中心越倾全力控制和剥削边缘,边缘对
  中心的反抗和违反就越激烈,中心越连续和无限地控制和剥削边缘,边缘对中心的反
  抗和违反就越持久和永恒。另一方面,\textbf{国家资本主义和国家把城市作为财富、决策、
  信息和空间组织的中心,伴随着中心的饱和、资源的匮乏等问题的出现,城市发展逐
  渐显现出衰退迹象,从而使中心化危机显露出来并不断扩展,甚至恶化。}美国的都市
  化进程最迅猛,相应的城市问题和城市危机也最先暴露出来。“\textbf{美国资本主义曾经面
  临极度痛苦的两难境地:是应该牺牲城市(纽约、芝加哥、洛杉矶等)并在别处组建决策
  中心(这是件很难的事) ,还是应该通过投入巨大的资源来保留这些城市,即使是美国
  社会自己所能支配的资源的总和也在所不惜。}”

  总之,国家资本主义的都市化进程生产了中心、边缘及其矛盾,\textbf{中心的衰落和中心与
  边缘矛盾的加剧引发的城市现象和城市危机使城市成为资本主义矛盾表现最激烈的场
  所。}列斐伏尔认为,资本主义抽象的都市空间中质与量的矛盾、交换价值与使用价值
  的矛盾、为非生产性消费和生产性消费进行的空间的生产之间的矛盾、暂时与稳定的
  矛盾、都市理性统治下抽象空间的意识形态化等矛盾是中心和边缘矛盾的征兆,同时
  也是其原因和结果。
\end{quotation}

\section{大卫·哈维}

大卫·哈维《叛逆的城市》\cite{harvey2012rebel}:

\begin{quotation}
  资本家资本主义永远都在生产城市化所要求的剩余产品。反之,资本主义也需要城市
  化(与此相关的还有其他一些现象,例如\textbf{军事开支}等等)来吸收无止境生产出来的
  剩余产品(\textbf{惊险地一跃})。因此,\textbf{资本主义发展与城市化之间呈现出一种内在的
    联系。}

  \textbf{贯穿于整个资本主义历史,城市化从来都是吸收剩余资本和剩余劳动力的关键手段,
    城镇化凭借不断变更空间和场所的使用功能,实现空间垄断及垄断地租,进一步推
    动资本积累。}

  如果劳动力短缺,劳动力工资太高,那么必须重新培训现存的劳动力(以技术手段引
  起失业,或打击有组织的工人阶级力量——如撒切尔和里根在20世纪80年代所采用
  的——是两种基本方法),或者找到新的劳动力。从一般意义上讲,必须要找到新的
  生产方式;从特定意义上来讲,必须要找到新的自然资源。

  1848年的经济危机是一个明显的无法利用剩余资本和剩余劳动力的危机。拿破仑公布
  了一个宏大的海内外基础设施投资项目……最重要的是重建巴黎的城市基础设施。奥
  斯曼使用了类似\textbf{凯恩斯}\footnote{不是经济学发明经济模式,而是经济学总结发现已有的
    经济模式}的体制,利用债务融资,改善城市基础设施,从而解决剩余资本的出路问
  题。

  奥斯曼清楚地认识到,他的使命是\textbf{通过城市化来帮助解决剩余资本和失业问题}。当
  建筑师雅克·伊尼亚斯·希托夫向奥斯曼展示他关于巴黎新林荫大道的设计方案时,
  奥斯曼驳回了这个方案,他说:“不够宽……\textbf{你的设计宽度是40米,我要的
    是120米}。”他吞并了郊区,重新设计了整个街区,如Les Halles。要做到这一点,
  奥斯曼需要\textbf{新的金融机构和债务工具},即建立在圣西门基础上的\textbf{流动信贷和不动
    产信贷}。实际上,他建立了一个 \textbf{原始凯恩斯体系,利用债务融资,改善城市基
    础设施,从而解决剩余资本的出路问题。}

  在随后的15年中,这个类似凯恩斯主义的体制运行良好……\textbf{但是这种过度扩张,以
    及日益具有投机性的金融和信贷制度最终于1868年崩溃。}奥斯曼被迫下台。绝望中
  的拿破仑三世发动了与俾斯麦德国的战争,并以法国的失败而告终。

  二战后,罗伯特·摩西在整个纽约都市区再现了奥斯曼在巴黎的作为。他改变了有关
  城市发展的思维尺度,\textbf{通过(由债务融资建设的)高速公路系统和基础设施改造,
    通过郊区化,通过对城市和整个都市区域的重新建设来吸收剩余产品,进而解决剩
    余资本的吸收问题。}当上述过程在全国范围内的都市区域推行时,对战后全球资本
  主义的稳定发挥了关键作用。

  尽管郊区化在战后发挥了巨大的作用,但郊区发展是以\textbf{掏空城市中心}为代价的。聚居
  在城市中心的少数族裔因此被隔离在新的繁荣之外,并受到极大的负面影响,从而导
  致他们的反抗,进而产生了人们所说的“\textbf{城市危机}”。

  直到2008年,美国人一直认为,\textbf{住宅市场是美国经济重要的稳定器},尤其是
  在20世纪90年代末高技术崩盘之后。通过新的建设,房地产市场直接吸收了大量剩余
  资本,而在历史性的低利率条件下,抵押贷款再融资的浪潮导致了住宅资产价格急剧
  飞涨,从而推动了美国国内消费品市场和服务市场。美国用平均日借款20亿美元来维
  系其不可满足的消费,同时以借贷融资开展阿富汗和伊拉克战争。

  \textbf{全球性城市化的繁荣一直依赖于建立新的金融体制和措施,以便组织起维持城市化
    的信贷需要。}开始于20世纪80年代的金融创新,特别是销售给世界范围投资者
  的\textbf{地方抵押贷款的证券化和打包},建立\textbf{持有债务抵押债券的新型金融机构},都
  发挥了关键的作用……它分散了风险,允许\textbf{剩余储备更容易地介入过剩的住房需求}。
  通过金融机构间的协调,这类金融创新导致\textbf{整体利率下降(给那些创造奇迹的金融
    中介机构带来了巨大的财富)}……但是,分散风险并不等于消除风险。而且,由于
  风险可以转嫁到其他地方,这样广泛的风险分散甚至鼓励了\textbf{更加风险的地方行为}。
  由于\textbf{没有适当的风险评估管理},抵押贷款市场已经失控……财政亏空,都再次呈现
  在2008年的次贷和住宅资产危机之中。

  地方政府债务的急剧上升和对投资公司借贷控制不力(多由政府主导)现在被认为是
  中国经济的主要风险,不仅给中国也给全世界的增长前景蒙上了深深阴影。截
  至 2011 年,中国政府估计市政债务约为2.2万亿美元,相当于“全国国内生产总值的
  近三分之一”,其中 80\% 的债务可能由地方政府融资平台公司持有,这些公司由市
  政府主导,但严格来说不是市政府的一部分。这些组织正在以极快的速度建设新的基
  础设施和标志性建筑,使中国城市如此壮观。但\textbf{市政当局的累积债务是巨大的。一
    波违约浪潮“可能成为中央政府的巨大负债,中央政府背负着约2万亿美元的债
    务:”崩溃之后是长期“日本式停滞”的可能性是非常真实的。}2011年中国经济增
  长机器的放缓已经导致进口减少,而这反过来又将反弹到世界上所有那些在中国原材
  料市场的支持下蓬勃发展的地区。

  吸收剩余已通过“\textbf{建设性摧毁}”引起了反反复复的城市重建。由于穷人、弱势群体
  和在政治权利上被边缘化的那些人总是首当其冲且受到最严重的影响,所以城市重建
  基本上总是具有阶级性的。新的城市是在旧城市的残骸上建立起来,因而需要暴
  力。

  \textbf{几乎在世界上的每一个城市都可以看到因有钱人而产生的建设高潮——而有
    钱人通常具有类似的令人沮丧的性格。与之相伴的是农民在农业工业化和商业化中
    变得一无所有,洪水般地涌向城市,沦落为贫困的城市移民。}

  贯穿整个资本主义历史,国家通过税收拿走一部分剩余资本。在其社会民主阶段,这
  一比例显著上升,国家控制了大部分剩余。最近30年新自由主义一直向\textbf{剩余资本私
    有化}发展……其主要成就是,阻止了国家按照20世纪60年代的方式来增加税收。更
  进一步的举措是\textbf{建立新的治理体系,将国家和企业利益相结合,并且在城市改造中
    应用货币权力,确保国家机构的剩余资本支出利好于企业资本和上层阶级。}

  \textbf{在过去的几十年里,新自由主义让富裕精英的阶级力量得以复原。}
\end{quotation}




%%% Local Variables:
%%% mode: latex
%%% TeX-master: "../main"
%%% End:
