\chapter{新自由主义札记}
\label{chap:neoliber}

\section{新自由主义的概念}


中国语境中常说的“\textbf{新自由主义}”实为Neoliberalism,即“\textbf{新古典自由主义}”,
它不同于Newliberalism\footnote{Newliberalism,也可称为\textbf{社会自由主义}(Social
  liberalism),与John Ruggie在1982年提出的\textbf{镶嵌型自由主义、嵌入式自由
    主义}(Embedded liberalism)相通。其代言人为\textbf{凯恩斯、罗尔斯和德沃
    金等,主张国家干预经济生活,可通过加大政府支出、投资来解决失业和消费不足
    经济危机,重视社会福利。}它允许社会和经济的不平等,但主张\textbf{政治自由
    权的平等优于经济自由权的平等},要求国家给予民众关怀和尊
  重。\cite{newneo}}。维基英文版对Neoliberalism的解释如下:

\begin{quotation}
  新古典自由主义,主要是指19世纪初与\textbf{自由放任的经济自由主义}相关的思想\footnote{古
    典自由主义(Classical liberalism)中的经济自由\textbf{部分}}在20世纪的复苏。这
  些想法包括\textbf{经济自由化}的一系列政策,如私有化,财政紧缩,放松管制,自由贸易
  和减少政府支出,以增加私营部门在经济和社会中的作用。这些\textbf{以市场为基础}的思
  想及其所激发的政策促成了从战后的凯恩斯主义(1945--1980)到新古典自由主义的
  范式转变。

  学者现在倾向于将其与朝圣山学派的经济学家弗里德里希·哈耶克、米尔顿·弗里德
  曼和詹姆斯·M·布坎南,以及玛格丽特·撒切尔,罗纳德·里根和艾伦·格林斯潘等
  政治家和政策制定者联系起来。
\end{quotation}

新自由主义理论内部并非铁板一块,充斥着各种学派,如\textbf{现代货币学派、理性预期学
  派、供给学派}等等,这些学派彼此之间也存矛盾、异议,各国政府的新自由主义实践
依时间、国情等的不同,对这些理论的侧重均有不同\cite{neoxuepai},也有一些早期的热
情拥护者和参与者如今也都\textbf{转向批判立场}。\improve[inline]{好像哈耶克也有转向和反
  思,最好能提供这方面资料。}


\section{哈维对新自由主义实践的批判}

本节内容主要参考大卫·哈维的《新自由主义简史》。

新自由主义理论口号所宣称的实则是一种\textbf{乌托邦}。它宣称市场放任的\textbf{经济自由}远比一
切其他方面的自由(尤其是政治自由)更为重要,认为经济自由是其它自由的唯一基
础。

新自由主义实践内容主要有:
\begin{enumerate}
\item 美国为首的发达国家向其它弱国推行新自由主义,倡导经济自由,\textbf{实则是使新自由
    化的弱国更易被强国资本全球化组织获取高利。}在收割过程中,以强国国力作为背
  书,\textbf{实现强国的超额利润和绝对主导地位,造成弱国的社会悲剧深渊}。还有强国对
  弱国放贷,往往以自由化为前提且要求债权人的至上权力和债务人的无限责任,其实即使是古典经济
  学、凯恩斯主义等也反对债权人的无风险。美国也可以依托美元的“世界货币”地位,
  直接在全球化中抽取高额利润。

  正如斯蒂格利茨的讽刺“这是多么古怪的世界啊,\textbf{反倒是贫穷的国家在补助最富裕
    的国家}\pagescite[][75]{davidneoliber}”。


\item 在放任的市场经济自由内部,本身也是\textbf{大资本(尤其是金融资本)的各方面霸权体
    现,而非经济自由和平等}。如经济精英对弱势者的剥削;垄断或寡头的形成;市民
  税收形成的政府投入成为精英阶层生财资本和工具;快速私有化过程中国有、集体资
  本被精英阶层严重折价收购从而使私人资本在收购结束时就已获取超量巨额利润;金
  融资本远远超越生产资本占据强势地位,一部分工业资本也建立金融化部门。

  即使是凯恩斯也鄙视食利者,提出“\textbf{食利者的安乐死}”,如今却是\textbf{食利者的迷醉
    狂欢}。

\item 新自由主义确实产生了一些其它自由,如择业自由、言论自由等。但这种附带自由很
  是有限,并且
  \begin{quotation}
    如卡尔·波兰尼所说“这些自由在很大程度是‘\textbf{市场经济的副产品,这同一种经
      济也要为那些恶的自由负责}’”。

    就不好的自由方面,波兰尼列出的有“\textbf{剥削他人的自由,或获得超额利润而不对
      社会做出相应贡献的自由,阻止技术发明用于公益事业的自由,或发国难财的自
      由}”。

    自由的理念由此“\textbf{堕落为仅仅是对自由企业的鼓吹}”,这意味着“那些其收入、闲
    暇和安全都高枕无忧的人拥有完全的自由,而人民大众仅拥有微薄的自由,尽管他
    们徒劳地试图利用自己的民主权利来获得某种保护,以免遭那些有钱人的权力的侵
    害”。但是——事情往往如此——如果“\textbf{没有权力和压制的社会是不存在的,强
      力不发挥作用的世界也是不存在的}”,那么维持这种自由主义乌托邦前景的唯一
    办法就是靠强力、暴力和独裁。在波兰尼看来,自由主义或新自由主义的乌托邦论
    调注定会为权威主义甚或十足的法西斯主义所
    挫。 \pagescite[][38-39]{davidneoliber}
  \end{quotation}

\item 金融和跨国集团等“实力自由派”,凭借新自由主义意识形态,并非要求国家无所作为,
  而是试图让国家全力为己服务,尤其是在难见市场效益的公共事务上的巨额投入,与
  法律法规的的倾斜。

  新自由主义赖以实现的并非是“否定国家干预”。恰恰相反,\textbf{它必须大力借助于国
    家的强力,甚至暴力和独裁,还有凯恩斯主义的赤字投入}。国家在此的作用并非是
  为了民众自由或普惠,而是被作为\textbf{上层精英获利手段},国家沦为企业资本和上层阶
  级的“\textbf{守夜人}”。新自由主义意识形态,借此成为最强且普遍的规训手段,为强大
  资本主义、现代性理性代言和背书。

  我国社科院的“新自由主义研究”课题组认为:
  \begin{quotation}
    该课题组将“新自由主义”的主要观点归纳和概括为以下三点:在经济理论方面大力
    宣扬\textbf{“三化”(自由化、私有化、市场化)},在政治理论方面特别强调和坚
    持\textbf{“三否定”(否定公有制、否定社会主义、否定国家干预)},在战略和政策方
    面\textbf{“极力鼓吹以超级大国为主导的全球经济、政治、文化的一体化,即全球资本
      主义化}”。\cite{newneo}
  \end{quotation}
  社科院这个课题组只看其表象、只听其宣称,竟没意识到新自由主义依赖国家干预这
  一点?!笔者对此是有愤怒的,怒其不争,劳民伤财。

\item \textbf{一切成为商品,人的肉体、精神以及权利也均被作为经济商品对待,无助于
    实现物质价值的便被认为是无价值的。}即使对资本精英来说,消费主义的盛行也造成表
  面满足、内心空洞、身份焦虑等。\pagescite[][179]{davidneoliber}

\item \textbf{社会、集体团结的意愿缺失},必然使人试图从他处寻得(很有限的)满足。
  催生\textbf{黑社会、边缘群落、非政府组织、以及宗教团体}等。

\end{enumerate}

\begin{quotation}
  总体而言,新自由主义化\textbf{无法刺激经济增长或提高人民生活}。第二,从上层阶级角
  度出发,新自由主义\textbf{进程而非其理论}确实是巨大的成功:它要么\textbf{重建了统治精英
    的阶级力量}(如美国和某种程度的英国),要么\textbf{为资产阶级形成创造了条
    件}(如中国、印度、俄罗斯等等)。……不管出现什么问题(不平等、低薪、失业
  等),都是因为缺乏竞争力,或因为个人、文化、政治上的缺陷。这样的论述宣称,
  在一个\textbf{社会达尔文主义}的新自由主义世界里,只有适者才应该也能够生
  存。\pagescite[][164]{davidneoliber}
\end{quotation}

\section[新自由主义的实质]{新自由主义的实质——哈曼对哈维的批判}

克里斯·哈曼\cite{chrisharmanneo1} \cite{chrisharmanneo2}和大卫·哈维都认为在新自由
主义国家的实践过程中均出现了与它所宣称的\textbf{背离}。在美国、中国等国的新自由主义
实践中,均采用了新自由主义所明确反对的\textbf{凯恩斯主义的国家干预}为经济发展提供保
障,例如\textbf{政府大规模财政赤字、不良银行债券资助基础设施建设和固定资本投资}等。
其他被灌输并实行较为彻底新自由主义化的落后国家,则被美国等国\textbf{利用资本全球化
  进行掠夺和积累}。

% 哈维认为寻求自由的新自由主义却要求国家干预是其\textbf{悖论},这一
% 悖论所产生的真正原因是“\textbf{新自由主义的主要实质成就不是生产财富和收入,而是对
%   财富和收入进行分配}\pagescite[][165-166]{davidneoliber}”,“\textbf{旨在重建资本
%   积累的条件并恢复经济精英(资产阶级)的权
%   力}\pagescite[][19-20]{davidneoliber}”。
哈曼以更为激进和传统的马克思主义视角对哈维提出了批评。哈曼认为“新自由主
义”这一意识形态提法具有\textbf{模糊性},它\textbf{弱化了国家支出}在新自由主义实践中的地
位,相比注重就业和福利的凯恩斯主义,新自由主义要求的国家投入更高;\textbf{抬高了积
  累过程中“暴力”的地位,忽视了资本在阶级生产方面的强大理性和连续性——资本
  主义自始自终在原始积累(包括斯大林农业集体化政策);过于强调了工业资本和金
  融资本的对立,忽略了工业资本的金融化;弱化了工人阶级的强大力量。}
\begin{quotation}
  “新自由主义”实际上并不是对今日资本运行的精确描述。我们没有面临向\textbf{自由市
    场资本主义}的回归,这种资本主义在一个世纪以前就\textbf{完结}了。我们面临的是这
  样一个体系,它尝试着在全球范围内\textbf{重建}它体系的各个单元来解决它自身的问题,
  这些单元出现于20世纪的进程中,马克思主义者称之为“\textbf{垄断资本主义}”、“\textbf{国
    家垄断资本主义}”或“\textbf{国家资本主义}”。\textbf{国家继续扮演重要角色},\textbf{想方设
    法为垄断资本提供便利或进行管理}……即使\textbf{生产的国际化}使得这样做比战后几
  十年更加困难。

  对马克思来说,原始积累不仅仅是早期资本家通过抢劫积累财富。它主要是\textbf{从农民
    手中掠夺土地,然后迫使他们寻找雇佣工人的工作。}它的特殊性不在于剥削阶级以
  武力增加他们的财富(这在各种阶级社会中都发生过)。至关重要的是,它允许发展
  一种特定的资本主义方式来扩大这种财富,通过创造一个“自由”工人阶级,他们别
  无选择,只能将他们的劳动力出卖给现在控制生产资料的人。

  这种形式的“原始”积累一直持续到今天。埃及的老地主、巴西的农业资本家、中国
  的共产党老板和印度新近建立起来的资本主义农民,都在不断地试图夺取当地农民的
  土地,在他们成功的地方,\textbf{一个新的无产阶级就诞生了。但哈维错误地认为这只是
    最近几十年的特征。}
\end{quotation}



似乎可以从哈曼的论述中得出这样一个结论:

从日不落英国到吸收了他国人力物力财力的美国初期:资本主义自由的胜利;从俾斯麦
时期的普鲁士 \footnote{对工人阶级实行鞭子加甜面包”政策。一方面解散工人组织,查禁进
  步报纸和刊物;另一方面增加医疗保险法、工伤事故保险法、残疾和老年保险法
  等。\pagescite[][75]{zhuzhaiwenti}}到国家资本初步发达的二十世纪初期德
国\footnote{1890年德国社会民主党占据议会27.2\%的席位,1912年提高到34.8\%。},再到一
战后、罗斯福新政、苏联、上世纪60年代末:古典自由资本主义衰退,国家垄断资本主义
步步兴起;从70年代美国至今:资本对国家力量的利用,国家日益成为服从于、服务于
资本逻辑的“守夜人”。虽然资本主义危机避无可避,但它却是总体向上发展的?!

只是,我的朋友,代价呢?未来呢?

以上只是笔者初步猜想,毕竟似乎有些神枪手谬误……欢迎交流和批判。


\section[联合国债务与人权独立专家报告]{联合国债务与人权问题独立专家报告摘抄}

布雷顿森林机构和发达国家常为向其它国家贷款或减债而附加\textbf{自由化、私有化、全球
  化}的条件,联合国多项机构和议题均涉及对此的强烈批判。因相关议题和文档过多,
笔者以较为随机的方式选择了\textbf{外债与人权独立专家的年度报告}作为切入点,以求管中
窥豹。

独立专家人选不同,其倾向、水平也有不同,希望大家能够批判辩证来看。笔者个人认
为,Fantu Cheru的报告有理有据,水平极高,可作重点研究。

联合国人权高级专员办事处官网链接:\url{https://www.ohchr.org/CH/Pages/Home.aspx}。

联合国外债问题独立专家年度报告链接:\url{https://www.ohchr.org/CH/Issues/Development/IEDebt/Pages/AnnualReports.aspx}


\textbf{人权事务高级专员办事处}(联合国人权高专办)是联合国\textbf{主要的人权实体}。联合
国大会赋予了高级专员及其办事处一个独特任务,即:促进和保护所有人的所有权利。

\textbf{联合国人权委员会}根据《联合国宪章》于1946年在联合国经济社会理事会第一次会议
上成立。2006年3月15日,联合国大会以170票支持、4票反对和3票弃权多数通过成
立\textbf{联合国人权理事会},取代联合国人权委员会。人权理事会是由47个成员国组成的政
府间机构,负责在全球范围内加强促进和保护人权的工作。\textbf{美国于2018年退出}人权理
事会。

人权理事会是\textbf{独立于}人权高专办的实体。这种划分源于联合国大会的分别授权。尽管
如此,人权高专办为人权理事会会议提供实质性的支持,并跟进理事会所作出的评议。

\textbf{联合国外债与人权问题独立专家}的职能是就国家外债、国际金融对人权\footnote{尤其是经
  济、社会和文化权利}的影响问题开展分析研究,进行国家访问任务,致力于与政府、
联合国、非政府行为者和其他利益相关者合作。

在人权委员会对充分享有人权的诸多议题中,\textbf{中国}几乎一直投赞成票,\textbf{切实表现出
  负责任的大国姿态}。英法意德韩日等常投反对票。

\begin{enumerate}
\item 1999年,独立专家Fantu Cheru向人权委员会提交报告E/CN.4/1999/50。报告
  认为,\textbf{货币基金组织、世界银行和七国集团(简称G7)的政府官员是第三世界负
    债发展的根源。}它们在债务国家未能及时还款时,一般要求债务国家以加紧实
  施\textbf{结构调整方案(全球化和自由化)}为条件,重订还款期限。它们的结构调整方
  案使债务国家陷入更为严重的经济和社会危机,指责货币基金组织和世界银行
  为“\textbf{新自由派反革命}”,“债务危机被用来作为打开第三世界市场,剥夺政府在
  国家发展中的作用的方便借口”。

\item 2000年,特别报告员Figueredo、Fanto Cheru,和独立专家Reenaldo向人权委员会提
  交报告E/CN.4/2000/51。报告开篇写到:
  \begin{quotation}
    将近20年来,国际金融机构和债务国债券国政府乐于一场自欺欺人的游戏,从远距离操
    纵第三世界的经济,强行让第三世界毫无力量的国家接受不得人心的经济政策,却自认
    为宏观经济调整的苦药最终将使那些国家走上繁荣的道路和摆脱债务。
  \end{quotation}

  布雷顿森林机构,如世界银行和货币基金组织,在\textbf{全球联盟压力下}于1996年秋天批
  准了\textbf{重债穷国计划}(the Heavily Indebted Poor Countries,缩写
  为HIPC)。HIPC要求债务国首先“必须在世界货币基金组织的强化结构调整方案
  (ESAF)下完成六年的结构调整,然后作出减免债务的决定要满足一些额外条
  件”。1999年春,货币基金组织和世界银行在\textbf{国际大庆2000运动的政治压力
    下}对HIPC作检讨。
  \begin{quotation}
    \textbf{简单地说,HIPC/ESAF是国际货币基金组织和世界银行通过后门继续控制穷国和债务国国家发展政策的一种手段。}
  \end{quotation}

\item 2001年,独立专家Fantu Cheru向人权委员会提交报告E/CN.4/2001/56。报告认真分
  析了非洲九国向IMF和世界银行提交的临时减贫战略文件
  (I--PRSP)。Cheru认可IMF和世界银行此举积极的一面,也提出一些批评,
  如I-RPSP是\textbf{根据捐助方设计的模版编制},“谈不上国家所有权的真实可靠
  性”,“仍导致一种把社会和人的发展以及公平方面的关注问题放在次于财政方面考
  虑因素的地位的局面”。希望IMF和世界银行能进一步改进。

  报告指出IMF和世界银行很大程度上服务于主要股东,即\textbf{七国集团(G7)的利
    益},“在这方面,也不可忽视\textbf{美国财政部}的作用”。批评G7不作为。

\item 2003年,独立专家Bernards Mudho向人权委员会提交报告E/CN.4/2003/10。报
  告中强调了一些成功的案例,也指出
  \begin{quotation}
    \textbf{非政府组织}认为,大量证据表明\textbf{结构调整战略是失败的},因为没有解决国际
    金融机构经济政策在世界各地造成的日益严重的贫困和不平等。……\textbf{重债穷国倡
      议、结构调整战略和减贫扶助信贷不仅不会成功,而且将使穷人的生活更加艰难},
    因为其目的是紧缩预算,实行的政策将缩小经济生产能力,减少可行和可持续性就
    业。

    \textbf{借款人和债权人}应该对重债穷国和最不发达国家日前无法持续的外债\textbf{承担共同责任。}
  \end{quotation}

\item 2004年,独立专家Bernards Mudho向人权委员会提交报告E/CN.4/2004/47。报
  告指出,
  \begin{quotation}
    独立专家赞同世界银行业务评价部的回顾的主要结论,即《重债穷国倡议》是一项有益
    但有限的手段,必须在债务国和国际社会需要对于整体的发展筹资方法作出更广泛承诺
    的范围内加以考虑。
  \end{quotation}

\item 2005年,独立专家Bernards Mudho向人权委员会提交报告E/CN.4/2005/42。报告延续
  上年宗旨,对IMF和世界银行总体持肯定态度。

\item 2006年,独立专家Bernards Mudho向人权委员会提交报告E/CN.4/2006/46。报告中提
  请人权委员会延缓“\textbf{八国集团(G8,俄罗斯为新加入国)倡议}”截止日期的进步性。
  \begin{quotation}
    由八国集团(G8,俄罗斯为新加入国)在 2005 年夏天提出的“\textbf{多边债务减免倡
      议}”\textbf{预测}IMF、世界银行国际开发协会(开发协会)和非洲开发基金(非发基
    金)会 100\%减免世界上负债最沉重穷国的债务,以帮助这些国家实现千年发展目
    标……

    但是首先,\textbf{只有成功完成重债穷国倡议的国家}(迄今只有 19 个)\textbf{才符合资
      格}。其次,只有三个多边开发银行参加债务减免倡议,使得特别是拉丁美洲和亚
    洲国家仍然承担沉重的债务。

    独立专家\textbf{遗憾}地指出,世界银行下设的国际开发协会将2003年定为取消合格债务
    的“\textbf{截止日期}”不符合G8的最初提议,并将导致债务减免大量损失。他请开发协
    会重新考虑其决定。
  \end{quotation}

\item 2007年,独立专家Bernards Mudho向人权理事会提交报告A/HRC/4/10。报告中
  \begin{quotation}
    遗憾的是,近二十年期间,贸易自由化在实施上往往\textbf{操之过急},并且\textbf{顺序安排
      不当}。有时更易为\textbf{经济教条}所左右,而不是就对之可能产生的\textbf{经济和社会
      影响}作出有事实根据的分析。
  \end{quotation}


\item 2008年,新任独立专家Cephas Lumina向联合国大会提交报告,A/63/289。报告
  中Cephas Lumina概述了自己的执政方针,着重指出\textbf{人权法的首要地位和人权的中心
    地位}。
  \begin{quotation}
    可以辩称,根据国际法,国家的人权义务凌驾于许多其他种类法律义务之上,因此,
    国家(以及作为国际法主题的国际组织)采取的所有行动都应与国际人权法一致。
  \end{quotation}

\item 2009年,独立专家Cephas Lumina向人权理事会提交报告A/HRC/11/10。报告中关于债
  务减免所附条件的脚注是
  \begin{quotation}
    例如,根据Eurodad最近一项研究,\textbf{国际货币基金每一笔低收入贷款通常附加13项条件;
    大多数条件要求实行私有化和自由论,对借款国家的穷人造成严重后果。}
\end{quotation}

另一个脚注提到世界银行一份报告指出
\begin{quotation}
  过去几十年里包括\textbf{世界银行}在内向贫穷国家提出的大多数政策忠告都强调参加\textbf{全
    球经济}的优势。\textbf{然而全球市场远非公平,其运作管理规则对发展中国家造成比例
    偏高的不利影响。}这些规则是经过复杂的谈判进程所取的结果,然而在这其中发展
  中国家却\textbf{没有什么发言权}”(加重强调)。世界银行,2006年世界发展报告:公平与
  发展(纽约:牛津大学出版社,2006年)。
\end{quotation}

此外
\begin{quotation}
  过高的债务偿还以及对债务减免和新贷款的附加条件通常\textbf{限制公共开支}(甚至有
  损于向教育和保健等基本公共服务提供资金),\textbf{促进经济自由化(包括国营企业私
    有化,解除投资管理并引进公共服务使用费)},以及\textbf{优先考虑债务偿还}而忽视
  满足基本需求,这些不仅使\textbf{贫穷恶化},而且对发展中国家获得\textbf{教育和保
    健}造成尤其严重影响。
  \end{quotation}

\item 2009年,独立专家Cephas Lumina向联合国大会提交报告A/64/289。报告中Lumina援
  引多方国家、组织、个人对“\textbf{非法债务}”的定义,希望依据公平、公正、持久、人
  权原则等,建立有权确立债务非法性的机构,建立负责任融资框架等。
  \begin{quotation}
    \textbf{债务人和债权人共同承担责任的原则是平等的全球金融系统的核心。}如同在《蒙特雷共
    识》中强调的,“债务人和债权人必须共同负责防止出现和解决债务不可持续的情况。”
  \end{quotation}

\item 2010年,独立专家Cephas Lumina向联合国大会提交报告A/65/260。基金组织
  和世界银行对贷款和债务减免机制常常附加私有化以及贸易和金融部门所需的\textbf{贸
    易自由化}政策。报告中认为这些贸易自由化使债务加剧和不可持续,并对人权,尤其是
  经济、社会和文化权利及发展权造成不良影响和灾难性后果。

\item 2010年,独立专家Cephas Lumina向人权理事会提交报告A/HRC/14/21。报告中
  定义了“\textbf{秃鹫基金}”的概念。秃鹫基金侵蚀了贫困穷国从《重债穷国倡议》和
  《多边债务减免倡议》(尽管存在种种缺陷)中获得的收益。呼吁各国立法限制秃鹫基金,
  其中美国、英国、比利时已经或者正在执行相关遏制法律。
  \begin{quotation}
    “秃鹫基金”一词用于描述私人商业实体通过购买、转让或其他交易
    形式,获得违约或不良债务,有时是实际的法庭裁决,以期获得高回报。从主权债务的
    角度来说,秃鹫基金(或如它们通常自称的“问题债务基金”)一般会在二级市场
    上\textbf{以远低于其面值的价格}获得穷国(其中很多是重债穷国)的\textbf{违约主权
      债务},然后企图通过\textbf{诉讼、扣押财产或施加政治压力}寻求获得\textbf{债
      务全额面值连带利息、罚金和法律费用}的偿付。根据非洲开发银行(非行),秃鹫基金
    的平均回收率为其投资的\textbf{3--20倍},相当于300--2000\%的利润率。非行把这种回
    收率描述为“可能是问题债务市场中\textbf{最高}的”。目前,既\textbf{没有法律限
      制}这种基金可通过诉讼获得的利息或利润总额,也没有管理框架要求披露这种基金
    的\textbf{购债成本}。

    秃鹫基金诉讼案件通常都在发达国家的法院上提出。这里可能是秃鹫基金的注册地或贷
    款协定中指定的管辖区。多数诉讼案件都是在\textbf{美利坚合众国、大不列颠及北爱
      尔兰联合王国和法国的法院}提起的,这些地方被视为“\textbf{有利于债权人
      的}”辖区。

    世界银行和基金组织承认商业债权人提起的诉讼“阻碍了向重债穷国提供完全的债务减免”。
  \end{quotation}

\item 2011年,独立专家Cephas Lumina向人权理事会提交报告A/66/271。报告中介绍
  了\textbf{出口信贷机构对国家可持续发展和人权造成的不利影响}。

  \begin{quotation}
    “\textbf{出口信贷}”一词系指一种保险、担保或融资安排,它使出口资本货物和/或服
    务的购买人能推迟一段时间付款(包括通常为两年以下的短期信贷,通常为两至五年
    的中期信贷,以及通常为五年以上的长期信贷)。出口信贷是出口信贷机构提供的主
    要贷款。

    \textbf{出口信贷机构}是公共实体,向母国的私营公司提供政府担保或补贴的贷款、担
    保、信贷和保险,以支助\textbf{出口和对外投资},尤其是对发展中国家和新兴经济体
    的出口和投资。大多数发达国家至少有一个出口信贷机构,通常是其政府的官方或准官
    方机构。

    出口信贷和投资保险机构一般称为出口信贷机构。这些机构作为一个整体,是为外国企
    业参与发展中国家\textbf{大规模工业和基础设施项目}、尤其是\textbf{采掘业部门项
      目}提供公共融资的主要来源。

    出口信贷机构\textbf{以低于私人市场的利率、保险费和手续费提供融资},且这些机构
    对提供支助提出的经济条件很低,只需有限度地遵守(或根本不用遵守)环境、社会和透
    明度标准,使金融交易得以更容易、更快捷地进行,但其风险也更高。然而,对发展中
    国家的借款人而言,出口信贷机构担保的贷款的利率仍\textbf{高于}开发银行或机构等
    其他官方来源提供的许多贷款的利率。

    大多数出口信贷机构则完全没有促进发展的任务。这些机构的\textbf{唯一目标}就是促
    进\textbf{本国的出口或对外投资}。
  \end{quotation}

\item 2012年,独立专家Cephas Lumina向联合国大会提交报告A/67/304。报告中介
  绍了国际金融机构在向借款国提供贷款、赠款和债务减免时所附加的经济改革条件,对借
  款国妇女权利造成的影响。如\textbf{削减政府开支,进行公共部门改革、公共服务私有
    化和贸易自由化}。\improve[inline]{以后做农业工业化时,可以参考这份报告。}
  \begin{quotation}
    IMF和世界银行估计,重债穷国的偿债金额占国民收入的百分比已从 2000 年的 4\%以上
    跌至2009 年的 1\%,而减贫支出占国民收入的百分比已从 2000 年的 7\%增至 2009 年
    的9\%。
  \end{quotation}

  此外报告中肯定了债务减免对穷国的一些积极作用,“然而,必须强调,债务减免通常并
  不降低重债穷国的脆弱性,因为许多国家仍然严重依赖外国贷款和投资。”

\item 2013年,独立专家Cephas Lumina向人权理事会提交报告A/HRC/23/37。

  \begin{quotation}
    虽然现行国际债务减免举措减轻了重债穷国的债务负担(从账面来看),但这些举措未能
    处理致使低收入国家的\textbf{债务无法持续}的根本原因,这些原因包括不公正的全球
    贸易条件,生产和出口基础狭窄,容易遭受外来冲击(包括国际资金量的减少),以及不
    负责任的放款等。实际上,这些举措侧重把债务降至债权人认为“\textbf{可持续}”的
    水平,这样做隐含的意思是:问题在于接受债务减免的国家在债务管理上欠谨慎而且治
    理不善。\textbf{债权人在举措中发挥主导作用,这是与共同责任原则相抵触的。}同时,
    与举措相关的附加条件损害了债务国的主权,在某些情形中妨碍了债务减免的减贫目标
    的实现。
  \end{quotation}

\item 2013年,独立专家Cephas Lumina向联合国大会提交报告 A/68/542。报告主要根据
  《联合国千年发展目标差距工作队2012年报告》,总结\textbf{千年发展目标8与实际进展的
    差距},希望建立更为强大的发展框架。报告中提出千年目标8实际进展过程中的不足和缺陷有:
  \begin{quotation}
    \textbf{发达国家持续的农业补贴也继续对发展中国家的农业贸易和生产产生不利影
      响。}2011年,经合组织国家的农业补贴增加到国内生产总值的 0.95\%。虽
    然\textbf{发达国家农业补贴一天为 10 亿美元},许多贫穷发展中国家无力补贴其农业,
    导致其产品价格较高,农民的贫困加剧和生活水平下降。发达国家还\textbf{对进口的制
      成品和加工产品征收高额税,使得发展中国家无法赚取更多收入,并使他们只限
      于原材料出口。}贸易谈判多哈回合停滞不前使得问题进一步复杂化。

    \textbf{二十国承诺抵制所有保护主义措施},并纠正任何在应对全球金融危机中所采取
    的保护主义措施。但2008世界贸易自危机开始以来\textbf{只废除了一小部分已经实施
      的贸易限制}。迄今实施的贸易限制已影响到将近 3\%的世界贸易。遭到\textbf{保护
      主义措施}的影响。

    如独立专家在提交人权理事会的报告(A/HRC/23/37)中所指出,\textbf{债务减免的直接
      财政影响难以衡量,债务减免与减贫支出增加之间的因果关系也难以确定}。

    债务减免机制已经\textbf{完全为债权人所主导},过度侧重于纠正被视为是受援国方面
    不慎重的债务管理,没有债务解决问题的根本原因,包括不公平的贸易条件、不负责任
    的贷款和国际金融机构不当的政策规定。
   \end{quotation}

 \item 2013年,独立专家Cephas Lumina向人权理事会提交报告A/HRC/25/52,说明不把非
   法资金归还来源国对享受人权的负面影响。
  \begin{quotation}
    “\textbf{非法资金}”一词泛指腐败、贿赂、贪污、逃税和其他犯罪行为的收益。
  \end{quotation}

\item 2014年,独立专家Cephas Lumina向人权理事会提交报告A/HRC/25/50。报告
  中Lumina总结了自己2008--2014任期内的工作。
  \begin{quotation}
    人权理事会的一些成员\textbf{没有为外债和人权这一任务提供支助,尤其是欧洲联盟和
      美利坚合众国。}
  \end{quotation}


\item 2015年,独立专家Juan Pablo Bohoslavsky向人权理事会提交报告A/HRC/28/60。报
  告主要涉及\textbf{非法资金流动}问题。
  \begin{quotation}
    根据全球金融诚信组织的最新\textbf{估计},发展中国家在2012年因非法资金外流损失
    了\textbf{9912亿美元},比2011年再增加1.8\%。2003年以来,非法资金外流实际每年增
    加9.4\%。通过将这些数字与发展中国家收到的官方发展援助比较即可表明这种资源
    流失的规模。2012年的官方发展援助为897亿美元,这意味着,2012年每支出\textbf{1美
      元的发展援助},就有超过\textbf{10美元}以\textbf{非法资金外流}的形式逃离发展中国家。
    根据全球金融诚信组织的资料,\textbf{过去十年官方发展援助和外国直接投资加在一起
      也抵不上发展中国家的非法资金外流。}

    根据保护记者委员会的资料,截至2014年12月31日,在1992年以来全球被谋杀
    的725记者中,有208人或29\%报道了\textbf{腐败问题}。记者无国界组织2011年报告,
    在2000--2010的十年中至少有141名报道\textbf{有组织犯罪和贩毒}——非法资金流动的
    另一个主要来源——的记者被杀害。
  \end{quotation}

\item 独立专家Juan Pablo Bohoslavsky向人权理事会提交报告A/HRC/28/59。报
  告题目为 \textbf{金融共谋向严重侵犯人权的国家提供贷款},提议对这些国家的贷款必须
  经过评估和验证。
  \begin{quotation}
    向严重侵犯人权的政权提供贷款可能有助于政权巩固、使不尊重人权行为得以延续以及
    增加严重侵犯人权的可能性。这些结论对官方和私人对政府的金融援助都适用。然而,
    私人贷款似乎更具有破坏性,因为与国家间的贷款和由国际金融机构分配的贷款相比,
    私人贷款的公共问责程度较低。
  \end{quotation}

  \improve[inline]{希望有人可以继续摘抄下独立专家历年报告。}
\end{enumerate}

% https://wallstreetcn.com/articles/230968 % https://en.wikipedia.org/wiki/Vulture_fund % https://en.wikipedia.org/wiki/Vulture_capitalist % http://www.xinhuanet.com/fortune/2016-03/01/c_1118200280.htm % https://botanwang.com/articles/201407/%E7%9C%8B%E7%BE%8E%E5%9B%BD%E7%A7%83%E9%B9%AB%E5%9F%BA%E9%87%91%E5%90%91%E9%98%BF%E6%A0%B9%E5%BB%B7%E9%80%BC%E5%80%BA%E7%9A%84%E6%89%8B%E6%B3%95.html % https://wiki.mbalib.com/wiki/%E7%A7%83%E9%B9%AB%E5%9F%BA%E9%87%91 % http://www.argchina.com/wx-index-content-id-3325.html

% http://cshan.mrecic.gov.ar/zh-hant/content/%E8%81%94%E5%90%88%E5%9B%BD%E5%A4%A7%E4%BC%9A%E6%89%B9%E5%87%86%E9%99%90%E5%88%B6%E7%A7%83%E9%B9%AB%E5%9F%BA%E9%87%91%E8%BF%90%E4%BD%9C%E5%87%86%E5%88%99

% https://news.un.org/zh/story/2010/04/129822

% http://genevese.mofcom.gov.cn/article/sqfb/201603/20160301270885.shtml

% https://documents-dds-ny.un.org/doc/UNDOC/GEN/G10/131/55/PDF/G1013155.pdf?OpenElement

% https://www.ohchr.org/CH/Issues/Development/IEDebt/Pages/Debtrestructuringvulturefundsandhumanrights.aspx % https://wallstreetcn.com/articles/230968 % https://en.wikipedia.org/wiki/Vulture_fund % https://en.wikipedia.org/wiki/Vulture_capitalist % http://www.xinhuanet.com/fortune/2016-03/01/c_1118200280.htm % https://botanwang.com/articles/201407/%E7%9C%8B%E7%BE%8E%E5%9B%BD%E7%A7%83%E9%B9%AB%E5%9F%BA%E9%87%91%E5%90%91%E9%98%BF%E6%A0%B9%E5%BB%B7%E9%80%BC%E5%80%BA%E7%9A%84%E6%89%8B%E6%B3%95.html % https://wiki.mbalib.com/wiki/%E7%A7%83%E9%B9%AB%E5%9F%BA%E9%87%91 % http://www.argchina.com/wx-index-content-id-3325.html

% http://cshan.mrecic.gov.ar/zh-hant/content/%E8%81%94%E5%90%88%E5%9B%BD%E5%A4%A7%E4%BC%9A%E6%89%B9%E5%87%86%E9%99%90%E5%88%B6%E7%A7%83%E9%B9%AB%E5%9F%BA%E9%87%91%E8%BF%90%E4%BD%9C%E5%87%86%E5%88%99

% https://news.un.org/zh/story/2010/04/129822

% http://genevese.mofcom.gov.cn/article/sqfb/201603/20160301270885.shtml

% https://documents-dds-ny.un.org/doc/UNDOC/GEN/G10/131/55/PDF/G1013155.pdf?OpenElement

% https://www.ohchr.org/CH/Issues/Development/IEDebt/Pages/Debtrestructuringvulturefundsandhumanrights.aspx




%%% Local Variables:
%%% mode: latex
%%% TeX-master: "../main"
%%% End:
