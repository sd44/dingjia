\chapter{市场经济体制时期 1992--}

之前一章已经说明,1992年邓小平南巡,江泽民在中央党校及十四大讲话中明确了建
立“社会主义市场经济体制”的改革目标,“八五”计划方针也响应十四大精神,调整
为“双加快”——加快改革开放、加快经济发展,中国正式进入市场经济体制时期。

本章涉及这样庞大的课题,基本不可能由笔者自己完成。写作之初时其实是希望本书能
够社会化编辑,有其他人的参与和贡献,所以铺开了太大的框架。但直至今日,基本上
也只有笔者自己独木苦支。只可论述二三事件,还请读者见谅。\todo[inline]{希望有
  人接手完善1992年之后的历史事件。}

\section{农业改革}


\subsection{粮食并轨}

1992年4月1日,继上年5月1日后,国务院再次决定\textbf{提高粮食统销价格},实
现\textbf{购销同价}。粮食统销价格提高后,粮食部门的经营费用仍由财政补贴;对城镇居
民口粮继续实行凭证、凭票、定量供应政策;对农村平价粮销售也继续实行计划供应。在提
高粮食统销价格的同时,国务院决定给城镇居民适当补贴。

1993年2月15日,国务院发布《国务院关于加快粮食流通体制改革的通知》,通知指出“粮食
流通体制改革要把握有利时机,在国家宏观调控下\textbf{放开价格,放开经营,增强粮食
  企业活力,减轻国家财政负担,进一步向粮食商品化、经营市场化方向推
  进。}” “1993年4月1日起,我国\textbf{取消了粮票和油票},实行粮油商品敞开供应。
从此,伴随城镇居民\textbf{38年历程}的粮票、油票等各种票证完成了谢幕演
出,\textbf{票证时代彻底终结}。”多数省份在此之后\textbf{取消了“粮棉三挂钩”}。
至此,\textbf{中国取消了长达40年的统销制度}。

“到1993年5月底,全国宣布放开粮价的县(市)超过总数的95\%以
上。”1993年11月5日,《中共中央、国务院关于当前农业和农村经济发展的若干政策措
施》中,决定“从明年起,国家定购的粮食全部实行‘\textbf{保量放价}’,即保留定购数量,
收购价格随行就市。……粮食价格和购销放开以后,国家对粮食实行\textbf{保护价制度},并
相应建立\textbf{粮食风险基金和储备体系}。”

据卢迈,1993年的政策,如1985年一样,目的也是\textbf{减轻国家财政负担,在供大于求
  的情况下将市场风险转嫁给农民。}\cite{lumaisg}

1993年10月底,南方沿海\textbf{粮价迅速上涨}并快速辐射蔓延全国,民间再次出现粮
食\textbf{抢购}。
\begin{quotation}
  1993年10--11月两个月粮食平均价格由0.935元/公斤上升到1.080元/公斤,涨幅约16\%。
  一些城市粮价出现一日一变甚至一日几价情况,为改革以来所仅见。1994年初粮价相对平
  静,但三月份以后重新上涨,6月份比2月上升了25.7\%.下半年上涨更猛,12月粮价比6月
  上涨31.8\%。1995年粮价仍在攀升,1995年粮价达到有史以来最高的2.155元/公斤,全年
  上升19.3\%。\cite{lufengsanci}
\end{quotation}

据卢锋,1993年底,关于粮价上升过快的原因,决策层和学界起初认为“\textbf{主要是由心
  理,投机因素等暂时因素推动}”,主要发生在\textbf{流通领域}。后来改变想法,认为主
因是\textbf{生产不足}。
\begin{quotation}
  1993年11月在京召开的“农村经济形势分析与展望”会上,专家们普遍认为“这次粮价猛
  涨,是\textbf{国营粮食企业带头抢购和哄抬粮价,而引起集市跟随涨价。}南方,特别是
  沿海地区粮价先期上涨的信息,通过各种途径迅速传播,助长了\textbf{粮食生产者和经
    营者普遍的涨价心理预期和惜售心理}。”

  (1994年春,)决策层已开始转而认为\textbf{生产不足}是粮食上涨的主要原因,并试图
  通过加强\textbf{在生产和流通领域的行政控制}来应对粮价上涨。
\end{quotation}

\textbf{以下绝大部分专家的论述只根据卢锋文章中的介绍作出,笔者没有查证原文,也不想
  查证……}

卢锋论文中整理了1994与1995两年中几位专家学者对粮价上涨原因的论断:陈锡文、杨
启先(1994)认为粮食总产量虽然上升,但\textbf{南方稻谷减产}导致粮价上涨;杨启先
(1995)也继承了一个\textbf{过往的传统观点},认为\textbf{粮价上涨导致通货膨胀};林毅夫、
李周(1995)认为地方产量发展不均衡有所扩大之下的\textbf{封闭粮食市场},与\textbf{不可阻挡
  的信息传播}之间的矛盾,从而产生农产品惜售行为,最终粮价上涨;戴根友(1995)
将主因归于经济加速发展下,\textbf{农村人口向城市的大规模流动},导致农业生产和供给出
现短缺……“不应把由改革所造成的通货膨胀当成宏观调控的主要目标”。

笔者认为,王建(1994)的论述需要单独提一下。王建将通货膨胀分为两种,\textbf{一种是
  货币超发和投资规模过大引起的通货膨胀,一种是我国转型期政策改革、产品结构调
  整所引起的通货膨胀}。王健认为侧重于农产品的改革型政策是通货膨胀主因,而粮价
提升在市场转轨过程中其实是种市场化体现,\textbf{只需有限制地调整}。笔者认为,王建漠
视货币超发和投资规模过大是粮食通膨主因之一,这点是明显错误的;但他对农产品结
构性调整的论述具有一定价值,可以引申为\textbf{农业在改革的结构型调整中倾向于真实市
  场价值,于是原本被抑制的粮价在面对无形之手时涨幅较高。}

另外笔者找到了温铁军(2001)一篇文章\cite{6cibodong}。温铁军认为粮价上涨的真实深
层原因有两条。温铁军所述第一条深层原因和戴根友相同,“1992年邓小平南巡讲话之后,
自1989年以来长期收入低下的农村劳动力在经济增长、基建投资增加的吸引下进城打
工,\textbf{打工人数约6000--8000万。他们需要增加的粮食消费大约为600亿斤/年,而国
  家粮食系统是没有这个储备的}”。第二条原因是,人民币汇率贬值导致“外贸和南方各省
突然从进口粮食转向在国内市场抢购以逐利……1993年秋季粮价已经开始上升,大量的粮食
经营单位,特别是南方的粮商,已经有\textbf{囤积居奇的投机行为}。接着1994年汇率调整
一步到位,\textbf{人民币贬值实际达到57\%},这就意味着刺激出口。本来1993年国内粮食
价格已经高于国际市场约20\%,但在人民币一次性贬值57\%的情况下,出口粮食就有可能得
到\textbf{约30\%的机会利润}。率先得到汇率调整信息的\textbf{南方粮商就从南到北抢购
  过来}。”

笔者认为温铁军也未能如他所说,找到“这次波动的真实原因”。现代社会大规模严重
饥荒的主因往往只是人祸,而严重粮食问题的主因往往也是政治或资本统治权力主导的
宏观调控。世界各国在粮食大问题上,常将囤积居奇的货商或自然条件等列为主要、本
质原因,只不过是避重就轻而已。

笔者支持卢峰观点\footnote{需要注意前文所述学者除温铁军外的论断均是在1994、1995年作出,
  卢锋的论断或许有一系列滞后性优势}。卢锋(1999)否定了“稻谷减产导致粮价上
涨”和“粮价上涨导致通货膨胀”的说法。他认为粮价上涨的主要原因是:粮食价格周
期性上升(\textbf{90年代初过往粮价被显著抑制后的反弹});\textbf{国际市场米价急剧上涨和人
  民币汇率贬值},大米进口需求降低,对国产大米需求增加;\textbf{投资需求(特别是房地
  产投资)激增和信贷膨胀(特别是通过民间非法集资和银行违规贷款实现的融资)导
  致经济过热和通货膨胀势头,从而农民倾向“惜售”和粮企调高合理库存量行为};政
府认为粮价上涨主因是生产不足后所采取的\textbf{宏观调控政策}反过来使市场参与者\textbf{预期
  粮价会进一步上涨},实际上\textbf{走向预期的反面,推动了粮价上涨}。

笔者妄论,除卢峰等人外,这都是些什么样的专家啊。简单来说,无非是谷贱伤农则农民
不愿种粮、卖粮,生产力不足。而超发货币、大搞基建的凯恩斯主义导致通膨,通膨导致被严重贬抑、
无法反映其真实价值的的粮价加剧反弹。这必然是\textbf{最主要}的原因,没有之一。

1994年5月9日,国务院下发《国务院关于深化粮食购销体制改革的通知》。
\begin{quotation}
  \textbf{粮食部门必须收购社会商品粮的70--80\%,即900亿公斤左右(贸易粮
    )。}……建立健全灵活的\textbf{粮食吞吐调节机制},适时平抑粮价,稳定粮食市场,
  促进生产,保证供应,是粮食部门的重要任务。……在粮食行政管理部门的统一领导下,
  粮食经营实行\textbf{政策性业务和商业性经营两条线运行}机制,业务、机构、人员彻底
  分开。
\end{quotation}

同日,国务院下发《国务院关于印发<粮食风险基金实施意见>的通知》\footnote{1994年11月中
  国农业发展银行挂牌成立,它是直属国务院领导的中国唯一的一家农业政策性银行。}。
这一次粮食双轨并轨、全面市场化、“保量放价”的尝试不到一年就\textbf{失败}了,我国重
新回到\textbf{粮食双轨制}。


\subsection{重回粮食双轨制}

以下内容引自王德文、黄济焜《中国粮食流通体制改革》一文。
\begin{quotation}
  1994年5月,《国务院关于深化粮食购销改革的通知》进一步明确规定:实行省、自治区、
  直辖市政府\textbf{领导负责制},负责本地区粮食总量平衡,稳定粮食面积、产量与库存,
  灵活运用地方粮食储备进行调节,保证粮食供应和价格稳定。这为1995年正式出
  台“\textbf{粮食省长负责制}”打下了基础。

  同时,为了缩小市场价格和定购价隔间的较大差距,刺激粮食生产,1994年和1996年,中
  央两次\textbf{提高了定购价格},两次粮食提价幅度均在\textbf{40\%以上}。

  政策干预和提价等因素刺激了粮食生产增长,1993年--1997年……粮食增产后,农民面临
  着“\textbf{卖粮难}”问题,国家一方面通过\textbf{保护价政策}来收购农民手中的余
  粮,另一方面通过建立\textbf{粮食风险基金}来填补政策运作中的各项成本。但是,由
  于1994年以来连续几年的粮食丰收、不适当的粮食进出口政策(常常是\textbf{丰年减少出
    口、增加进口}),以及为遏制通货膨胀而采取的\textbf{通货紧缩政策}措施制约了国内
  需求扩张,\textbf{粮食市场价格随之下跌}。而按市场价销售收购的粮食将面
  临\textbf{巨大亏损},粮食部门不仅没有将旧帐减少,反而又添巨额新帐,摆脱粮食收购
  中的\textbf{财政补贴压力}成为燃眉之急。
\end{quotation}

为什么在粮价涨幅较大的情况下,依然在1996年大幅提高了粮食收购价格呢?综合几份
资料,笔者认为可以简单总结为:\textbf{对市场信息反应迟滞,错判粮食生产不足。}

卢锋提到两件事:一,“国家计委‘九五’规划预测2000年才达到9800--10000亿斤”;
二,国家怀疑统计部门上报的1996年粮食总产量超过一万亿斤的数据可靠性,认为水分
较大。王小鲁关于粮食波动的论述能较为清晰直白地佐证这点。
\begin{quotation}
  由于(1994年的大幅)提价大致是从6月份开始,已经\textbf{错过了播种季节,因此对当年
    的粮食生产并未发生积极影响。}1994年产量反而\textbf{下降了2.5\%}。在产出未作出明
  显反应的情况下,1995、1996和1997年实际定购价格又连续提高了4.4\%、8.1\%,从
  而使总产量扩过了5亿吨。由于\textbf{供给过剩},市场价格和议购价格在1997--1998年大
  幅下跌,而定购价则在1997年继续上涨,1998年也只有小幅下跌,这导
  致\textbf{在1997--1998年定购价超过了市场价}\footnote{按照王小鲁收集整理的价格数据(\cref{fig:dingyishi}),大豆、
    小麦、玉米三大主粮中其实只有小麦的定购价在1997--1998年超过市场
    价。}。\cite{wangxiaoluliangshi}
\end{quotation}

\improve{大米小麦玉米保护价的数据到底是多少?资料太难寻。}

\subsection{混沌双轨制}

1996年10月份,国家院在大连召开部分地区粮食工作座谈会,会上提出“\textbf{四分开,
  一并轨}”的改革思路——政企分开、中央和地方责权分开、储备与经营分开、新老财务帐目
分开,粮食定购价格与市场价格并轨。

1997和1998年,国家又持续调低了粮食定购价和议购价,并且保护价下调至定购价之下。有
资料显示,1997--1998两年,国家对粮农其实是负保护水平。

1998年5月10日,国务院下发《国务院关于进一步深化粮食流通体制改革的决定》,提出“四
分开,一完善”,将原来的“\textbf{粮食定购价格与市场价格并轨}”替换为“\textbf{完
  善粮食价格机制}”。

本次改革的原因和目标是:
\begin{quotation}
  现行粮食流通体制仍然没有摆脱“\textbf{大锅饭}”的模式,国有粮食企业管理落
  后,\textbf{政企不分,人员膨胀,成本上升};同时又\textbf{严重挤占挪用粮食收购资
    金},导致\textbf{经营亏损和财务挂帐剧增},超出国家财政的承受能力。这些都说明,
  现行粮食流通体制已越来越不适应社会主义市场经济的要求,到了非改不可、不改不行、
  刻不容缓的时候了。不改革,中央和地方的责权关系不清,\textbf{中央财政不堪重负};
  不改革,\textbf{国有粮食企业就难以扭转亏损},不能担当粮食流通主渠道的重任;不改
  革,\textbf{不利于保护农民的生产积极性},必将影响粮食生产的持续稳定增长。
\end{quotation}

具体政策和解释如下:
\begin{enumerate}
\item 政企分开。“实行政府粮食行政管理职能与粮食企业经营的分离”。国营粮企具体政策方
  面也包括“国有粮食企业要实施\textbf{下岗分流、减员增效和再就业工程}。直接从事粮食
  收储业务的人员要逐步减少到\textbf{现有人员的一半左右}。”

\item \textbf{中央和地方的粮食责权分开},全面落实粮食省长负责制。

\item \textbf{储备与经营分开}。分权责建立\textbf{中央和省份的两级储备体
    系},\textbf{储备粮与企业经营周转粮实行分开管理}。

\item \textbf{新老财务账目分开}。“新老财务帐目分开的核心是要正确核定应由财政补贴的挂账额
  度,同时要认真制定和落实消化新老挂账的措施,并要求在新体制下不得再出现新的挂
  账。”\cite{caobaoming01}

\item \textbf{完善粮食价格体制}。收购保护价和市场价的动态关系;销售限价;进出口和储备
  粮的宏观调控。

\item “要充分发挥\textbf{国有粮食企业收购粮食的主渠道作用},农村粮食收购主要由国有粮
  食企业承担,\textbf{严禁私商和其他企业直接到农村收购粮食}。”
\end{enumerate}

“四分开、一完善”政策鲜明地继承了94分税制思路,\textbf{政企分离、减轻中央财政负
  担、中央和省级责权分离}等。

1998年6月3日,国务院召开全国粮食购销工作电视电话会议,朱镕基出席并发表重要讲话,
正式提出“\textbf{三项政策、一项改革}”。

下文接着引用王德文、黄济焜《中国粮食流通体制改革》一文:
\begin{quotation}
  到1998年,国务院又出台了在学术界和地方很有争议的“\textbf{三项政策、一项改革}”方案,即
  按\textbf{保护价敞开收购余粮、实行顺价销售、收购资金封闭}运行三项政策,\textbf{加快粮食
    企业自身改革},转化经营机制,提高市场竞争力。2年来的实践证明,\textbf{预期的改革目标不仅
    难以实现,而且更加大了走出双轨制度的难度。}\cite{shuangguizhi}
\end{quotation}

专家学者普遍认为“四分开、一完善”与“三项政策、一项改革”政策的主要目的
是\textbf{使粮企止损盈利并迎接市场化、消化历史亏损挂账。}笔者根据各方文献,结合自
己思考对此一时期政策的理解如下:

\begin{enumerate}
\item 关于“保护价敞开收购余粮”的内在逻辑,朱镕基曾在1998年粮食工作会议上如是说:
  \begin{quotation}
    从我国的基本国情出发,为了\textbf{保障农民收入的稳定增长},对农民出售的余粮,只
    能由国有粮食收储企业按保护价敞开收购,\textbf{不能侈谈什么‘放开’}。在这方面,
    过去的教训是深刻的。1992年秋收后,一些地方盲目放开粮食收购和价格,结果大量私商
    进农村抢购粮食,粮食市场混乱,\textbf{粮价猛涨}。国家不得不采取抛售专储粮等措施,
    才把粮价稳住,为此付出了巨大的代价。\cite{zhuchangkai}
  \end{quotation}

  笔者认为朱的谈话除防止粮价过低外其实还有三个含义,\textbf{一、虽然目前粮食供给过
    剩,仓储也过剩,但如果以市场来主导,则粮食产量可能会降低。我国目前仍要保
    证产量。二、如果粮食产量降低,供给不足,从而粮价暴涨,国家就要花费巨大代
    价“稳住粮价”。三、禁止私商粮贩收购,形成国家垄断性市场,便于宏观调控和
    实现国营粮企的顺价销售\footnote{顺价销售:指国有粮站、粮库等粮食购销企业出售的原
      粮及其加工的成品粮,必须以粮食收购价格为基础,加上合理费用和最低利润形
      成的价格进行销售,不允许以任何形式向任何粮食加工、批发和零售企业亏本销
      售。}。}


  笔者还有一个相当主观,缺乏科学论证,很可能错误的揣测,希望读者能够批评指正。根据
  卢锋所言
  \begin{quotation}
    我国粮食生产继1995年丰收和1996年特大丰收之后,1997和1998两年仍是较大丰收
    年,这是一个很异常的现象。因为过去专家和官员谈粮食,通常有“\textbf{两年一
      歉”或“两丰两歉一平”}一类的说法,\textbf{不可能有三年连续丰收},认为这是我
    国粮食产量\textbf{受气候制约}的周期波动规律。然而\textbf{1995年以来四年连续丰收}% \footnote{卢
      % 锋此处所说连续丰收,应该是指总产量,而非增长率。(可
      % 见\cref{fig:liangchanliang}、\cref{fig:liangzengzhang})}
    ,以往的经验和规律似乎突然失去了灵验。反常现象必有反常原因,这应当主要
    是\textbf{价格保护政策}的功劳……
  \end{quotation}
  那么,是不是可以换一个角度来想。国家1998年制定“保护价敞开收购”政策时其实
  是预期我国粮食减产?如果粮食如预期一样减产,那么这一政策对国家财政收入
  将\textbf{不是利损,而是利好}。这可能是错判96年粮价之后2年的又一次错判。

  % 另外,我国小农生产状况下的粮价,较世界粮食市场价格偏高。但也因此\textbf{我国不能
  %   大规模进口国外粮食,必须保障最基本的农民卖粮价}。如果大量进口低价粮食,即使不考
  % 虑\textbf{国际战略层面的粮食安全问题},我国也\textbf{无法吸收被国内供应过剩、自身生产成本
  %   过高所抛离出来的大量农村剩余劳动力},从而造成严重社会问题。进口粮食同整
  % 体粮食政策一样,都是为了\textbf{保底价、限高价}。

  王德文、黄济焜评价此政策造成的弊端是:
  \begin{quotation}
    保护价敞开收购导致的\textbf{国家仓储设施、信贷资金和粮食风险基金负担沉重}。(陆
    文强等调查资料显示)\textbf{1998年国家粮食储备率高达60\%,属超安全储备},成
    为\textbf{严重的经济负担,国家仅支付保管和利息就高达500亿元,财政已不堪重负。}”
  \end{quotation}

  \textbf{虽然保护价落实折损后的价格并非那么如意,“敞开收购”也未能真正贯彻执行,
    但提高了粮农一定的安全感,粮农的生产积极性没有立即减弱,}随着保护价的持续降低和
  取消保护政策,我国抗压农民生产积极性才真正降低。因农村和国家有大量库
  存,\textbf{我国粮食过剩问题存在到2000年。}

\item 顺价销售的内在逻辑是通过严禁私商和其他企业(非国有粮企)直接收购粮食形成的
  垄断性地位使\textbf{国家掌控市场},节约成本和增加利润,最终使\textbf{粮企止损盈利、消化
    历史亏损挂账},

  但是此时的垄断性国营粮企也增加了各项成本。(1)\textbf{高昂垄断行政成本}:如王和黄
  所说政府主管部门、公安、工商、税务等有关部门、建立加工企业台账和粮食稽查队伍等
  的财政支出。(2)\textbf{高昂经营性成本}(收购资金利息、仓储成本、人力成本、损
  耗、政策失误等)。(3)国营粮企将一些\textbf{经营性亏损转嫁}至下游粮农、上游粮
  食需求单位如面粉厂等、提供财政补贴的国家(\textbf{经营性亏损转嫁为政策性亏损}),
  伴生吃拿卡要的\textbf{权力寻租}。

  笔者认为,这一切导致的结果只能是\textbf{农民粮食经粮企收购折算后的实际收购价}仍然
  较低。此时我国仍有2亿多农村生产单位,加之以上原因,\textbf{无法有效管控逐利私商和
    非国企粮贩},从而“\textbf{粮食部门与私商粮贩在价格竞争上并没有太大的区别}”。
  最终\textbf{“市场仍是竞争性市场状态”},\textbf{国家耗巨资赔钱运行“三项政策、一项改
    革”。}

  曹宝明的论述比较实在一些:
  \begin{quotation}
    事实上广大农民已经习惯了将自己的粮食出售给服务\textbf{远远好于}国有粮食部门的
    私人粮商,在1998--1999年的粮食收购中工商行政管理部门及国有粮食部门已经自
    认\textbf{无能为力}了。\cite{caobaoming01}
  \end{quotation}

  农民并未获得实惠,王和黄的文章可以提供论据:实际执行过程中\textbf{农民收入不升反
    降}。“1999年农民人均纯收入为2430.0元,其中来自种植业收入为342.3元,比上
  年下降10.5\%(农村固定观察点办公室,2000)”……

  竞争性市场状态下,农民实际卖粮价必然低于市场价,可通过\cref{fig:liangjia}获
  知此时期粮农真实收入上限。

\item “收购资金封闭”政策的内在逻辑是为\textbf{杜绝收购资金的挪用和超期占用,以
    控制财政贴息}。

  笔者没有找到“资金封闭”政策的宏观深刻论述,仅列举几个例子吧。地方粮企指责
  地方财政\textbf{挤占挪用粮企补贴},指责农发行只管收购所需\textbf{封闭资金},限制了粮企
  生产运营的\textbf{动态要求},如企业缺乏所需信贷、企业间资金流转受到限制等。农发行
  指责地方粮企\textbf{挤占挪用收购资金},粮企\textbf{亏损挂账}等。

\end{enumerate}

98粮改政策当年就无法贯彻实施,此后保护价开始放开,但98粮改精神尚在。

1999年5月30日,国务院发文《国务院关于进一步完善粮食流通体制改革政策措施的通知》。
《通知》中提出:
\begin{quotation}
  黑龙江、吉林、辽宁省以及内蒙古自治区东部、河北省北部、山西省北部的春小麦和南方
  早籼稻、江南小麦,从2000年新粮上市起\textbf{退出保护价收购范围}……1999年暂不退
  出保护价收购范围,但要\textbf{较大幅度地调低收购保护价格水平}。具体办法由各有关
  省级政府根据本地的实际情况研究决定。
\end{quotation}

2000年2月10日,棋盘乡党委书记李昌平向时任国务院总理的朱镕基写信,“\textbf{农民真
  苦,农村真穷,农业真危险!}”这几年间我国个别地区也发生个别粮农群体性事件。

2000年春天,经国务院批准,浙江成为全国第一个实行\textbf{粮食购销市场化改革}的省
份。

2000年2月2日,国务院办公室发文《关于部分粮食品种退出保护价收购范围有关问题的通
知》。《通知》中指出“从2000年新粮上市起,\textbf{长江流域及其以南地区的玉米退出
  保护价收购范围}。”

2001年我国继续缩小实行粮食保护价政策的范围和品种,将实行粮食收购保护价政策的地区局
限在粮食主产区,同时\textbf{赋予粮食主产区省级人民政府自主决策的权力,即自行确定实行保护价
  收购的品种、范围和办法。}

2001年7月31日,国务院发文《国务院关于进一步深化粮食流通体制改革的意
见》。《意见》中提出:
\begin{quotation}
  特别是东南沿海的浙江、上海、福建、广东、海南、江苏和北京、天津等地区的经济相对
  比较发达,农业和农村经济结构调整的潜力较大,粮食市场发育较好,粮食购销形势已发
  生很大变化,完全可以\textbf{放开粮食收购,粮食价格由市场调节}。
\end{quotation}


从2000年到2004年5年,当年生产粮食其实无法满足当年消费所需,缺口是由库存来补充,
借此我国每年消耗约400亿--700亿斤的库存\footnote{任职于国务院发展研究中心农村经济研究
  部的韩俊所作《当前我国粮食供求形势分析》一文中提出“自2000年以来我国粮食消
  费需求大致在9600—9800亿斤之间”,笔者结合我国粮食年产量计算出此一阶段粮食
  库存消耗数据。}……

\textbf{邓大才所作《粮改30年:农民、市场与国家的博弈与利益重
  构》}\cite{dacailianggai}对粮改的分析非常深刻客观,并具有完整逻辑链条的,这在
粮改论文中是\textbf{相当难得和罕见}的。为读者观看方便,笔者直接大篇幅引用邓大才的文
章吧。

\begin{quotation}
  1999年开始国家每年都\textbf{下调收购价格及保护价格},而且还允许部分收购价格低于
  保护价格。另外退出保护价格收购的粮食又缺少其他收购主体,\textbf{粮食价格大幅下
    滑},\textbf{“卖粮难”再次出现,}而且有些地区种粮还出现了亏损,农民纷
  纷\textbf{以脚投票,弃田抛荒,粮食播种面积不断降低,}1999年至2001年分别下
  降0.55\%、4.15\%、2.2\%,\textbf{粮食产量也逐年下降},三年分别下
  降0.76\%、9.09\%、2.06\%,特别是2000年粮食减产4621.1万吨,2001年粮食总产量只
  有45263.7万吨,达到了历史新低。\textbf{但是此时,中央对粮食供给仍然比较乐观,根
    据库存与粮食价格判断,粮食仍然供过于求,}其实此时已经开始酝酿\textbf{新一轮的粮食紧
    张。}

  2002年虽然粮食播种面积减少,但是粮食产量有稍许的增长,减缓了改革的压力,更重要地
  是\textbf{老一届的政府领导即将卸任,没有大改的动机,但是粮食的根本问题没有解决,
    粮农增收的环境没有解决,}2002年粮食价格下滑到了\textbf{谷底},2003年粮食播种面
  积继续减少,粮食再次减产2636.3万吨。2003年底粮食供给形势发生了重大的变化,粮食价
  格开始上涨,累计的粮食问题开始爆发。老的粮食政策已经无法适应当时的经济社会的需
  要,\textbf{新一轮粮改势在必行}。
\end{quotation}

2004年,粮食市场化改革开始。

\subsection{有限制的市场化改革}


2004年5月26日《粮食流通管理条例》正式对外颁布,赋予了粮食行政管理部门管理全社会的
粮食流通和对市场主体准入资格审查的职能,\textbf{个体工商户和非国企具有了直接收购
  粮食的资格,具有了合法性。}

2004年5月31日国务院召开全国粮食流通体制改革工作会议,发布的《国务院关于进一步深化
粮食流通体制改革的意见》明确宣布,\textbf{2004年全面放开粮食收购市场,实现粮食购
  销市场化和市场主体多元化。}


邓大才深入客观地介绍了2004年粮改起初几年的情况,笔者已无力再将农业问题继续写
下去,推荐读者尽量阅读邓大才的完整文章。为方便读者,以下仅直接引用邓大才原文
吧,
\begin{quotation}
  2004年中央以农民增收为主题发布了1号文件,要求“\textbf{国家将全面放开粮食收购和销
    售市场,实行购销多渠道经营}”。2月全国人大会议为配合和落实中央1号文件精神,
  政府工作报告以粮食为主题——“\textbf{调动农民积极性,增加粮食生产}”,推出了一系
  列的粮食生产优惠政策:\textbf{一是直接对农民实施粮食补贴,这是几千年中国政府第一
    次对粮食生产者进行直接补贴;二是加大粮食主产区减免农业税的力度,}11个粮食
  主产区省市降低3个百分点,其他地区降低1个百分点;\textbf{三是扩大良种补贴试点范围
    和规模,鼓励农民进行粮食结构调整;四是对重点粮食品种实行最低收购价格制度,
    实施“地板价格”保证农民利益;五是稳定农业生产资料价格。}这些政策都是\textbf{历
    史性的突破},表明国家\textbf{从向农民索取剩余转向休养生息,从“以农补
    工”转向“以工补农”,}政策的出发点和目标已经发生了根本性的变化。

  2004年粮食改革政策和粮食生产支持政策,使\textbf{粮价开始回升},连续两年粮食播种
  面积增长,粮食产量增长。但是由于多年粮食减产的供给压力,\textbf{粮食价格回升比
    较快}。

  为了培育粮食生产能力,2005年中央1号文件强调“\textbf{提高农业生产能力}”,一是继
  续加大“\textbf{两减免、三补贴}”等政策实施力度,即减免农业税、取消除烟叶以外的农
  业特产税,对种粮农民实行直接补贴,对部分地区农民实行良种补贴和农机具购置补
  贴。二是切实加强对粮食主产区的支持。三是建立稳定增长的支农资金渠道。

  2006年中央1号文件要求“\textbf{稳定发展粮食生产}”,\textbf{全国人大会议决定取消农业税},
  几千年以来加在粮食生产上面的税收被取消。2007年中央1号文件同样也强调“\textbf{继续
    促进粮食稳定生产}”,加大农业生产的补贴范围和水平,2008年中央1号文件要
  求“\textbf{高度重视发展粮食生产}”。由于粮食政策已经在“04粮改”调整到位,2005年
  至2008年基本没有出台有关粮食改革的专门文件,只是在四年的中央1号文件中强调粮
  食安全,确保粮食产量,减轻粮农的负担,增加粮食生产经营的支持和保
  护。\cite{dacailianggai}

  2007年至2008年出现的一些新情况检验着“04粮改”政策框架的弹性。这两年国际粮食价
  格大涨,但是\textbf{国内粮价}却在国家抛售储备粮的调节下\textbf{“非常稳定”},
  而且\textbf{粮食生产资料更是随国际粮价水涨船高},国内国际粮价差异巨大、粮食生产
  成本不断侵蚀农民的粮食收益,粮食生产是否又在隐藏着\textbf{下一轮的粮食危机}和农
  民“\textbf{以脚投票}”呢?
\end{quotation}


\todo[inline]{希望有人续写,最好说明下WTO影响}

\section{分税制}

\improve[inline]{2013年至今实行营改增,\url{http://tax.hexun.com/2013-09-02/157625914.html}}

\subsection{分税制历史资料简单汇总}

自1985年3月21日确定\textbf{财政包干制}以来,因利益关系,我国地方政府往往采用地方收
款(预算外收入,地方独享)代替税收(预算内收入)、减免或隐瞒税收、以行政收费、
集资、摊牌和赞助等代替税收等手段,实际控制了有效税率和税基。为应对中央财政收
入困难,中央政府向地方政府借款,并在还款上采用了拖延、欺骗、耍赖、强霸等手段
补贴中央财政,使地方政府不愿再借钱给中央。\cite{majuncaigai}中央在80年代末的分税
尝试均未有效实施。

\begin{quotation}
  1982—1992年,地方预算外收入年均增长30\%,远超过预算内收入年均19\%的增
  速。1992年,地方预算外收入达到了预算内收入的86\%,相当于“第二财政”了。\cite{zhishenshinei}
\end{quotation}


以下资料汇总主要来源为江大桥所作《我们地方没钱:分税制改革下中央与地方的博弈 》
\cite{difangmeiqian}与张曙光所作《中国经济学风云史》。

中央财政收入困难,90年代初期我国两任财政部长由“穷得只剩下背心和裤衩了”到“我
连背心都没有,只剩下裤衩了”。加之上章提到的汇率改革,造成的人民币持续大幅贬值、通货膨胀:

\begin{quotation}
  1992年,党的十四大提出确立建设社会主义市场经济,“\textbf{要逐步实行税利分流和分
    税制}”。6月5日,财政部开始在天津、辽宁、沈阳、大连、浙江、武汉、重庆、青
  岛和新疆九个试点试行分税制。

  1993年,朱镕基正式接手分税制的改革,当时\textbf{财政收入}在国内生产总值的比重
  从1979年的28.4\%降到1993年的12.6\%,\textbf{中央财政在全国财政的比
    重}从46.8\%降到31.6\%,也就是所谓“\textbf{双降}”的局面。
\end{quotation}

据张曙光所述,最具影响力并引起中央强烈重视、正式采取分税制改革的学术源头为当
时留美归国博士(学位)王绍光、胡鞍钢所著《中国国家能力报
告》\pagescite[][756]{fengyunshi1b},该报告发表于1993年5月。报告中认为当时实
行财政包干制的我国国情为“\textbf{弱政府、弱中央、强地方、财政收支最为分散}”。

\begin{quotation}
  1993年夏季,中国最高当局决定实行财税体制改革。7月23日,朱镕基副总理在全国财政工
  作会议上宣布,中央决定取消中央与地方财政大包干的制度。从1994年起在全国实施统一
  的财税体制。即建立\textbf{中央和地方的分税制}。从此,税制改革进入了快车道。

  为应对地方对即将施行的分税制的不满情绪,朱镕基亲自带队,
  从1993年9月9日--11月21日,先后分10站走访了17个省、市、
  区。\pagescite[][758]{fengyunshi1b}
\end{quotation}

1993年12月25日,国务院发布《国务院关于实行分税制财政管理体制的决
定》:

\begin{quotation}
  从一九九四年一月一日起改革现行地方\textbf{财政包干体制},对各省、自治区、直辖市
  以及计划单列市实行\textbf{分税制财政管理体制}。

  中央固定收入包括:关税,海关代征消费税和增值税,消费税,中央企业所得税,地方银
  行和外资银行及非银行金融企业所得税,铁道部门、各银行总行、各保险总公司等集中交
  纳的收入(包括营业税、所得税、利润和城市维护建设税),中央企业上交利润等。外贸
  企业出口退税,除一九九三年地方已经负担的20\%部分列入地方上交中央基数外,以后发
  生的\textbf{出口退税全部由中央财政负担。}

  地方固定收入包括:\textbf{营业税}(不含铁道部门、各银行总行、各保险总公司集中交纳的营业
  税),\textbf{地方企业所得税}(不含上述地方银行和外资银行及非银行金融企业所得税),地方
  企业上交利润,个人所得税,\textbf{城镇土地使用税,固定资产投资方向调节税,城市
    维护建设税(不含铁道部门、各银行总行、各保险总公司集中交纳的部分),房产税,
    车船使用税,印花税},屠宰税,农牧业税,对农业特产收入征收的农业税,(简称农业
  特产税),耕地占用税,契税,遗产和赠予税,\textbf{土地增值税},国有土地有偿使用收入等。

  中央与地方共享收入包括:增值税、资源税、证券交易税。增值税中央分享\textbf{75\%},
  地方分享\textbf{25\%}。资源税按不同的资源品种划分,大部分资源税作为地方收入,\textbf{海
    洋石油资源税}作为中央收入。证券交易税,中央与地方各分享50\%。

  一九九四年以后,\textbf{税收返还额}在一九九三年基数(即1993年的\textbf{消费
    税 + 75\% 的增值税 - 中央下划收入})上逐年递增,递增率按全国增值税和消费税的
  平均增长率的1:0.3系数确定,即上述两税全国平均每增长\textbf{1\%},中央财政对地方的
  税收返还增长\textbf{0.3\%}。
\end{quotation}

此外,为实行分税制,\textbf{财税分家},原财务系统下的税务系统独立为国税、地税两套机构;
中央通过分税制集中起来的财力,不断加大对落后地区的“\textbf{转移支付}”。

此后又对税权划分进行了一系列调整和完善:
\begin{quotation}
  连续6次调整中央与地方证券交易税分享体制,到2003年达到\textbf{98\%};金融保险业营
  业税率,从1997年11月起由\textbf{5\%}提高到\textbf{8\%},增加的收入全部归\textbf{中央财
    政}。2001年起分三年恢复到\textbf{5\%};1999年恢复开征\textbf{利息所得税},收入归\textbf{中
    央财政};从2002年起,除部分中央企业和机构缴纳的所得税继续作为中央收入
  外,\textbf{其他企业所得税和个人所得税收入}以2001年地方实际的所得税为基数,中央与
  地方\textbf{增量分成},分成比例2002年为5:5,2003年为6:4,以后年度再根据实际收入情
  况商议;\textbf{新办企业的所得税}由国家税务局征收,\textbf{新增税种的收入}都由国家税务
  局征收;来自新批转为\textbf{非农用地的国有土地有偿使用收入}上缴中
  央;2004年10月对\textbf{出口退税机制}进行重大改革,适当降低出口退税率,从2004年起
  以2003年出口退税为基数,起基数部分的应退税额由中央与地方
  按\textbf{75\%}和\textbf{25\%}的比例共同承担。\cite{eryuancaizheng}
\end{quotation}

“\textbf{财权与事权相匹配}”成为分税制基本原则及政策合理性诉求。胡锦涛
在2012年11月8日中国共产党第十八次代表大会上所作政府报告中改为“\textbf{财力与事权相
  匹配}”。“财权”(与税基相关)变“财力”(政府拥有的可支配财产),这一变动
是为减轻各地区发展不平衡现象,便于向落后地区转移支付。

\subsection{关于分税制的部分后果}

分税制毫无疑问地是中国自1992年确立社会主义市场经济体制至今,影响最为深远的政
策,并且这一影响持续至今。

\improve[inline]{撤县设区,三级财政没有写,朱镕基所说未完成的。因中国财政以对
  企业征收间接税为主, 不仅九成的税收征收自企业,税收之外的其他政府收入基本也
  都征收自企业,比如土地转让费和国有资本经营收入等。社保费中个人缴纳的比例也
  低于企业缴纳的比例。所以在分税制改革后的头些年,地方政府在财政支出上向招商
  引资倾斜(如基础设施建设、企业补贴等),而民生支出(教育、医疗、环保等)相
  对不足。2002年,中央提出“科学发展观”,要求“统筹经济社会发展、统筹人与自
  然和谐发展”,要求更加重视民生支出。由于第一章中讨论过的规模经济、信息复杂
  性等原因,民生支出基本都由地方政府承担,所以地方支出占比从2002年开始快速增
  长,从70\%一直增长到了85\%。

  总的来看,分税制改革后,地方政府手中能用来发展经济的资源受到了几方面的挤压。
  首先,预算内财政支出从重点支持生产建设转向了重点支持公共服务和民
  生。20世纪90年代中后期,财政支出中“经济建设费”占40\%,“社会文教费”(科
  教文卫及社会保障)只占26\%。到了2018年,“社会文教费”支出占到了40\%,“经
  济建设费”则下降了。(17)其次,分税制改革前,企业不仅缴税,还要向地方政府缴
  纳很多费(行政收费、集资、摊派、赞助等),这部分预算外收入在改革后大大减
  少。90年代中后期,乡镇企业也纷纷改制,利润不再上缴,基层政府的预算外收入进
  一步减少。最后,2001年的税改中,中央政府又拿走了所得税收入的60\%,加剧了地
  方财政压力。地方不得不另谋出路,寻找资金来源,轰轰烈烈的“土地财政”就此登
  场。\cite{zhishenshinei}}

张维迎书中写到:
\begin{quotation}
  新财税体制的实施达到了政府的目标,实施的第一年,虽然\textbf{整个财政收入占GDP的比
    例}从1993年的12.31\%下降到1994年的10.83\%,但\textbf{中央收入占财政收入的比
    例}却从22\%提高到55.7\%。此后,财政收入以年平均18.84\%的速度增长,相当
  于GDP增长速度的2倍,到2010年,这两个比例分别是20.71\%和51.1\%。形成了\textbf{强政
    府、富中央、穷百姓、地方苦乐不均}的格局。
\end{quotation}

\begin{quotation}
  分税制改革的直接结果是中央地方财政收入的初次分配比例急剧变化,地方财政收入占
  全国总财政收入的比重由1993年的78\%迅速下降到1994年的44.3\%,而与此同时地方财
  政支出比重一直维持在70\%左右,2006年达到75.3\%,分税制改革就形成了地方财政几
  乎30\%的巨大收支缺口。\cite{JJXU201304002}
\end{quotation}

% 朱镕基的反应非常强烈,指责“攻击分税制,说分税制掏空地方财政,造成农民贫穷的
% 人,“根本就是无知!无知还透顶”。他用2010年的财政数据举例,92年、93年的中央财政
% 比例是28\%、27\%,而2010年,扣除财政转移支付3万3000亿,中央财政不过是1万5900亿只
% 占20\%左右,地方政府有的是钱。

% 朱镕基认为根子在于当时把买地的钱全都划给了地方政府,而且不纳入预算,所以地
% 方政府开始疯狂卖地。但是他也坦承,分税制也有问题主要在于转移支付,造成
% 了“跑部进钱”。


\subsubsection{分税制的缺点}

\improve[inline]{分税制优缺点还需进一步补充修改!}

\begin{enumerate}
\item \textbf{增值税}是财政收入大头,其75\%被中央拿走,地方财政收入面对一个很大缺口。
  地方政府为弥补缺口最终走向了“\textbf{土地财政}”——一部份是非税收入,包括土地出
  让金、土地租金、新增建设用地有偿使用费等;一部份是分税制中地方政府享有的税
  收——耕地占用税、房地产和建筑业营业税与所得税、土地增值税、城镇土地使用税,
  房产税等。

  问题是,分税制是不是“土地财政”的最主要或根本原因?分税并转移支付及其他分
  配后,地方政府仍然确实没钱吗?还是说“土地财政”的根本原因并非分税制制度本
  身,而是重视任期内GDP发展速度的\textbf{官僚考核制度}呢?或是权力寻租呢?

  2011年清华校庆,朱镕基回校参观,借着师生座谈会的机会说到分税制
  \begin{quotation}
    指责“攻击分税制,说分税制掏空地方财政,造成农民贫穷的人,“根本就是无知!
    无知还透顶”。他用2010年的财政数据举例,92年、93年的中央财政比例
    是28\%、27\%,而2010年,扣除财政转移支付3万3000亿,\textbf{中央财政不过
      是1万5900亿只占20\%左右,地方政府有的是钱。}

  ……

  (财政收入)总共8万亿,一来一回(地方)收回来7万3千亿,还少啊?还没钱?现在地方
  有的是钱。这房地产(问题)根子就是\textbf{房改政策错误}……我们制定了一个错误的政策,
  就是\textbf{房地产的钱,都收给地方政府,而且不纳入预算,这不得了。这个钱就是搜刮
    民膏,所以把地价抬得那么高。}这个绝对不是分税制的错误。地方没少收钱……
  \end{quotation}

\item “\textbf{财权上移,事权下放}”是对分税制认识的普遍共识。我国实行的是中央、
  省、市、县、乡五级财政。根据财政局科研所领导贾康、白景明研究:
  \begin{quotation}
    自1994年以来,中央的资金集中度实际在\textbf{下降}(从1994年的55.7\%下降
    到2000年的52.2\%),而省级政府的集中程度不断加大,\textbf{年均提高2\%}(从1994年
    的16.8\%提高 到2000年的28.8\%)。市一级政府同样在想方设法增加集中程度。2000年
    地方财政净结余134亿元,\textbf{而县、乡财政赤字增加}。这些说明\textbf{实际上财
      力在向省、市集中。}\cite{xianxiangfenshui}

    省以下政府层层向上集中资金,基本事权却有所下移,\textbf{特别是县、乡两级政府,履
      行事权所需财力与其可用财力高度不对称,成为现在的突出矛盾。}
  \end{quotation}

  据腾霞光研究:
  \begin{quotation}
    1994年分税制改革几乎没有触及省以下转移支付,各级包干体制下的转移支付得以延
    续。……地方转移支付以省级为主,地市级规模很小,县级几乎微不足
    道。1994--1997年全国28个地区均等化转移支付,中央占37\%、省级占47\%、地市级
    占15\%、县级只占\textbf{\%1}。
  \end{quotation}

  官僚考核制度与官僚权力寻租变现明显也是造成五级政府\textbf{层层盘剥克扣}的主要原
  因,并且因县、乡政府财权事权两极分化,事实上加重了当时的城乡二元对立,县乡困
  难。

  胡鞍钢在2014年接受采访时也说到,“中央现在只是对省级政府转移支付,确实\textbf{没有
  解决省以下的转移支付问题}。”其实在王绍光、胡鞍钢1993年所著《国家发展报告》
  中,确实已经提出建立\textbf{四级政府运行结构,实行三级财政征收和四级财政}使用。但
  是“\textbf{当时中央来不及解决这一根本问
    题}”\footnote{\href{http://business.sohu.com/20140428/n398892838.shtml}{胡鞍钢:
      中央转移支付可直接到县}}。2006年后我国逐步提出“\textbf{省管县}”,最终目标建
  立\textbf{财政实体三级架构:中央、省和市县}。

\item 中央所拥有的集中财力及转移支付权力,造成“\textbf{跑部进京}”现象。各级地方政府、
  机构、企事业单位机构均设立驻京办事处;部委官员权力巨大,其个人与地区的政治
  人文关联,同地区所得政策支持相关,也存在官僚权力寻租变现问题。

  如朱镕基所说 “当然我们还有缺点,主要是\textbf{返还支付}的方式……要靠地方\textbf{‘跑
    部钱进’},求爷爷告奶奶才能拿到,分税制有缺点,但我负的责任不是主要的,因
  为我当时就说,分税制改革没有完,要继续进行。”

\item 张曙光书中列出了几人对分税制的不同意见,认为“张曙光对《报告》的批评带有
  根本的和理论的性质”,也提及胡鞍钢曾说“张曙光对《报告》的批评最详尽”,“是
  建立在认真读书的基础之上的”\pagescite[][765]{fengyunshi1b}。笔者在此简单片面
  的论述下张曙光当时对《报告》的批评意见。

  张曙光的批评意见主要集中在国家与市民社会的不同上,“\textbf{强国家}”不等于“\textbf{强
    社会}”,其中贯穿经济学中对市场“\textbf{看不见的手}”的推崇。强国家政府也会造
  成国家权力的过份膨胀和对个人经济自由的侵害,应当重在市场基础结构的建立和市
  场关系的发育。张曙光所倾向的既不是强中央,也不是强地方,而是\textbf{强市场、弱政
    府}。

  一些老经济学家说话,真是听君一席话,如听一席话……
\end{enumerate}


\subsubsection{分税制的优点}

分税制的优点很难单列,他的有机性太强,因此我就放在一起来说吧。

80年代国有企业“利改税”的实施,并未达到国有企业利税两分的预期。分税制完善了一系
列税制,建立了以增值税为主体,消费税和营业税为补充的流转体制。
\begin{quotation}
  1994年实行了分税制,暂免交利,名义看是利税合一,企业向国家交税,实际上把利润和
  税收两个范畴分开了。一是……结束了国有企业不纳税的历史……;二是利税合一后的国
  有企业在交税的同时,国家不再向企业投资,企业的资金来源采取了“拨改贷”的方
  式。\footnote{\href{http://www.aisixiang.com/data/28752.html}{张曙光:从利改税
      到收租交利――对国有企业改革及上交利润的评论}}
\end{quotation}
国有企业逐渐融入市场,开始发展,大国企受到重视,小国企退出或者转为非国有,也
就是“抓大放小”。“抓大放小”也使我国失业人数、失业率持续增长。\footnote{国家统计局
  失业率统计的真实目的不是显示真实失业人数,而是为由国家负担的城镇职工相关投
  入数据(如失业保险金、补贴等)提供参数。}至1998年,国有企业迎来了根本性的剧
变——\textbf{大下岗}。其实官方保守口径中,1992年已经开始“下岗分流、减员增效”,1997
年一年下岗人数已是1000万。

邓小平认为的我国体制优势——“\textbf{集中力量办大事}”得以建立。当然,“集中力量办
大事”未必就使大事有个好结果,也可能造成更大恶果,但分税制确实赋予了中央政府更多
的可能与力量,得以指导整个市民社会朝某方向发展。

胡鞍钢认为“中央集权——地方分权混合模式得以建立,且以中央政府为主导,中央确
立了自己的主导地位”,避免了省级政府权力过重、各自为政。弱中央、强诸侯在中国
历史上没有好结果。

笔者认为,就全球化环境下,繁荣地区、都市本身就获得了更多自主权,与世界各国、
地区建立了更多联系,这些繁荣地区、都市在一些层面或部分上甚至可以影响或超越国家。

近三十年世界各国基本都处在民族国家与全球化这一具有内部强大张力的框架之下,一
个强力国家的管控,能够使繁荣地区不至于太过出格,同时为落后地区多输血,并使国
内意识形态较为统一,地区的\textbf{多样性}仍主服从于中央的\textbf{统一性}\footnote{多样性和统一性
  的提法为胡鞍钢所述}。

繁荣地区、都市为了自身利益,不会过于积极主动输血,而国家可以通过转移支付或强
权照顾地区发展不平衡,照顾落后地区,调动整个社会力量。事实上,中国政府
在1998、2008年经济危机中的表现抢得世界关注,引发世界上各阵营各界人士艳羡我国
整体力量,查尔斯·德伯一篇文章的题目挺有意思《\textbf{中国:超级社会主义还是超级资本
  主义?}》。



% 朱镕基是坦诚的,地方政府财政在民生上的捉襟见肘和土地财政问题 并不是完全由分税制造成的。李郇、洪国志、黄亮雄考察1999-2008年间,240个地级市地方财政预算内缺口和土地财政的关系发现,地方财政预算内缺口并不能完全解释土地财政的关系。

% 只有1999-2003年间,随着地方预算内缺口扩大,土地财政的规模才开始在不断扩大;而2003年之后,地方预算内的切口是比较稳定的,但是土地财政的规模却在飞速膨胀。这就说明20003年后,财政压力并不能解释地方对于土地财政的疯狂。

% 2003年是极为关键的一年,土地财政的起飞更多是因为2002年有两项政策,《招标拍卖挂牌
% 出让国有土地使用权的规定》和 企业和个人所得税改革。前者开了口子,地方政府可以通过
% 土地挂牌拍卖,从市场拿钱;后者就给了地方政府一鞭子,中央和地方以60\%:40\%的比例
% 来分企业和个人所得税增量。

% 2001年,地方所得税是1636亿,占地方总收入的21\%。被中央拿掉大头之后,地方自然对营业税的依赖进一步上升。 而建筑业又是营业税的第一大户,自此之后,地方政府盯着上房地产,疯狂的发展房地产也就不奇怪了。

% 其实,省级地方政府角色并没有那么无辜。我国是五级财政制度,由于分税制主要是中央和省级财政之间实行,所以很多的中央的财政转移支付必须通过省级,才能到达市级,再到达市级、县级和乡级,可能会面临层层的盘剥和挪用。

% 贾康、白景明研究发现,从1994-2000年间,中央的资金集中度从55.7\%下降到52.2\%,而省级资金集中度却从16.8\% 提高到28.8\%,年均提高2\%。什么意思?就是国家给省里的钱越来越多了,而省里发到县里的钱越来越少了。

% 2003年,个人所得税改革和土地挂牌拍卖合法化,潘多拉的魔盒正式被打开,土地财政和各个地方政府领导的晋升冲动产生化学反应,固定资产投资飙升,房地产开始起飞,中国经济逐渐开始出现过热的情况。 2003-2006年,每年的 GDP 增长都在10\%以上,中央又开始和地方展开新一轮的博弈,宏观调控,为过热的经济降温。

% 2010年前后,东部开始讲产业升级,腾笼换鸟,中西部就开始蠢蠢欲动。

% 2007年,郑州就已经成立专门针对富士康的招商工作领导小组,市长亲自任组长。可惜,富士康仍在鼎盛,省级领导时常来亲自拜访,市级领导还常常被拦在门外,郑州并没有任何特殊的待遇,只能时断时续的联系着。2010年,富士康跳楼事件发酵,传出北迁的消息,郑州又兴奋了起来,不过它只是其中的一个,当时还有廊坊、武汉、成都、烟台等许多内陆城市翘首以盼。

% 最后,郑州给出了极为优厚的条件,五免五减半;降低每年的社保和其他费用1亿美元,享受出口退税。5年之后,当传出中央开始清理税收优惠的消息,郭台铭这回就积极主动地多,还亲自去郑州拜会市长马懿,确认有关50亿的财政补贴的落实。


% 2012年,三星传出要考虑在海外投资300亿美元建厂, 西安就曾靠着天量的补贴,在北京和重庆的手中,抢下了三星300亿美金的高端储存芯片项目。

% 在最后2012年4月2日,三星确定项目落户西安的时候,一片哗然,众人纷纷猜测地方政府的补贴力度到底有多强。

% 媒体一度流传出,2000亿数字的嫁妆:投资30\%的补贴,“十免十减半”,修建配套措施,无偿提供土地。 众人惊呼,赔本做买卖。娄勤俭亲自上了凤凰卫视的《神州问答》节目解释,不做亏本买卖,没有“十免十减半”的政策,只是有“五免五减半”,而且是国家的政策,西安只承担运输成本费用。娄省长最后话里还是留了余地,即使不赚钱,

% “但是将有260多家配套企业过来,现在就已经有近100多家企业,跟着来投资,这就是一种增值效应,产业带起来了”。

% 2012年,中央以减税的名义提出营业税改增值税的政策,理由是消除第三产业重复征税,总理亲自赴沪督军。
% 中央如此劳心费神,自然也别有深意。中国经济的顽疾,是地方政府对于房地产、建筑业的依赖。房地产和建筑业除了能快速拉高 GDP 之外,还是营业税的主要来源,而营业税又是地方政府的主要税种,所以造成了地方政府受到税收激励偏爱房地产。

% 2016年5月1日,房地产、建筑业等营业税的主体税种纳入营改增,中央和地方的博弈达到高潮。 2016年3-4月,总理在一个月之内,连开了三次会来讲营改增的重要性。

% 中央拿了营改增拿掉营业税,自然对地方也要所交代,这也就是给地方开的房地产税。房地产税并不是房产税,房产税是一个已经存在的税种,只是一直免征罢了,而房产地产税是指整合房产税和城市土地使用税的新税种。

% 2017年4月7日,教育部就出台多校划片的政策以朝阳区为试点,规定6月30日之后拿到房产证的,参加多校划区的曾策,希望能够通过增加购买学区房的不确定性来降低房价, 可是刚刚买房的家长,就去教育部门抗议,指责教育部违反义务教育法中,“免试就近入学、学区制和九年一贯对口招生”的规定, 后来多校划片的政策也就不了了之了。

% 财权与事权相匹配”是分税制的基本原则,有多少钱办多少事。但这种政策有着严格的限制条件,即辖区内经济发展水平的同质化。在中国这样一个大国,各地的经济发展条件和能力并不均衡。2007年,党的十七大报告采纳了“财力与事权相匹配”的提议。

% 另一句叫做“上面千根线,基层一根针”——上面出政策,下面对口执行,任务最终都压到基层政府,形成“财权层层上收,事权层层下移”的局面。比如“中央点菜,基层埋单”,中央部委下发文件要搞新农合医保、农村危房改造等,地方政府就要掏钱。

% “最弱小的一级政府却承担着具有全局意义的支付责任。”天津财经大学教授李炜光评论说。这种体制使县乡两级政府承担了地方的多项公共服务责任。全国两千多个县级组织,曾经有一半以上拖欠教师和离退休人员工资,或大面积拖欠银行贷款和建筑工程款,以至于中央财政亲自出手,出台“三奖一补”政策(指中央对地方缓解县乡财政困难奖励和补助的办法)为此兜底。

% “从决策权来看,中国是全世界最集权的国家,从执行权来看,我们又是最分权的国家。”刘尚希说。

% 在刘尚希看来,造成这一现象的根本原因是,“地方政府只对本级财政负责,它对下没有责任,不愿下移财力,还‘市刮县’向下摊派等”。因此,他开出的改革药方是摈弃“层级财政责任制”,建立“辖区责任制”,即上级政府要对辖区内各级政府的横向和纵向财政平衡负责。比如县一级财政发不出工资,首先应该问责市政府或省政府,而不是由中央财政埋单。



% \begin{figure}[!p]
  \centering

  \begin{tikzpicture}[baseline]
    \begin{axis}[width=15cm,height=6cm,% no markers,
      every axis/.append style={line width=.8pt},
      cycle list={
        {blue,mark=*},
        {red,mark=square},
        {orange,mark=o},
        % {loosely dotted,mark=+},
        {green!90!black,mark=otimes*,
          mark options={fill=brown!40},
        }% <-- don’t add a comma here
      },
      xtick=data,
      % xticklabels={1975,1980,...,2020},
      xmin=1979,
      xmax=2018,
      xticklabel style= {/pgf/number format/1000 sep=,xshift=1ex, rotate=60,anchor=east,},
      ytick distance=0.5,
      % ytick=data,
      % yticklabels={0.1, 0.2, 0.3, 0.4, 0.5, 0.6},
      % extra y ticks={0.1,0.2,0.3,0.6},
      % extra y tick style={grid=major},
      % extra y tick labels={0.1, 0.2, 0.3, 0.6},
      xlabel=年份,
      ylabel=产量(亿吨),
      % grid=major,
      ymajorgrids,
      ]
      % \addplot+ [point meta=explicit symbolic, every node near coord/.append style={color=red}, nodes near coords,]
      \addplot table [x=totalyear, y=total, col sep=comma] {figures/econohistory/realliangjia.csv};
    \end{axis}
  \end{tikzpicture}
  \vspace{-14pt}
  \caption{1980--2017年粮食产量}
  \label{fig:liangchanliang}
  \capsource{资料来源:国家统计局,《中国统计摘要--2018)》}
  \bigskip\bigskip

  \begin{tikzpicture}[baseline]
    \begin{axis}[width=15cm,height=6cm,% no markers,
      every axis/.append style={line width=.8pt},
      cycle list={
        {blue,mark=*},
        {red,mark=square},
        {orange,mark=o},
        % {loosely dotted,mark=+},
        {green!90!black,mark=otimes*,
          mark options={fill=brown!40},
        }% <-- don’t add a comma here
      },
      xtick=data,
      % xticklabels={1975,1980,...,2020},
      xmin=1979,
      xmax=2018,
      xticklabel style= {/pgf/number format/1000 sep=,xshift=1ex, rotate=60,anchor=east,},
      % ytick=data,
      % yticklabels={0.1, 0.2, 0.3, 0.4, 0.5, 0.6},
      % extra y ticks={0.1,0.2,0.3,0.6},
      % extra y tick style={grid=major},
      % extra y tick labels={0.1, 0.2, 0.3, 0.6},
      xlabel=年份,
      ylabel=增长率(\%),
      % grid=major,
      ymajorgrids,
      ]
      % \addplot+ [point meta=explicit symbolic, every node near coord/.append style={color=red}, nodes near coords,]
      \addplot table [x=totalyear, y=totalpercent, col sep=comma] {figures/econohistory/realliangjia.csv};
    \end{axis}
  \end{tikzpicture}
  \vspace{-14pt}
  \caption{1980--2017年粮食产量增长率(较上一年)}
  \label{fig:liangzengzhang}
  \capsource{资料来源:国家统计局,《中国统计摘要--2018)》}
  \bigskip \bigskip

  \begin{tikzpicture}[baseline]
    \begin{axis}[width=13cm,height=6cm,% no markers,
      every axis/.append style={line width=.8pt},
      cycle list={
        {blue,mark=*},
        {red,mark=square},
        {orange,mark=o},
        % {loosely dotted,mark=+},
        {green!90!black,mark=otimes*,
          mark options={fill=brown!40},
        }% <-- don’t add a comma here
      },
      xtick=data,
      % xticklabels={1975,1980,...,2020},
      xticklabel style= {/pgf/number format/1000 sep=,xshift=1ex, rotate=60,anchor=east,},
      ymin=0.3,
      ymax=0.6,
      ytick distance=0.05,
      % ytick=data,
      % yticklabels={0.1, 0.2, 0.3, 0.4, 0.5, 0.6},
      % extra y ticks={0.1,0.2,0.3,0.6},
      % extra y tick style={grid=major},
      % extra y tick labels={0.1, 0.2, 0.3, 0.6},
      xlabel=年份,
      ylabel=价格(元/公斤),
      % grid=major,
      ymajorgrids,
      ]
      % \addplot+ [point meta=explicit symbolic, every node near coord/.append style={color=red}, nodes near coords,]
      \addplot table [x=x, y=y, col sep=comma] {figures/econohistory/realliangjia.csv};
    \end{axis}
  \end{tikzpicture}
  \vspace{-14pt}
  \caption{1980--1998年粮食市场真实价格(1978年不变价格)}
  \label{fig:liangjia}
  \capsource{资料来源:卢锋,《三次粮食过剩(1984--1998)》}
\end{figure}


% 我国是农业大国,农业是立国之本。农业相关问题处理不好,国家必然动荡不堪,但这又是
% 一个现实的巨大难题。我国总体来说经济水平不发达、非农产业所能提供就业岗位少、农业
% 生产力低下、农村人口多且生活水平很差等等。种种现状错综复杂,且彼此交织、充满张力
% 和矛盾,并无鲜明出路。

% 列宁在1905年左右


% 农业大国的含义不止是农产品,还包括农村、农民问题,
% 即“三农”。三农可覆国。三农问题的深层逻辑不为民间掌握。有关三农问题的论述往往趋
% 于简单化、表面化、片面、避重就轻。

% 我国过往经济差,生产水平低,农民人口多且贫穷,



% 他们只能是被抛离至更弱势地位,被抛离的一部分人中将
% 成为愿做艰辛工作的劳动力红利,国家的合法性也将在重重问题中受到损害。
% 也希望读者能够冷静客观思考,体谅国家实际困难,寻得妙方。

%%% Local Variables:
%%% mode: latex
%%% TeX-master: "../main"
%%% End:
