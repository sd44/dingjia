\chapter{有关中国}

本章只是非常粗陋的第一版草稿,准备以后整合进我正在做的开源免费电子书中,结构内容都会有较大变动。

而在中国,有的迷梦表现在人民希望“上下五千年中华传统文明”的回归上,认为借此可使中国在纷繁世界中寻得一条救世之路;有的迷梦表现在人民对于国家力量的不切实际的希望上,认为国家拥有无比强大力量和可能去应对各种挑战,国家的兴衰纯粹依托于国家的举措是否得当;更多的迷梦则表现在人民希望一位横空出世,或者寄托于过去历史中的某位领袖人物。这位圣主在为国为民的道路上,必然与旧的、压迫的、有损人民的强大体制做斗争。在这种艰苦斗争的过程中,他披肝沥胆、远见卓识,虽数经艰险却屡屡化险为夷,实则是无往不胜,他带领全国人民走向一个更为美好的图景。当其半道崩殂,继承他衣钵的人自然不会像他一样有能力,甚至可能是一个反对他的反动派,这都使其巨大贡献受到折损甚至毁于一旦。

笔者在这里并不是说没有具有开创性和心怀天下的领袖或着强而有力的国家,而是希望读者注意\textbf{历史现实的限度},特别是\textbf{当代全球化与民族国家的巨大张力}下的历史限度。这种限度使个人或少数几人,甚至国家的能力均是有限的。不管领袖或国家如何强力,想要超越限度的前提是必须尊重限度,凭空去违抗这种现实客观规律既不明智也会带来恶果。

\section{削弱迷梦------论尼采的酒神和日神思想}
\label{sec:nicai}

在尼采的美学概念中,日神和酒神,即阿波罗(梦)和狄俄尼索斯(醉)两者均
为“\textbf{迷醉的类型}”\pagescite[][125]{ouxianghuanghun}。

\begin{quotation}
  按其词根来讲,阿波罗乃是“闪耀者、发光者”,是光明之神,他也掌管着内心幻想世界
  的\textbf{美的假象}。这种更高的真理,这些与无法完全理解的日常现实性相对立的状态
  的完满性,还有对在睡和梦中其治疗和帮助作用的自然的深度意识,同时也是预言能力的
  象征性类似物,一般地就是使生活变得可能,变得富有价值的各门艺术的象征性类似物。
  然而,有一条柔弱的界线,梦境不可逾越之,方不至于产生病态的作用,\textbf{不然的
    话,假象就会充当粗鄙的现实性来欺骗我们}。

  真实存在者和原始统一性,作为永恒受苦和充满矛盾的东西,为了自身得到永远的解脱,
  也需要\textbf{迷醉的解脱},也需要\textbf{迷醉的幻景}、\textbf{快乐的假象}。

  阿波罗以崇高的姿态向我们指出,这整个痛苦世界是多么必要,它能促使个体产生出具有
  解救作用的\textbf{幻景},然后使个体沉湎于幻景的关照中,安坐于大海中间一叶颠簸不
  息的小船上。\cite{beijudansheng}
\end{quotation}

每一人都不可避免、程度不一地受到日神阿波罗式的“迷梦”的影响。而在中国,有的迷梦表现在人民希望“上下五千年中华传统文明”的回归上,认为借此可使中国在纷繁世界中寻得一条救世之路;有的迷梦表现在人民对于国家力量的不切实际的希望上,认为国家拥有无比强大力量和可能去应对各种挑战,国家的兴衰纯粹依托于国家的举措是否得当;更多的迷梦则表现在人民希望一位横空出世,或者寄托于过去历史中的某位领袖人物。这位圣主在为国为民的道路上,必然与旧的、压迫的、有损人民的强大体制做斗争。在这种艰苦斗争的过程中,他披肝沥胆、远见卓识,虽数经艰险却屡屡化险为夷,实则是无往不胜,他带领全国人民走向一个更为美好的图景。当其半道崩殂,继承他衣钵的人自然不会像他一样有能力,甚至可能是一个反对他的反动派,这都使其巨大贡献受到折损甚至毁于一旦。

笔者在这里并不是说没有具有开创性和心怀天下的领袖或着强而有力的国家,而是希望读者注意\textbf{历史现实的限度},特别是\textbf{当代全球化与民族国家的巨大张力}下的历史限度。这种限度使个人或少数几人,甚至国家的能力均是有限的。不管领袖或国家如何强力,想要超越限度的前提是必须尊重限度,凭空去违抗这种现实客观规律既不明智也会带来恶果。

《悲剧的诞生》一书的中译者孙国兴在译后记中总结到尼采所探求的问题是“人何以承受悲苦人生”。尼采的答案是日神与酒神所融合的“悲剧------形而上学的慰藉”,即“变幻不居的现象背后坚不可摧的、永恒的生命意志”。也就是尼采所说的“预感到太一怀抱中一种至高的、艺术的原始快乐”。在这种形而上学意义上,“原始痛苦”与“原始快乐”\textbf{根本是合一}的。

笔者认为,尼采试图以纵情、忘我从而达到“\textbf{普遍性}”和“\textbf{至高意蕴}”的酒神精神中和现代性中过强的日神(时常作为“\textbf{守着种种界限和适度原则}”的、被规训和限于迷梦的\textbf{个体化神化})精神,来实现生命和身体在原始期就具有的强力意志。

笔者所说个人社会学,部分受到尼采影响。个人社会学首先也是建立在人类全体之上------社会伦理(虽然尼采在一定程度上反对社会伦理,但也鼓励对他个人的背离)。其次它本身不可避免带有\textbf{空想或试验的乌托邦性质}。从事个人社会学的个人,承担这种悲剧的同时也带有对悲剧的享受。为达个人社会学目的,也必然要去削弱强大现代性中日神阿波罗的迷梦。《有关中国》这一章正是致力于此。

\section{简论汪晖《当代中国的思想状况与现代性问题》}

汪晖在其名文《当代中国的思想状况与现代性问题》中写到,“\textbf{反现代性的现代化理论}\ldots{}\ldots{}是晚晴以降中国思想的主要特征之一”,而毛泽东思想中这一“\textbf{悖论式}”的特征表现更为明显突出:

\begin{quotation}
对于毛泽东来说,他一方面以集权的方式建立了现代国家制度,另一方面又对这个制度本身进行“文化大革命”式的破坏;他一方面用公社制和集体经济的方式推动中国经济的发展,另一方面他在分配制度方面试图避免资本主义现代化所导致的严重的社会不平等;他一方面以公有方式将整个社会组织到国家的现代化目标之中,从而剥夺了个人的政治自主权,另一方面他对国家机器对人民主权的压抑深恶痛绝。总之,中国社会主义的现代化实践包含着反现代性的历史内容。{[}\cite{wangxiandai}
\end{quotation}

毛泽东时期“以消灭工人和农民、城市和乡村、脑力劳动与体力劳动的'三大差别'这一\textbf{平等目标}为主要目的的”社会主义公有制,它的城乡关系却步入了“计划经济时期\textbf{严格的城乡二元分割}。\textbf{城市高度剥夺农村剩余},除了少量工业品进入农村,大部分劳动用于支持工业化和国防建设,本身亦维持在较低的生活消费水平”\cite{yangkongjian}的阶段。

汪晖可以认识到问题,但很可能因其主观意识形态倾向未能更为深入本质。毛时代的问题,正是一种“\textbf{苏联的今天就是我们的明天}”\footnote{上世纪50年代初中苏关系未破裂时流传的一句话}式的问题,中国从列宁、斯大林处吸取了大量内容,这使中国在一些方面不可避免地继承了苏联的一些错误;更深层和根本的原因则在于马克思科学社会主义本身就蕴含着对于马克思历史唯物主义的\textbf{背离},从而带有\textbf{空想乌托邦}的性质。即使资本主义要求大量劳动者“战战兢兢,畏缩不前,像在市场上出卖了自己的皮一样,只有一个前途---------\textbf{让人家来鞣}”;即使资本主义蕴含着反对自身先进生产力发展的矛盾;即使资本主义将不断产生短波、长波的经济危机\ldots{}\ldots{}在历史和理论的结合之下,我们可能仍要悲痛地承认这一事实:\textbf{资产阶级仍代表着最为先进的生产力,资本主义仍是最为先进的生产关系}\ldots{}\ldots{}

当人们(不管其是不是马克思主义者)寻求社会伦理大前进时,往往要面对无能为力从根本
上对抗资本的悲剧,从而微观、空想、希望或者脱离实际,甚至造就苦难。

按照霍华德和King的说法,传统社会主义国家的建成原因有类似于苏联由政治层面发起的自
上而下革命;有城市知识分子和农民联合起来推翻旧体制;有其他社会主义强国压倒性的政
治、军事影响下的改朝换代\ldots{}\ldots{}笔者认为,霍华德和King并未直言的意思应当
是:很可能并无一国是由无产阶级、工人阶级领导并自发建成。这一点甚至可能说明了传统
社会主义的不可能成功\ldots{}\ldots{}


当社会主义国家建成后,为应对世界范围内资本主义或现代性的强大理性,社会主义国家也不得不运用这一理性。特别是社会主义革命家中,有的人为实现社会大同的美好理想,因\textbf{现实实践的种种困境}似乎是\textbf{不得不}采用极端手段,这种极端手段又常常取代目的,从而成为\textbf{反社会伦理},在某些方面反而比资本主义社会还要\textbf{人为加重和扩大了掠夺性积累}(大卫·哈维将前资本主义社会的\textbf{原始积累}加以拓展,改为资本主义常用手段------掠夺性积累)的苦难。这便是真正的“\textbf{悖论式}”\ldots{}\ldots{}(笔者的详细阐述请见另一章《中早期苏俄科社实践》)。

\begin{quotation}
1989年,\textbf{一个历史性的界标}。将近一个世纪的社会主义实践告一段落。两个世界变成了一个世界:\textbf{一个全球化的资本主义世界}。中国没有如同苏联、东欧社会主义国家那样瓦解,但这并没有妨碍中国社会在经济领域迅速地进入\textbf{全球化}的生产和贸易过程。中国政府对社会主义的坚持并未妨碍下述结论:中国社会的各种行为,包括经济、政治和文化行为甚至政府行为,都深刻地\textbf{受制于资本和市场的活动}。
\end{quotation}

汪晖之所以将1989年这一硬性年份定为“历史性的界标”,除世界上“西方体制最终胜利的
历史终结论”外,也在相当程度上参考了中国的社会转向。关于89之后的中国,汪晖在另一篇文章写到:

\begin{quotation}
  专政手段(相对于以前的意识形态手段)与经济改革的结合,它标志着旧的国家意识形态的
  基本效能已经丧失。正是在这样的条件下,“\textbf{新自由主义”才能取而代之成为一
    种新的统治意识形态},并为国家政策、国际关系和媒体的价值取向提供基本的方向和合
  理性,为某些新自由主义知识分子在国内和国外媒体中扮演双重角色(即国家政策的鼓吹者
  和所谓“民间知识分子”)提供了制度的和意识形态的前提。
\end{quotation}

\textbf{新自由主义}是什么,它的表现有哪些,产生什么影响,笔者放在下一篇来探讨。

\section{中国新自由主义简介}

关于新自由主义相关定义,请参阅本书\cref{chap:neoliber}。

金融危机可能是由资本冲击、资本逃逸或金融投机引起的,或者,金融危机是精心设计出来
协助掠夺性积累的。

 




将自由化约为企业自由。

规训强力的地方工会,如以提倡个体劳动者的个人自由神圣不可侵犯为名,立法和治安策
略,用以驱散或镇压反对企业力量的集体组织形式。削弱(如英美)、绕过(如瑞典)或暴
力摧毁有组织劳工的势力,是一个必要的前提。去工业化,空间转移

秃鹰资本

阿根廷经济危机\pagescite[][108]{davidneoliber} 


汪晖 《中国“新自由主义”的历史根源》
http://wen.org.cn/modules/article/view.article.php/c8/2560
http://www.aisixiang.com/data/40003.html
http://history.sina.com.cn/his/zl/2014-01-13/175379935.shtml


缩小。过去几十年,全球相对不平等程度稳步下降,相对基尼系数从1975年的0.74下降
到2010年的0.63,其主要推动因素为:经济飞速增长(主要是中国和印度)引起的国家间不平等
程度的下降55。

中国等国家的贡献是全球中产阶级规模迅速扩大的主要原因,在中国,中产阶级家庭(年收入
在11,500–43,000美元)从2000年的500万增加到2015年的2.25亿72。但不同国家对中产阶级的
定义各不相同(虽然都采用将收入和支出与社会平均值进行比较的方法)73。

阶级已不是一个稳定的社会形态



出口导向型,进口替代型。不均衡地理发展


大卫·哈维在《新自由主义简史》一书中,认为中国的新自由主义转向

%%% Local Variables:
%%% mode: latex
%%% TeX-master: "../main"
%%% End: