\chapter{罗斯福新政与凯恩斯主义}

% http://ias.cssn.cn/cbw/mgyjjj/1991/dyq_119068/201506/t20150616_2688427.shtml
% https://www.sohu.com/a/197986540_567589
% http://old.civillaw.com.cn/Article/default.asp?id=27826
% https://www.aisixiang.com/data/110057.html

1929年美国爆发严重经济危机,经济大萧条,并扩散至世界,主张自由放任市场调节完
善性的经济学彻底倒台。1933年当选的美国罗斯福总统发布一系列政策,试图依靠国家宏观
调控能力缓解经济周期中下行期的萧条影响,史称罗斯福新政。

不少人(包括经济学家)认为罗斯福新政主要是受英国经济学家约翰·梅纳德·凯恩斯
影响;也有不少人对此反驳,认为罗斯福受凯恩斯的影响较小。争议综述可参考刘绪
贻\cite{roosevelt}。

笔者认同刘绪贻、张小鲁\cite{bijiao202002}、张世明\cite{JJFX200100010}等人观点,罗
斯福新政与凯恩斯关系并不大,凯恩斯指导罗斯福新政是个“童话”。
单是两人行动时间节点就可以说明不少问题。另外早期制度经济学派代表凡勃伦、康芒
斯对于新政的影响力也远比凯恩斯大得多。

根据布鲁和格兰特《经济思想史》摘抄如下:
\begin{quotation}
  康芒斯的第一本著作《财富的分配》(1893)并没有获得充分认同。批评家认为,这是
  康芒斯为其\textbf{社会主义思想}确立科学基础的一次令人不满尝试。然而,康芒斯并不是
  一个试图改变私人财产和自由企业社会结构的革命家。他认为,\textbf{资本主义的本质可
    以并且应当保持完整无缺,但是,经济秩序的运转规则需要变革,以消除自由放任
    经济体的明显缺陷。}在威斯康星大学,他的观点获得了州长拉·弗利特(La
  Follette)的支持。

  被一些人称作\textbf{威斯康星学派}(Wisconsin school)的经济学方法,主要是在\textbf{康芒斯}的
  影响下在威斯康星大学得到发展。\textbf{这种方法支撑了美国的非正统经济理论,发动了
    改变美国经济结构与功能的改革}。

  威斯康星州政府广泛利用威斯康星大学教职工充当新思想的智囊团、法律的起草者以
  及指定委员会的成员。

  体现在罗斯福新政社会立法中的很多思想来自于威斯康星州,这一点很少有人怀疑。毫
  无疑问,1932 年,很多在麦迪逊接受培养的经济学家和其他人都搬到了华盛顿特区。
\end{quotation}

% 其实凯恩斯也了解过马克思,根据《经济思想史》,凯恩斯在《通论》第一稿中有对马
% 克思关于资本循环的论述和注解。另外
% \begin{quotation}
%   凯恩斯的学生莫里斯·多布还是一位研究生时,曾在凯恩斯的房间里读到一篇论述马克
%   思与剑桥政治经济学俱乐部的论文。多布回忆道,凯恩斯很赞许这篇论文,因为“他
%   年轻时在一定程度上喜欢非正统思想”。
% \end{quotation}

缘何造就出力挽狂澜于美国大萧条的“凯恩斯童话”,笔者不明就里,只敢妄加揣测:
难道是因为凯恩斯作为英国政府高级代表参加1944年的布雷顿森林会议,制定了当代国际金
融体系,并与美国财政部官员哈里·德克斯特·怀特(Harry Dexter White)一起创立
了国际货币基金组织(IMF)和世界银行(World Bank)?


对罗斯福新政和凯恩斯主义的论述有助于理解国家干预政策,有助于构建一条资本主义
发展史链条,以下两节笔者对两者分别论述。


\section{罗斯福新政}

本节大量参考王小鲁《美国大萧条与新政再思考》\cite{bijiao202002}。罗斯福新政主要
还是应用经济,是应对当时大萧条的实用政策,主要包括以下内容。

\begin{enumerate}
\item 保障劳动者的基本权利,制定最低工资标准、实行8小时工作制、禁止使用童工。这
  些劳工保障措施于1935年被最高法院宣布违宪,随国家复兴署的裁撤而终止,直
  到1938年美国国会通过了新的《公平劳动标准法案》后才重新得以实行。

\item 大幅度增加政府的社会保障和救济等福利支出,建立了劳动者的养老保险和失业保险制
度,逐步替代了临时性社会救助的功能。

\item 改善收入分配状况,减少过大贫富差距。如提高个人所得税的累进率,征收了累进的
  遗产税和财产税。据王小鲁:
  \begin{quotation}
    在这一变化中减少的主要是个人财产收入的份额,\textbf{对企业利润份额没有影响,因
    此企业家的积极性也受到了保护}……到1950年没有太大变化,其中财产收入份额进
    一步下降,公司利润份额上升了。
\end{quotation}

\item 规范工农业生产和产品价格,防止过度竞争,促进价格止跌回升。

\item 新政未使用凯恩斯主义要求的扩大政府借债和支出政策,大搞基建。

  据王小鲁,当时已经货币严重超发,罗斯福自然不可能依据凯恩斯的请求继续超发货币。
  \begin{quotation}
    大萧条之前美国已确实经历了长期的货币宽松。但并非如罗斯巴德所说主要发生
    在20世纪20年代,而是持续了更久。更宽松的时期是在20世纪10年代……
    1917-1919年,美国参加一战,政府以\textbf{巨额赤字}支持了军事支出扩张,而该支出的大
    部分都是通过借债和货币发行筹集的。这解释了20世纪10年代货币快速扩张的原因

    (20年代)货币事实上超量供应。而且这是在20世纪10年代货币严重超发基础上的
    继续。\textbf{长期货币扩张最后导致股市崩盘、引发萧条的说法是有充分根据的。}
  \end{quotation}

  据列宁,金融和垄断资本发动的一战使自由资本主义从垄断阶段走向国家垄断资本主义。
  \begin{quotation}
    \textbf{世界托拉斯和银行资本为争夺世界市场的统治权而引起的世界大战}, 使物质财富遭
    到巨大破坏, 使生产力消耗殆尽, 使军事工业蓬勃发展, 以致绝对必需的、 最低限
    度的消费品和生产资料的生产无法进行。

    战前在最发达的先进国家中无疑已经具备的社会主义革命的客观前提(笔者注,这
    里列宁还是过于乐观了),由于战争而更加成熟,并且继续在异常迅速地成熟。中
    小经济更加迅速地遭到排挤和破产。资本的积聚和国际化正在大大地加强。\textbf{垄断
      资本主义正在向国家垄断资本主义转变},由于情势所迫,许多国家实行生产和分
    配的社会调节,其中有些国家进而采取普遍劳动义务制。\pagescite[][441]{lenin29}
  \end{quotation}

  据克里斯·哈曼:
  \begin{quotation}
    正如希法亭、布哈林和列宁所指出的,当“自由市场资本主义”开始让位于“垄断
    资本主义”及其产物帝国主义时,\textbf{经济“自由主义” 已在实践中被替代了。国家
      干预被认为是为资本主义生产提供基础设施所必需的}(铁路在德国长期以来就是
    国有化的,在英国,保守党政府将电网和航空公司国有化)。后来的战时国民经济
    组织——最初出现在德国和日本,后来又出现在英国和美国——表明国家干预可以
    为收益率和积累的复苏提供基础。
  \end{quotation}

  据王小鲁,罗斯福也没有大搞基建:
  \begin{quotation}
    除了田纳西工程,新政期间政府还投资了3万多个公共工程项目,小型项目居
    多……大多采取以工代赈的方式。这些项目对改善基础设施条件、减少失业、减贫、
    环境治理和带动经济增长发挥了一定作用。但政府投资并不是新政的核心。
  \end{quotation}

\end{enumerate}

\section{凯恩斯主义}

虽然大萧条救市和凯恩斯关系不大,但年凯恩斯于1936年发表的《就业、利息和货币通
论》创建了现代宏观经济学。

凯恩斯经济学\cite{jahan2014keynesian}的立足点是,\textbf{总需求}(以家庭、企业和政府
的\textbf{总支出}来衡量,包括消费、投资、政府购买和净出口,凯恩斯没有对支出进行细
分。)是经济中最重要的驱动力。自由市场没有导致充分就业的自我平衡机制---“长期
来看,我们都死了”。失业和经济危机的原因是\textbf{有效需求不足}。

经济周期中的下行期,随着总体需求的持续下降,消费者信心不足减少\textbf{消费支出},进
而企业减少\textbf{投资支出}。\textbf{政府支出的积极干预}便有助于缓和经济周期的繁荣上升和
萧条下行期幅度,并且是必要的。

具体举措是政府针对经济周期情况\textbf{适时干预}:在需求测(总支出)下降时,对\textbf{劳动
  密集型基础设施项目}进行\textbf{赤字支出和降低利率},增加社会福利,以刺激就业和稳
定工资,修复消费和生产循环。在需求测增长充足时,提高税收以冷却经济并防止通货
膨胀。


二战后世界各国政府常常应用凯恩斯主义,加强对经济生活的全面干预,直
至20世纪60年代末、70年代初,西方世界出现经济停滞或衰退、企业倒闭、工人失业和
通货膨胀并存的现象——“\textbf{滞胀}”,凯恩斯主义受到其他学派巨大挑战,面临危机。
此后有一些学者对凯恩斯主义进行不同改造。

一个有意思却未必正确的观点:凯恩斯主义是劫小康、济大贫、修缮大权贵。

它是应用政策经济学,而非学院理论经济学。它是\textbf{短期}需求管理政策,漠视长期经济周
期,\textbf{坚持长期使用将成为越来越毒的毒药},在未来市场出清、结算总账时带来更加巨大
的灾难,包括但不限于“滞胀”;如同任何经济学说一样,它不能消除资本主义经济固
有的基本矛盾。

\section{从垄断资本主义到国家垄断资本主义}

在马恩看来,国家是阶级矛盾不可调和的产物,是统治阶级的统治工具,用于“缓和经
济利益互相冲突的阶级,\textbf{不致在无谓的斗争中把自己和社会消灭}”。资本主义以降,
市民社会是国家的基础,市民社会的阶级张力与矛盾制约和决定了国家,而不是相反。
国家不是社会的主宰物,恰恰相反,国家是社会发展的\textbf{产物}。

关于市民社会中的统治阶级,可能是政府内阁;更可能是“统治阶级不统治”,藏在幕
后的大佬、甚至是国外大佬。经济学家们五花八门、各圆其说、常彼此矛盾的妙计锦囊,
只被用作统治阶级的政策支持武器,哪个趁手拿哪个,加以改造,用完即弃。分析现实
经济政策,切忌用XX经济思想标签这些武器来套取,更要像福尔摩斯一样关注背后的获
利群体。

笔者认可刘绪贻的论断,是社会的变革诞生出这时各种观点
\begin{quotation}
  \textbf{罗斯福新政与凯恩斯主义同是垄断资本主义向国家垄断资本主义转变过程的典型产物。}
\end{quotation}


自由放任的垄断资本主义自一战碰壁后,逐渐向国家垄断资本主义转变,以求自保。它,
资本主义,进化了!



% 我国铸币税个别参考文献
% 1995年禁止中央财政直接向人民币透支:

% 1995年《中国人民银行法》公布以前,财政部可以向人民银行借款和透支,用于弥补中
% 央财政赤字和解决专项支出。《中国人民银行法》关于“中国人民银行不得对政府财政
% 透支,不得直接认购、包销国债和其他政府债券”的规定出台后,中央财政不能再从人
% 民银行借款。同时《预算法》规定政府财政赤字只能通过发行国债来弥补。自2003年初
% 以来,为对冲因外汇储备增长过快而导致基础货币投放过多的影响,人民银行在加大公
% 开市场回购操作力度的同时,大量发行央行票据,回笼货币,以保证货币供应适度增长。
% 随着人民银行持有的债券大量到期,其公开市场操作工具不足的问题日益突出,实施货
% 币政策的操作压力不断加大。截至2003年7月末,广义货币和狭义货币供应量同比分别增
% 长20.7\%和20\%,大大高于年初确定的调控指标,货币超经济供应现象逐渐显现。为解
% 决人民银行公开市场操作工具不足问题,确保现阶段货币供应与经济发展相适应,有必
% 要将中央财政这部分历史借款转换为标准的、可交易的国债。2006年,吴汉洪和崔
% 永[4]对我国铸币税的研究进行了小部份综述。

% 参考文献
% [1] 张怀清. 论中央银行铸币税和通货膨胀税的关系[J]. 南方金融, 2007(10): 28–30.
% [2] 张健华, 张怀清. 人民银行铸币税的测算和运用:1986–2008[J]. 经济研究, 2009(07): 79–90.
% [3] 张怀清. 商业银行铸币税研究[J]. 金融发展研究, 2008(04): 12–15.
% [4] 吴汉洪, 崔永. 中国的铸币税与通货膨胀:1952–2004[J]. 经济研究, 2006(9): 27–38.

% 在时任总理温家宝(1953-)的领导下,中国党/国家对2008年全球金融危机的反应实际
% 上是提振“总需求”,政府采取措施通过“量化宽松”计划刺激经济,其中包括对全国
% 基础设施的广泛投资,其中包括4万亿元人民币(5860亿美元)的刺激计划。 被描述
% 为“凯恩斯博士的中国病人”(见《经济学人》2008:1)。然而,问题在于储蓄和消费
% 之间的“结构性失衡”是否可以得到解决(Fang and Gang 2009:149)。“乘数效
% 应”可能起效缓慢,提振内需可能说起来容易做起来难。中国的收入不平等程度很高,
% 官方的基尼系数为0.47,但实际上可能要高得多,可能超过0.60(见Warner
% 2013:177)

% ,“供给侧改革”成为引领经济政策的新动向,它主张通过促进分工深化来提升全要素
% 生产率进而提高潜在产出水平。显然,“供给侧管理”的理论基础与其说是新古典经济
% 学(包括凯恩斯主义经济学),不如说是古典经济学和马克思主义经济学。
