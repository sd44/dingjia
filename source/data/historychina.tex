\chapter{计划经济时期 1949--1978}

\section{社会主义过渡时期 1949--1956}
\label{sec:hongguodu}

1949年10月新中国成立到1956年社会主义改造基本完成,是新民主主义社会向社会主义
社会过渡时期。

经济恢复3年,1952年底,中共中央提出了党在过渡时期的总路线,明确规定:党在这个过
渡时期的总路线和总任务,是要在一个相当长的时期内,\textbf{逐步实现国家的社会主义工业
  化,并逐步实现国家对农业、对手工业和对资本主义工商业的社会主义改造。}”

\begin{enumerate}

\item 1950年2月13日至15日,第一次全国财经会议,中央统管全国财政收支、国家国营贸
  易和物资调度,稳定金融物价,具体实施中实行了严格的公粮入库制度。薄一波认为
  这是新中国财经战线三大战役中的第一次大战役。

\item 1950年3月16日,第一次全国统战会议,人民民主统一战线。针对党内左倾倾向,毛
  泽东提出现阶段是工人阶级、农民阶级、小资产阶级和民族资产阶级四个阶级的联盟,
  应求得四个阶级的共同解放。

\item 1950年夏,淮河发大水,灾情严重。农田受灾面积达4687万亩,灾民约1300多万人,
  倒塌房屋89万余间。以淮河为起点开展了对全国大江大河的治理。

  史实部分参考樊宪雷《20世纪50年代至60年代初我国兴修水利的探索实践及其基本经验》。
  \begin{quotation}
    在1950年至1952年3年中,国家用于以大江大河治理为主的水利建设的投资共约7亿
    元,占预算内基本建设投资的10%,对全国2.4万多公里的重要堤防进行了修
    整,3年内直接参加水利工程的人员有2000万人,完成土方17亿多。如此大规模的以
    大江大河治理为主的水利建设,改变了新中国成立前河道失修的状态,基本上解除
    了我国人民几千年所受洪水灾害的严重威胁,保证了大部地区农业生产和人民生活
    的安全。
  \end{quotation}

  防汛为主的水利工程之后延伸至以增产增收为目的的农田灌溉水利建设。
  \begin{quotation}
    水利部提出,“1953年以及今后的农田水利工作的方向”,“应该着重开展群众性
    的各种小型水利,整顿现有水利设施,加强灌溉管理,发掘潜在力量,以扩大灌溉
    面积”。
  \end{quotation}
  从治淮到全国大兴农田水利,最后犯了大跃进错误。

  新中国经济相当薄弱,城市经济自身都面临劳动力过剩问题,无法吸纳小农经济下农
  村严重过剩劳动力。大兴农田水利是\textbf{巨型以工代赈}工程。一方面,从农村中析出大
  量剩余劳动力参与基础工程建设,对这部分人发放劳务报酬,农村农民生活压力得以
  减轻。另一方面,以过剩劳动力替代基础建设中耗资巨大的固定资本投资。笔者认为,
  这是一次后发国家中国尝试\textbf{弯道超车的巨型社会制度实验}。如果可行,可将这
  种\textbf{以工代赈的群众性运动}推广至其他全国性大规模基建,并大力发展经济。可惜尝
  试的结果可能是失败的。

  大兴水利至今已有70余年,不管中国的意识形态有何变化,大规模固定资本投资始终
  是中国求得经济发展的主要依赖路径之一,1992年之后,房地产也列入了其中。

  笔者还认为读者还需关注一点,毛泽东在他整个执政生涯里虽犯了一些错误,但始终
  没有开展直接的、血淋淋的创造大量失地农民,将其挤入城市的资本原始积累。


\item 1952年3月,“五反”运动中,全国范围内形成了一个反对资本家“五毒”行为的斗争
  高潮。1952年10月,五反运动结束,中共中央转批的廖鲁言报告中指出
  \begin{quotation}
    根据华北、东北、华东、西北、中南5大区67个城市和西南全区的统计,参加“五
    反”运动的工商户总计999707户,受到刑事处分的只有1509人,仅占总数的0.15\%,
    其中判处死刑和死缓的仅19人,占判刑总数的1.26\%。
  \end{quotation}

\item 中国计划经济时代医疗制度是贯穿计划经济时期时期的,因此做下全面陈述。以下内
  容根据王晓玲《中国医疗市场政府管制的历史演进及制度反思》\cite{yiliaoshi}:

  1951年起开始对工商业职工及其供养的直系家属实行源自企业纯收入的\textbf{劳保医
    疗}。1952年底,全国90\%以上的地区建立了县级卫生机构,县级卫生院达到2123所。
  政府在1952年,1960年和1972年三次大幅降低医疗价格收费标准,低于成本部分进行
  财政补贴。1952年,开始实行针对国家工作人员的\textbf{公费医疗},后于1953年扩大至大
  学和专科院校。1955年在山西省高平县率先实行了医疗合作社和生产合作社相结合
  的\textbf{集体医疗}保健制度,标志着我国农村正式出现具有保险性质的\textbf{合作医疗}制
  度。1968年,毛泽东批示了湖北省长阳县乐园人民公社举办合作医疗的经验,\textbf{合作
    医疗}制度在全国蓬勃发展起来。计划经济时期禁止私人资本进入医院。

  毛泽东以领袖强力推动了农村医疗水平。主要是医疗巡回队下农村和培养保健员(赤
  脚医生)。据时任卫生部长钱信忠回忆:
  \begin{quotation}
    在1965年1月 三届全国人大第一次会议期间,毛主席又当面指示过,并批评“\textbf{卫生
    部想不想面向工农兵}”,“\textbf{为什么把医学教育年限搞的那么长!}”……我于当月20日,
    向毛主席呈上报告,决定在15所 医学院中开办三年制的班,为农村培养医生,毛主
    席于第二天就作了“同意照办”的批示。

    1965年2月6日中共中央下发了对卫生部关于组织高级医务人员下农村和为农村培养
    医生报告的批示,至4月初,全国各地就有1500个医疗队,18600名医务人员下到农
    村,卫生部还派了两名司局长分赴四川、广西等地检查贯彻执行情况,全国上下掀
    起了一个\textbf{下农村巡回医疗}和\textbf{为农村培养卫生人员}的高潮,其声势之大,是前
    所未有的。
  \end{quotation}

  医疗全覆盖至农村,这在中国历史上史无前例,乃至于后无来者。1980年,在世界卫
  生组织和世界银行共同发布的一份考察报告中称,“中国农村实行的合作医疗制度
  是\textbf{发展中国家群体解决卫生保障的唯一范例}”。80年代初,由于农村联产承包制的
  推行,农业最小单位由农村合作社转为家庭,“赤脚医生”和农村合作医疗退出历
  史。

\item 第一个五年计划(1953-1957)出台,首要重点发展重工业。
  \begin{quotation}
    “一五”计划的主要指标是:五年内经济和文化教育建设总支出为766.4亿元……其
    中,基本建设的投资总额为427.4亿元,占55.8\%,其余44.2\%的资金计339亿元,
    用于基础建设所要的资源勘探、工程设计和器材储备等;工业交通运输和邮电业的
    设备大修、技术组织措施、新产品试制等;各经济部门的流通资金;经济和文化事
    业,培养专业干部等。

    基本建设投资427.4亿元的分配是:工业部门为248.5亿元,占58.2\%;农、林、水
    部门为32.6亿元,占7.6\%;运输邮电业为82.2亿元,占19.2\%;文教卫生部门
    为30.8亿元,占7.2\%。
  \end{quotation}

\item 1953年2月15日,中央正式决议,重点发展以土地入股为特点的农业生产合作社,左
  倾右倾均要不得,另外推广国营农场。1953年底到1955年春这一年半的时间里,合作
  社数量就由1.4万个发展到67万个。1956年1月,中央发布《1956--1957年全国农业发
  展纲要(草案)》,强调完成初级合作社到高级合作社的升级。1956年底,全国建
  成75.6万个农业生产合作社,高级社农户占农户总数的88.8\%。这标志着中国农村在
  生产资料所有制方面的社会主义改造基本完成,实现了对农村史无前例的全面管控。

\item 1953年10月,国家对粮、油、棉统购统销。1955年3月3日,国务院发出《关于迅速布
  置粮食购销工作,安定农民生产情绪的紧急指示》,决定实行粮食“三定”(定产、
  定购、定销)。薄一波写到:
  \begin{quotation}
    继稳定物价、统一全国财经工作之后,被称为新中国财经战线上的第二次大战役的,
    就是1953年开始的对粮食等主要农产品实行\textbf{统购统销}。到1985年改行\textbf{粮棉合同
      定购}制度为止,这个在特定条件下开始实行的农产品统购制度,持续时间长
    达32年之久。而统销制度的一些基本内容现在还在持续进
    行。\cite[255]{boyibo}
  \end{quotation}

  关于统购统销的研究存在多方论述,田锡全论文《粮食统购统销制度研究的回
  顾与思考》有对各方观点的理论综述。

  统购统销出台的背景如薄所说:1952年粮食丰收,但粮食收支赤字较大;经济作物比
  粮食作物利润高;私营粮商的行价比政府征购牌价高20\%--30\%。笔者想,显而易见,
  征购不畅最主要的原因归根到底还是征购价格低。

  1953年10月2日晚,陈云在政治局扩大会议上作报告,其中最重要一点是\textbf{粮食征购不
    足将导致物价整体上涨、工人工资上涨、继而使工业生产、预算、建设计划均受损。
    且人人囤积,形势会日益紧张严峻。}笔者认为陈云的分析是正确的,并且符合马克
  思、前期苏联的认识。

  陈云在会上直言采用这种方式的代价是可能会有小部分农民闹事。可见《陈云文集》
  第二卷1953年10月10日《实行粮食统购统销》,报告中也陈述了他预想的另七种方法
  的优劣,可见陈原文,薄一波也赞美陈云这份报告的有力。
  \begin{quotation}
    我这个人不属于"激烈派",总是希望抵抗少一点。我现在是挑着一担"炸药",\textbf{前
      面是“黑色炸药”,后面是“黄色炸药”。}如果搞不到粮食,整个市场就要波动;如
    果采取征购的办法,农民又可能反对。两个中间要选择一个,都是危险家伙。现在
    的问题是要确实把粮食买到,如果办法不可行,落空了,我可以肯定地讲,粮食市
    场一定要混乱。这可不是开玩笑的事情。
  \end{quotation}

  任何国家政策的制定首先要立足于现实状况,而现实状况有时无法给出正确答案。关
  于农业,新中国与二十世纪初俄国面临同样问题,我们都没有大规模资本原始积累形
  成农业工业化(“美国式道路”)这一条件。陈云所述、薄一波评价“两个炸弹中的
  选择”便是为了发展工业化而苦力维持小农经济的现实实践。这真的难以说是国家的
  错误,而更像是整个人类世界的局限和绝望,从全球到各国家到每个个人,有时候都
  会面对这种可怕可悲选择的境地……

  也如薄一波所说,虽然党内外批评和反对苏联利用剪刀差的一些具体做法,但仍
  然“\textbf{在实际上无法同剪刀差真正决裂}”。\cite[281]{boyibo}黄宗智认为“三农政策
  不仅把小农家庭农场经济纳入国家计划,实际上还强有力地把农民推向集体化的道
  路。”\cite[175]{sanjiaozhou} ,国家依靠农村合作社等集体模式获得了对农村前所未
  有的强力管理,另外在国家对合作社提供优惠政策及农民个人无法承担征购巨大压力
  的情况下,使农民主动融入合作社。

\item 1953年6月,中共中央起草《关于利用、限制、改造资本主义工商业的意见
  》。1954年9月2日,政务院颁布《公私合营工业企业暂行条例》。1955年10月全国工
  商联合会议通过了《告全国工商界书》。1956年初,全国范围出现社会主义改造高潮,
  资本主义工商业实现了全行业公私合营。这也是薄一波所说新中国财经战线三大战役
  中的第三次大战役。\footnote{1966年9月,在原定的向资本家支付定息的年限已满后,政府
    决定不再支付定息,公私合营的企业由此变成了完全的全民所有制企业。}

  % 1966年9月,当局按照原定的向资本家支付定息的年限已满[可疑 –讨论],决定不再支付定% 息,公私合营的企业就变成了完全社会主义性质的全民所有制企业。有报道称,按现时的% 概念,即一夜之间股民股票归公,房奴房产归公。未经任何合法手续,私营股份被“没% 收”为国有,公私合营企业全部变成了国营企业。[2]

  % 1979年1月,中央出台《党对民族资产阶级政策问题》规定:“公私合营时股票股息发 放% 到1966年9月结束,现有资产阶级工商业者要求领取在此前应领未领股息是可以 的”。但% 国家财政部又在当年下发文件,确定不再清退私股股金。[3]

  % 1983年2月,中共中央统一战线工作部和商业部联合发文规定:“国家已按年息五厘发 给% 定息,发至1966年3季度,公私合营资产(包括核定投资房屋)已属国家所有,不应退还% 本 人”。此后全国发生多例私股定息或股权的诉讼,皆因上述政策文件的原因而败诉。有% 学者对这一“不应退还”政策提出了质疑。既然向私股股东支付“定息”,就说明 “公% 家”承认私股股东对于合营财产的所有权。自1966年9月之后不再支付定息,并不说明一夜% 之间这些财产收归国有。[2]

\item 1956年是“一五”计划第四年,在这一年全国工农业总产值已经完成整个“一五”计
  划指标。工业方面总产值平均年增长19.2\%,超过目标14.7\%。在1957年“一五计
  划”完成时,农业方面:
  \begin{quotation}
    农业增长率为4.5\%,虽然完成了计划指标(4.3\%) ,但是与工业高速增长相比明显
    滞后。这也造成了农村购买力增长缓慢,原计划农村购买力增长100\%,实际上只增
    长20\%,大大低于计划目标。\cite{shiyiwu}
  \end{quotation}
\end{enumerate}

1956年的国民收入中,有92.9\%来自于公有制经济(国营、合作社、公私合营),这标
志着中国已经从新民主主义社会过渡到社会主义社会。

\section{红色社会主义社会时期 1957--1978年}


笔者将1957--1978年称为红色社会主义社会时期。

\subsection{财政体制改进}

关于这一时期的具体历史事件可看胡安全《中共八届三中全会与经济体制改
进》\cite{DZSY201701010},中央和地方权利分配变化可看周飞舟和谭飞智所著《当代中国
中央地方关系》\cite{yangdi}。

关于中央集权和给地方放权、分权的内在张力:
\begin{quotation}
  中国政体的突出特点是\textbf{以中央政府为中心的一统体制},即中央政府对其广大国土及
  居住其上的民众、各个领域和方面有着最高和最终的决定权。在这一前提下,中国国
  家治理的一个深刻矛盾是\textbf{一统体制与有效治理之间的矛盾},集中表现在\textbf{中央管辖
    权与地方治理权间的紧张和不兼容}:前者趋于权力、资源向上集中,从而削弱了地
  方政府解决实际问题的能力,即这一体制的有效治理能力;而后者又常常表现为各行
  其是、偏离失控,对一统体制的中央核心产生威胁。在一统体制中,这一矛盾无法得
  到根本解决,只能在动态中寻找某种暂时的平衡点。国家治理逻辑在很大程度上是针
  对这一矛盾而演化发展起来的,体现在一整套制度设施和应对机制之
  上。\cite{zhililuoji}

  地方权力过大,容易造成中央的政令不通,在极端的情况下造成“诸侯政治”甚至地
  方割据和国家分裂;而中央实行过度的集权又容易使得整个政治和经济体制陷于僵化,
  难以对地方出现的问题进行灵活和适当的反应和处理,地方的小问题往往容易蔓延和
  发展成为全国性的大问题,从而也威胁到政权和国家的稳定。所谓“\textbf{一放就乱、一
    收就死}”就是指的这种状况。\cite{yangdi}
\end{quotation}

根据万其刚《当代中国财税法制的历史及其发展》:
\begin{quotation}
  政务院1950年3月颁布的《关于统一国家财政经济工作的决定》、《关于统一管
  理1950年度财政收支的决定》,核心内容就是把财政收支权集中于中央。随着整个经
  济形势的好转,逐步实行“划分收支、分级管理体制”。1951年3月政务院颁布《关
  于1951年度财政收支系统划分的决定》,把国家财政分为中央、大行政区和省(市)
  三级。初步建立起“\textbf{统收统支、总额分成}”的集权型财政体制。
\end{quotation}


1957年11月,国务院颁发《关于改进财政管理体制的规定》, 决定从1958年起,实
行“以收定支,三年不变”(1958年4月改为“\textbf{五年不变}”)的体制。中央政府开始
全面下放权限,建立起了“\textbf{块块为主、条块结合}”的计划体制,规定各省市自治区可
以对本地区的工农业生产指标进行调整,可以安排本地区的建设投资和人力、财力、物
力以及公共事业项目。这是新中国历史上的\textbf{第一次大规模放权}的变革。各省开始追求
建立独立完整工业体系,地方可以自主基建投资等。企业管理方面,央企下放88\%。对
企业管理的指标性计划从12个减少为4个。


官僚主义抬头,地方第一次开始\textbf{指标锦标赛},1958--1960年,各地纷纷上马各大中型
项目,重复基建,重工业增长了2.33倍,轻工业增长了47\%,农业下降了22.7\%。大跃
进开始,“浮夸风”等五风盛行,造成恶劣后果。

因五风盛行、大跃进、地方大干快上等,
\begin{quotation}
  “以收定支,五年不变”仅执行不到一年就难以为继。1958年9月,国务院通过《关于
  进一步改进财政管理体制和改进银行信贷管理体制的几项规定(草案修改稿)》,决
  定从1959年起,实行“收支下放,计划包干,地区调剂,总额分成,一年一变”的财
  政管理体制,简称“\textbf{总额分成,一年一变}”。

  对地方的机动财力进行了适当压缩,将特定收入之外的其余地方财政收入捆在一起按
  照核定的比例由中央和地方实行总额分成,把基本建设支出由中央专案拨款改为列入
  地方预算支出,参与收入分成,加上预算指标一年一定,中央适当地集中了财力,属
  于\textbf{“高度集中”体制下的“部分分权”模式}。但是,由于“大跃进”对经济工作的
  影响、中苏关系逐渐交恶及严重的自然灾害,1959年地方上仍然存在财权相对下放较
  多、财力分散、财政管理松软等问题。\cite{DZSY202204015}
\end{quotation}

从1961年到1966年,中央在行政和经济管理上\textbf{重新“收权”},对于恢复国民经济的作
用明显。

“文化大革命”开始后,中央提出以“块块”为主的管理国民经济的基本思路,是“大
跃进”时期之后的\textbf{第二次大规模“放权”}。大量精简、合并中央机构,下放85\%的部属
企业到地方。扩大地方基建投资的管理权;中央统配和部管物资由1966年的579种减
为1972年的217种,同时下放企业的物资分配和供应权限给地方。

1973年1月2日,国家计委向国务院提交《关于增加设备进口、扩大经济交流的请示报
告》,正式提出“四三方案”。建议利用西方资本主义危机红利,引进43亿美元的西方
成套设备来调整中国工业结构,布局相对集中的沿海主要工业城市,以调整国内“缺重
少轻”的工业结构。由此开始了新中国的第二轮仍然维持自主性的对外开放。
\begin{quotation}
  统计表明了周恩来总理提出的“四三方案”的实施情况:中国从六十年代中后期到七
  十年代用延期付款和利用中国银行外汇存款等方式,大规模引进的机械设备价值高
  达42.4亿美元。同期,中国马上就出现了与20世纪50年代初期第一次面向苏东“对外
  开放”的“一五”计划完成后类似的问题:国家进行扩大再生产的投资能力严重不足。
  特别是1974年以后,财政赤字连续突破100亿元,而当时的财政总规模还不到800亿
  元。\cite{wenbaci}
\end{quotation}
1974年,财政困难,\textbf{中央收权},回归“\textbf{总额分成,一年一变}”。第三次知青上山下乡。

\subsection{赫鲁晓夫的秘密报告}

对于新中国这一历史阶段的理解,必须加入对赫鲁晓夫秘密报告影响的认识。

1956年2月25日,赫鲁晓夫在苏共第二十次代表大会上作了《关于个人崇拜及其后果》
的“秘密报告”,大力批判了斯大林,斯大林模式中一些残酷问题暴露出来,共产主义
世界受到极大冲击,波兰发生波兹南事件,匈牙利发生十月事件,拥有绝大部份法国知
识分子的法共有半数以上党员退党,其中一些理论家后来成为后现代主义的中坚力量。

毛泽东对斯大林的评价是功过三七分,功大于过。\footnote{此类文献较多,可简单参
  考\url{https://www.wxyjs.org.cn/mzdyj/201802/t20180208_236685.htm}。}笔者认
为,毛泽东与斯大林之间存在分歧和矛盾,但这种分歧和矛盾是建立在两人同一个社会
理想的框架之下,并且在当时“\textbf{苏联的今天就是我们的明天}”语境之下,斯大林的被
批判使毛泽东感觉到了沉重的危机,阶级斗争的这根弦更加紧绷起来,整治党内外,达
成建设社会主义、迈向共产主义的统一共识成为毛泽东的首要目标。但是也要考虑在历
史语境下,当时世界各国各派领导人的神经都高度敏感和紧张。

\subsection{整风到反右派}

据中共中央文献研究室编写的《毛泽东传(1949-1976)》,1956年下半年开始,
\begin{quotation}
  在半年内,全国各地,大大小小,大约有一万多工人罢工,一万多学生罢课。
  从1956年10月起,广东、河南、安徽、浙江、江西、山西、河北、辽宁等省,还发生
  了部分农民要求退社的情况。
\end{quotation}

1957年4月27日,中央起草了《中央关于整风运动的指示(初稿)》,决定在全党的范围
内“重新进行一次普遍的、深入的反官僚主义、反宗派主义、反主观主义的整风运
动”。5月4日,毛泽东起草《关于继续组织党外人士对党政所犯错误缺点展开批评的指
示》,邓小平、薄一波对于此时毛泽东的评价是正确的——毛泽东此时确实是将党内官
僚、宗派、主观问题当作社会波动的主因,想依靠开放批评来进行党内整治。

不久党外人士的批评中开始出现很小部分反共反社会主义思想,5月15日,毛写了《事情
正在起变化》,认为一些右派正在猖狂进攻。6月8日,毛为中央起草了《关于组织力量
准备反击右派分子进攻的指示》,直言中国如果不反右,将可能面临匈牙利十月事
件……到1958年8月为止,这一年半的时间里,本是党外批评人士采用的“大鸣、大
放”形式被借用过来,成为“\textbf{大鸣、大放、大争辩、大字报}”,发动反右派斗争。据
官方公布,这一运动中全国实际划归右派分子55万多人,其中绝大多数人被错划。

薄一波认为,1956年9月党的第八次全国代表大会正确提出的“\textbf{国内的主要矛盾,已经
  是人民对于建立先进的工业国的要求同落后的农业国的现实之间的矛盾,已经是人民
  对于经济文化迅速发展的需要同当前经济文化不能满足人民需要的状况之间的矛
  盾}”被1957年9月到10月9日召开的第八届中央委员会第三次会议上“\textbf{无产阶级和资
  产阶级的矛盾,社会主义道路和资本主义道路的矛盾,毫无疑问,这是我国社会的主
  要矛盾。}”错误取代。自社会主义改造完成后,更应依靠法制解决社会矛盾,而非群
众性阶级斗争。自1958年到1978年十一届三中全会的20年,党和国家错误实行了群众
性“以阶级斗争为纲”的方针。\cite[620-632]{boyibo}

\subsection{浮夸风、命令风、共产风、生产瞎指挥风、干部特权风}

在经历了一年半左右的对“反冒进”的批评和提出大跃进之后,1958年5月5日至23日召
开的中共八大二次会议上提出“鼓足干劲、力争上游、多快好省地建设社会主义”的
\textbf{总路线},在钢铁等重要工业品的产量上赶超英美成为一个目标,笔者认为这
标志着在中央政府层面上正式开展大跃进运动。虚报目标并次次层层地加码,“浮夸
风”盛行。薄一波认为农业上的“浮夸风”形成的盲目乐观又导致了国家转向工业,形
成了工业上的“浮夸风”,导致举国上下“以钢为纲”的大炼钢铁,农村砸锅进行土法
小高炉炼钢,钢企不重安全和质量地快速出钢。

\begin{quotation}
  1958年8月中共中央通过了国家计委重新拟定的《关于第二个五年计划的意
  见》(1958--1962年的国家计划,这也是真正开始实施的计划),提出了天方夜谭的
  高指标,冒进指数\footnote{原文注释:冒进指数是指本计划期的指标值相当于上一个计划
    期实际值的百分比,该比值越高,说明制定的计划越冒进,反之,越保
    守。}达到354.6\%,基本建设投资规模是“一五”时期的7.8倍,工业总产值增长
  速度是4.9倍,农业总产值增长速度是6.7倍。\cite{shiyiwu}

  由于“大跃进”浮夸风的影响,1959年全国定产指标为5000亿斤原粮,
  而1959、1960、1961年的实产量分别只为3400亿斤、2870亿斤、2950亿斤。三年平
  均实产比1957年减少827.6亿斤,但平均年征粮食却比1957年增加了95.8亿斤,相当
  多的地方购了农民的“过头粮”。\footnote{过头粮:中国在粮食争购工作中,对农业生
    产单位征购超过其实际负担能力的那部分粮食。1991年版胡绳所著《中国共产党
    的七十年》数据与薄相同。}\cite[278]{boyibo}
\end{quotation}


浮夸风的盛行,使中央盲目乐观,认为“共产主义在我国的实现已经不是什么遥远将来
的事情了”,又刮起了“共产风”。浮夸风导致的现实具体因素方面,薄一波书中认为
是大兴农田水利要求有\textbf{更为行之有效的基层组织结构管理},对基层组织结构规模提出
了较高要求。

1958年3月20日成都会议通过,4月8日政治局会议批准的《中共中央关于把小型的农业合
作社适当地合并为大社的意见》佐证了这点。\cite[728-730]{boyibo}
1958年8月17日到30日,中央政治局扩大会议通过《关于在农村建立人民公社问题的决
议》。9月1日《红旗》杂志第七期中的嵖岈山卫星农业社模式推广至全国。生产生活资
料公有,公社命令式调拨农民人力、物力、财力,吃饭不要钱\footnote{薄一波写道,国家统计
  局1960年1月报告,参加公共食堂吃饭的约4亿人,占人民公社总人数的72.6\%。},个
别公社收缴了农户土地、房屋、资金、粮食……邓书杰写到
\begin{quotation}
  到1958年10月底,全国农村就实现了人民公社化。全国原有的74万多个农业生产合
  作社,此时改组成了2.6万多个人民公社,加入公社的农户达1.2亿,占总农户数
  的99\%以上。
\end{quotation}

农村人民公社兴起的同时,城市人民公社也开始兴起。据邓书杰:
\begin{quotation}
  到1960年7月底,在全国190个大中城市中,已经建立了1064个人民公社,基本上实
  现了城市人民公社化。
\end{quotation}

为实现“15年赶超英国”的强国梦,新中国实现了对农村前所未有的集权管理——政社合一的“人民公
社”。

人民公社所出现问题的宏观根源究竟是什么?其实早在1958年11月2日至10日召开的第
一次郑州会议上已经由毛泽东本人提出答案,并在1958年11月28日至12月10日的中共
八届六中全会得到深化。八届六中全会作出《关于人民公社若干问题的决议》,提出
了问题的根源。笔者认为时至今日这仍然接近标准答案,并且标志着中央正式开始“纠左”。
\begin{quotation}
  人民公社目前基本上仍然是\textbf{集体所有制}的经济组织。农业生产合作社变为人民公社,\textbf{不
  等于由集体所有制变为全民所有制,更不等于由社会主义变为共产主义。生产关系一
  定要适合生产力的性质。}无论由社会主义的集体所有制向社会主义的全民所有制过渡,
  还是由社会主义向共产主义过渡,都\textbf{必须以一定程度的生产力的发展为基础}。
\end{quotation}

笔者在苏联一章阐述过马克思的科学社会主义的目的论和空想成分,马克思的历史唯物
主义足以对他的共产主义作出批判,在此不再赘述。除此之外,这一时期的冒进无视了
历史唯物主义中所要求的\textbf{生产力和生产关系等物质基础};混淆了\textbf{社会主义初级阶
  段}和\textbf{共产主义社会};混淆了\textbf{集体所有制}和\textbf{全民所有制}。

人民公社的实质仍是集体所有制,集体所有制的权力主体不是全民,而是集体组织。这
一环境之下政社合一的集体组织是公社,以及可直接命令调拨公社财产的县级以上国家机构,
官僚具有的强大权力和自身欲望是人民公社问题一个需要探讨的重要问题。

另一方面,马克思的一些文章中已经作出一些判断,特别是在《哥达纲领批判》这一
科学社会主义重要纲领文件中,沉重批判了拉萨尔为首的德国社会主义工人党。《纲
领中》明确提出共产主义第一阶段(即社会主义阶段)仍是按劳分配和“带有经济、
道德、精神方面的资本主义痕迹”,“仍带有资产阶级权利”,“权利决不能超出社
会的经济结构以及由经济结构制约的社会的文化发
展。”。\cite[435]{maenwen3}苏联列宁、斯大林均犯过冒进错误,很遗憾当
时中国未能从苏联历史经验中吸取足够教训。

更为遗憾的是党中央未能发展和坚持《决议》的正确看法。1959年7月2日至8月1日,
中共中央政治局在江西庐山召开扩大会议。会议初期朱德、周小舟、周慧等对公共食
堂提出一些反对意见。7月14日,因彭德怀写信给毛泽东,更为激进提出左倾错
误。7月21日,张闻天在会议上讲话肯定和支持彭德怀观点。7月23日,毛泽东发表长
篇讲话,对左右两派“各打五十大板”。23日当晚,支持“纠左”的彭德怀、张闻天、
黄克诚、周小舟、周惠等人于黄克诚住处发牢骚,张闻天说出“像斯大林晚年”。后
被互相检举揭发,自此本次大会正式从“纠左”转向“反右”,左倾错误继续扩大
化。

另外笔者想提一点,据《重整河山1950-1959》与《动荡年代1960-1969》一书,
\begin{quotation}
  (农村人民公社)大办公共食堂、幼儿园、托儿所、幸福院等公共事业。截
  至1958年末,全国农村共建立公共食堂340多万个,托儿组织340多万个,幸福
  院15万所。

  (城市人民公社)在这些基本实现了城市人民公社化的大中城市中,共计有850万闲
  散劳动力被安置就业,占这些城市闲散劳动力总人数的87\%;共计兴办了7.6万个居
  民公共食堂,就餐人数达1700万;8.8万个托儿所,入托儿童为365万;还建立
  了8.9万个服务站。
\end{quotation}
虽然这些“大跃进”充满了主观期望与客观生产力不足的矛盾、大规模调用各方民众
资源等问题,但也由此可见毛泽东的政治理想和抱负。

1959--1961年\footnote{一说是1958--1962年大饥荒。},三年困难时期,岁大饥,人相食,饿
死者以千万计。

\subsection{大跃进中后期的安徽省}

笔者之所以选择安徽大跃进问题单独论述,原因有四:一、机缘巧合,笔者最先接触这
方面的完整史料。二、综合各方观点,安徽省总饿死人数不及四川省、位居全国各省第
二,但是饿死人比例是各省最高。三、安徽在大跃进前后出过两个上达天听的大人物,
一个张恺帆,一个曾希圣,两人均具各方面代表性。四、安徽省官方尊重这段历史,地
方志等文献资料较为健全。

曾希圣于1952年1月--1962年2月任职安徽省委第一任书记,1960年10月,兼任中共中央
华东局第二书记、中共山东省委第一书记,\footnote{1961年1月曾希圣主动申请辞去山东省委
  第一书记职务。普遍说法是曾希圣希望大搞“责任田”,无心兼任两省第一书记。另
  一说法是中央监察委员会针对安徽省饿死人进行的调查促使曾希圣辞职,这一说法可
  见尹曙生所作《曾希圣是如何掩盖严重灾荒
  的》\url{http://www.yhcqw.com/33/10039_2.html}}。也曾任济南军区政治委员等职。
  大跃进时期曾希圣在土法炼钢、水利和粮食方面均大搞“浮夸风”。

\begin{quotation}
  1958年安徽产粮\textbf{167.9亿斤},却被浮夸虚报成\textbf{450亿斤}(指标更高,是494亿斤),谎报
  了2.68倍。\cite{zhangfandang}

  在上下不讲真话的氛围中,1959年安徽粮食生产任务,于3月30日向全省宣布:“超额
  完成\textbf{720亿斤}”!这一严重浮夸的高指标,是1959年实际产
  量(\textbf{140.2亿斤})的5.14倍!虽然当年征购粮是\textbf{70.94亿斤}还不到指标的9.9\%,但
  却占实际产量的50.6\%。为此,安徽人民蒙受了巨大的痛苦,出现了严重的饿、病、
  逃、荒、死。\cite{zhang1959}
\end{quotation}

张恺帆1959年时任安徽省委书记处书记之一,安徽省副省长,7月去巣县和老家无为县考
察饿死人情况后,未经组织程序、自作主张、暂停无为县公共食堂,放赈救济粮。无为
县问题也得到了时任安徽省书记处候补书记、副省长陆学斌的支持。此事被上报至庐山
会议后,受到严厉批评。在随后8月、9月召开的两次安徽省委会议上两人被定性为“张
恺帆、陆学斌反党联盟”。

据张回忆录的记录整理者宋霖:
\begin{quotation}
  7 月 9 日,张恺帆给省委和曾希圣写报告,报告无为情况和即将实行的“\textbf{三还
    原}”措施:1.吃饭还原,停办公共食堂; 2.自留地还原; 3.房屋还原,让农民回自
  己家居住。并于当晚严令县委立即付诸实施,他说: “救人要紧!”

  7月10 日至 12 日,库存的 150 万斤大米和 300 万斤稻谷,迅速发往农村; 30 万斤
  黄豆加工成豆腐、豆浆,供应浮肿病人和没有奶喝的婴儿;设法弄来的一批肉食品,供
  应给病人。此举拯救了数十万濒临死亡的人民的生命。

  为此,两个多月后,张恺帆全家跌入苦难,六个亲人惨死……后来平反时统计:仅无
  为一县,因张恺帆事件受株连被批斗、被处理的县、社、队党员、干部和群众,共
  达28741人。\cite{zhang1959}
\end{quotation}

因笔者资料有限,不知酷爱仕途的曾希圣为何突然在1960年底开始转向,顶着重压积极
实施“责任田”,笔者愿意相信曾希圣此时已无法再承受良心的拷问。

老农刘庆兰父子1956年起先后上山独立垦荒,4年“共向集体无偿缴纳上交粮食4716斤
(平均每年1179斤)”。\cite{anhuiliushi}刘庆兰的事迹引发曾希圣极大兴趣,后经毛泽
东“同意试验”,在湖北进行“计划统一、分配统一、大农活和技术性农活统一、用水
和管水统一、抗灾统一”等“五个统一”之下的“责任田”推广。

\begin{quotation}
  它一问世就很受农民欢迎,全国不少地方都程度不同地实行起来。比如,当时搞各种
  形式包产到户的,安徽全省达80\%,甘肃临夏地区达74\%,浙江新昌县、四川江北县
  达到达70\%,广西龙胜县达42.3\%,福建连城县达42\%,贵州全省达40\%,广东、湖
  南、河北和东北三省也都出现了这种形式。据估计,当时全国实行包产到户的约
  占20\%。\cite[1078]{boyibo}
\end{quotation}

据张恺帆回忆:
\begin{quotation}
  我想,造成这种局面,高征购、浮夸风是主要原因,但办食堂,层层克扣,也是重要
  原因之一。江苏早就把食堂停办了,与各省比较,安徽食堂持续的时间太久了,后果
  是惨痛的——在1962年七千人大会上,刘少奇同志参加安徽组讨论,追问安徽饿死了
  多少人,第一次报40万,后来追问紧了,报到400万。实际上约有500万
  人。\cite[344]{zhangkaifanhuiyi}
\end{quotation}

张恺帆的500万数据可由以下官方数据佐证,据《安徽省志·人口志》所载表1--1--14,
安徽省总人口在1959--1961年间\textbf{负增长406万},仅1960年就比1959年负增长11.21\%,
减少\textbf{3839979人},\cite[27]{anhuishengzhi};表2--1--18记载1960年实际死亡人数据
为\textbf{2218280人},死亡率为68.58‰,书中还就此写到“\textbf{此为年报统计数,人口实际损
  失更大}”,1960年的死亡比例远高于之前的建国后最高
峰——1959年的16.72‰。从1961年至1985年,安徽省年死亡人数再也未超过300000人,
死亡比例最高为1964年的8.6‰。\cite[95-96]{anhuishengzhi}

安徽省志1958年数据显示溺婴死亡10159人,1959-1961年未提供溺婴数
据……\cite[108]{anhuishengzhi}根据周曼《三年困难时期安徽人口变动研究》,“关于
安徽省的婴儿死亡率,也没有找到确切的数据。安徽与河南同属重灾区,再根据两省死
亡率的比较,安徽省的婴儿死亡率可能会高于河南省(河南省1960年婴儿死亡率
为\textbf{276.8‰}),但确切数据无法估计。”

\begin{quotation}
  1960年,人口死亡异常,死亡率在10\%以上的有太和县(163.47‰)、无为县
  (158.29‰)、宣城县(147.26‰)、毫县(145.95‰)、宿县(130.32‰)、凤阳
  县(119.46‰)、阜阳县(118.31‰)、肥东县(113.31‰)、五河县(108.71‰)
  等9个县,死亡率最低的是合肥市(11.27‰)。\cite[98]{anhuishengzhi}

  (安徽省)各市县地方志关于人口死亡率的数据普遍(比《人口志》)\textbf{更高}。以宣城
  县为例,安徽省《人口志》显示,1960年全县人口的出生率为4.57‰,死亡率
  为163.47‰,两相比较,人口自然增长率为\textbf{-142.69‰}。《宣城县志》没有显示当年全
  县人口出生率和死亡率数据,但提供了人口自然增长率数据,
  为\textbf{-205.88‰}。\cite{zaihuangchayixing}
\end{quotation}


笔者认为同样要引起注意的是,据《安徽省志·人口志》,在“大跃进”拨乱反正3年后
的1964年仍出现了1.81\%的负增长。书中解释是1964年全国第二次人口普查时,各地虚
报人口获取利益的现象被纠正。如果将1964年纠正的负增长给到之前几年,死亡数据可能
更加惨烈。

据姚宏志论文《1959--1961年安徽灾荒的差异性分析》,安徽省大饥荒时期烤烟、棉花
等经济作物国家统购价格下降20\%,改种粮食较多。文中引述的韩敏、王朔柏和陈意新
等人的调查结果显示,宗族关系中官僚多的村庄受灾影响小,外派官僚取代原生宗族领
袖的村庄受灾影响大,原生宗族领袖和宗族关系未变的村庄受灾影响小(采用偷稻种、
瞒产、藏粮、对外守口如瓶等方式)。

曾希圣在1962年七千人大会上受到批判,较多人——包括薄一波和陈者香等,说曾希圣
是因这时的“单干风”而被调离安徽,笔者持不同意见。当时情景是1962年1月30日七千
人大会续开“出气会”,分派刘少奇、周恩来、邓小平、朱德、陈云等领导人再次分赴
几个省区大组,组织开展地委和县委干部对省委书记的“出气会”,且主要矛头均集中
在“大跃进”错误。各省委书记捶胸顿足者有,嚎啕大哭者有,安徽省委第一书记曾希
圣被批判的主要错误是他的饿死人掩盖子问题。

邓子恢、陈云等人在七千人大会同样主张搞“单干风”,毛泽东后派田家英进行责任田
调查,此时并没有“单干风”一事;“单干风”是在同年8月北戴河会议转向继续阶级斗
争路线后才被定性的。笔者就此认为曾希圣被调离的主要原因绝不是“责任田”,就
是“掩盖子”。

曾希圣力主推行“责任田”有功,但又何必将其粉饰为神呢?“出气会”初期依然死扛
强硬“掩盖子”,他能算是个神吗?即使经过了中国历史几千年的教训,国民仍常常打
破一个偶像后,又树立起一个新的偶像来。什么时候,我们才能学会客观求真、实事求
是与就事论事呢?其实就算是张恺帆,笔者也觉要敬佩他停办食堂、开仓赈粮等英雄事
迹和操守,但也莫要让其成神。我们要看的要评价应当是人所做之事,而非赋予人一个
刻板标签……对于神圣者和偶像的崇拜,无外乎是将本人无力之事寄托于他人,个人人
格在此情境下无论如何也是残缺的。

这次始于1月30日的“出气会”,高层领导的参与除了为打破僵局、对省委书记展开批判
外还有一个作用,就是防止批判扩大至对一些省委书记更为严重的追责,如入狱、枪毙,
隐含得更深层原因可能是将部分中央应付责任转移到省委一级。笔者一家之言,不可轻
信,还要批判性吸收为好。

只是,曾希圣,你后来真的悔过了么?

最后,笔者也向曾希圣道一声歉意。本书前后虽批评很多人,但只细述了他一人的过失
并不留情地抨击。在时代的大背景下,曾希圣也只是个浮萍。笔者细述他的深层目的不
是将安徽省大饥荒归咎于他个人,而是他前后的故事均较有代表性,希望官民皆以史为
鉴,引以为戒。

\subsection{对“大跃进”的补救措施及反思}

\begin{enumerate}

\item 李若建论文《权力与人性:大跃进时期公共食堂研究》将此时国人分为六个阶层:高
  级、中级、基层官员、利益相关民众和利益不相关民众,分别就权力和人性两方面展
  开论述。笔者认为这是一篇不可多得的客观雄文文,建议大家阅读。李若建另一篇
  《理性与良知:“大跃进”时期的县级官员》则说明了县官的生态环境、务虚正职与
  务实副职的差别和一些县长的英雄事迹,前文于无意中也引用了李若建《困难时期的
  精简职工与下放城镇居民》。我们还有一些负责任的官员、专家、学者和人民背负着
  国计民生的伟大理想和目标,向他们致敬。因笔者知识面狭窄,无法将他们姓名一一
  道出,还请读者自我发掘。

\item 1960年11月3日,中共中央向各级党组织发出《关于农村人民公社当前政策问题的紧
  急指示信》(简称十二条)和《中共中央关于贯彻执行“紧急指示信”的指示》,反
  五风,清理“一平二调”,反对五风,明确“以生产队为基础的\textbf{三级所有制}(公社、
  生产大队、生产队),是现阶段人民公社的根本制度”,城乡精简,“允许社员经营
  少量的自留地和小规模的家庭副业”等。被庐山会议中断的“纠左”重新起步,笔者
  不再赘述。
  % 郑文中认为,“确切意义上
  % 的调整即后退(全年基建、钢产量、粮食产量指标下调),是从1961年9月庐山工作
  % 会议开始的。”

  经过“五风大跃进”,国家吸取了蔑视客观生产力的教训,即使在“文化大革命”时
  仍要“促生产”,各行业、特别是农业虽再发生过动荡,但再无这样强度的惨剧发
  生。


\item 城乡二元结构对立进一步加强,城市向农村逆向迁移。

  \begin{quotation}
    1956年秋天,由于过激的合作化运动加上自然灾害,导致不少省份粮食歉收、农民
    吃饭成问题,安徽、河南等省出现大量农民外流,进城寻求就业机会。在这种情况
    下,12月《国务院关于防止农村人口盲目外流的指示》出台,防止农村农民进城就
    业。

    (出台一系列此类政策之后,)1958年1月9日,全国人民代表大会常务委员会第九
    十一次会议通过并颁布了新中国第一部户籍制度《中华人民共和国户口登记条
    例》……正式确立了户口迁移审批制度和凭证落户制度。以这个条例为标志,中国
    政府开始对人口自由流动实行严格限制和政府管制……第一次明确将城乡居民区分
    为“农业户口”和“非农业户口”两种不同户籍。严格限制农村农民迁往城镇,限
    制城市间人口流动,在城市与农村之间构筑了一道高墙,城乡分离的“二元经济模
    式”因此而生成。

    户籍制度变化第二阶段:(1958年-1978年),这一时期包括大跃进、三年困难时期和
    十年“文革”。严格限制户口迁移,特别是严格限制农民向城市迁移,控制农村人
    口流人城市,压缩城市人口,包括精简职工、知识青年上山下乡、干部下放农村等,
    出现了人类历史上罕见的人口\textbf{从城市迁往农村的反向运动},形成了一整套严格的户
    籍管理制度。\cite{quxiaohuji}
  \end{quotation}

  精简职工方面,
  \begin{quotation}
    (大跃进时期的大招工)使得工人数从1957年的3101万增加到1960年的5969万,增
    长92.5\%。职工人数的增加,特别是从农村招收的职工,给城镇带来了大批的人
    口,1957-1960年间,中国的城镇人口从9949万增加到13073万,其中由农村迁入城
    镇的大约2218万。

    当粮食危机越来越严重时候,许多城市已经面临几乎没有库存的窘境,1960年底全
    国82个大中城市的库存粮食只有正常水平的 $\sfrac{1}{3}$ 。1960年6月北京、天
    津和辽宁的几个主要城市的库存粮食几乎没有,只能维持不到10天的供应,上海的
    大米库存已经没有,天天告急。

    有关的统计,在1961-1963年间,压缩下放2500万城镇人口,精减职工1833万人,被
    精减的职工中,大部分也被下放到农村,少数转为城镇集体企业工人,还有少数流
    浪到边疆地区,在当地谋生。\cite{jingjianzhigong}
  \end{quotation}

\item 1962年1月11日至2月7日,扩大的中央工作会议在北京举行,俗称“七千人大会”。
  七千人大会是一次纠左的、带有很多正面影响的会议,同时也是一场涉及政治、经济、
  权力纠葛的复杂会议。

  刘少奇代表中央政府引一老农的话“三分天灾,七分人祸”,也引用毛泽东说过几次
  的“指头论”对左的做法进行了批评,毛泽东带头进行了自我批评,众高层官员也开
  展批评和自我批评。

  1962年2月21日至23日,刘少奇主持召开中共中央政治局常委扩大会议(西楼会议)。
  深化和发展了七千人大会的“民主集中制”之风。七千人大会和西楼会议奠定了此后
  半年的进步纠左基调。

  邓子恢会后继续着力于推广“责任田”和“包产到户”。7月作出《关于农业问题的报
  告》。

  1962年2月27日,王稼祥、刘宁一、伍修权联名给中央写信,分析了资本主义和社会主
  义阵营间、社会主义主义阵营内部的缓和,提倡对国外援助要量力而行。

  “1962年4月27日,中共中央根据扩大的中央工作会议的精神,发出《关于加速进行党
  员、干部甄别工作的通知》”,邓小平主持了这次会议,并推动平反工作展
  开,”到1962年8月,全国已有600万干部和党员得到了平反”。

  1962年6月,彭德怀上书“八万言书”。
\end{enumerate}

\subsection{继续阶级斗争}

虽然此后一直坚持阶级斗争,但再也不敢像“五风”盛行时那样破坏生产了。

\begin{enumerate}
\item 七千人大会的胜利成果只维持很短时间,1962年8月在北戴河召开的中共中央会议
  和9月召开的中共八届十中全会上,重提“无产阶级和资产阶级之间的阶级斗争、社会
  主义道路和资本主义这两条道路的斗争。”

  陈云、邓子恢和田家英经调查后支持“包产到户”“责任田”,因此被定为“单干
  风”;刘少奇、周恩来、陈云等对经济困难形势的判断等被定为“黑暗风”;王稼
  祥等被定为“三和一少”;之前的右派分子甄别平凡工作、彭德怀此时的申诉信、受《刘志丹》一
  书牵连的习仲勋、贾拓夫、刘景范等分别被定为“翻案风”。

\item 此后的历史,笔者认为已经乏善可陈。1963--1966年指向基层的“四清”,即城乡社
  会主义教育运动。1966--1976年的“文化大革命”。阶级斗争,一抓就灵;发动群众
  斗群众、官僚,发动官僚斗官僚……有人说本意是借人民群众“大民主”的觉醒摧毁
  国家官僚机器,只是“二月逆流”
\end{enumerate}

\subsection{笔者个人总结}

\begin{enumerate}
\item 对唯物史观有基本了解的人都可以认识到,客观事物的发展不以人的主观意志为转
  移,1965年 2 月 11 日,毛泽东在与苏联大使柯西金谈话时,主动提到“我对不以人
  民、党派和政府意志为转移,甚至不以个人意志为转移的历史进程感兴趣。事件的过
  程都是被生存的客观法则所决定的。”毛泽东也知道。

  但理论是抽象、枯燥、片面、书面的;而现实是综合、立体、具体、生动鲜活的;政
  治又是现实实践的,交织着种种人性私欲。正如扉页引用的黑格尔话语“人民和政府
  从来就没有从历史中学到任何东西……”

  中国这一段历史犯有一些与苏联异曲同工的错误,可见前
  文\ccref{sec:marxkexue},\ccref{sec:sushijian}。

\item 空怀理想无论如何也接不了地气,它也不能成为善的代名词,实事求是是行动主义的
  必要前提、目标和标准。为了遥远伦理道德理想(空想社会主义)无视当前阶段的伦
  理道德——甚至肆无忌惮剥夺生命,这本身就是一个悖论。如钱理群所说
  \begin{quotation}
    至善至美的理想社会,只能存在于彼岸世界,是一个可望不可及的远方目标。它的存在,
    使人们永远也不会满足于现状,从而加强对现实的批判和改造力量。但这种改造、变革
    的努力,只能使此岸世界不断改善,不断趋向、接近理想,却永远也不会达到至善至美
    的理想境界。任何将彼岸世界此岸化的努力,都会给人类带来灾难,\textbf{天堂的现
      实化就是地狱}。
  \end{quotation}

\item “民主集中制”是显明的“民主的集中制”,集中制为主。笔者认可中国必须实行集
  中制。诸侯纷争一直是中国历史上朝代灭亡前的必然现象,乃至动因,不可不吸取这
  个教训。

  党和国家的高层领导人的回忆录中常提加强法制和民主(“法制”今日应当变更
  为“法治”),但是问题在于,集中制为主下,民主与法治的成份到底该有个怎样的
  度,怎样去确定度?历史唯物主义是极为宏观的理论,它缺乏对微观层面的考量,实
  质上历史唯物主义本身就轻视上层建筑。

\item 官僚考核制度长远以来一直是各个国家的重大问题,它常常是唯上而不唯下、爱官而
  不爱民。我们该怎样发展官僚考核制度,是党和国家的重大课题。另外,这一阶段官
  员腐败和其他犯罪现象往往是被过于低估的。

\item 雪崩之下,没有一片雪花是无辜的。这段时期中,雪崩是主因、是制度,雪花是次次
  因、是人性。这段动荡年代的种种人性体现足以给我们太多的警醒和教训,一些人是那样
  的残忍、残酷、蔑视生命、伦理、公理。人的本质就是会这样么?

  再怎样的铁房子里,总有几线阳光照入,也总有并不算多的人如星般闪耀。联合国多
  年来一直倡导个人主观能动性,便是想要个人突破各种枷锁限制,最终使一个个进步
  的个人们彼此聚合为有力的、进步的和超越的群体。我们应当发展个人主观能动性,
  保持对人性恶的一面的警惕并敢于去对抗恶一些,有时甚至要为此抱有一种悲剧意识,
  只是这确实太理想化、太苛求了……
\end{enumerate}

\section{计划经济时期的成就}

本节部分内容来自李慎明文章《正确评价改革开放前后两个历史时期》。

\begin{enumerate}

\item “实现和巩固了全国范围(除台湾等岛屿以外)的国家统一,根本改变了旧中国四分五
  裂的局面。实现和巩固了全国各族人民的大团结,形成和发展了五十多个民族平等互
  助的社会主义民族关系。战胜了帝国主义、霸权主义的侵略、破坏和武装挑衅,维护
  了国家的安全和独立,胜利地进行了保卫祖国边疆的斗争。”

\item 排除种种干扰重返联合国。由于毛泽东关于“三个世界”划分理论的正确指导,我国
  与美国、欧洲诸国和日本等主要国家的外交关系取得突破性进展,成功打破外部霸权
  主义和强权政治对我国的严酷封锁,真正跨入了大国的行列,并迎来和平与发展的时
  代主题。

\item 1949年,中国人口5.4亿,人均预期寿命不足35岁。1976年,人口9.3亿,人均预期寿
  命64.6岁,人口数量增长近4亿,预期寿命增长近30岁。当然,这里也有之前战乱动荡,
  底子过于薄弱的加成。

  新中国成立后,中央将扫盲列为成人教育的首位,并在1952年,成立了“中央扫除文
  盲工作委员会”,积极推行“速成识字法”。1956年3月29日,中共中央和国务院发布
  《扫除文盲的决定》,将扫盲提高到了空前的高度,第一次把扫盲作为国家发展大
  计。1960年5月11日,中共中央发布了《关于推广注音识字的指示》。

  解放初期文盲率80\%,1964年15岁以上人口文盲率56.8\%,1982年34.5\%。\footnote{参考黄
    晨熹《1964\~2005年我国人口受教育状况的变动》}

\item 在经济落后的条件下,以严重城乡二元对立为代价,保证了高积累和优先快速发展重
  工业,建立了比较完整的独立的工业体系和基础设施。

  根据中共十一届六中全会《关于建国以来党的若干历史问题的决议》:
  \begin{quotation}
    一九八〇年同完成经济恢复的一九五二年相比,全国工业固定资产按原价计算,增长
    二十六倍多,达到四千一百多亿元;棉纱产量增长三点五倍,达到二百九十三万吨;
    原煤产量增长八点四倍,达到六亿二千万吨;发电量增长四十倍,达到三千多亿度;
    原油产量达到一亿零五百多万吨;钢产量达到三千七百多万吨;机械工业产值增长五
    十三倍,达到一千二百七十多亿元。在辽阔的内地和少数民族地区,兴建了一批新的
    工业基地。国防工业从无到有地逐步建设起来。资源勘探工作成绩很大。铁路、公路、
    水运、空运和邮电事业,都有很大的发展。

    ……

    我们现在(80年代初)赖以进行现代化建设的物质技术基础,很大一部分是这个期间
    建设起来的;全国经济文化建设等方面的骨干力量和他们的工作经验,大部分也是在
    这个期间培养和积累起来的。这是这个期间党的工作的主导方面。
  \end{quotation}
  在毛泽东时期,我国从一个落后的农业国跻身为世界第六大工业国。

\item 独立自主、自力更生研发出“两弹一星一潜艇”。1964年10月,我国第一颗原
  子弹爆炸成功。1966年12月,我国第一颗氢弹原理试验爆炸成功。1970年4月,我国
  第一颗人造卫星发射成功。1971年9月,我国第一艘核潜艇下水,并于1974年8月,
  正式加入人民海军战斗序列。在成熟的核潜艇的基础上,1981年4月,我国第一艘战
  略核潜艇下水,使陆海空全都具备了第二次核反击能力。

  邓小平1988年明确指出,“如果六十年代以来中国没有原子弹、氢弹,没有发射卫
  星,中国就不能叫有重要影响的大国,就没有现在这样的国际地位。”



\item 政治层面上前所未有的管控能力。计划经济时期通过“政社合一”人民公社体制实现
  了对农村前所未有、甚至可能后不见来者的资源管控能力。一切资源,包括“国有”和“集
  体”企业、土地,事实上均掌握在各级政府手中,为之后的市场经济时期提供了多方面
  可以运作的巨额红利空间。

  因城市经济不可承受的压力,20世纪60年代初和三次知青下乡等以千万人为单位的城
  市向农民的劳动力逆向转移,没有这样管控能力和“信仰”加成其实是办不到的,只
  是“信仰”在现实面前逐次衰减。
\end{enumerate}


下一章我们来探讨中国市场经济体制过渡时期1978--1992年。
% http://econ.cssn.cn/jjx/xk/jjx_yyjjx/jjx_slyjsjjx/201310/t20131024_516814.shtml % 中国财政支出结构的演进研究(上) 穷富地区支出差异,行政管理比重的曲线。

% 下图反映的是中国20世纪50年代至80年代城镇人口数的变化情况。其中,导致从C到D变化% 的主要原因是 A.人民公社化运动B.“大跃进”的影响C.国民经济的调整D.自然灾害% % 的影响





%%% Local Variables:
%%% mode: latex
%%% TeX-master: "../main"
%%% End:
