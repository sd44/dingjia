\chapter{维基解密、阿桑奇与西方政治}

\section{维基解密简述}

维基解密官网的域名wikileaks.org注册于2006年10月4日,朱利安·保罗·阿桑奇一般
被视为其创始人。自维基解密成立之初,就着力于解密大批文档。创始之初采用公共编
辑方式,任何人都可以发布、修改页面,提供各行业领域的秘密信息,所提交文件需经
匿名维基解密工作人员的审阅。因超出审核人员处理能力,后来改为只接受具有政治、
外交、历史或伦理意义的文件。

维基解密在十年多的时间里,多次对世界造成巨大影响。如公布肯尼亚原总统腐败案;
阿尔及利亚政府与石油公司合作,破坏另一家石油公司的设备造成石油大面积泄露;伊
拉克战争美军直升机射杀平民,包括两名路透社记者和儿童;阿富汗战争的大量文件及
关塔那摩虐俘;美国、俄罗斯的可以侵入几乎所有系统的网络黑客工具等。

\section{希拉里邮件门事件}
\label{sec:podestamail}

在2016年,维基解密先后发布10多万封有关于希拉里的邮件,一般被称为“邮件门”,
影响更是巨大。本节内容有借鉴以下网页。
\begin{enumerate}
\item \href{https://www.zhihu.com/question/41676600}{知乎:DNC邮件中有哪些美国民主党不可告人的内容?}

\item \href{https://www.zhihu.com/question/51362588}{知乎:如何看待The Podesta Emails?}

\item \href{https://github.com/zhouningyi/us_selection_crack}{GitHub:希拉里邮件门数据}
\end{enumerate}


为力求客观表述,克林顿基金会连续杀人、撒旦教披萨门等经由网友讨论演绎出的论点
不计算在内,只节选其中极少数邮件和与之相关的后续发展如下。

2016年3月16日,维基解密发布30322封希拉里邮件,邮件收发时间跨度
为2010年6月至2014年8月。2016年7月22日,维基解密发布19252封DNC(民主党全国委员
会)邮件,邮件收发时间跨度为2015年至2016年5月25日。笔者翻译部分邮件如下:

\begin{enumerate}
\item emailid/25 希拉里竞选团队用邮件向DNC确认已收到几张DNC支票,被Jordan
  Kaplan严厉斥责:“不要再像这样发邮件。你认识Alex(直接跟他说)。不要犯
  蠢。”(根据网友分析,希拉里竞选团队通过HVF和DNC违法获取超过政治献金额度的
  捐赠,然后将超额捐赠化整为零,将这笔钱投入到竞选广告或者分化为小额筹款以躲避
  监管)

\item emailid/1041 DNC中的Luis Miranda提供了造谣污损特朗普的几个方向。如特朗普危
  险、暴力、侮辱女性、穆斯林、墨西哥人、反对言论自由等。

\item emailid/17065 富人Liz为HVF(希拉里胜利基金会)开具了一张20万美元的支票,要
  求希拉里参加美国驻联合国人权理事会大使Eileen Donahoe举办的私人晚宴。

\item emailid/20352 Jordan Kaplan索要捐赠人名单,要求将名单发给Scott Comer。这些
  人可进入USPS, NEA, NEH等实权董事会,也有可能进入不怎么好的董事会、理事会,
  如美国妇女历史委员会。

\item emailid/658 Scott Comer提供了一个23名捐赠人名单。(据知乎
  帖 \url{https://www.zhihu.com/question/41676600},有部分人的捐赠数额,除去
  捐款最少的25美元和2600美元外,数额均在4万美元以上,最多捐款额为334000美元
  。)
\end{enumerate}

2016年10月7日,维基解密又发布58000余封John Podesta的邮件。Podesta于1998-2001年
任比尔·克林顿的白宫幕僚长,2014-2015年任奥巴马总统顾问,2015-2016年任希拉里
竞选团队主席。笔者翻译部分邮件如下:

\begin{enumerate}
\item emailid/8396 2011年,卡塔尔邀请克林顿参加了纽约一个5分钟的小会,并承诺捐
  赠给克林顿基金会100万美元,作为克林顿生日礼物。另外,卡塔尔感谢希拉里提出
  的海地投资意见,并表示会加以考虑。

\item emailid/7452 比尔·克林顿的幕僚长Tina Flournoy致信Podesta:外国政府捐赠的
  钱已经入账。

\item emailid/22030 摩洛哥提出向克林顿基金会支付1200万美元,但有条件——希拉里要
  于2015年5月出席在摩洛哥古城马拉喀什为其召开的“克林顿全球倡议大会”,并在大
  会发表演讲。(希拉里当时在美国国务院工作,此捐赠已构成受贿行为,但希拉里仍
  然收下这笔钱。因担心影响选情,后由其丈夫前总统比尔·克林顿和女儿出席。)

\item emailid/6775 沙特的谢赫·穆罕默德酋长想对克林顿提供飞机带其参
  加埃塞俄比亚的会议一事表示感谢,要求克林顿亲自致电给他。Podesta同意Doug
  Band的意见,同时要求穆罕默德酋长向克林顿基金会捐赠600万美元。

\item emailid/4635 Podesta在疑与普京高度相关的Joule Unilimited公司持有75000股股
  票。(后于奥巴马任期2014年时将股票转让给一家匿名私人控股公司)

\item emailid/57027 民主党国家委员会的临时主席,也曾任职于CNN的Donna Brazile,向
  希拉里泄露其与桑德斯\textbf{党内辩论}时要被问到的两个问题。在其他邮件中,Brazile表
  示想在希拉里总统胜选后做Podesta的代理人。

\item emailid/39107 Alphabet公司(Google母公司)董事长Eric Emerson Schmidt的一个
  小团队为希拉里竞选团队制作竞选页面工具,并搜集整理捐赠者信息、信用卡号数据
  库等,团队工作人员认为Eric Emerson Schmidt暗示他可以做的更“全面”。(
  据
  \href{https://www.opensecrets.org/federal-lobbying/clients/summary?cycle=2019&id=D000067823}{opensecrets
    网站}统计,Alphabet/Google政治献金数额极大,2015-2023年年均游说费用
  为1485万美元,年均使用说客99.4人。)

\item emailid/8190 2008年10月6日,时任花旗银行高管的Michael Froman发给Podesta一
  封主题为“Lists”的邮件,要求名单上的人应当优先考虑出任政府高官。内有三个附
  件,附件1: 92个女性官员提名名单;附件2: 222个非白种或残疾美国人提名名单;
  附件3: 31个内阁级别职位的提名名单(附优先级与候选人)。(2008年大选投票日期
  是11月4日,奥巴马当选后,有近半数入选奥巴马内阁。)

\item 其他,近百位媒体工作者、领导被Podesta招待。多位记者向Podesta表忠心,还有人
  预先告知将会提问希拉里的问题或者在发文章前请希拉里竞选团队过目。
\end{enumerate}


\section{维基解密的理念}

综观维基解密历史,它的理念应当是显而易见的。

吸收多渠道泄密出来的大批量文件,不管信息来自哪个渠道——黑客、政府和企业工作
人员、某一派系的敌对方都可以,只要解密文件是真实的便可接受。当信息量大到一定
量级时,这个系统就可以高度容错,泄密者、甚至网站管理人员的个人主观就不起作用
了。诸多事件就会一个个结合起来,构成为更富普遍性、一般性、客观性的权力和金钱
光谱。世人由此可见光谱中的反伦理反人类思想,从而谴责、批评、改造、审判这种反
伦理,促进相关信息的更加公开化,以让世界变得更加阳光美好、公正平等。

以邮件门为例,\$illary Clinton和她 ``still dicking bimbos at home'' 的丈夫比
尔·林顿本来只是具体两个人,在竞选中则代表一方竞选势力。但随着邮件量级的增长,
我们可以将其抽象到民主党、共和党;再到美国政治、资本生态;再到西方,直至抽象
到人类整个的——可以是当前的,也可以是历史的——权力、资本生态,甚至抽象到人
欲本身。

\section{维基解密的缺陷及问题}

维基解密仍有些问题和缺陷:

\begin{enumerate}
\item 存在公器私用的可能,如阿桑奇与小特朗普的生意。

  美国总统特朗普的长子小特朗普在推特上发布过这样一条信息:从2016年9月美国大选
  期间到2017年7月之间,他与维基解密在Twitter上的私聊记录。其中一条信息显示,
  维基解密希望小特朗普帮忙——让特朗普总统建议澳大利亚指派阿桑奇为澳驻美大使。
  笔者个人认为阿桑奇向特朗普的倾斜出自现实的考虑,以使自己不必腹背受敌,谋求
  大使职位是希望借助大使的外交豁免权来保护自己不受伤害。

  有人指责维基解密与俄罗斯政府有关联,如DNC邮件泄露事件被怀疑有俄罗斯政府官员参与
  其中,向维基解密递交了邮件。也有人指责维基解密与特朗普有关联。如阿桑奇就曾
  在某次电视节目上自带“Vote Trump”的胸标,在个别采访中也倾向于特郎普。但在
  一次采访中当主持人问他要投票给特朗普还是希拉里时,阿桑奇又说“霍乱还是淋病
  的选择吗?就我个人而言,我一个都不喜欢。”。

\item 不受制约的权力:

  维基解密试图借助公开秘密文件,打击不受制约和暗箱化操作的权力,使政治、军事、
  外交、伦理奔向更为阳光的一面。在维基解密的历史中,它显然具备极高可信度。但
  同时,维基解密的内容发布权限只集中在少数几个人的委员会,甚至阿桑奇自己手中。
  在试图制约其他权力的同时,维基解密的“发布委员会”自身也具备了极高的权力,
  这种权力由维基解密各种直接或间接的用户所赋予。

  那么,由谁来制约维基解密自身的权力呢?在现实中,因其倾向于特朗普,也在一定
  程度上影响了大选结果。勇者要谨防成为恶龙,而我们却不能提供预防恶变的措施,
  只能单薄无力地寄希望于阿桑奇自己的个人人格。


\item 核心机密文件的筹码困境:

  维基解密在因特网上数次释放了数十GB加密文件,解密密码被认为掌握在阿桑奇手中,
  数年来从未公开。这些大量加密文件被认为足以造成国家动荡的核心机密,用来解密
  这些文件的密码也是阿桑奇的“保命筹码”。凭此筹码,美国等国家不敢轻易直接对
  阿桑奇采取极端措施。但维基解密掌握的这些最为有力的文件,在非极端情况下却几
  乎永不会为人所知。

  即使阿桑奇已于2014年被英国逮捕,这些核心文件仍未被解密。

\item 体制外权力的限度:

  维基解密是一种反体制规训的强大力量,这种力量无法融入体制。它寄希望于通过公
  众的知情权,以“自下而上”的方式来打破规训,走向美好,至于怎样“走向”,是
  它所不能提出的。也就是说,它具有极强批判性,但是建设性仍是有限,且会受到多
  个强权集团的强力打击,极难以生存下来。自阿桑奇被捕后,维基解密网站的活跃度、
  真实性、影响力都大幅下降。

  知乎的 @阿伽陀 认为体制,就现实来说,仍是重要且必须的。维基解密太过于反对体
  制,只能被反对派、寄生虫或别有用心者利用,我个人对此持有限赞同态度。
\end{enumerate}


\section{希拉里邮件门之后}

希拉里邮件门,特别是Podesta邮件曝光后,美国媒体先是相当沉默,直至此事在公众网
络上越炒越热后才真正介入。CNN居然有主持人说公众直接去看泄密邮件是违法的,公众
应当从媒体、从CNN获取“权威报道”。Quora, 4Chan,Google等网站均有删除维基解密
相关内容的行为。本是希拉里激烈对立面的特朗普也往往避重就轻,顾左右而言他,皮
尽管扯,里子却是触动不得的,两党保持相当默契。媒体、政治地被规训在此表现的淋
漓尽致。

那维基解密理念中所在意的世界大众的力量呢?结果同样令人失望,浪潮过后并未起什
么大的波澜……历史上类似的政治腐败曝光事例,又有多少激起了民众非表面的、积极
行动的强烈反抗呢?

那么,我们是否可以说维基解密充其量是理想,而不具有现实的行动性呢?虽然我们的
现实太过贫困,很可能无法提出任何行之有效的总体建设方案,但笔者认为不可无视星
火,正因为有这些理想的火种散播在部分人心中,那就始终是有燎原希望的。不然整个
人类社会都将持续向边沁、功利、利己的的方向发展,走向毁灭。当潘多拉打开魔盒,
放出一切邪恶与困病时,“希望”还留在里面。

\begin{quotation}
  愿中国青年都摆脱冷气,只是向上走,不必听自暴自弃者流的话。 能做事的做事,能
  发声的发声。 有一分热,发一分光,就令萤火一般,也可以在黑暗里发一点光,不必
  等候炬火。 此后如竟没有炬火:我便是唯一的光。

  \raggedleft
  —— 鲁迅\quad 《热风·随感录四十一》 \qquad
\end{quotation}
