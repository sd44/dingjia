\chapter{再论土地金融及相关经济概念}

近30年来,土地金融如此深刻地影响了全国全民,并且下一步改革重点似乎还会是土地,
土地已和国运紧密相联。笔者认为有必要在前文基础上,再做总结和展望。

\begin{table}[h]
\centering
\resizebox{\textwidth}{!}{%
\begin{tabular}{@{}cccccccccc@{}}
\toprule
\multirow{3}{*}{年份} &
  \multicolumn{2}{c}{GDP} &
  \multicolumn{4}{c}{土地出让收入} &
  \multicolumn{3}{c}{财政收入} \\ \cmidrule(lr{0.5em}){2-3} \cmidrule(lr{0.5em}){4-7}\cmidrule(lr{0.5em}){8-10}
 &
  当年价格 &
  \begin{tabular}[c]{@{}c@{}}较上年\\ 增长\end{tabular} &
  收入 &
  \begin{tabular}[c]{@{}c@{}}较上年\\ 增长\end{tabular} &
  \begin{tabular}[c]{@{}c@{}}占比\\ GDP\end{tabular} &
  \begin{tabular}[c]{@{}c@{}}占比地方\\ 财政收入\end{tabular} &
  全国 &
  中央 &
  地方 \\ \midrule
1988 & 15180   & 15180   & 4.16  & 24.69\%   & 0.03\% & 0.26\%  & 2357   & 775   & 1582   \\
1989 & 17180   & 13.17\% & 4.47  & 7.45\%    & 0.03\% & 0.24\%  & 2665   & 823   & 1842   \\
1990 & 18873   & 9.86\%  & 10.52 & 135.35\%  & 0.06\% & 0.54\%  & 2937   & 992   & 1945   \\
1991 & 22006   & 16.60\% & 11.37 & 8.08\%    & 0.05\% & 0.51\%  & 3149   & 938   & 2211   \\
1992 & 27195   & 23.58\% & 525   & 4517.41\% & 1.93\% & 20.97\% & 3483   & 980   & 2504   \\
1993 & 35673   & 31.18\% & 558   & 6.25\%    & 1.56\% & 16.45\% & 4349   & 958   & 3391   \\
1994 & 48638   & 36.34\% & 639   & 14.56\%   & 1.31\% & 27.64\% & 5218   & 2907  & 2312   \\
1995 & 61340   & 26.12\% & 388   & -39.33\%  & 0.63\% & 12.99\% & 6242   & 3257  & 2986   \\
1996 & 71814   & 17.07\% & 349   & -9.96\%   & 0.49\% & 9.32\%  & 7408   & 3661  & 3747   \\
1997 & 79715   & 11.00\% & 462   & 32.37\%   & 0.58\% & 10.44\% & 8651   & 4227  & 4424   \\
1998 & 85196   & 6.88\%  & 508   & 9.87\%    & 0.60\% & 10.19\% & 9876   & 4892  & 4984   \\
1999 & 90564   & 6.30\%  & 514   & 1.31\%    & 0.57\% & 9.19\%  & 11444  & 5849  & 5595   \\
2000 & 100280  & 10.73\% & 596   & 15.80\%   & 0.59\% & 9.30\%  & 13395  & 6989  & 6406   \\
2001 & 110863  & 10.55\% & 1296  & 117.58\%  & 1.17\% & 16.61\% & 16386  & 8583  & 7803   \\
2002 & 121717  & 9.79\%  & 2417  & 86.50\%   & 1.99\% & 28.38\% & 18904  & 10389 & 8515   \\
2003 & 137422  & 12.90\% & 5421  & 124.32\%  & 3.95\% & 55.04\% & 21715  & 11865 & 9850   \\
2004 & 161840  & 17.77\% & 6412  & 18.28\%   & 3.96\% & 53.91\% & 26396  & 14503 & 11893  \\
2005 & 187319  & 15.74\% & 5884  & -8.24\%   & 3.14\% & 38.96\% & 31649  & 16549 & 15101  \\
2006 & 219439  & 17.15\% & 8078  & 37.29\%   & 3.68\% & 44.13\% & 38760  & 20457 & 18304  \\
2007 & 270092  & 23.08\% & 12217 & 51.24\%   & 4.52\% & 51.83\% & 51322  & 27749 & 23573  \\
2008 & 319245  & 18.20\% & 10260 & -16.02\%  & 3.21\% & 35.81\% & 61330  & 32681 & 28650  \\
2009 & 348518  & 9.17\%  & 17180 & 67.45\%   & 4.93\% & 52.69\% & 68518  & 35916 & 32603  \\
2010 & 412119  & 18.25\% & 27464 & 59.87\%   & 6.66\% & 67.62\% & 83102  & 42488 & 40613  \\
2011 & 487940  & 18.40\% & 32126 & 16.97\%   & 6.58\% & 61.14\% & 103874 & 51327 & 52547  \\
2012 & 538580  & 10.38\% & 28042 & -12.71\%  & 5.21\% & 45.91\% & 117254 & 56175 & 61078  \\
2013 & 592963  & 10.10\% & 43745 & 56.00\%   & 7.38\% & 63.39\% & 129210 & 60198 & 69011  \\
2014 & 643563  & 8.53\%  & 34377 & -21.41\%  & 5.34\% & 45.31\% & 140370 & 64493 & 75877  \\
2015 & 688858  & 7.04\%  & 31221 & -9.18\%   & 4.53\% & 37.61\% & 152269 & 69267 & 83002  \\
2016 & 746395  & 8.35\%  & 36462 & 16.79\%   & 4.89\% & 41.79\% & 159605 & 72366 & 87239  \\
2017 & 832036  & 11.47\% & 51984 & 42.57\%   & 6.25\% & 56.83\% & 172593 & 81123 & 91469  \\
2018 & 919281  & 10.49\% & 65096 & 25.22\%   & 7.08\% & 66.49\% & 183360 & 85456 & 97903  \\
2019 & 986515  & 7.31\%  & 72581 & 11.50\%   & 7.36\% & 71.81\% & 190390 & 89309 & 101081 \\
2020 & 1015986 & 2.99\%  & 84142 & 15.93\%   & 8.28\% & 84.02\% & 182914 & 82771 & 100143 \\
2021 & 1143670 & 12.57\% & 87051 & 3.46\%    & 7.61\% & 78.36\% & 202555 & 91470 & 111084 \\ \bottomrule
\end{tabular}%
}
\caption{1988--2021财政及土地出让收入(单位:亿元)}
\label{tab:19882021}
\capsource{财政收入中不包括国内外债务收入。\par 土地出让收入数据,1988-1998来源于张清勇,1999-2017来源于《国
  家土地统计年鉴》,2018-2021来源于财政部。其余数据来源于国家统计局。}
\end{table}

\section{王小鲁《土地财政的昨天、今天和明天》}

关于土地金融的总结,笔者尚未见到比王小鲁论述更为直接和深刻的文章,本节除特殊
说明,完全引用王小鲁《土地财政的昨天、今天和明天》\cref{wxljintian}。碍于篇
幅等,王小鲁没有论述具体历史,可见本书\cref{sec:80tudi}、\cref{92tudi}。

全国土地财政格局迎来了一个大转折。2022年上半年,全国国有土地出让收入2.36万亿
元,比上年同期大幅度下降31\%。要判断其原因和未来趋势,需要对土地财政的来龙去
脉做一个清晰的梳理。

{\heiti 一、昨天:土地财政从何而来?}

在旧的计划经济体制下,我国土地不是商品,需要的建设用地由政府从农民手里征用,
并进行\textbf{无偿划拨}。

上世纪80年代,少数城市开始尝试\textbf{土地有偿出让}和\textbf{住房商品化改
  革},1990和1992年国务院分别发布了土地出让的暂行条例和暂行管理办
法,1993—1994年开始推行住房商品化改革,土地和住房才开始成为商品。

但土地交易从一开始并没有按市场原则来设计。地源仍来自政府征收农民的土地,只给
很少的补偿(起初按被征土地农作物年产值的三到六倍计算补偿,后有增加,但多数情
况下仍然\textbf{严重违背等价交换原则})。

政府征收的土地进行初步开发后、通过\textbf{协议方式}(后房地产和商业服务业用地改
为“\textbf{招拍挂}”方式)有偿出让,\textbf{地方政府是唯一的建设用地垄断供应者}。扣除征
地补偿和拆迁费用后的土地收入、全部归政府所有。

按照早期的设计,住房改革实行\textbf{双轨并行}的体制,为中低收入居民提供\textbf{保障性住
  房},为高收入居民提供\textbf{商品住房}。但在实行过程中,地方政府的利益导向使商品
房比重越来越大,保障性住房受到挤压。2000年以后,\textbf{商品房用地}的收入越来越成为地
方政府重要的收入来源,从此开始了土地财政时期。

土地财政体制的形成有某种必然性。我国在计划经济体制下,所有资源集中控制在政府
手中,企业没有活力,居民收入被压低且增长缓慢。改革之初,为改变这种僵死局面,
政府实行了大规模的放权让利和其他市场化改革。此后,企业收入和居民收入迅速增长,
市场初步形成,经济增长加速、而\textbf{全口径政府收入(起初包括财政预算收入和政府预
  算外资金收入,后加入土地出让收入和社保基金收入)占GDP的比重
  从1978年的40.2\%直线下降到1995年的15.9\%(据国家统计局和财政部数据计算)。}在
经济逐步走向繁荣的同时,也出现了政府财政困难。\textbf{为缓解政府财政资金不足,土地出
让收入自然成为地方政府收入的一个补充来源。}

所罗门的魔瓶一旦打开,就很难让其回到瓶子里了。\textbf{伴随城市化快速发展,土地收入越
来越大,成了地方政府的第二财政。}1998年,全国国有土地出让成交价款约500亿元,相
当于当年地方财政预算收入的10\%。此后土地出让收入以不可抑制的高速度膨
胀——2001年1300亿元,接近地方财政收入的17\%;2011年3.3万亿元,相当于地方财政
收入的63\%;2021年8.7万亿元,相当于地方财政收入的78\%(均不包括在地方财政预算
内)。(笔者注:具体数据可见笔者整理的\ccref{tab:19882021})

多年来,它已成为地方政府不可缺少的巨大财源。

从2001到2021这20年,现价GDP年均增长12.4\%,现价全国财政预算收入年均增长13.4\%,
而土地出让收入年均增长23.4\%,远超过了经济增长和财政收入增长(据财政部、国土
资源部、国家统计局公布数据计算)。

计算平均地价,2001年每公顷土地出让收入143万元,2011年为978万
元,2021年为2393万元。\textbf{20年间地价上涨15.7倍。}

同期消费价格指数(CPI)只上涨58\%,工业生产者价格指数(PPI)只上涨38\%(据财
政部、国土资源部、国家统计局、华经产业研究院数据计算)。

过高的地价大幅度推高了房价,并改变了国民收入分配格局。由于不包括地价和房价,
消费者价格指数(CPI)和生产者价格指数(PPI)已不足以反映真实的通货膨胀率。

而地价和房价飞速上涨,是几方面影响因素合成的结果。

从供给侧来说,一个正常因素是\textbf{城市规模经济的溢出效应带来的城市用地和城市周边待
开发土地升值。}

从需求侧而言,\textbf{城市化快速发展造成的房地产需求上升,而土地资源相对有限,持续
  拉动了房价和地价上涨。}

第三个因素的作用可能更加突出,即长期以来\textbf{货币增长大幅度快于经济增长,货币超发
造成的多余购买力大量流向房地产,造成越买越涨、越涨越买的趋势,成了长期不破的
资产泡沫。}

第四个重要影响因素是\textbf{地方政府}的作用。对地方政府来说,\textbf{土地出让收入来得容易,
  用得方便,监管不严,透明度低,}一方面可以用来搞各种地方政府想搞但没有其他资
金来源的投资项目,既包括城市基础设施和其他公共建设,也包括政府楼堂馆所等各种
自我服务设施,另一方面还可以\textbf{增加包括“三公消费”在内的党政机关行政管理费用。}

更严重的是,\textbf{在政府垄断地源的情况下,大量的官商勾结腐败案件都与土地和房地产
  交易有关,给腐败官员带来了巨额收入,造就了一大批隐性富豪}——在种种利益驱动
之下,地方政府可以利用其独占地位,通过控制土地投放量,借“招拍挂”尽量抬高地
价,以获得最大收益。这也导致政府在资源配置中所占份额不断扩大,挤压了市场配置
资源的份额,形成与改革开放前面20年的放权让利截然相反的趋势。

{\heiti 二、今天:土地财政的4大负面影响}

由于地价房价不断上涨,土地似乎成了取之不尽用之不竭的财源。\textbf{土地财政的基础条
  件是城市化带来的效率提高,产生了巨大的溢出效应。随着土地资源商品化、资本化,
  成为巨大的财富源泉,政府的土地收入也确实在过去的经济建设中起了巨大的作用,
  推动了城市建设和基础设施条件快速发展完善,过去20年间对保持经济快速增长的作
  用不可忽视。}

但同时,土地财政体制的负面作用已经越来越大。

其一,随着地价不断上涨,推动房价迅速上升,\textbf{普通百姓购房、租房负担越来越重。}

2001年,全国房地产企业的商品住宅平均售价2017元/平米,2021年升至10396元/平米,
已是原来的5倍以上。但这远远没有反映出大城市房价的涨幅。在一线城市,房价涨了几
十倍。而此期间,城镇居民人均可支配收入的名义涨幅不到6倍。

北京市四环路到五环路间的房价目前大约在5—10万元/平米,2020年北京市私营企业职
工年平均工资9.06万元,只够负担1—2平米的房价。

目前大城市有房居民和无房居民的生活冰火两重天,天价住房让外来年轻人望而却步或
最终不得不离开。留不住年轻人的城市,未来可能是人口老化、人力资源枯竭的城市。

其二,\textbf{农民合法权益受到侵犯。}宪法规定农村土地为农民集体所有,因此\textbf{农民在符
  合国家的土地规划用途的条件下、应当有处置属于自己的土地和获得合法收入的权利。
  城市周边土地大幅度增值,是城市规模经济的溢出效应所致,不应全归农民集体或个
  人,政府有理由通过税收提取土地增值的一个合理部分用于公共目的。}但在现行制度
下,如政府以获利为目的,用征地的方式把土地收归国有,独占土地收入,实际是违反
宪法的。目前,除一些大城市周边征地有高额补偿,在很多情况下农民并未获得足够的
补偿。这对农村发展和提高农民的财产收入有严重的不利影响。

其三,\textbf{地价房价的持续畸形上涨扭曲了国民财富分配格局。}

伴随中国几十年来的经济快速发展,居民收入迅速提高,中等收入阶层不断扩大,90年
代的住房改革又把原来的公有住房以低价卖给城镇职工,使他们获得了属于自己的房产。
这些本来创造了一个改善收入分配和财富分配、缩小贫富差距的良好条件。

但过去20年,地价房价的过度上涨,在很大程度上打断了这一进程,\textbf{实际上形成了对
  新一代中低收入群体的掠夺}。中年以下的中等收入人群或者背上了沉重的房债,或者
成为永久的无房户。房贷和房租大大压缩了他们的消费能力,使他们很多人只在名义上
属于中等收入阶层,实际生活水平无法提高甚至下降。

过高的地价和房价还推高了商业、服务业和制造业的租金水平,把成本转嫁给各行各业
和消费者,而占人口少数、囤有多套住房的富裕阶层,则可以轻松凭借房价上涨使财富
快速升值。因此\textbf{土地财政实际造成了财富的逆向再分配,是导致财富占有差距持续扩
  大和国民经济各行业间苦乐不均的重要原因。}

近年来,全口径政府收入(包括财政预算收入、以土地出让收入为主的政府性基金收入、
社保基金收入、国有资本经营收入)和全口径政府支出已分别上升到GDP的35\%左右
和40\%左右,\textbf{政府配置资源的程度大体回到了改革前的水平},其中土地出让收入扮演了
重要角色。

\textbf{市场配置资源的比重与上世纪80--90年代相比、被大大压缩,出现了背离改革大方向
  的趋势。}这种情况与发达国家政府收入占比高的情况差异很大,后者占比高是因为政
府承担了大量公共服务和收入再分配的功能,而我国这样高的占比中很大部分是政府直
接参与资源配置。

其四,\textbf{地价房价持续上涨和相关的金融扩张导致了越来越大的金融风险。}

2020年,房地产业创造增加值7.3万亿元,占GDP的7\%,但\textbf{房地产相关贷款已超过67万
  亿元,占银行贷款总额的39\%,相当于GDP的66\%。2020年我国房地产企业的资产负债
  率已经高达80.7\%,居民的房贷规模也越来越大}\footnote{笔者注:2020年个人住房贷款余
  额34.44万亿元,相当于GDP的34\%。}(数据来自国家统计局和郭树清:“完善现代金
融监管体系”,2020年12月)。

\textbf{地方政府的高负债也与土地有密切关联。}这样高额的负债还不仅仅是因为\textbf{地价和房
  价}越来越高,也是因为\textbf{融资的放大效应}。因为地价不断上涨,变相鼓励了房地产
企业用外部融资囤地囤房,地方政府还可以用升值后的土地做抵押,借更多的贷款进行
新的土地开发和其他投资。\textbf{资产泡沫和金融泡沫促使经济继续脱实向虚,对实体经济
  发展产生挤压作用,并形成了巨大的风险。}

这些越滚越大的债务虽然有价格不断上涨的土地和房屋为抵押,表面上安全,但一旦地
价房价由涨转跌,就可能引发坏债连锁反应,引爆金融危机。

\textbf{综上所述,土地财政已经成为一个加剧经济结构失衡和财富分配失衡的因素,土地相
  关制度亟待改革。}天下没有不散的宴席,今年的土地收入大幅下跌说明,现有的土地
财政格局已经维持不下去了。

{\heiti 三、明天:调整政策,重振改革,化硬着陆为软着陆}

据国家统计局数据,从1990年到2021年,全国已累计销售新建商品住宅202亿平米。房地
产企业目前正在施工的房屋面积还有97亿多平米,此外,在住房商品化之前和来自非商
品房的城镇居民住宅存量估计90--100亿平米。待目前施工中的住宅完工后,全国城镇住
宅城镇居民住宅存量总共将达到380--390亿平米。

按现有城镇人口计算,人均拥有住房面积41--43平米(按42平米计),已接近发达国家水平,城镇住房建设已经接近饱和。

假设今后到2035年,城镇化率再提高10个百分点,达到75\%,城镇人均住房面积从目前
的42平米提高到50平米,城镇充其量还需要新建150亿平米住宅。未来14年,平均每年只
需新建10.7亿平米住宅,城镇住房就将完全饱和。

而过去3年,平均每年新开工住宅面积高达15.9亿平米。这意味着今后房地产业哪怕只保
持现有建设规模不再增长,未来也将出现\textbf{非常严重的住房过剩,每年都会有约1/3的
新建住房(5亿平米)卖不出去。相应地,房地产业对土地的需求量也将至少下
降1/3(实际下降幅度可能更大,因为房地产企业目前还囤有相当数量的待开发土地),
使地方政府的土地收入大幅度缩水。}

据这些情况判断,今年上半年出现的土地出让收入大幅度下降,并非短期波动,很可能
代表了土地和房地产市场走势的根本拐点。只是由于今年经济形势不佳,土地需求颓势
更加突出。

在住房和土地需求大幅度下降的情况下,未来经济会发生什么?

大致会有\textbf{硬着陆}和\textbf{软着陆}两种可能。这里先对硬着陆的情况做一些大胆推测。作
者希望这只是杞人忧天,推测错误。

\textbf{土地成交可能转向量价齐跌。}地方政府收入大幅度减少,以前那种不计成本、不算回
报、寅吃卯粮、大拆大建、为了一时的政绩大手笔投资的格局难以为继,政府消费靠土
地收入挥霍、钱权交易靠土地收入大行其道的局面也无法维持。靠政府大规模投资拉动
经济增长的模式,不再是一个现实选择。

\textbf{地价下降到一定程度,大量以土地为抵押的政府贷款将会成为不良贷款。银行的坏账率
会大幅上升。政府债券到期不能兑现的情况可能大量发生。出现较普遍的偿债危机。}

土地收入下降的直接原因是房地产业已进入衰退。今年上半年,房地产业的土地购置面
积和土地成交价款分别下降了48\%和46\%。同期商品住宅销售面积和销售额分别下
降27\%和32\%,意味着房价和销量同步进入下降期。地价下降还有一定的滞后期,但估
计坚持不下去。房屋销售和房价大幅下降将引发一系列连锁反应,会有相当多的房地产
公司倒闭,坏债大量发生。

房价降到一定程度,一些贷款买房的居民会发现与其继续还贷,还不如\textbf{直接违约,因
  为市场上的现房可能变得更便宜。}这与当年美国次贷危机的情况相仿,会使银行的处
境雪上加霜,坏账进一步增加。政府可以禁止房地产公司降价售房,但房子卖不出去,
房地产企业倒闭可能更多,对银行的打击可能更大。

\textbf{大银行有政府支持,不会倒闭,但坏债会严重影响资金周转,把影响扩散到实体经济。
中小银行将面临严重挑战,对实体经济影响可能更大,可能拖累整体经济进入较长时期
的衰退。}

\textbf{有些人也许会主张重走大规模货币放水的老路来救经济、拉增长,但可能是一个致命的
错误。因为最终需求在消费者一端,没有最终需求回升,大量增加流动性、只会加剧坏
债螺旋形攀升。}

上述情况只是一个沙盘推演,作者希望不会发生。现实情况永远比理论推演更复杂多
变。

但一旦发生上述情况,需要有理性的宏观政策应对,同时痛下决心,推进改革。如果应
对得当,有可能使硬着陆变成软着陆,可以考虑以下措施。

\textbf{延长还贷,清理坏债,收缩房地产规模,整顿金融,稳定储户,防止挤兑。}

如果发生全面的资金周转困难,货币当局可能需要释放一定量的货币维持资金周转,同
时面对企业面临的困难,货币当局可以考虑降息作为一个选项来缓解企业的负担,但同
时\textbf{必须坚持货币总量控制,坚决不搞大水漫灌。}利率与货币量的变化之间未必是线性负
相关关系,不同政策工具的效果可能有很大差异,这方面的问题还有待更深入的研究。

\textbf{财政政策不宜把重点放在扩大政府投资。保护失业者、低收入者和稳住最终需求是当务
之急。}强化失业保障,对未被失业保险覆盖的失业者(包括符合常住人口标准的失业外
来农民工)进行普遍救济,促进消费需求尽快回升,引领经济回升。

\textbf{政府收购适用的滞销房产,转为廉租房和公租房,并通过政府采购扩大保障性住房建设};
一方面减轻房地产业和银行业受到的冲击,另一方面(更重要的方面))把保障性住房
覆盖面扩大到合理水平,分步使保障性住房能够满足城镇低收入和中下收入居民的居住
需要,应覆盖未取得户籍的城镇常住人口。但整个过程必须公开透明、建立规范,防止
幕后交易、私相授受。

改革土地制度,政府征地仅限于公益性和必须的基础设施建设,并参照市场地价予以补
偿。符合土地利用规划的商业性土地需求通过市场满足,不必通过政府征地。允许集体
建设用地直接入市。允许农户闲置宅基地在不违反土地规划用途的条件下进行使用权的
有偿转让,取消对需求方身份的限制。

改革土地增值税制度,政府对土地交易中的大幅度增值部分征收适当比例的土地增值税,
用于补充公共服务和社会保障资金的不足。

政府要过紧日子,压缩不必要和不急需的政府支出,转向量入为出的理财观念,把更多
的资源留给市场配置,尽量减轻企业负担,帮助企业渡过难关。实现政府职能转变,政
府退出竞争性市场活动,转向以提供公共服务为中心任务。

多用普惠政策,少用特惠政策,对民营企业要一碗水端平,鼓励公平竞争,改善营商环
境,坚持市场化改革方向。


\section{土地财政还是土地金融?}

笔者认为相较于土地财政或者“土地、财政、金融三位一体”,还是直称土地金融为
好,这更能说明核心本质。

赵燕菁:
\begin{quotation}
  (80年代初)开发和建设深圳的故事显示,并不是只要有经济活动,就会自发形成基
  础设施;而是要政府先提供基础设施。中国城市化伟大成就背后的重要原因,就是创
  造性地发展出一套\textbf{将土地作为信用基础}的制度——“\textbf{土地财政}”,也正因如
  此,\textbf{“土地财政”乃是一种金融活动。将土地收入视作“财政收入”,暴露出传统
    经济理论对真实世界的错误观察和认识。}

  土地肯定“不具有天然的信用”,所有资产的价值都是未来收益的贴现。土地及附着
  其上的不动产也是如此。土地的信用来自附着其上的公共服务给土地带来的现金流。

  只有政府提供了重资产(笔者注:企业不愿投入的、耗资巨大、长周期、低回报等难
  见市场效益的公共服务),其他社会主体才可能以轻资产启动各类商业模式。\textbf{城市
    化就是资本不断聚集的过程}。
\end{quotation}

自70年代末深圳开始实际实施国有土地有偿使用之初,地方就采用了土地入股、利润分
成、卖楼花(房屋预售制)等方式(见\cref{sec:80tudi});而在1992年海南岛、北海
等房地产泡沫时,金融资本成分更是加重,批条、图纸、信贷、炒楼花、炒地皮、炒项
目,楼还未建,已可层层转手数次,击鼓传花;之后一直延续低价或零低价出让工业用
地换取财政收入,再将之用于七通一平等基建;1998--2004年不计入赤字的长期建设国
债9100亿,房地产开始作为超发货币蓄水池,并成为中国面对经济危机的固有路
径;2004年招拍挂8·31大限抬高土地转让价格,“\textbf{从此开发商(消费者)之间为买地
  而展开竞争,政府(生产者)坐享生产者剩余。土地成为地方政府最主要的资本来
  源}”;2008年4万亿投资刺激形成的大量地方融资平台,以地方政府信用背书,以土
地作为信用和货币支撑,以债养债,则进入了完全体的土地金融时代,地方政府债务也
节节攀升,面临难以偿付、甚至事实上破产的系统性风险。(
见\cref{sec:92tudi},\cref{tab:19882021})


赵燕菁对土地金融成就比较肯定:
\begin{quotation}
  中国之所以能“和平崛起”,原因恰恰离不开“土地财政”这种融资模式,这使
  得\textbf{中国不必借由外部征服,就可以获得原始资本积累所必需的“初始信用”。高效
    率的资本生成,缓解了原始资本积累阶段的信用饥渴,确保了中国经济成为开放的
    和在全球化中获利的一方。}因此,即使处于发展水平较低的城市化初始阶段,中国
  也比其他任何国家更希望维持现有国际经济秩序,更有动力推动经济全球化。“土地
  财政”的成功,确保了“和平崛起”成为中国模式的内置选项。
\end{quotation}

但赵其实也知道\textbf{大多数城市的衰退是必然}:
\begin{quotation}
  城市发展的衰退与企业经营的失败在原理上是相同的。能够获得资本,完成市政建设
  的城市很多;\textbf{但能创造足够的收益,持续运转下去的城市却很少。}当城市化转入运
  营阶段,问题的焦点不再是资本的多少,而是经常性收入是否足以覆盖一般性公共服
  务支出。如果无法获得足够的现金流性收入,之前所有的投资就会转变为无法偿还的
  债务。正如能盖起厂房的企业很多,但能赚钱的企业没几家。\textbf{城市的道理相同,能
    建设起来的城市很多,但能通过运营最终获利的却很少。}
\end{quotation}

土地金融更深层的内容是政府财政赤字货币化,依赖土地成为货币蓄水池。而中国超发
货币之路,至少从上世纪70年代末的“洋跃进”就已经开始。

不管采用何种形式的超发货币,其实质就是广义政府负债如雪球般滚动,寅吃卯粮;财
富极具分化、阶级固化、分配不公。发展期看起来欣欣向荣,却是埋下一个随着时间增
长而不停膨胀的超级炸弹,待到下行期时爆发……



对大国大城批判,所谓自由,正是国家充当……原始积累白手套,为服务。



关于土地金融的内在逻辑,笔者认为可通过赵燕菁的书籍文章\cite{dajueqi}窥得不少较为
广泛和直接的认识,但笔者并不赞同其对“土地金融”的一些辩护意见。



% 中国的城市化其实在历史演变过程中逐渐囊括了“四化”:城镇化、工业化、信息化和农业现代化。

如果我们将“城市”视作一组公共产品(安全、教育、交通、绿化……)的集合,实际
上也就从制度的角度给出了城市的定义:城市是一组通过空间途径赢利的公共产品和服
务。或者按照规划师的习惯,将城市定义为“公共产品和服务赖以交易的空间”。

奥尔森认为,诸如“和平”这样的公共产品(如城墙)是垄断的竞争者(常驻的匪帮)
出于自私的目的强加给居民的。而居民通过纳税获得保护,进而与常驻的匪帮分享“和
平秩序”带来的巨大好处

在城市的制度原型里,政府乃是以空间(行政边界)为基础提供公共产品的“企业”。

城市政府同企业家一样,其核心工作就是发现并设计最优的公共产品提供模式。具体的形式体现在不同的公共产品和各式各样的收费模式上[如税收以及诸如建设-运营-转让(BOT)等基础设施建设模式]。公共产品(比如消防、路灯、治安)由于通常无法以排他的方式提供给付费的消费者,因此,必须以向特定空间使用者收费的方式提供。

城市化的过程,就是一个城市(更准确地说是公共服务水平)从谱系的低端向谱系的高端移动的过程。

地方政府的土地收入在第一阶段本质上是“金融”,只有完成招商引资后获得持续性税收才来到第二阶段成为“财政”。

在传统经济中,一次性投资的获得主要是通过过去剩余的积累。

工业化和城市化的启动,都必须跨越原始资本的临界门槛。一旦原始资本(基础设施)积累完成,就会带来持续性税收。这些税收可以再抵押,再投资,自我循环,加速积累。

计划经济遗留下来的这一独特制度,使土地成为中国地方政府巨大且不断增值的信用来
源。不同于西方国家抵押税收发行市政债券的做法,中国土地收入的本质,就是通过出
售土地未来的增值(70年),为城市公共服务的一次性投资融资。中国城市政府出售土
地的本质,就是直接销售未来的公共服务。如果把城市政府视作一个企业,那么西方国
家城市是通过发行债券来融资,中国城市则是通过发行“城市股票”来融资。(超发货
币,债券)

因此,在中国,居民购买城市的不动产,相当于购买城市的“股票”。这就是中国城市
的积累效率远高于土地私有化国家的重要原因。也正是依靠这一做法,中国得以一举完
成工业化和城市化两个进程的原始资本积累。

反倾销历来是发达国家对付其他发达国家的经济工具,现在却被用来对付中国这样的发
展中国家;以前从来都是城市化发展快的国家出现资本短缺,完成城市化的国家出现资
本剩余,现在却反过来了,是中国向发达国家输出资本。在这些“反经济常识”的现象
背后,实际上都有赖于“土地财政”融资模式的超高效率。

在中国,“土地财政”的本质是“融资”,其替代者必定是另一种对等的信用。而要把税收变为足以匹敌土地的另一种信用基础,就必须突破一个重要的技术屏障——以间接税为主的税收体制。中国的税负水平并不低,其增速远超GDP。2012年完成税收收入11万亿元,同比增长了11.2%。在此基础上,继续大规模加税的基础根本不存在。

《福布斯》杂志根据边际税率,曾连续两次将中国列为“税负痛苦指数全球第二”。
数据显示,2011年,我国全部税收收入中来自流转税的收入占比达70\%以上,而来自所得税和其他税种的收入合计占比不足30\%。来自各类企业缴纳的税收收入占比更是高达92.06\%,而来自居民缴纳的税收收入占比只有7.94\%。如果再减去由企业代扣代缴的个人所得税,个人纳税创造的税收收入不过占2\%。2012年个税起征点上调后,2013年个人直缴的比例更低。这就是税收高速增长,居民税负痛感却不敏感的重要原因。

任何一种改革,如果想成功,前提都是纳税人的负担不能恶化。如果按照某些专家的建议,通过直接增加财产税等新的地方税种来补偿土地收入损失,可能会引发社会骚乱。这种非帕累托改进,对任何执政者而言,都是巨大的风险。

如果你把不同城市的房价视作该“城市公司”的股价,你就会发现中国“城市公司”股
票市场的增长速度和中国经济的增长速度十分一致,一点也不反常,并通过免交财产税
的方式分红。由于土地市场的融资效率远高于股票市场,因此,很多产业都会借助地方
政府招商,以类似搭售(tie-in sale)的方式变相通过土地市场融资。中国大量企业是
在土地市场而不是在股票或债券市场完成融资的。

通过抵押或直接出让“平衡用地”,是地方政府基础设施建设融资和招商引资的主要手段。这也间接反驳了那些认为“土地财政”抑制了实体经济的指责。

美元通过与大宗商品,特别是石油挂钩,重新找到了“锚”,使得美元可以通过大宗商品涨价,消化货币超发带来的通货膨胀压力。欧元试图以碳交易为基准,为欧元找到“锚”,但迄今仍未成功。日元则基本上以美元为“锚”,它必须不断大规模囤积美元,其货币超发,只能依靠美元升值消化。[12]

而“土地财政”却给了人民币一个“锚”。土地成为货币基准,为中国的货币自主提供
了基石。“锚”就是不动产:不动产升值,货币发行应随之上升,否则就会出现通货紧
缩;货币增加,而不动产贬值,则必然出现通货膨胀。\improve[inline]{实际的通货
  膨胀,到底高不高?土地蓄水池的边界到底几多?}

同日本一样,囤积的大量美元是人民币信用的另一个来源。美元升值,人民币就可以多
发。如果人民币贬值,不动产就必须升值,否则,就会导致通货膨胀。因此,在美元贬
值的背景下,打压房价,就是打压人民币。房价下跌,必定导致通货膨胀,其后果可能
远比我们大多数人想象的巨大、复杂。之所以没有发生通货膨胀,乃是因为房价上升导
致全社会信用规模膨胀的速度比货币更快。[13]

任何一种改革,如果想成功,前提都是纳税人的负担不能恶化。这种非帕累托改进,对
任何执政者而言,都是巨大的风险。

历史上,直接税的征收比间接税的征收要艰难得多。发达经济体为了建立起以直接税为基础的政府信用,无不经历了漫长痛苦的社会动荡。

“土地财政”只是专门用来解决城市化启动阶段原始信用不足问题的一种特殊制度。随着原始资本积累的完成,“土地财政”也必然会逐渐退出,并转变为更可持续的增长模式。

当城市化进入新的发展阶段,就要及时布局不同模式间的转换。在这个意义上,放弃“土地财政”绝不是简单的财政改革,而是一场剧烈的社会改革。如果这场改革发生在城市化完成之后,可能是再一次的制度升级;而如果发生在城市化完成之前,很可能会导致巨大的社会风险。模式的过渡,没有简单的切换路径可循,必须经过复杂的制度设计并花费几代人的时间。在还没有找到替代方案之前就轻率抛弃“土地财政”,是不明智的。

真正用来满足需求并成为经济稳定之锚的,是保障房供给。\improve[inline]{经济适
  用房、报障房到底归了谁?}

由于住房最终可以上市,因此土地(及附着其上的保障房)就可以成为极其安全有效的抵押品。通过发行“资产担保债券”(covered bonds)等金融工具,利用社保、养老金、公积金等沉淀资金获得低息贷款,只需政府投入(贴息)少许,就可以一举解决“全覆盖”式保障房的融资问题。[6]近年来,社保基金、养老基金和公积金进入股票市场的呼声不绝于耳。但低迷的收益和有限的规模,使得股票市场难以满足保值的需要。如果我们拓宽视野,就会发现,房地产(特别是保障房)市场其实是比股票市场更大、更安全的资本市场。[7]

“先租后售”模式,看似解决的是住房问题,实际上却意味着“土地财政”的升
级——都是以抵押作为信用获得原始资本,地方政府很难主动实施。。以往“土地财
政”是通过补贴地价来直接补贴企业,而“先租后售”保障房制度,则是通过补贴劳动
力来间接补贴企业。2008年以后,制约企业发展的最大瓶颈已经不是土地,而是劳动
力。

“土地财政”的另一个后果就是“空间的城市化”并没有带来“人的城市化”——城市到处是空置的住宅,农民工却依然在城乡间流动。现在很多研究都把矛头指向户口,似乎取消户籍制度就可以在一夜之间消灭城乡差距。取消户籍制度,如果不涉及背后的公共服务和社会福利,等于什么也没做;但如果所有人自动享受公共服务和社会福利,那就没有一个城市负担得起。
要想取消户籍制度,就必须改间接税为直接税。户籍制度同公共产品付费模式密切相关。改变税制,如前所述,制度风险较大。

今天因为缺少财产而无法拥有城市户籍的非城市人口,明天也一样会因为缺少财产而无
法成为合格的纳税人。坠入“中等收入陷阱”国家的一个共同特征,就是大量进入城市
的居民不为城市公共产品付费。城市贫民窟同小产权房一样,本质都是为了\textbf{逃避为公
  共产品付费}。一旦坠入“中等收入陷阱”,城市化就会半途而废,除非将这些不为公
共产品付费的人口也作为城市人口。

现在财政界有一种普遍的看法,认为中国的税制结构已经到了非调整不可的地步。理由
是,间接税使每个购买者成为无差别的纳税人,无法像直接税那样,通过\textbf{累进制}使高收
入者承担更多的税负来调节贫富差距。

正确的做法,不是回到土地私有的原始状态再启动城市化(这样只能让城市周围的农民
获得城市化的最大好处),而是要利用这一制度遗产,通过企业补贴、保障房“先租后
售”等制度,让远离城市地区、更大范围内的农民,一起参与原始资本的积累,共同分
享这一过程创造的社会财富。

房价的上涨对冲了其他产品价格上涨的压力。在某种意义上,正是因为土地的超级通货
膨胀,才避免了整个经济的超级通货膨胀。一旦房价暴跌,土地就会大幅贬值,信用就
会崩溃,从而引发金融动荡。

防止土地大幅贬值的关键,在于防止房价暴跌。防止房价暴跌的有效办法,就是控制供
给规模。唯有大幅降低商品房供地规模并切断信贷从银行流向房地产的路径,才能减少
土地信用在市面上的流通,从而避免资产价格暴跌。

保障房直接和真实需求挂钩,如果有了这个“锚”,只要保障房需求(预先登记并实际居住)是真实的,政府的“土币”就不会超发。中央政府则可以像调节银行准备金那样,通过调节商品房与保障房的比例,来调节市场上信用的多寡,进而调控经济的发展速度。

有了保障房这个“锚”,我们就可以像调整银行的货币准备金那样,调节商品房和保障房的比例,从而控制地方政府信用发行规模——如果我们希望经济增速快一点,就可以提高商品房相对保障房的比例;反之,则可以降低商品房的“发行规模”。宏观调控工具因此会更加丰富,经济政策就可以更加精确。

“土地财政”是一把双刃剑,它既为城市化提供了动力,也为城市化积累了风险。放弃是一种容易的选择,但找到替代模式却绝非易事。没有十全十美的模式。“税收财政”演进了数百年,导致了世界大战、大萧条、次贷危机、主权债务等无数危机,其破坏性远超过“土地财政”,但西方国家并没有轻言放弃。它之所以仍然被顽强地坚持、探索,盖因其\textbf{积累模式的内在逻辑}。

深圳经济并没有因为无地可卖而“不可持续”,深圳“土地财政”已经悄然退出。王建表示:“深圳的实践表明,我们可能根本无须为不治自愈的‘病’吃药。”

1992年以后,中国的市场化进程发展到了把土地变成资本的阶段,土地财政才能大行其道,
既然土地的市场化是推动中国工业化、城市化与经济高增长的起源,当这一块资源用尽的时候,就是依靠土地财政、信用增长方式转型的时候,目前东部发达地区已经没有土地资源可用,新的财政体系建设是必然趋势。

中国土地资本的高增值,其主因是在新全球化时代外需所带来的中国经济高增长,实现了极高的利润增长率,从而使包括土地在内的所有生产要素不断增值,因为说到底,只有利润才能决定资产的价值。

城市也不是自发集聚的结果,而是经济活动的参与者(最初是“常驻的匪帮”)有目的地设计并提供的。

将政府视为一类生产公共产品的企业后,接下来就会涉及政府垄断的话题。在仅考虑私人产品生产的完全竞争范式下,价格是无数消费者和无数生产者共同面对的一个市场结果,没有单个消费者或生产者可以决定价格。而一旦将公共产品的生产者——政府引入市场,价格就必然会被“有形的手”操控,完全竞争范式就会崩溃。于是,经济学家造了一个叫作“市场失灵”的概念来兼容这一现实。

在新古典微观经济学中,价格竞争基于相同产品,尽管这些产品的生产区位不同,但由于运费和地租等空间变量被忽略不计,因此,这些产品仍被视为无差异的。在这种完全竞争范式下,采用空间收费方式的公共产品(服务)自然无法定价。笔者将这类相互具有替代性的不同产品之间的竞争,定义为“哈耶克竞争”。这种竞争与迪克西特-斯蒂格利茨模型(D-S模型)的效用替代相似,但更加接近霍特林模型所描述的区位竞争,本章将其称为哈耶克—霍特林竞争。只要向哈耶克—霍特林竞争模型引入空间要素,就可以发现公共产品(服务)根本不会像新古典经济学预期的那样形成“垄断”(详见第十四章)。

按照竞争范式,城市公共服务完全可以以竞争的方式由市场提供。只要把政府视作一个企业,就可以很好地解释公共服务是怎样提供的。由于公共服务收费的特殊性,拥有空间定价权(税收和地租——根据租税互补原理,税收属于广义的地租)的政府,就成为这类服务最有效率的提供者。一旦把空间要素引入竞争,政府就像企业一样,通过空间竞争(也就是所谓的“用脚投票”)在市场上提供公共服务,公共服务的市场定价问题就可以迎刃而解。

在现实中,公共产品竞争自古就有,只是“土地财政”使得这一现象在中国地方政府之间体现得更为淋漓尽致。这一竞争最早由美国经济学家蒂伯特提出,张五常在“中国县域竞争”实证中对其进行了检验。笔者在更早的一篇文章《对地方政府行为的另一种解释》中,也提出过类似的观点,拥有土地收益权的地方政府可以像“企业”一样竞争。
只有在新的竞争范式里,土地的“价值—价格”才可以得到完整的解释。

批评文章认为“逻辑与事实不符”:第一,“扣除城市建设开发和对农民的征地补偿,地方政府并不赢利”;第二,“地方政府对土地市场与规划权的垄断,极大地增加了地方债务违约风险”。

政府卖地(使用权)的一次性收益本质上是一种融资——“土地金融”,用于“城市建设开发和对农民的征地补偿”。政府卖地(使用权)的持续性收益来自未来的税收——“土地财政”。只要能够覆盖“城市建设开发费用和对农民的征地补偿”,“土地金融”的使命就已经完成。

方政府“在无地(使用权)可卖的情况下,只能走向名义上的破产财政”。这类批评看似直观,但却一点也不“专业”,笔者当时对马光远博士的回答,现在可以原封不动地用来回答这个问题:

在金融不发达的时代,基础设施建设的规模,取决于“过去”劳动剩余的积累。但如果借助金融体系,则可以抵押“未来”的收益,突破基础设施建设的资金瓶颈。这就是为什么发达国家政府出现了这么多次金融危机,负的债要远比中国地方政府多,却没有一个国家干脆立法,禁止政府融资,以防“走向名义上的破产财政”。

在计划经济时代,虽然政府财政没有破产,但基础设施建设却“欠账累累”。可以说,“负债”是现代经济的主要特征。\textbf{负债越多,表明政府信用越好}。

土地财政”本质上也是一种融资模式,它极大地扩张了地方政府的信用,盘活了“未来”的资产,提高了政府的负债能力。

有强大的信用才有资格“负债”,负债高反过来也可以证明信用好。“土地财政”就是
\textbf{帮助地方政府创造信用的工具}。至于“债务”的使用是否合理、是否赢利,那不是\textbf{工具
本身}的问题,而是\textbf{如何使用工具}的问题。


那种认为“政府要获得基础设施建设的外溢收入,可以租用农民的土地,而不必把农村
的土地国有化”的观点乃是纸上谈兵,土地的升值永远只会落到\textbf{产权人手中}而不是佃农
或租客手中。\textbf{只有当公共服务提供者和土地所有者一致时,地租才会公平地落在创造地
租的人手中。}

严崇涛从1970年到1999年,先后担任新加坡财政部、贸工部、总理公署等多个政府部门的常务秘书,参与了新加坡的快速崛起。他在《新加坡发展的经验与教训:一位老常任秘书的回顾和反思》一书中认为:

没有基础设施的土地只不过是沙土罢了。因为对土地进行的基础设施投资主要由国家进行,我们认为,公共建设的道路、地铁系统、学校等基础设施带来的大部分土地升值收益,应该归属于国家。因此,当政府征用土地从事公共建设如组屋、工业厂房的时候,赔偿标准应基于毫无基础设施的原始未开发土地的价格。买主必须向市建局保证将进一步开发土地,并支付发展费用。市建局每年会在政府公报上刊登不同土地的发展收费,征收该费用则是为了抵消政府建公路和排水系统等部分的开支。如果没有征收发展费用,投资者或地主不就能免费从这个国家的投资中受益吗?

土地国有化可以使公共服务(比如道路)外溢出来的价值完全体现在地价上,从而避免了征税的困难。开发前土地国有化就成为效率更高的价值捕获工具。这也就是严崇涛宣称“我们工业用地和公共住房发展的基石,就是土地征用法令”的原因。

批评文章写道:“近年来征地成本日益高涨,有的地区达到每亩地上百万元”,“强制
征收具有较高的成本,带来了与农民的矛盾和大量上访事件,影响了社会秩序与政府信
誉”。这些也是近年来公共项目成本暴增,进而导致城市政府负债骤增、固定资产投资
速度降低、经济增速下滑的重要原因。这也从相反方向证明,当年国家强制征收能力对
建立“现代社会秩序”多么重要。\improve[inline]{正因为土地金融不可持续,寅吃
  卯粮,才会这样。而不是因为土地金融发展不足。}

批评文章写道:“近年来征地成本日益高涨,有的地区达到每亩地上百万元”,“强制征收具有较高的成本,带来了与农民的矛盾和大量上访事件,影响了社会秩序与政府信誉”。这些也是近年来公共项目成本暴增,进而导致城市政府负债骤增、固定资产投资速度降低、经济增速下滑的重要原因。这也从相反方向证明,当年国家强制征收能力对建立“现代社会秩序”多么重要。

土地国有化必定效率低”乃是主流经济学臆造出来的结论。新加坡在独立前,国有土地
面积约占国土总面积的60\%;新加坡独立以后,通过强行征用,使国有土地面积占比
于2006年达到90\%。以色列国有土地面积占比更是高达93\%(樊正伟和赵准,2009),
其城市土地和农村土地都是“国有的”,其余大部分也是归地方政府所有。真正影响土
地使用效率的,是实际制度的设计,而非字面上的“国有”还是“私有”。


由于农地收益流极低,农地作为信用基本没有太大价值,自然无法用来抵押。

那些“法律不允许”抵押的农地,乃是位于城郊的土地。由于靠近城市,这些农地并非
纯粹的“农地”,上面附着了城市外溢出来的公共服务。城市的基础设施大部分是政府
建的。如果法律允许这些土地改变用途,就等于把城市的公共财富免费送给郊区“地
主”,还记得前文提到美国人建运河时想避免的情况吗?“拥有产权意味着这些土地所
有者即使袖手旁观,也能获得土地升值带来的利润……出于这个原因,法院和议会/经常
介入……”政府公共服务的收益流失,基础设施的建设成本无法收回,就会造成发展中
国家普遍面临的公共产品“搭便车”困局。\improve[inline]{对大国大城的抨击}

发债模式马上就会因为无力还贷出现违约;卖地(使用权)模式则不用补偿地价下跌的损失,因而不会出现违约。“融资平台”模式绕开了国家发债的限制,虽然可以扩大信用的规模,但也会带来违约的风险。这些年地方政府积累的巨额债务,恰恰不是卖地(使用权)带来的。也正是因为“土地财政”这一直接融资模式,中国地方政府的违约概率要比成熟的市场经济国家低得多。


\textbf{一旦开征财产税,中国土地市场的资本属性也会像其他国家的一样显著萎缩。}

对于信用货币而言,真正需要担心的不是通货膨胀,而是通货紧缩——一旦信用消失,货币随之消失,结果一定是伴随通货紧缩而来的大萧条。

\textbf{过去四十余年,中国的高速城市化在很大程度上是建立在大规模的货币创造基础上的。}
而这些货币的创造离不开信用的创造。将土地作为一种信用,各国皆然,但像中国这样
将土地市场发展成为一个巨大的资本市场,并使之成为货币信用主要来源的却再也找不
到第二个国家。这不能不说是“土地财政”制度的又一个创新。

在很多学者看来,没有通货膨胀是因为货币都“涌向住房和土地市场”,而不理解现在
市场上的大多数货币本身就是由“住房和土地市场”生成的。若房地产市场泡沫破裂,
我们不会看到多余的货币决堤而出,而是会看到市场上的流动性突然干涸。大家在批判
房地产泡沫时,其实也在享受资产泡沫带来的好处。没有这些泡沫,就不可能创造出足
够的货币,泡沫崩溃时,前所未有的商业繁荣就会烟消云散。


真正将“土地财政”风险“升维”的,是各种以政府信用作为担保的融资平台。这些融资平台以地方政府垄断的土地一级市场作为隐性担保,一旦房地产价格下跌,就会触发债务违约。

第一,不要轻易将土地融资工具从风险较低的直接融资,升维到风险较大的间接融资;

第二,不要像杰克逊那样简单地“去杠杆”,因为“去杠杆”的副产品——消灭货
币——给经济带来的危害,远大于债务违约本身。(不可能的童话)

在传统的经济学理论中,只要总体收支平衡,就是可以的。

但如果按照两阶段增长模型,“缺口”和“剩余”不能相抵。这是由于依照金融学原理,
资本性剩余与现金流性剩余的\textbf{“量纲”不同}。前者是通过金融创造的,是将未来收益
贴现进行资金的跨期配置,在本质上属于\textbf{向未来借贷}的行为。而现金流性支出在时间
上的持续性,导致再大的资本存量也难以弥补\textbf{流量缺口},由于更多的资本性支出意味
着更高的债务及利息,\textbf{如果未来创造的现金流不足以偿还债务,就会陷入债务危机。}欧
洲的债务危机,本质上就是用资本性收入覆盖“现金流缺口”的结果。欧盟有些国家并
没有创造出足够的收入,却想维持和其他国家相同的福利水平,于是就用举债获得的资
本性收入弥补养老金缺口,由此陷入债务循环……


在这一点上,国家(中央政府)、城市(地方政府)、企业和家庭的道理相同。用资本
性收入弥补现金流缺口的最终结果,无不指向庞氏循环。

资本缺口不能用现金流剩余来平衡,反之亦然。笔者称之为“不可替代规则”。

所谓的金融危机,不是资本型增长阶段(Ri0-Ci0=Si0)没有完成,而是运营型(劳动型)增长阶段(Rik-Cik=Sik)没有完成,使得第三个公式不成立,导致从资本型增长向运营型(劳动型)增长过渡的过程中陷入增长崩溃。

所谓泡沫,就是对未来现金流的估值,反映在贴现倍数上。在实物商品货币时代,按照
格雷欣法则会出现“劣币驱除良币”的现象;但在信用货币时代,则是“高泡沫驱除低
泡沫”。美国凭借其制度、军事和文化,在债市、股市等资本市场上,将更多的未来收
益贴现,给同样的现金流以更高的估值,创造并支撑全球最大的资本市场泡沫,助力其
经济独步全球。现在,中国在美国之外创造了一个具有更高贴现倍数的资本市场。中国
房价行情网站数据显示[2],2018年北京楼市的售租比为55年,即买房出租要660个月才
能收回成本。而在国际上,房地产的售租比一般界定为17.25年。


在以企业信用为支撑的股市和债市等资本市场,中国的发展同发达国家相比长期处于落后状态,这导致高风险、长周期但高收益的产业根本无法获得昂贵的资本。但在以政府信用为支撑的房地产市场,由于地方政府是这一市场中的核心“企业”,其快速发展在很短的时间内创造了大量的“廉价”资本。在中国目前的资本结构下,面对巨额的高风险投资,只有“廉价”的政府资本才能承受巨大的风险,因此中国经济只有政府“重资产”——“国进”,民营企业才可能“轻资产”——“民进”。也即在政府投资成功之后,民营经济的“轻资产”再嫁接到政府的“重资产”上,因此政府的公共投资是企业私人投资的基础。无论是芯片的研发还是高铁的建设,都是同样的道理。我们可以把民营经济比作各种各样的电器,民营企业无须自建煤矿、电厂和电网,只需插到政府这个“插座”上,就可以轻资产运营。

这也就很好地解释了中美贸易摩擦中,美国一定要将中国划分为“非市场经济国家”以
及针对“中国制造2025”的原因。中国“政府+土地金融”的制度设计挑战了欧
美“私企+股票债券”的市场模式,中国的资本市场(以房地产市场为主)挑战了美国的
资本市场(以股票市场为主),但无论是政治制度还是城市化所处的阶段,都决定了美
国无法效仿中国的做法,因此美国只能依靠所谓市场经济国家的“准则”迫使中国“自
断双臂”,放弃自己的“有形之手”。发达国家地方政府的财政收入几乎完全靠税收支撑,其资本生成能力远远低于中国地方政府。它们可能补贴一些制造业企业,但补贴的规模完全不能与中国地方政府相匹敌。

。而“市场—商品”经济必须使用货币分工。在中国历史上市场经济长期发育不良,一个根本的原因就是货币不足,所以必须依赖出口顺差换取分工所必需的货币。而发达国家,无一例外都是通过货币的信用化,摆脱了流动性不足的约束。改革开放的成功,特别是过去十几年的经济增长,很大程度上是因为土地金融创造的巨大信用为货币从商品货币转向信用货币创造了条件。信贷成为货币生成的重要途径。



在城市化需要巨大融资的资本型增长阶段,因为有足够的信用,贷款需求不成问题。但当城市化从资本型增长转向运营型增长时,贷款需求迅速减少,如果此时家庭、企业和政府三个部门同时“去杠杆”,银行资产负债收缩,结果就是货币供给的减少。如果再加上国际局势动荡导致的顺差生成货币减少正好与“去杠杆”同步,经济就会面临更大的萎缩风险。

在中国历史乃至世界历史上,货币不足导致的社会动荡屡见不鲜。这才是城市化转型的最大风险。

对于现代经济而言,转型成功的前提,就是要解决伴随转型而来的流动性不足问题。这就意味着必须有另外新的商业模式开始进行资本型增长。


1750年是人类财富增长的转折点,在此之前世界人均GDP长期处于停滞状态;在此之后,世界财富出现陡然增长。[3]是什么导致了人类从“大停滞”走向“大增长”?笔者认为这是缘于制度创新,即:通过金融革命,人类发明了一种能将未来收益贴现过来,弥补经济增长资本缺口的制度,从此摆脱了依靠过去剩余积累的桎梏,经济增长的原始资本积累阶段得以迅速完成,不同商业模式从传统的序贯增长转入现代的平行增长。

如果说常态增长的特征是以私人部门的运营型增长为主,那么危机增长的特征就是以公
共部门的资本型增长为主,例如“\textbf{新基建}”。不过比“新基建”更准确的提法应当
是“资本深化”或“公共服务深化”。\improve[inline]{这里的公共服务部门究竟是什么意思?}

第二,要有足够的资本供给。如果一个国家不能创造足够的“资本”(比如所谓的“不
发达经济体”),无论有多少闲置、过剩的生产要素,都无法将其动员起来。转向危机
增长的前提,就是要绕过已经坍塌的资本废墟,重建新的资本渠道——通过公共服务领
域向市场大规模注入流动性。只要市场上有充足的流动性,那些搁浅的资产就会重新漂
浮起来。

\textbf{真正导致经济危机的原因,是随内外投资需求下降带来资产负债表缩表并发的流动性枯竭。}

\textbf{系统性风险(所有市场主体都选择持有货币而不消费)}

虚拟财富与真实财富之间的关系意味着降息将导致经济中虚拟财富(资本)的估值提高,占社会总财富的比重提高,而真实财富(劳动)的估值下降,占社会总财富的比重下降。二者的比值一旦越过临界点,资本形态财富的增长速度就会超过实体形态财富的增长速度,实体经济的货币就会被逆向吸入资本市场,从而进一步推高资本的价格。

在现代经济中,社会总财富是虚拟财富和真实财富的加总……随之而来的是股价和房价
飙升。一旦虚高的资本估值得不到\textbf{未来真实财富}的支持,泡沫就会破灭,并引发更大的
流动性危机。理论上,货币体系乃是不同流动性组合的信用。一旦位于顶端的高流动性
货币不再被信任,取而代之的次级货币会导致社会商品分工水平下降。危机下的货币宽
松,不过是饮鸩止渴。因此,危机货币供给的核心不是数量而是方式——通过什么渠道
注入流动性比注入多少流动性更重要。

在债务不变的条件下,修复资产负债表有两种方式:一种是提高资产的估值;另一种是
增加资产的\textbf{数量}。前者是常态增长阶段的主要工具,后者则适用于危机增长阶段。

在常态货币供给渠道失灵的危机状态下,央行可绕过商业银行系统,通过直接对基础性战略资产(BSA)进行投资向市场投放货币。史正富先生将这个新的货币发行通道称为“长期资本注资便利”(long-term capital funding facility)渠道。

央行通过投资基础性战略资产生成基础货币与通过财政发债生成基础货币最大的不同,就是前者的抵押品是投资的基础性战略资产市值,而后者的抵押品是政府的未来税收。前者直接创造货币需求,后者本质上还是需要得到资本市场的支持。由于在危机中政府的税收能力下降,通过财政赤字发债将会导致赤字扩大,融资成本也会更高。而基于新增基础性战略资产的货币,无须以税收增加作为信用的基础。只要成本足够低,即使是较低的回报,也一样可以生成正的信用。

央行通过投资基础性战略资产生成基础货币与通过财政发债生成基础货币最大的不同,
就是前者的抵押品是投资的基础性战略资产市值,而后者的抵押品是政府的未来税收。
前者直接创造货币需求,后者本质上还是需要得到资本市场的支持。由于在危机中政府
的税收能力下降,通过财政赤字发债将会导致赤字扩大,融资成本也会更高。而基于新
增基础性战略资产的货币,无须以税收增加作为信用的基础。只要成本足够低,即使是
较低的回报,也一样可以生成正的信用。\improve[inline]{可通过 财政赤字货币化理论与实
  践 刘思源 了解下货币政策。}

在现代政府职能分工中,财政主要是负责\textbf{运营型增长},体现的是\textbf{现金流的收和支};
央行主要是负责\textbf{资本型增长},体现的是\textbf{资本的收和支}。

由于这一功能是传统央行所没有的,因此需要成立专门的国家“自然资源银行”。其职
能是:(1)代表央行收购、持有、转让国家基础性战略资产;(2)负责资产的保值、
增值;(3)建立相应的基础性战略资产交易市场,对自身代央行持有的基础性战略资产
进行定价。

例如,建立国家建设用地指标交易市场,使得央行持有的新增或改进的耕地可以在市场
上定价;类似地,可以建立水资源交易市场、大气质量交易市场等。通过这样的措施为
以基础性战略资产为信用的货币提供流动性。

判断经济是否恢复的一个关键指标,就是失业率。央行的货币政策要从盯住通货膨胀率
转向盯住失业率。一旦劳动力实现就业,危机增长的条件(要素闲置)将不复存在,增
长也要随之切换回常态增长。

不仅是公路,理论上所有的公共资产(铁路、电网、供水网格、通信等系统)都可以通过此路径实现大规模资本深化(而不是建设更多的无效资产)。

央行也可以启动“先租后售”计划,通过收购开发商烂尾/违章项目,将其纳入基础性战略资产计划,将住房低成本租给无房的新就业者,一定年限后出售,收回资本。收购居民断供的物业,将其纳入“共有产权”计划,居民回购或交易时,收回资本。协助家庭部门完成城市化阶段的重资产化。


在农户自愿的基础上,大规模收购弃耕的耕地,通过完善耕作基础设施,建立农村基础
公共服务(借鉴东亚地区的农业合作组织),建立从播种到收割、从融资到销售的一系
列服务,使自耕农成为轻资产的国家专业农户。(佃农?)


央行还应该尽快建立国家“最终雇佣者”(employer of last resort, ELR)计划,以
最低工资和基本社保(the basic public sector wage, BPSW)为就业条件,大规模提
供公共就业岗位,将货币尽快滴灌到市场中最微观的细胞——家庭上。政府为就业兜底
还可以和企业救助相结合,通过将我国劳动合同法中企业对员工义务的社会化,恢复劳
动力市场的弹性,减轻企业困难时期的压力。就业岗位计划可以由政府提出,也可以由
劳动力本人提出后经劳动部门审核。(金融资本低劳动力高资本,没有受到冲击,反而
受到扶持。)

今天,大量过剩产能、闲置资产、积压库存、失业劳动力给中国经济带来了巨大挑战,但也为中国启动新一轮危机增长提供了机会和条件。

一个国家实现资本型增长对应的条件是资本密集,完成运营型增长对应的条件是劳动密
集。资本最便宜的美国和劳动最便宜的中国成为在这一轮全球化进程中获利最多的两大
赢家,与此对应的是,美国的劳动和中国的资本则成为这一分工模式的受害者。由于美
国拥有资本和劳动的不同群体之间的贫富差距加大,巨大的国内分裂和阶级冲突必然投
射到美国与其他国家的关系上。

换句话说,一旦经济进入现代增长,传统的“积累—消费”增长两难就会转变为“\textbf{资
  本—劳动}”(或称“\textbf{虚拟—实体}”)增长两难。宏观经济政策本质上都是通过\textbf{改
变贴现倍数}来改变两者间的分配关系。所谓“中性”的宏观政策并不存在——有利于资
本增长的政策,就会损害劳动增长;有利于劳动增长的政策,就会牺牲资本增长。所有
的经济政策其实都意味着要对两个增长阶段进行权衡取舍——是发展\textbf{资本密集型}产业,
还是发展\textbf{劳动密集型}产业。

中国的土地金融拥有一个非常重要的特点:\textbf{它是世界上唯一与美元周期脱钩的大型资
  本市场。特别是中国土地市场的资本估值(售租比)比世界上最强大的股票市
  场——美国股市的资本估值(市盈率)更高,泡沫更大。}按照格雷欣法则,廉价的资
本赋予了中国资本密集型企业更大的竞争优势。

总体而言,人类社会一直处于资本不足的状态,资本剥削劳动是世界经济史的主线。

只要中国不重蹈当年日本打压房地产市场的覆辙(比如大规模征收财产税),仅这个市场本身就可以为资本生成提供源源不断的信用。

对于资本密集型产业而言,需要比对手更廉价的巨额资本,因此宏观经济政策就要求宽货币/财政(贷款生成货币)、降息、加税(货币增信)、强货币(将人民币作为储备货币)、货币国际化(输出资本)、低关税(输出货币)、高杠杆、通货膨胀(有利于债务人)……所有这些政策的后果,都会提高虚拟财富在总财富中的估值。而对于劳动密集型产业而言,则要求货币中包含更多现金流,以免在与虚拟财富(未来收益贴现)的兑换中吃亏。由于现代货币是以信用为基础的,信用越高,货币中虚拟财富的比重就越高,劳动在交易时就越吃亏。而提高货币中现金流的含量,就需要紧货币/财政、加息、减税、弱货币(输出产品)、商品国际化(输出商品)、高关税(保护市场)、低杠杆、通货紧缩(有利于债权人)……所有这些政策的后果,都会提高真实财富(现金流)在总财富中的估值,如图6-5所示。

\todo[inline]{解决方案多是滑稽的或者邪恶的}

贴现倍数越高的政策,越有利于资本;贴现倍数越低的政策,越有利于劳动。加息有利
于实体(劳动),降息有利于虚拟(资本);加税有利于实体(虚拟),减税有利于虚
拟(资本);贬值有利于实体(劳动),升值有利于虚拟(资本)……

其中房地产市场创造的信用扮演了决定性角色。低息货币环境中孵化出大量新科技公司,它们的商业模式开始从以往的追随美国变为与之并驾齐驱。

过去20余年,中国房地产市场是世界上唯一脱离美元周期的大型资本市场。它为巨量人
民币提供了强大的信用,使中国成为唯一在低息资本(远低于银行利息)上能和美国一
拼的经济体。

正是依靠房地产市场,我国渡过了1997年和2008年两大金融险滩。

房地产市场同其他资本市场一样,是否崩盘取决于是否可以维持正的信用冗余。特朗普依靠减税为资本市场补充现金流,显著地增加了股票市场的信用冗余;而中国如果此时给房地产市场加税,将进一步减少其所剩无几的信用冗余。

债券市场主要是由中央政府信用创造的,房地产市场主要是由地方政府信用创造的,股票市场则主要是由企业信用创造的。

止血,就是要立即停止和减少不能马上带来现金流的新投资。现在有人一提到阻止经济下滑,就想到以前最有效的一招——固定资产投资。这些越多,维持其运转流失的现金流规模就越大,固定资产投资会通过折旧、付息等创口持续地给地方政府财政放血。
根据李嘉图等价定理,成功的融资背面就是令人痛苦的偿还要求。\textbf{硬约束和财经纪
  律}(好厉害,哈哈,狗头),要成为问责地方政府的头号优先依据。

造血,就是尽一切可能扩大、扶持现金流性收入。其中,最主要的就是企业税收。中国
政府绝对不能放弃对产业的支持尤其是在中国的股票市场没有超过发达国家的股票市场
之前。在私人资本无力进入的领域,国有资本必须带头进入。这不是因为国有资本更有
效率,而是因为中国资本市场的主体是土地,资本市场的性质决定了地方政府的市场参
与者角色。无为政府根本不可能把这些信用传递给市场。国有资本不是与民争利,而是
开疆拓土,打下市场后,再由民营经济跟进。\improve[inline]{是的,不是与民争利,
  而是给企业家,鱼肉百姓。}

对于人口增长减慢的城市,要迅速停止一切不能带来现金流的政绩型固定资产投资。扶贫、对口支援、边疆和民族政策等,都应将是否能增加受益对象的现金流作为衡量成败的标准;对于人口增长强劲的城市,则要放开约束(包括人口和土地限制),加大能带来增量现金流的固定资产投资;对于超级明星城市(比如深圳、苏州等),还可以考虑行政范围扩大、行政等级升级等手段,鼓励其全速增长。

输血,就是增大地方政府在现金流分配中的比例。。加强向地方政府的转移支付,甚至调整央地分税的比例,都要立即提上议事日程。当地方政府的资产负债表上升为中美贸易战的主战场,在关键时刻,甚至要不惜投入国家信用给地方政府背书。

这并不意味着应当给予地方政府无差别的支持,支持应当向那些能创造最多现金流的城市倾斜。所有城市都必须将创造现金流而不是GDP增长作为核心的经济指标。输血不是目的,造血才是目的。
地方政府为房地产市场构筑的第一道防火墙,就应当是迅速建立全覆盖的保障体系,全面接管资本市场现在还在承担的“住”的职能。

利率是货币的价格,反映的是货币“供不应求”的程度,是货币供需关系的晴雨表。

该文认为,“正是房地产抽干了货币,导致货币大量流向国企和地方融资平台,而实体
经济,主要是民营中小微企业,严重失血”。殊不知,由于流通中的住房价格同时也为
非流通住房定价,因此房地产不仅没有抽干货币,相反,它还为货币生成提供了巨大的
信用——相当一部分货币本身就是由房地产的信用生成的。没有房地产,那些货币根本
就不会存在,民营企业和中小微企业也就根本无血可失。\improve[inline]{对中小企
  业,利弊几何可能需要专业客观分析,笔者能力不足。}

按照“货币数量理论”,高贴现倍数创造出的货币可以将原来因收益低、风险高而无法投资的商业活动也卷入市场分工。相反,低贴现倍数意味着资本不足,也意味着开展高风险商业活动不合算。货币的真实贴现倍数k给所开展的商业活动提供了临界贴现倍数,货币泡沫越大,商业活动的临界贴现倍数也就越高,研发、创业这类高风险商业活动的信用冗余Δk值也就越大。

一个合理的推测就是,高贴现倍数资本市场虽然不利于已有的实体经济,却特别有利于研发、创新和创业这类高风险投资活动。

笔者从来没有说依靠高房价就可以多发行货币,而是认为房地产只有存在“\textbf{广泛的供不应求}”,才具有参与货币创造的资格。
\textbf{对于货币生成而言,高房价不是问题,高房价可能带来的流动性丧失才是最大的问题。}

只要中国地方政府的信用不断转移给企业,中国企业就可能在与美国企业的竞争中笑到最后。

如果我们把“泡沫”和“杠杆”等同于真实贴现倍数,就会知道,在现代经济中,这两个概念对资本来说是同义词。

资本市场管理有两个目标:一是降低融资成本,这通常意味着更低的利息(债市)、更
高的市盈率(股市)、更高的资产价格(房地产市场);二是降低市场风险,这通常意
味着更高的利息(债市)、更低的市盈率(股市)、更低的资产价格(房地产市场)。
如何管理这两个相反的目标?一个主要的手段就是在不同的资本市场之间重新配置信用,
这就涉及资本市场的信用结构分层假说。

根据“格雷欣效应”,在资本市场上“劣信用”会驱除“良信用”,具有高贴现倍数的
资产B会通过收购具有低贴现倍数的资产A的方法来套利。最终的结果是,具有低贴现倍
数的资产A被迫转变为具有高贴现倍数的资产B[1],整体经济随之“脱实向虚”。

自诞生之日起,“土地财政”便在咒骂声中顽强而丑陋地生长,褒贬不一,毁誉参半,
在帮助中国迅速崛起为世界级的创新体和高技术玩家的同时,也在不断掏空实体经济;
在帮助更多的人卷入货币分工的同时,也造成了财富分布的巨大落差;在帮助货币减轻
对美元信用依赖的同时,也使中美在经济上开始分离、在政治上开始对立;高信用带来
的高贴现倍数也导致中国面对和美国类似的问题——整个经济不断“脱实向虚”[3],出
现“泡沫”和“高杠杆”等症状。资本市场存在的系统性风险,像达摩克利斯之剑一样,
高悬在高速增长的中国经济之上。

这是最好的时代,也是最坏的时代。当以千年为尺度的时候,我们很难意识到历史的拐
点。洪水来袭,避险财富最后会流向能够提供最大临界贴现倍数的货币。要想在危机中
幸存,要做的就是比对手跑得更快。

在信用货币制度下,加息、降息会产生和实物货币制度下完全不同的财富分配效应:若
加息(去杠杆),财富会从虚拟部门向实体部门转移,实现资本支持实体经济发展的政
策效果;降息(加杠杆)则会导致财富从实体部门向虚拟部门转移,出现资本剥削实体
经济的现象。市场利率低于影子利率的幅度越大,资本越便宜,实体经济就会越萎缩,
虚拟经济就会越膨胀。在全球化分工的格局下,通过不断降低利率,资本大国就可以实
现财富从劳动大国(实体经济)净转移。而负利率更是意味着哪怕社会总财富缩水(资
本不创造新增财富),虚拟经济也可以通过从实体经济“吸血”获得更多的财富。

在中国,股市同房市、矿产开发和征地拆迁一起,构成了社会财富向少数人转移的四大渠道,急需加以改进。

一方面,从效率的角度来看,中国目前仍处于工业化阶段,实体经济是中国立足世界的核心竞争力和比较优势,这就要求中国资本市场能够有效地为实体经济配置资源。

通过资本市场分配财富对社会均富具有不可替代的作用。寄希望于通过二次分配(高遗产税、所得税、财产税)转移财富来缩小贫富差距,已被发达国家的实践证明无效。
鉴于资本市场在财富分配上的巨大效应,资本市场制度设计的一个核心原则,就是资本
市场的最大收益应当归公众,而不是造就越来越多的超级富豪。\improve[inline]{多
  么的天方夜谭啊,比直接税更甚。}

由于离开村集体的原成员不能将其拥有的集体经济份额转让、变现,因此新加入村集体的成员自然无法获得有法律保障的集体份额。就投资而言,除非投资者是本村成员,否则投资可能处于产权不受法律保护的风险之中,而原集体成员利用集体身份几乎没有违约风险,这种不对称导致资本投入方和资本接收方互相信任的成本极高。

未来农村制度改革的方向,既不是彻底私有化,把孤立的农户抛向“市场”,也不是重回集体经济,再次剥夺农户的经营自主权;而是构建一种公共服务和个体经济分成机制,恢复基层公共服务,使每个农户都可以以最低的成本快速响应市场需求发出的信号。

如果简单地赋予农民更大的土地权利(比如“农地入市”和“同地同权”),结果就一定是农地不断转为生产率更高的非农地。这是因为,农村土地、住房等资产的价值,是农业未来收益的贴现值,如果农业收益少,农地、农宅入市价值也就很小。

要按照现代企业标准改造农村集体组织,最重要的就是使传统农村资产获得合法的资本接口。
要实现这一目标,就要在农村地权分置的基础上分别资本化(股权化)——开放资本市场,使城市资本在满足国家政策目标的前提下,通过收购集体股份获得村庄运营权;个人和企业可以通过购买股权加入村集体,并享有宅基地和耕地份额,分享村集体公共服务以及相应股份的分红。这种政策设计在确保资本进入农业,从而将农村经济接入高价值分工链条的同时,可以避免“田底”的资本化导致“田面”所有者产权的丧失(比如土地兼并)。升级改造后的乡村集体所有制,通过开放的“田底”权益与引入国家和社会资本,完成农业基础设施的升级,重建农村公共服务。


,农村改革的制度设计方向,不应是把集体产权分解为更小的私有产权,而是推动集体产权向现代产权结构转变,在农村重建公共服务,完成农业发展所需的重资产投资与建设。农村制度改革应抛弃批判土地私有化的教条,建立强大、开放的集体平台,通过接入非农产业链,完成农业的资本化,将建立农村的自我“造血”机制置于乡村振兴的核心。




彭波

https://www.guancha.cn/pengbo/2023_02_25_681392_1.shtml

改革开放以来,中国还没有遭遇过一次真正的完整意义上的经济周期,整个社会存在较强的投机倾向,很多地方政府与企业似乎早就忘记经济是有周期的,在繁荣过后是会出现危机的。此次如果再次出现饮鸩止渴的结果,未来要调控房地产及金融市场就将更加困难。而房价涨得越高越快,未来的危机也就越严重越猛烈。


所有资本主义经济制度的国家都必须是负债增长,而其本质上是国家和普通民众向资本负债来维持运转。在同样的资源约束下,产生了同样的经济成果,但所有权和收益权因为债务关系而发生了变化。因为资本控制了社会的核心资源与生产要素。
举例说明。
比如现在的房地产。我们在现有的资源条件下,修建了这么多房子,这些房子最终归结为各种资本出资修建。也就是说这个社会本质上具有修建这么多房子的经济资源条件,但是社会民众必须负债来消费。
比如,中国各级政府负债修建了大量的基础设施。这些基础设施是现有的经济资源条件下修建出来的,但是中国各级政府本身并不对这些经济资源拥有所有权,因此只能负债修建。
这就是私有化条件下市场配置资源的必然结果。国家和社会的绝大部分经济资源配置给了资本,因此其所有权和收益权归资本,但是使用权可以用债务的形式配置给国家和社会民众。

平台企业私有化必然导致不可逆的贫富分化,因此要防止平台企业被私人资本所挟持。
是社会主义还是资本主义,不取决于是否对资本征税,而取决于是否对资本拥有所有权。
(何其乐观)

税制的设计者终于明白房地产的本质是为地方政府提供的公共服务定价,对应的是地方事权。

实践表明,资本价格高并不是坏事,崩盘才是坏事。相对而言,资本廉价的经济可以抵抗更强烈的冲击,承受更大的亏损,并在对手资本耗尽后抄底对手的资产和市场。

但央行的这些货币政策工具只能用来创造基础货币。不通过银行贷款,基础货币就不能真正进入市场流通。一旦房地产市场萎缩,市场缺少优质的抵押物,贷款生成货币的渠道就会阻塞,基础货币再多也只能堆积在银行系统里。换句话说,货币政策只能在基础货币供不应求时起作用,一旦市场因缺少抵押物消失而出现货币需求不足,再宽松的货币政策也创造不出足够的货币。美联储面对的就是这样的困局:在逆全球化和新冠肺炎疫情冲击下,美元需求急剧萎缩,美联储大规模扩表与美元需求萎缩同步,美联储不得不通过前所未有的Taper回收滞留在银行系统内的流动性。

将“田底”和“田面”分开,大体上有两种路径:

一种路径是自下而上,与业主选择物业公司类似,由拥有承包权(“田面”)的村民自主选择公共服务的最优提供者。另一种路径是自上而下,国家收回土地所有权(“田底”),并由其在市场上为村民选择最优的公共服务提供者。

仅是腾讯,其他数据平台企业也都具有类似的特征,即低利润、低分红、低纳税,但却
有高估值、高市盈率。

世界货币与主权货币不同,它无须由“锚”资产背书,无须由税收驱动,也不要与特定商品挂钩,驱动它的唯一原因,就是流动性本身(使用范围最大)。多数交易使用世界货币来定价就是其价值所在。货币作为所有交易的价格集,所包含的价格信息本身就是其价值所在。

主权政府之所以看上去可以无限量负债,关键在于主权货币发行者总是可以通过提高其未来税收的估值来稀释当前的债务。在现代信用货币的生成机制下,发行主权货币的本质就是政府以税收为标的直接融资(股权融资),货币就是政府发行的“股票”。政府可以通过增发货币(稀释股权)向货币持有者(持币者)强行融资,政府赤字货币化其实是通过“增资扩股”由所有持币者(股东)分摊债务。




%%% Local Variables:
%%% mode: latex
%%% TeX-master: "../main"
%%% End:
