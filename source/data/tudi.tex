\chapter{土地金融总结}

空间的生产不是资本主义生产方式自然而然的再生产,而是被构想的和深思熟虑的结果。
空间具有政治性,它是政治统治的工具,空间的生产是现代主义国家的政治战略。 “空间
与政治国家的关联比曾经的领土与民族国家的关联更牢固。它不仅被生产力、生产关系
和所有权生产;而且它是一种政治产品,具有行政和残暴统治性的产品、由政治国家上层
统治关系和战略决定的产品。并且,这不是在某一政治国家范围内,而是在国际和全球范
围内,在全球国家体系范围内的生产。”

从列斐伏尔的观点来看,资本主义生产方式生产了它自己的空间,在空间的
生产的进程中资本主义的生产方式改变为 “国家生产方式”。同时,空间不仅成了生产
力要素、生产关系和所有权关系要素,还完全成为政治性的,政治性的空间在当今资本
主义社会中成为主导性的。空间的政治性主要表现为,首先,空间是意识形态的,它是
社会技术专家治国制的表象。其次,空间是实践的,是政治权力的工具和手段。最后,
空间是战略性的,它从属于政治目标,被纳入了剩余价值的生产。

在过去资本主义的长期发展中,土地曾被视为封建地主阶级的残余而被忽视,建筑
业的重要性曾远远不及钢铁生产、制糖工业等。而现代资本主义生产实践则相反,土地
进入了生产关系再生产的范畴,在新资本主义的结构性生产关系中处于中心地位。显
而易见,政府的住房规划就促进了这种以 “不动产” 的动产化为特征的空间的生产。住
房建筑与土地不可分割,土地构成住房价值的一部分,于是,被分割的一块一块土地成
了空间性的产品。“因而,资本投资在房地产部门中找到了一个避难所,一个补充性和
互补性的剥削领域。......在一些国家中,比如西班牙和希腊,房地产部门已经成为由相
当熟悉的政府干预形式所构成的经济的一个必不可少的组成部分。在其他国家,比如
日本,求助于房地产部门来弥补通常的生产-消费循环带来的困境并增加利润,这已是
稀松平常之事:甚至对房地产部门进行事先预测和规划。”[48] 基于作为整体的空间的
生产的新资本主义的增长战略为了实现空间的生产而进行的空间动员开始于土地,然
后,这种动员延伸到地下空间和地上空间,从地下的能源、原材料资源、地面的土地资
源到地上的被建筑或各种需要分隔出来的空间容量都被赋予了交换价值,作为整体的
空间成了一个更庞大的 “商品世界”。过去我们买卖或租赁的是土地,而今是房屋、楼
层、公寓、停车场、游泳池等各种各样的可交换可计量的碎片化的空间。 “空间成为商
品,把空间中的商品特征发展到了极致。”

)
 ,他说: “如果我没记错的话,在 20 世纪 60 年代初,有一个关于空间
战略的高层决策;不是欧洲的,不是欧洲的空间战略,而是一个法国的空间战略。换句
话说,它描绘着中心化,巴黎的中心化。巴黎必须变成像鲁尔或英国的巨大都市一样
的财富和权力的都市核心。这是关于空间政策的政治决定。”[50] 因此,中心化需要更
高的政治理性,也就是需要国家或者叫作都市理性以更有效的方式,也就是在全球范
围和整体上生产空间,通过这种空间秩序的中心性驱逐边缘要素,强有力地集中财富、
行为手段、知识、信息和文化。同时,因为中心化是一种政治决定,它还需要技术和知
识的代理人,也就是规划者为中心化提供不证自明的合理性。

工业化带来的破坏——史无前例的大规模地重建,即在整个社会的规模上进行重
建。这一过程的推进,伴随着许多越来越深刻的矛盾。现存的生产关系被推广、扩张了;
在同时把农业和都市的存在整合起来的过程中,它们又带来了一些新的矛盾:一方面,
拥有某些未知的权力的决策中心已经形成,因为这些中心集中了财富、压迫性的权力
和信息;另一方面,对过去的城邑的破坏,使得各种形式的隔离成为可能,各种社会
力量无情地将人们在空间中分隔开来。由此,一种广泛意义上的社会关系解体了,而
与之相伴随的,则是和所有制关系密切相关的那些关系,被集中化了。”[51] 总体来说,
资本主义国家和政治权力主持整合历史城市和农业,把地下、地面和地上的空间以及
世界范围的空间作为整体进行规划,为寻求日益稀缺的能源、水、光等资源而被重组。
在这个过程中形成的都市空间既是统一的又是分离的,都市空间被分割和分隔成彼此
分离又相互叠加的异常复杂的空间碎片,国家和政治权力保证碎片化的都市空间相互
联系,同时,国家和政治权力正是通过历史城市的碎片化和中心化建构来保证都市空
间的统一性。

当国家和政治权力占据它所生产的空间时,日常成了政治建筑屹立其
上的土壤。权力处心积虑地联合技术和实证知识,小心翼翼地维持着日常生活的连续
性,把都市社会伪装成具有虚假透明性的 “抽象空间”,结果是,这层神秘的面纱把都
市社会的日常生活笼罩在恐怖主义之中, “对社会成员来讲,到处弥漫着恐怖,暗藏着
暴力,压力来自四面八方,只能通过超人的努力来避免或转移这种压迫;每个成员都
是恐怖分子,因为他们都想掌权;因此,无须有一个独裁者;每个成员都自我背叛和自
我惩罚;恐怖不能被定位,因为它来自四面八方,来自每一件事; ‘系统’ (如果能被称
为 ‘系统’ 的话)掌控着每个单独的成员,并使每个成员服从整体,也就是,服从一个
战略,一个隐藏的结局,这些目标除了掌权者外无人知晓,也无人质疑。”[

在法国就存在着一个过于庞大的中
心,这就是法国的首都巴黎。巴黎作为决策和舆论中心统治、剥削着分布在巴黎周围
的从属性和被等级化的空间,从而在法国内部建立起了一种新殖民主义,形成了 “超发
达、超工业化、超都市化的地区” 与欠发达和贫困状况日益加剧地区的不平衡发展的矛
盾。同时,他也指出,在现代世界里, “边缘” 具有多重含义。首先,边缘在广义上包括
资本主义生产方式下被剥夺生产工具的世界无产阶级。狭义上来讲包括世界范围内的
不发达国家特别是前殖民地国家。其次,在资本主义国家内部,边缘指那些远离中心的
区域。比如,法国的布列塔尼,大不列颠的爱尔兰、威尔士和苏格兰,意大利的西西里
岛和南部地区等。再次,边缘指城市的边缘地区、城郊的居民等。最后,边缘还指那些
社会和政治的边缘群体,特别是青年和妇女、同性恋者、绝望的人、“精神错乱” 的人、
吸毒者等[15] 。中心和边缘的矛盾不仅仅表现为单方面的中心对边缘的控制和剥削,以
及中心和边缘发展不平衡的加剧。同化和同质化的过程必然伴随着激烈的反抗,中心
越倾全力控制和剥削边缘,边缘对中心的反抗和违反就越激烈,中心越连续和无限地
控制和剥削边缘,边缘对中心的反抗和违反就越持久和永恒。另一方面,国家资本主
义和国家把城市作为财富、决策、信息和空间组织的中心,伴随着中心的饱和、资源的
匮乏等问题的出现,城市发展逐渐显现出衰退迹象,从而使中心化危机显露出来并不
断扩展,甚至恶化。美国的都市化进程最迅猛,相应的城市问题和城市危机也最先暴
露出来。“美国资本主义曾经面临极度痛苦的两难境地:是应该牺牲城市(纽约、芝加
哥、洛杉矶等)并在别处组建决策中心(这是件很难的事)
 ,还是应该通过投入巨大的
资源来保留这些城市,即使是美国社会自己所能支配的资源的总和也在所不惜。”[16]

总之,国家资本主义的都市化进程生产了中心、边缘及其矛盾,中心的衰落和中心
与边缘矛盾的加剧引发的城市现象和城市危机使城市成为资本主义矛盾表现最激烈的
场所。列斐伏尔认为,资本主义抽象的都市空间中质与量的矛盾、交换价值与使用价
值的矛盾、为非生产性消费和生产性消费进行的空间的生产之间的矛盾、暂时与稳定
的矛盾、都市理性统治下抽象空间的意识形态化等矛盾是中心和边缘矛盾的征兆,同
时也是其原因和结果。



对大国大城批判,所谓自由,正是国家充当……原始积累白手套,为服务。


关于土地金融的内在逻辑,笔者认为可通过赵燕菁的书籍文章\cite{dajueqi}窥得不少较为
广泛和直接的认识,但笔者并不赞同其对“土地金融”的一些辩护意见。
\begin{quotation}
  中国城市化伟大成就背后的重要原因,就是创造性地发展出一套\textbf{将土地作为信用基
    础}的制度——“\textbf{土地财政}”,也正因如此,\textbf{“土地财政”乃是一种金融活动。
    将土地收入视作“财政收入”,暴露出传统经济理论对真实世界的错误观察和认识。}

  只有政府提供了重资产(笔者注:企业不愿投入的、耗资巨大、长周期、低回报等难
  见市场效益的公共服务),其他社会主体才可能以轻资产启动各类商业模式。\textbf{城市
    化就是资本不断聚集的过程}。
\end{quotation}
% 中国的城市化其实在历史演变过程中逐渐囊括了“四化”:城镇化、工业化、信息化和农业现代化。

如果我们将“城市”视作一组公共产品(安全、教育、交通、绿化……)的集合,实际上也就从制度的角度给出了城市的定义:城市是一组通过空间途径赢利的公共产品和服务。或者按照规划师的习惯,将城市定义为“公共产品和服务赖以交易的空间”。

奥尔森认为,诸如“和平”这样的公共产品(如城墙)是垄断的竞争者(常驻的匪帮)
出于自私的目的强加给居民的。而居民通过纳税获得保护,进而与常驻的匪帮分享“和
平秩序”带来的巨大好处

在城市的制度原型里,政府乃是以空间(行政边界)为基础提供公共产品的“企业”。

城市政府同企业家一样,其核心工作就是发现并设计最优的公共产品提供模式。具体的形式体现在不同的公共产品和各式各样的收费模式上[如税收以及诸如建设-运营-转让(BOT)等基础设施建设模式]。公共产品(比如消防、路灯、治安)由于通常无法以排他的方式提供给付费的消费者,因此,必须以向特定空间使用者收费的方式提供。

城市化的过程,就是一个城市(更准确地说是公共服务水平)从谱系的低端向谱系的高端移动的过程。

地方政府的土地收入在第一阶段本质上是“金融”,只有完成招商引资后获得持续性税收才来到第二阶段成为“财政”。

在传统经济中,一次性投资的获得主要是通过过去剩余的积累。

工业化和城市化的启动,都必须跨越原始资本的临界门槛。一旦原始资本(基础设施)积累完成,就会带来持续性税收。这些税收可以再抵押,再投资,自我循环,加速积累。

计划经济遗留下来的这一独特制度,使土地成为中国地方政府巨大且不断增值的信用来
源。不同于西方国家抵押税收发行市政债券的做法,中国土地收入的本质,就是通过出
售土地未来的增值(70年),为城市公共服务的一次性投资融资。中国城市政府出售土
地的本质,就是直接销售未来的公共服务。如果把城市政府视作一个企业,那么西方国
家城市是通过发行债券来融资,中国城市则是通过发行“城市股票”来融资。(超发货
币,债券)

因此,在中国,居民购买城市的不动产,相当于购买城市的“股票”。这就是中国城市
的积累效率远高于土地私有化国家的重要原因。也正是依靠这一做法,中国得以一举完
成工业化和城市化两个进程的原始资本积累。

反倾销历来是发达国家对付其他发达国家的经济工具,现在却被用来对付中国这样的发
展中国家;以前从来都是城市化发展快的国家出现资本短缺,完成城市化的国家出现资
本剩余,现在却反过来了,是中国向发达国家输出资本。在这些“反经济常识”的现象
背后,实际上都有赖于“土地财政”融资模式的超高效率。

中国之所以能“和平崛起”,原因恰恰离不开“土地财政”这种融资模式,这使得中国不必借由外部征服,就可以获得原始资本积累所必需的“初始信用”。高效率的资本生成,缓解了原始资本积累阶段的信用饥渴,确保了中国经济成为开放的和在全球化中获利的一方。因此,即使处于发展水平较低的城市化初始阶段,中国也比其他任何国家更希望维持现有国际经济秩序,更有动力推动经济全球化。“土地财政”的成功,确保了“和平崛起”成为中国模式的内置选项。

[16]周其仁对此批注道:“其实中央政府更依靠税收财政,地方政府更多的是靠土地财政。”

之所以没有发生通货膨胀,乃是因为房价上升导致全社会信用规模膨胀的速度比货币更快。[13]

土地财政”虽给中国经济带来诸多好处,但也引发了许多问题。这些问题不解决好,很可能会给整个经济带来巨大的系统性风险。

第一,“土地财政”必定使得不动产变成投资品。
“土地财政”的本质是融资,这就决定了土地乃至为土地定价的住宅必定是投资品。

第二,拉大贫富差距。

“土地财政”不仅给地方政府带来巨大财富,同时也给企业和个人快速积累财富提供了通道。没有机会投资城市不动产的居民与早期投资城市不动产的居民的贫富差距迅速拉开。房价上涨越快,贫富差距越大。房地产如同股票,会自动分配社会增量财富。正是这一功能,阻碍了不同社会阶层上下流动的渠道。

第三,占用大量资源。

如果说中国经济“不协调、不平衡、不可持续”,房地产市场首当其冲。同虚拟的股票甚至贵金属不同,以不动产为信用基础的融资模式,会超出实际需求制造大量只有信用价值而没有真实消费需求的“鬼楼”甚至“鬼城”。为了生产这些信用,需要占用大量土地,消耗掉本应用于其他发展项目的宝贵资源。资本市场就像水库,可以极大地提高水资源的配置效率,灌溉更多的农田。但是,如果水库的规模过大并因此而淹没了能真正带来产出的农田,水库就会变为一项负资产。

第四,带来金融风险。
土地“净收益”已经成为很多企业,特别是地方政府信用的基础。一旦房价暴跌,如此规模的抵押资产贬值将导致难以想象的金融海啸。广泛的破产不仅会摧毁地方政府的信用,而且会席卷每一个经济角落,规模之大会使中央财政无力拯救。


在中国,“土地财政”的本质是“融资”,其替代者必定是另一种对等的信用。而要把税收变为足以匹敌土地的另一种信用基础,就必须突破一个重要的技术屏障——以间接税为主的税收体制。中国的税负水平并不低,其增速远超GDP。2012年完成税收收入11万亿元,同比增长了11.2%。在此基础上,继续大规模加税的基础根本不存在。

《福布斯》杂志根据边际税率,曾连续两次将中国列为“税负痛苦指数全球第二”。
数据显示,2011年,我国全部税收收入中来自流转税的收入占比达70%以上,而来自所得税和其他税种的收入合计占比不足30%。来自各类企业缴纳的税收收入占比更是高达92.06%,而来自居民缴纳的税收收入占比只有7.94%。如果再减去由企业代扣代缴的个人所得税,个人纳税创造的税收收入不过占2%。2012年个税起征点上调后,2013年个人直缴的比例更低。这就是税收高速增长,居民税负痛感却不敏感的重要原因。

任何一种改革,如果想成功,前提都是纳税人的负担不能恶化。如果按照某些专家的建议,通过直接增加财产税等新的地方税种来补偿土地收入损失,可能会引发社会骚乱。这种非帕累托改进,对任何执政者而言,都是巨大的风险。





%%% Local Variables:
%%% mode: latex
%%% TeX-master: "../main"
%%% End:
