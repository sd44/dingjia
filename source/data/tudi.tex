\chapter{土地金融总结}

空间的生产不是资本主义生产方式自然而然的再生产,而是被构想的和深思熟虑的结果。
空间具有政治性,它是政治统治的工具,空间的生产是现代主义国家的政治战略。 “空间
与政治国家的关联比曾经的领土与民族国家的关联更牢固。它不仅被生产力、生产关系
和所有权生产;而且它是一种政治产品,具有行政和残暴统治性的产品、由政治国家上层
统治关系和战略决定的产品。并且,这不是在某一政治国家范围内,而是在国际和全球范
围内,在全球国家体系范围内的生产。”

从列斐伏尔的观点来看,资本主义生产方式生产了它自己的空间,在空间的
生产的进程中资本主义的生产方式改变为 “国家生产方式”。同时,空间不仅成了生产
力要素、生产关系和所有权关系要素,还完全成为政治性的,政治性的空间在当今资本
主义社会中成为主导性的。空间的政治性主要表现为,首先,空间是意识形态的,它是
社会技术专家治国制的表象。其次,空间是实践的,是政治权力的工具和手段。最后,
空间是战略性的,它从属于政治目标,被纳入了剩余价值的生产。

在过去资本主义的长期发展中,土地曾被视为封建地主阶级的残余而被忽视,建筑
业的重要性曾远远不及钢铁生产、制糖工业等。而现代资本主义生产实践则相反,土地
进入了生产关系再生产的范畴,在新资本主义的结构性生产关系中处于中心地位。显
而易见,政府的住房规划就促进了这种以 “不动产” 的动产化为特征的空间的生产。住
房建筑与土地不可分割,土地构成住房价值的一部分,于是,被分割的一块一块土地成
了空间性的产品。“因而,资本投资在房地产部门中找到了一个避难所,一个补充性和
互补性的剥削领域。......在一些国家中,比如西班牙和希腊,房地产部门已经成为由相
当熟悉的政府干预形式所构成的经济的一个必不可少的组成部分。在其他国家,比如
日本,求助于房地产部门来弥补通常的生产-消费循环带来的困境并增加利润,这已是
稀松平常之事:甚至对房地产部门进行事先预测和规划。”[48] 基于作为整体的空间的
生产的新资本主义的增长战略为了实现空间的生产而进行的空间动员开始于土地,然
后,这种动员延伸到地下空间和地上空间,从地下的能源、原材料资源、地面的土地资
源到地上的被建筑或各种需要分隔出来的空间容量都被赋予了交换价值,作为整体的
空间成了一个更庞大的 “商品世界”。过去我们买卖或租赁的是土地,而今是房屋、楼
层、公寓、停车场、游泳池等各种各样的可交换可计量的碎片化的空间。 “空间成为商
品,把空间中的商品特征发展到了极致。”

)
他说: “如果我没记错的话,在 20 世纪 60 年代初,有一个关于空间
战略的高层决策;不是欧洲的,不是欧洲的空间战略,而是一个法国的空间战略。换句
话说,它描绘着中心化,巴黎的中心化。巴黎必须变成像鲁尔或英国的巨大都市一样
的财富和权力的都市核心。这是关于空间政策的政治决定。”[50] 因此,中心化需要更
高的政治理性,也就是需要国家或者叫作都市理性以更有效的方式,也就是在全球范
围和整体上生产空间,通过这种空间秩序的中心性驱逐边缘要素,强有力地集中财富、
行为手段、知识、信息和文化。同时,因为中心化是一种政治决定,它还需要技术和知
识的代理人,也就是规划者为中心化提供不证自明的合理性。

工业化带来的破坏——史无前例的大规模地重建,即在整个社会的规模上进行重
建。这一过程的推进,伴随着许多越来越深刻的矛盾。现存的生产关系被推广、扩张了;
在同时把农业和都市的存在整合起来的过程中,它们又带来了一些新的矛盾:一方面,
拥有某些未知的权力的决策中心已经形成,因为这些中心集中了财富、压迫性的权力
和信息;另一方面,对过去的城邑的破坏,使得各种形式的隔离成为可能,各种社会
力量无情地将人们在空间中分隔开来。由此,一种广泛意义上的社会关系解体了,而
与之相伴随的,则是和所有制关系密切相关的那些关系,被集中化了。”[51] 总体来说,
资本主义国家和政治权力主持整合历史城市和农业,把地下、地面和地上的空间以及
世界范围的空间作为整体进行规划,为寻求日益稀缺的能源、水、光等资源而被重组。
在这个过程中形成的都市空间既是统一的又是分离的,都市空间被分割和分隔成彼此
分离又相互叠加的异常复杂的空间碎片,国家和政治权力保证碎片化的都市空间相互
联系,同时,国家和政治权力正是通过历史城市的碎片化和中心化建构来保证都市空
间的统一性。

当国家和政治权力占据它所生产的空间时,日常成了政治建筑屹立其
上的土壤。权力处心积虑地联合技术和实证知识,小心翼翼地维持着日常生活的连续
性,把都市社会伪装成具有虚假透明性的 “抽象空间”,结果是,这层神秘的面纱把都
市社会的日常生活笼罩在恐怖主义之中, “对社会成员来讲,到处弥漫着恐怖,暗藏着
暴力,压力来自四面八方,只能通过超人的努力来避免或转移这种压迫;每个成员都
是恐怖分子,因为他们都想掌权;因此,无须有一个独裁者;每个成员都自我背叛和自
我惩罚;恐怖不能被定位,因为它来自四面八方,来自每一件事; ‘系统’ (如果能被称
为 ‘系统’ 的话)掌控着每个单独的成员,并使每个成员服从整体,也就是,服从一个
战略,一个隐藏的结局,这些目标除了掌权者外无人知晓,也无人质疑。”[

在法国就存在着一个过于庞大的中
心,这就是法国的首都巴黎。巴黎作为决策和舆论中心统治、剥削着分布在巴黎周围
的从属性和被等级化的空间,从而在法国内部建立起了一种新殖民主义,形成了 “超发
达、超工业化、超都市化的地区” 与欠发达和贫困状况日益加剧地区的不平衡发展的矛
盾。同时,他也指出,在现代世界里, “边缘” 具有多重含义。首先,边缘在广义上包括
资本主义生产方式下被剥夺生产工具的世界无产阶级。狭义上来讲包括世界范围内的
不发达国家特别是前殖民地国家。其次,在资本主义国家内部,边缘指那些远离中心的
区域。比如,法国的布列塔尼,大不列颠的爱尔兰、威尔士和苏格兰,意大利的西西里
岛和南部地区等。再次,边缘指城市的边缘地区、城郊的居民等。最后,边缘还指那些
社会和政治的边缘群体,特别是青年和妇女、同性恋者、绝望的人、“精神错乱” 的人、
吸毒者等[15] 。中心和边缘的矛盾不仅仅表现为单方面的中心对边缘的控制和剥削,以
及中心和边缘发展不平衡的加剧。同化和同质化的过程必然伴随着激烈的反抗,中心
越倾全力控制和剥削边缘,边缘对中心的反抗和违反就越激烈,中心越连续和无限地
控制和剥削边缘,边缘对中心的反抗和违反就越持久和永恒。另一方面,国家资本主
义和国家把城市作为财富、决策、信息和空间组织的中心,伴随着中心的饱和、资源的
匮乏等问题的出现,城市发展逐渐显现出衰退迹象,从而使中心化危机显露出来并不
断扩展,甚至恶化。美国的都市化进程最迅猛,相应的城市问题和城市危机也最先暴
露出来。“美国资本主义曾经面临极度痛苦的两难境地:是应该牺牲城市(纽约、芝加
哥、洛杉矶等)并在别处组建决策中心(这是件很难的事)
 ,还是应该通过投入巨大的
资源来保留这些城市,即使是美国社会自己所能支配的资源的总和也在所不惜。”[16]

总之,国家资本主义的都市化进程生产了中心、边缘及其矛盾,中心的衰落和中心
与边缘矛盾的加剧引发的城市现象和城市危机使城市成为资本主义矛盾表现最激烈的
场所。列斐伏尔认为,资本主义抽象的都市空间中质与量的矛盾、交换价值与使用价
值的矛盾、为非生产性消费和生产性消费进行的空间的生产之间的矛盾、暂时与稳定
的矛盾、都市理性统治下抽象空间的意识形态化等矛盾是中心和边缘矛盾的征兆,同
时也是其原因和结果。



对大国大城批判,所谓自由,正是国家充当……原始积累白手套,为服务。


关于土地金融的内在逻辑,笔者认为可通过赵燕菁的书籍文章\cite{dajueqi}窥得不少较为
广泛和直接的认识,但笔者并不赞同其对“土地金融”的一些辩护意见。
\begin{quotation}
  中国城市化伟大成就背后的重要原因,就是创造性地发展出一套\textbf{将土地作为信用基
    础}的制度——“\textbf{土地财政}”,也正因如此,\textbf{“土地财政”乃是一种金融活动。
    将土地收入视作“财政收入”,暴露出传统经济理论对真实世界的错误观察和认识。}

  只有政府提供了重资产(笔者注:企业不愿投入的、耗资巨大、长周期、低回报等难
  见市场效益的公共服务),其他社会主体才可能以轻资产启动各类商业模式。\textbf{城市
    化就是资本不断聚集的过程}。
\end{quotation}
土地肯定“不具有天然的信用”,所有资产的价值都是未来收益的贴现。土地及附着其
上的不动产也是如此。土地的信用来自附着其上的公共服务给土地带来的现金流(租
金)。

% 中国的城市化其实在历史演变过程中逐渐囊括了“四化”:城镇化、工业化、信息化和农业现代化。

如果我们将“城市”视作一组公共产品(安全、教育、交通、绿化……)的集合,实际上也就从制度的角度给出了城市的定义:城市是一组通过空间途径赢利的公共产品和服务。或者按照规划师的习惯,将城市定义为“公共产品和服务赖以交易的空间”。

奥尔森认为,诸如“和平”这样的公共产品(如城墙)是垄断的竞争者(常驻的匪帮)
出于自私的目的强加给居民的。而居民通过纳税获得保护,进而与常驻的匪帮分享“和
平秩序”带来的巨大好处

在城市的制度原型里,政府乃是以空间(行政边界)为基础提供公共产品的“企业”。

城市政府同企业家一样,其核心工作就是发现并设计最优的公共产品提供模式。具体的形式体现在不同的公共产品和各式各样的收费模式上[如税收以及诸如建设-运营-转让(BOT)等基础设施建设模式]。公共产品(比如消防、路灯、治安)由于通常无法以排他的方式提供给付费的消费者,因此,必须以向特定空间使用者收费的方式提供。

城市化的过程,就是一个城市(更准确地说是公共服务水平)从谱系的低端向谱系的高端移动的过程。

地方政府的土地收入在第一阶段本质上是“金融”,只有完成招商引资后获得持续性税收才来到第二阶段成为“财政”。

在传统经济中,一次性投资的获得主要是通过过去剩余的积累。

工业化和城市化的启动,都必须跨越原始资本的临界门槛。一旦原始资本(基础设施)积累完成,就会带来持续性税收。这些税收可以再抵押,再投资,自我循环,加速积累。

计划经济遗留下来的这一独特制度,使土地成为中国地方政府巨大且不断增值的信用来
源。不同于西方国家抵押税收发行市政债券的做法,中国土地收入的本质,就是通过出
售土地未来的增值(70年),为城市公共服务的一次性投资融资。中国城市政府出售土
地的本质,就是直接销售未来的公共服务。如果把城市政府视作一个企业,那么西方国
家城市是通过发行债券来融资,中国城市则是通过发行“城市股票”来融资。(超发货
币,债券)

因此,在中国,居民购买城市的不动产,相当于购买城市的“股票”。这就是中国城市
的积累效率远高于土地私有化国家的重要原因。也正是依靠这一做法,中国得以一举完
成工业化和城市化两个进程的原始资本积累。

反倾销历来是发达国家对付其他发达国家的经济工具,现在却被用来对付中国这样的发
展中国家;以前从来都是城市化发展快的国家出现资本短缺,完成城市化的国家出现资
本剩余,现在却反过来了,是中国向发达国家输出资本。在这些“反经济常识”的现象
背后,实际上都有赖于“土地财政”融资模式的超高效率。

中国之所以能“和平崛起”,原因恰恰离不开“土地财政”这种融资模式,这使得中国不必借由外部征服,就可以获得原始资本积累所必需的“初始信用”。高效率的资本生成,缓解了原始资本积累阶段的信用饥渴,确保了中国经济成为开放的和在全球化中获利的一方。因此,即使处于发展水平较低的城市化初始阶段,中国也比其他任何国家更希望维持现有国际经济秩序,更有动力推动经济全球化。“土地财政”的成功,确保了“和平崛起”成为中国模式的内置选项。

[16]周其仁对此批注道:“其实中央政府更依靠税收财政,地方政府更多的是靠土地财政。”

之所以没有发生通货膨胀,乃是因为房价上升导致全社会信用规模膨胀的速度比货币更快。[13]

土地财政”虽给中国经济带来诸多好处,但也引发了许多问题。这些问题不解决好,很可能会给整个经济带来巨大的系统性风险。

第一,“土地财政”必定使得不动产变成投资品。
“土地财政”的本质是融资,这就决定了土地乃至为土地定价的住宅必定是投资品。

第二,拉大贫富差距。

“土地财政”不仅给地方政府带来巨大财富,同时也给企业和个人快速积累财富提供了通道。没有机会投资城市不动产的居民与早期投资城市不动产的居民的贫富差距迅速拉开。房价上涨越快,贫富差距越大。房地产如同股票,会自动分配社会增量财富。正是这一功能,阻碍了不同社会阶层上下流动的渠道。

第三,占用大量资源。

如果说中国经济“不协调、不平衡、不可持续”,房地产市场首当其冲。同虚拟的股票甚至贵金属不同,以不动产为信用基础的融资模式,会超出实际需求制造大量只有信用价值而没有真实消费需求的“鬼楼”甚至“鬼城”。为了生产这些信用,需要占用大量土地,消耗掉本应用于其他发展项目的宝贵资源。资本市场就像水库,可以极大地提高水资源的配置效率,灌溉更多的农田。但是,如果水库的规模过大并因此而淹没了能真正带来产出的农田,水库就会变为一项负资产。

第四,带来金融风险。
土地“净收益”已经成为很多企业,特别是地方政府信用的基础。一旦房价暴跌,如此规模的抵押资产贬值将导致难以想象的金融海啸。广泛的破产不仅会摧毁地方政府的信用,而且会席卷每一个经济角落,规模之大会使中央财政无力拯救。


在中国,“土地财政”的本质是“融资”,其替代者必定是另一种对等的信用。而要把税收变为足以匹敌土地的另一种信用基础,就必须突破一个重要的技术屏障——以间接税为主的税收体制。中国的税负水平并不低,其增速远超GDP。2012年完成税收收入11万亿元,同比增长了11.2%。在此基础上,继续大规模加税的基础根本不存在。

《福布斯》杂志根据边际税率,曾连续两次将中国列为“税负痛苦指数全球第二”。
数据显示,2011年,我国全部税收收入中来自流转税的收入占比达70\%以上,而来自所得税和其他税种的收入合计占比不足30\%。来自各类企业缴纳的税收收入占比更是高达92.06\%,而来自居民缴纳的税收收入占比只有7.94\%。如果再减去由企业代扣代缴的个人所得税,个人纳税创造的税收收入不过占2\%。2012年个税起征点上调后,2013年个人直缴的比例更低。这就是税收高速增长,居民税负痛感却不敏感的重要原因。

任何一种改革,如果想成功,前提都是纳税人的负担不能恶化。如果按照某些专家的建议,通过直接增加财产税等新的地方税种来补偿土地收入损失,可能会引发社会骚乱。这种非帕累托改进,对任何执政者而言,都是巨大的风险。

如果你把不同城市的房价视作该“城市公司”的股价,你就会发现中国“城市公司”股
票市场的增长速度和中国经济的增长速度十分一致,一点也不反常,并通过免交财产税
的方式分红。由于土地市场的融资效率远高于股票市场,因此,很多产业都会借助地方
政府招商,以类似搭售(tie-in sale)的方式变相通过土地市场融资。中国大量企业是
在土地市场而不是在股票或债券市场完成融资的。

通过抵押或直接出让“平衡用地”,是地方政府基础设施建设融资和招商引资的主要手段。这也间接反驳了那些认为“土地财政”抑制了实体经济的指责。

美元通过与大宗商品,特别是石油挂钩,重新找到了“锚”,使得美元可以通过大宗商品涨价,消化货币超发带来的通货膨胀压力。欧元试图以碳交易为基准,为欧元找到“锚”,但迄今仍未成功。日元则基本上以美元为“锚”,它必须不断大规模囤积美元,其货币超发,只能依靠美元升值消化。[12]

而“土地财政”却给了人民币一个“锚”。土地成为货币基准,为中国的货币自主提供
了基石。“锚”就是不动产:不动产升值,货币发行应随之上升,否则就会出现通货紧
缩;货币增加,而不动产贬值,则必然出现通货膨胀。\improve[inline]{实际的通货
  膨胀,到底高不高?土地蓄水池的边界到底几多?}

同日本一样,囤积的大量美元是人民币信用的另一个来源。美元升值,人民币就可以多
发。如果人民币贬值,不动产就必须升值,否则,就会导致通货膨胀。因此,在美元贬
值的背景下,打压房价,就是打压人民币。房价下跌,必定导致通货膨胀,其后果可能
远比我们大多数人想象的巨大、复杂。之所以没有发生通货膨胀,乃是因为房价上升导
致全社会信用规模膨胀的速度比货币更快。[13]

任何一种改革,如果想成功,前提都是纳税人的负担不能恶化。这种非帕累托改进,对
任何执政者而言,都是巨大的风险。

历史上,直接税的征收比间接税的征收要艰难得多。发达经济体为了建立起以直接税为基础的政府信用,无不经历了漫长痛苦的社会动荡。

“土地财政”只是专门用来解决城市化启动阶段原始信用不足问题的一种特殊制度。随着原始资本积累的完成,“土地财政”也必然会逐渐退出,并转变为更可持续的增长模式。

当城市化进入新的发展阶段,就要及时布局不同模式间的转换。在这个意义上,放弃“土地财政”绝不是简单的财政改革,而是一场剧烈的社会改革。如果这场改革发生在城市化完成之后,可能是再一次的制度升级;而如果发生在城市化完成之前,很可能会导致巨大的社会风险。模式的过渡,没有简单的切换路径可循,必须经过复杂的制度设计并花费几代人的时间。在还没有找到替代方案之前就轻率抛弃“土地财政”,是不明智的。

真正用来满足需求并成为经济稳定之锚的,是保障房供给。\improve[inline]{经济适
  用房、报障房到底归了谁?}

由于住房最终可以上市,因此土地(及附着其上的保障房)就可以成为极其安全有效的抵押品。通过发行“资产担保债券”(covered bonds)等金融工具,利用社保、养老金、公积金等沉淀资金获得低息贷款,只需政府投入(贴息)少许,就可以一举解决“全覆盖”式保障房的融资问题。[6]近年来,社保基金、养老基金和公积金进入股票市场的呼声不绝于耳。但低迷的收益和有限的规模,使得股票市场难以满足保值的需要。如果我们拓宽视野,就会发现,房地产(特别是保障房)市场其实是比股票市场更大、更安全的资本市场。[7]

“先租后售”模式,看似解决的是住房问题,实际上却意味着“土地财政”的升
级——都是以抵押作为信用获得原始资本,地方政府很难主动实施。。以往“土地财
政”是通过补贴地价来直接补贴企业,而“先租后售”保障房制度,则是通过补贴劳动
力来间接补贴企业。2008年以后,制约企业发展的最大瓶颈已经不是土地,而是劳动
力。

“土地财政”的另一个后果就是“空间的城市化”并没有带来“人的城市化”——城市到处是空置的住宅,农民工却依然在城乡间流动。现在很多研究都把矛头指向户口,似乎取消户籍制度就可以在一夜之间消灭城乡差距。取消户籍制度,如果不涉及背后的公共服务和社会福利,等于什么也没做;但如果所有人自动享受公共服务和社会福利,那就没有一个城市负担得起。
要想取消户籍制度,就必须改间接税为直接税。户籍制度同公共产品付费模式密切相关。改变税制,如前所述,制度风险较大。

今天因为缺少财产而无法拥有城市户籍的非城市人口,明天也一样会因为缺少财产而无
法成为合格的纳税人。坠入“中等收入陷阱”国家的一个共同特征,就是大量进入城市
的居民不为城市公共产品付费。城市贫民窟同小产权房一样,本质都是为了\textbf{逃避为公
  共产品付费}。一旦坠入“中等收入陷阱”,城市化就会半途而废,除非将这些不为公
共产品付费的人口也作为城市人口。

现在财政界有一种普遍的看法,认为中国的税制结构已经到了非调整不可的地步。理由
是,间接税使每个购买者成为无差别的纳税人,无法像直接税那样,通过\textbf{累进制}使高收
入者承担更多的税负来调节贫富差距。

正确的做法,不是回到土地私有的原始状态再启动城市化(这样只能让城市周围的农民
获得城市化的最大好处),而是要利用这一制度遗产,通过企业补贴、保障房“先租后
售”等制度,让远离城市地区、更大范围内的农民,一起参与原始资本的积累,共同分
享这一过程创造的社会财富。

房价的上涨对冲了其他产品价格上涨的压力。在某种意义上,正是因为土地的超级通货
膨胀,才避免了整个经济的超级通货膨胀。一旦房价暴跌,土地就会大幅贬值,信用就
会崩溃,从而引发金融动荡。

防止土地大幅贬值的关键,在于防止房价暴跌。防止房价暴跌的有效办法,就是控制供
给规模。唯有大幅降低商品房供地规模并切断信贷从银行流向房地产的路径,才能减少
土地信用在市面上的流通,从而避免资产价格暴跌。

保障房直接和真实需求挂钩,如果有了这个“锚”,只要保障房需求(预先登记并实际居住)是真实的,政府的“土币”就不会超发。中央政府则可以像调节银行准备金那样,通过调节商品房与保障房的比例,来调节市场上信用的多寡,进而调控经济的发展速度。

有了保障房这个“锚”,我们就可以像调整银行的货币准备金那样,调节商品房和保障房的比例,从而控制地方政府信用发行规模——如果我们希望经济增速快一点,就可以提高商品房相对保障房的比例;反之,则可以降低商品房的“发行规模”。宏观调控工具因此会更加丰富,经济政策就可以更加精确。

“土地财政”是一把双刃剑,它既为城市化提供了动力,也为城市化积累了风险。放弃是一种容易的选择,但找到替代模式却绝非易事。没有十全十美的模式。“税收财政”演进了数百年,导致了世界大战、大萧条、次贷危机、主权债务等无数危机,其破坏性远超过“土地财政”,但西方国家并没有轻言放弃。它之所以仍然被顽强地坚持、探索,盖因其\textbf{积累模式的内在逻辑}。

深圳经济并没有因为无地可卖而“不可持续”,深圳“土地财政”已经悄然退出。王建表示:“深圳的实践表明,我们可能根本无须为不治自愈的‘病’吃药。”

1992年以后,中国的市场化进程发展到了把土地变成资本的阶段,土地财政才能大行其道,
既然土地的市场化是推动中国工业化、城市化与经济高增长的起源,当这一块资源用尽的时候,就是依靠土地财政、信用增长方式转型的时候,目前东部发达地区已经没有土地资源可用,新的财政体系建设是必然趋势。

中国土地资本的高增值,其主因是在新全球化时代外需所带来的中国经济高增长,实现了极高的利润增长率,从而使包括土地在内的所有生产要素不断增值,因为说到底,只有利润才能决定资产的价值。

城市也不是自发集聚的结果,而是经济活动的参与者(最初是“常驻的匪帮”)有目的地设计并提供的。

将政府视为一类生产公共产品的企业后,接下来就会涉及政府垄断的话题。在仅考虑私人产品生产的完全竞争范式下,价格是无数消费者和无数生产者共同面对的一个市场结果,没有单个消费者或生产者可以决定价格。而一旦将公共产品的生产者——政府引入市场,价格就必然会被“有形的手”操控,完全竞争范式就会崩溃。于是,经济学家造了一个叫作“市场失灵”的概念来兼容这一现实。

在新古典微观经济学中,价格竞争基于相同产品,尽管这些产品的生产区位不同,但由于运费和地租等空间变量被忽略不计,因此,这些产品仍被视为无差异的。在这种完全竞争范式下,采用空间收费方式的公共产品(服务)自然无法定价。笔者将这类相互具有替代性的不同产品之间的竞争,定义为“哈耶克竞争”。这种竞争与迪克西特-斯蒂格利茨模型(D-S模型)的效用替代相似,但更加接近霍特林模型所描述的区位竞争,本章将其称为哈耶克—霍特林竞争。只要向哈耶克—霍特林竞争模型引入空间要素,就可以发现公共产品(服务)根本不会像新古典经济学预期的那样形成“垄断”(详见第十四章)。

按照竞争范式,城市公共服务完全可以以竞争的方式由市场提供。只要把政府视作一个企业,就可以很好地解释公共服务是怎样提供的。由于公共服务收费的特殊性,拥有空间定价权(税收和地租——根据租税互补原理,税收属于广义的地租)的政府,就成为这类服务最有效率的提供者。一旦把空间要素引入竞争,政府就像企业一样,通过空间竞争(也就是所谓的“用脚投票”)在市场上提供公共服务,公共服务的市场定价问题就可以迎刃而解。

在现实中,公共产品竞争自古就有,只是“土地财政”使得这一现象在中国地方政府之间体现得更为淋漓尽致。这一竞争最早由美国经济学家蒂伯特提出,张五常在“中国县域竞争”实证中对其进行了检验。笔者在更早的一篇文章《对地方政府行为的另一种解释》中,也提出过类似的观点,拥有土地收益权的地方政府可以像“企业”一样竞争。
只有在新的竞争范式里,土地的“价值—价格”才可以得到完整的解释。

批评文章认为“逻辑与事实不符”:第一,“扣除城市建设开发和对农民的征地补偿,地方政府并不赢利”;第二,“地方政府对土地市场与规划权的垄断,极大地增加了地方债务违约风险”。

政府卖地(使用权)的一次性收益本质上是一种融资——“土地金融”,用于“城市建设开发和对农民的征地补偿”。政府卖地(使用权)的持续性收益来自未来的税收——“土地财政”。只要能够覆盖“城市建设开发费用和对农民的征地补偿”,“土地金融”的使命就已经完成。

方政府“在无地(使用权)可卖的情况下,只能走向名义上的破产财政”。这类批评看似直观,但却一点也不“专业”,笔者当时对马光远博士的回答,现在可以原封不动地用来回答这个问题:

在金融不发达的时代,基础设施建设的规模,取决于“过去”劳动剩余的积累。但如果借助金融体系,则可以抵押“未来”的收益,突破基础设施建设的资金瓶颈。这就是为什么发达国家政府出现了这么多次金融危机,负的债要远比中国地方政府多,却没有一个国家干脆立法,禁止政府融资,以防“走向名义上的破产财政”。

在计划经济时代,虽然政府财政没有破产,但基础设施建设却“欠账累累”。可以说,“负债”是现代经济的主要特征。\textbf{负债越多,表明政府信用越好}。

土地财政”本质上也是一种融资模式,它极大地扩张了地方政府的信用,盘活了“未来”的资产,提高了政府的负债能力。

有强大的信用才有资格“负债”,负债高反过来也可以证明信用好。“土地财政”就是
\textbf{帮助地方政府创造信用的工具}。至于“债务”的使用是否合理、是否赢利,那不是\textbf{工具
本身}的问题,而是\textbf{如何使用工具}的问题。


那种认为“政府要获得基础设施建设的外溢收入,可以租用农民的土地,而不必把农村
的土地国有化”的观点乃是纸上谈兵,土地的升值永远只会落到\textbf{产权人手中}而不是佃农
或租客手中。\textbf{只有当公共服务提供者和土地所有者一致时,地租才会公平地落在创造地
租的人手中。}

严崇涛从1970年到1999年,先后担任新加坡财政部、贸工部、总理公署等多个政府部门的常务秘书,参与了新加坡的快速崛起。他在《新加坡发展的经验与教训:一位老常任秘书的回顾和反思》一书中认为:

没有基础设施的土地只不过是沙土罢了。因为对土地进行的基础设施投资主要由国家进行,我们认为,公共建设的道路、地铁系统、学校等基础设施带来的大部分土地升值收益,应该归属于国家。因此,当政府征用土地从事公共建设如组屋、工业厂房的时候,赔偿标准应基于毫无基础设施的原始未开发土地的价格。买主必须向市建局保证将进一步开发土地,并支付发展费用。市建局每年会在政府公报上刊登不同土地的发展收费,征收该费用则是为了抵消政府建公路和排水系统等部分的开支。如果没有征收发展费用,投资者或地主不就能免费从这个国家的投资中受益吗?

土地国有化可以使公共服务(比如道路)外溢出来的价值完全体现在地价上,从而避免了征税的困难。开发前土地国有化就成为效率更高的价值捕获工具。这也就是严崇涛宣称“我们工业用地和公共住房发展的基石,就是土地征用法令”的原因。

批评文章写道:“近年来征地成本日益高涨,有的地区达到每亩地上百万元”,“强制
征收具有较高的成本,带来了与农民的矛盾和大量上访事件,影响了社会秩序与政府信
誉”。这些也是近年来公共项目成本暴增,进而导致城市政府负债骤增、固定资产投资
速度降低、经济增速下滑的重要原因。这也从相反方向证明,当年国家强制征收能力对
建立“现代社会秩序”多么重要。\improve[inline]{正因为土地金融不可持续,寅吃
  卯粮,才会这样。而不是因为土地金融发展不足。}

批评文章写道:“近年来征地成本日益高涨,有的地区达到每亩地上百万元”,“强制征收具有较高的成本,带来了与农民的矛盾和大量上访事件,影响了社会秩序与政府信誉”。这些也是近年来公共项目成本暴增,进而导致城市政府负债骤增、固定资产投资速度降低、经济增速下滑的重要原因。这也从相反方向证明,当年国家强制征收能力对建立“现代社会秩序”多么重要。

土地国有化必定效率低”乃是主流经济学臆造出来的结论。新加坡在独立前,国有土地
面积约占国土总面积的60\%;新加坡独立以后,通过强行征用,使国有土地面积占比
于2006年达到90\%。以色列国有土地面积占比更是高达93\%(樊正伟和赵准,2009),
其城市土地和农村土地都是“国有的”,其余大部分也是归地方政府所有。真正影响土
地使用效率的,是实际制度的设计,而非字面上的“国有”还是“私有”。


由于农地收益流极低,农地作为信用基本没有太大价值,自然无法用来抵押。

那些“法律不允许”抵押的农地,乃是位于城郊的土地。由于靠近城市,这些农地并非
纯粹的“农地”,上面附着了城市外溢出来的公共服务。城市的基础设施大部分是政府
建的。如果法律允许这些土地改变用途,就等于把城市的公共财富免费送给郊区“地
主”,还记得前文提到美国人建运河时想避免的情况吗?“拥有产权意味着这些土地所
有者即使袖手旁观,也能获得土地升值带来的利润……出于这个原因,法院和议会/经常
介入……”政府公共服务的收益流失,基础设施的建设成本无法收回,就会造成发展中
国家普遍面临的公共产品“搭便车”困局。\improve[inline]{对大国大城的抨击}

发债模式马上就会因为无力还贷出现违约;卖地(使用权)模式则不用补偿地价下跌的损失,因而不会出现违约。“融资平台”模式绕开了国家发债的限制,虽然可以扩大信用的规模,但也会带来违约的风险。这些年地方政府积累的巨额债务,恰恰不是卖地(使用权)带来的。也正是因为“土地财政”这一直接融资模式,中国地方政府的违约概率要比成熟的市场经济国家低得多。


\textbf{一旦开征财产税,中国土地市场的资本属性也会像其他国家的一样显著萎缩。}

对于信用货币而言,真正需要担心的不是通货膨胀,而是通货紧缩——一旦信用消失,货币随之消失,结果一定是伴随通货紧缩而来的大萧条。

\textbf{过去四十余年,中国的高速城市化在很大程度上是建立在大规模的货币创造基础上的。}
而这些货币的创造离不开信用的创造。将土地作为一种信用,各国皆然,但像中国这样
将土地市场发展成为一个巨大的资本市场,并使之成为货币信用主要来源的却再也找不
到第二个国家。这不能不说是“土地财政”制度的又一个创新。

在很多学者看来,没有通货膨胀是因为货币都“涌向住房和土地市场”,而不理解现在
市场上的大多数货币本身就是由“住房和土地市场”生成的。若房地产市场泡沫破裂,
我们不会看到多余的货币决堤而出,而是会看到市场上的流动性突然干涸。大家在批判
房地产泡沫时,其实也在享受资产泡沫带来的好处。没有这些泡沫,就不可能创造出足
够的货币,泡沫崩溃时,前所未有的商业繁荣就会烟消云散。


真正将“土地财政”风险“升维”的,是各种以政府信用作为担保的融资平台。这些融资平台以地方政府垄断的土地一级市场作为隐性担保,一旦房地产价格下跌,就会触发债务违约。

第一,不要轻易将土地融资工具从风险较低的直接融资,升维到风险较大的间接融资;

第二,不要像杰克逊那样简单地“去杠杆”,因为“去杠杆”的副产品——消灭货
币——给经济带来的危害,远大于债务违约本身。(不可能的童话)

在传统的经济学理论中,只要总体收支平衡,就是可以的。

但如果按照两阶段增长模型,“缺口”和“剩余”不能相抵。这是由于依照金融学原理,
资本性剩余与现金流性剩余的\textbf{“量纲”不同}。前者是通过金融创造的,是将未来收益
贴现进行资金的跨期配置,在本质上属于\textbf{向未来借贷}的行为。而现金流性支出在时间
上的持续性,导致再大的资本存量也难以弥补\textbf{流量缺口},由于更多的资本性支出意味
着更高的债务及利息,\textbf{如果未来创造的现金流不足以偿还债务,就会陷入债务危机。}欧
洲的债务危机,本质上就是用资本性收入覆盖“现金流缺口”的结果。欧盟有些国家并
没有创造出足够的收入,却想维持和其他国家相同的福利水平,于是就用举债获得的资
本性收入弥补养老金缺口,由此陷入债务循环……


在这一点上,国家(中央政府)、城市(地方政府)、企业和家庭的道理相同。用资本
性收入弥补现金流缺口的最终结果,无不指向庞氏循环。

资本缺口不能用现金流剩余来平衡,反之亦然。笔者称之为“不可替代规则”。

所谓的金融危机,不是资本型增长阶段(Ri0-Ci0=Si0)没有完成,而是运营型(劳动型)增长阶段(Rik-Cik=Sik)没有完成,使得第三个公式不成立,导致从资本型增长向运营型(劳动型)增长过渡的过程中陷入增长崩溃。

所谓泡沫,就是对未来现金流的估值,反映在贴现倍数上。在实物商品货币时代,按照
格雷欣法则会出现“劣币驱除良币”的现象;但在信用货币时代,则是“高泡沫驱除低
泡沫”。美国凭借其制度、军事和文化,在债市、股市等资本市场上,将更多的未来收
益贴现,给同样的现金流以更高的估值,创造并支撑全球最大的资本市场泡沫,助力其
经济独步全球。现在,中国在美国之外创造了一个具有更高贴现倍数的资本市场。中国
房价行情网站数据显示[2],2018年北京楼市的售租比为55年,即买房出租要660个月才
能收回成本。而在国际上,房地产的售租比一般界定为17.25年。


在以企业信用为支撑的股市和债市等资本市场,中国的发展同发达国家相比长期处于落后状态,这导致高风险、长周期但高收益的产业根本无法获得昂贵的资本。但在以政府信用为支撑的房地产市场,由于地方政府是这一市场中的核心“企业”,其快速发展在很短的时间内创造了大量的“廉价”资本。在中国目前的资本结构下,面对巨额的高风险投资,只有“廉价”的政府资本才能承受巨大的风险,因此中国经济只有政府“重资产”——“国进”,民营企业才可能“轻资产”——“民进”。也即在政府投资成功之后,民营经济的“轻资产”再嫁接到政府的“重资产”上,因此政府的公共投资是企业私人投资的基础。无论是芯片的研发还是高铁的建设,都是同样的道理。我们可以把民营经济比作各种各样的电器,民营企业无须自建煤矿、电厂和电网,只需插到政府这个“插座”上,就可以轻资产运营。

这也就很好地解释了中美贸易摩擦中,美国一定要将中国划分为“非市场经济国家”以
及针对“中国制造2025”的原因。中国“政府+土地金融”的制度设计挑战了欧
美“私企+股票债券”的市场模式,中国的资本市场(以房地产市场为主)挑战了美国的
资本市场(以股票市场为主),但无论是政治制度还是城市化所处的阶段,都决定了美
国无法效仿中国的做法,因此美国只能依靠所谓市场经济国家的“准则”迫使中国“自
断双臂”,放弃自己的“有形之手”。发达国家地方政府的财政收入几乎完全靠税收支撑,其资本生成能力远远低于中国地方政府。它们可能补贴一些制造业企业,但补贴的规模完全不能与中国地方政府相匹敌。

。而“市场—商品”经济必须使用货币分工。在中国历史上市场经济长期发育不良,一个根本的原因就是货币不足,所以必须依赖出口顺差换取分工所必需的货币。而发达国家,无一例外都是通过货币的信用化,摆脱了流动性不足的约束。改革开放的成功,特别是过去十几年的经济增长,很大程度上是因为土地金融创造的巨大信用为货币从商品货币转向信用货币创造了条件。信贷成为货币生成的重要途径。



在城市化需要巨大融资的资本型增长阶段,因为有足够的信用,贷款需求不成问题。但当城市化从资本型增长转向运营型增长时,贷款需求迅速减少,如果此时家庭、企业和政府三个部门同时“去杠杆”,银行资产负债收缩,结果就是货币供给的减少。如果再加上国际局势动荡导致的顺差生成货币减少正好与“去杠杆”同步,经济就会面临更大的萎缩风险。

在中国历史乃至世界历史上,货币不足导致的社会动荡屡见不鲜。这才是城市化转型的最大风险。

对于现代经济而言,转型成功的前提,就是要解决伴随转型而来的流动性不足问题。这就意味着必须有另外新的商业模式开始进行资本型增长。

城市发展的衰退与企业经营的失败在原理上是相同的。能够获得资本,完成市政建设的
城市很多;但能创造足够的收益,持续运转下去的城市却很少。当城市化转入运营阶段,
问题的焦点不再是资本的多少,而是经常性收入是否足以覆盖一般性公共服务支出。如
果无法获得足够的现金流性收入,之前所有的投资就会转变为无法偿还的债务。正如能
盖起厂房的企业很多,但能赚钱的企业没几家。城市的道理相同,能建设起来的城市很
多,但能通过运营最终获利的却很少。\improve[inline]{WTF!大多数城市的衰颓和失
  败}

1750年是人类财富增长的转折点,在此之前世界人均GDP长期处于停滞状态;在此之后,世界财富出现陡然增长。[3]是什么导致了人类从“大停滞”走向“大增长”?笔者认为这是缘于制度创新,即:通过金融革命,人类发明了一种能将未来收益贴现过来,弥补经济增长资本缺口的制度,从此摆脱了依靠过去剩余积累的桎梏,经济增长的原始资本积累阶段得以迅速完成,不同商业模式从传统的序贯增长转入现代的平行增长。

如果说常态增长的特征是以私人部门的运营型增长为主,那么危机增长的特征就是以公
共部门的资本型增长为主,例如“\textbf{新基建}”。不过比“新基建”更准确的提法应当
是“资本深化”或“公共服务深化”。\improve[inline]{这里的公共服务部门究竟是什么意思?}

第二,要有足够的资本供给。如果一个国家不能创造足够的“资本”(比如所谓的“不
发达经济体”),无论有多少闲置、过剩的生产要素,都无法将其动员起来。转向危机
增长的前提,就是要绕过已经坍塌的资本废墟,重建新的资本渠道——通过公共服务领
域向市场大规模注入流动性。只要市场上有充足的流动性,那些搁浅的资产就会重新漂
浮起来。

\textbf{真正导致经济危机的原因,是随内外投资需求下降带来资产负债表缩表并发的流动性枯竭。}

\textbf{系统性风险(所有市场主体都选择持有货币而不消费)}

虚拟财富与真实财富之间的关系意味着降息将导致经济中虚拟财富(资本)的估值提高,占社会总财富的比重提高,而真实财富(劳动)的估值下降,占社会总财富的比重下降。二者的比值一旦越过临界点,资本形态财富的增长速度就会超过实体形态财富的增长速度,实体经济的货币就会被逆向吸入资本市场,从而进一步推高资本的价格。

在现代经济中,社会总财富是虚拟财富和真实财富的加总……随之而来的是股价和房价
飙升。一旦虚高的资本估值得不到\textbf{未来真实财富}的支持,泡沫就会破灭,并引发更大的
流动性危机。理论上,货币体系乃是不同流动性组合的信用。一旦位于顶端的高流动性
货币不再被信任,取而代之的次级货币会导致社会商品分工水平下降。危机下的货币宽
松,不过是饮鸩止渴。因此,危机货币供给的核心不是数量而是方式——通过什么渠道
注入流动性比注入多少流动性更重要。

在债务不变的条件下,修复资产负债表有两种方式:一种是提高资产的估值;另一种是
增加资产的\textbf{数量}。前者是常态增长阶段的主要工具,后者则适用于危机增长阶段。

在常态货币供给渠道失灵的危机状态下,央行可绕过商业银行系统,通过直接对基础性战略资产(BSA)进行投资向市场投放货币。史正富先生将这个新的货币发行通道称为“长期资本注资便利”(long-term capital funding facility)渠道。

央行通过投资基础性战略资产生成基础货币与通过财政发债生成基础货币最大的不同,就是前者的抵押品是投资的基础性战略资产市值,而后者的抵押品是政府的未来税收。前者直接创造货币需求,后者本质上还是需要得到资本市场的支持。由于在危机中政府的税收能力下降,通过财政赤字发债将会导致赤字扩大,融资成本也会更高。而基于新增基础性战略资产的货币,无须以税收增加作为信用的基础。只要成本足够低,即使是较低的回报,也一样可以生成正的信用。

央行通过投资基础性战略资产生成基础货币与通过财政发债生成基础货币最大的不同,
就是前者的抵押品是投资的基础性战略资产市值,而后者的抵押品是政府的未来税收。
前者直接创造货币需求,后者本质上还是需要得到资本市场的支持。由于在危机中政府
的税收能力下降,通过财政赤字发债将会导致赤字扩大,融资成本也会更高。而基于新
增基础性战略资产的货币,无须以税收增加作为信用的基础。只要成本足够低,即使是
较低的回报,也一样可以生成正的信用。\improve[inline]{可通过 财政赤字货币化理论与实
  践 刘思源 了解下货币政策。}

在现代政府职能分工中,财政主要是负责\textbf{运营型增长},体现的是\textbf{现金流的收和支};
央行主要是负责\textbf{资本型增长},体现的是\textbf{资本的收和支}。

由于这一功能是传统央行所没有的,因此需要成立专门的国家“自然资源银行”。其职
能是:(1)代表央行收购、持有、转让国家基础性战略资产;(2)负责资产的保值、
增值;(3)建立相应的基础性战略资产交易市场,对自身代央行持有的基础性战略资产
进行定价。

例如,建立国家建设用地指标交易市场,使得央行持有的新增或改进的耕地可以在市场
上定价;类似地,可以建立水资源交易市场、大气质量交易市场等。通过这样的措施为
以基础性战略资产为信用的货币提供流动性。

判断经济是否恢复的一个关键指标,就是失业率。央行的货币政策要从盯住通货膨胀率
转向盯住失业率。一旦劳动力实现就业,危机增长的条件(要素闲置)将不复存在,增
长也要随之切换回常态增长。

不仅是公路,理论上所有的公共资产(铁路、电网、供水网格、通信等系统)都可以通过此路径实现大规模资本深化(而不是建设更多的无效资产)。

央行也可以启动“先租后售”计划,通过收购开发商烂尾/违章项目,将其纳入基础性战略资产计划,将住房低成本租给无房的新就业者,一定年限后出售,收回资本。收购居民断供的物业,将其纳入“共有产权”计划,居民回购或交易时,收回资本。协助家庭部门完成城市化阶段的重资产化。


在农户自愿的基础上,大规模收购弃耕的耕地,通过完善耕作基础设施,建立农村基础
公共服务(借鉴东亚地区的农业合作组织),建立从播种到收割、从融资到销售的一系
列服务,使自耕农成为轻资产的国家专业农户。(佃农?)


央行还应该尽快建立国家“最终雇佣者”(employer of last resort, ELR)计划,以
最低工资和基本社保(the basic public sector wage, BPSW)为就业条件,大规模提
供公共就业岗位,将货币尽快滴灌到市场中最微观的细胞——家庭上。政府为就业兜底
还可以和企业救助相结合,通过将我国劳动合同法中企业对员工义务的社会化,恢复劳
动力市场的弹性,减轻企业困难时期的压力。就业岗位计划可以由政府提出,也可以由
劳动力本人提出后经劳动部门审核。(金融资本低劳动力高资本,没有受到冲击,反而
受到扶持。)

今天,大量过剩产能、闲置资产、积压库存、失业劳动力给中国经济带来了巨大挑战,但也为中国启动新一轮危机增长提供了机会和条件。

一个国家实现资本型增长对应的条件是资本密集,完成运营型增长对应的条件是劳动密
集。资本最便宜的美国和劳动最便宜的中国成为在这一轮全球化进程中获利最多的两大
赢家,与此对应的是,美国的劳动和中国的资本则成为这一分工模式的受害者。由于美
国拥有资本和劳动的不同群体之间的贫富差距加大,巨大的国内分裂和阶级冲突必然投
射到美国与其他国家的关系上。

换句话说,一旦经济进入现代增长,传统的“积累—消费”增长两难就会转变为“\textbf{资
  本—劳动}”(或称“\textbf{虚拟—实体}”)增长两难。宏观经济政策本质上都是通过\textbf{改
变贴现倍数}来改变两者间的分配关系。所谓“中性”的宏观政策并不存在——有利于资
本增长的政策,就会损害劳动增长;有利于劳动增长的政策,就会牺牲资本增长。所有
的经济政策其实都意味着要对两个增长阶段进行权衡取舍——是发展\textbf{资本密集型}产业,
还是发展\textbf{劳动密集型}产业。

中国的土地金融拥有一个非常重要的特点:\textbf{它是世界上唯一与美元周期脱钩的大型资
  本市场。特别是中国土地市场的资本估值(售租比)比世界上最强大的股票市
  场——美国股市的资本估值(市盈率)更高,泡沫更大。}按照格雷欣法则,廉价的资
本赋予了中国资本密集型企业更大的竞争优势。

总体而言,人类社会一直处于资本不足的状态,资本剥削劳动是世界经济史的主线。

只要中国不重蹈当年日本打压房地产市场的覆辙(比如大规模征收财产税),仅这个市场本身就可以为资本生成提供源源不断的信用。

对于资本密集型产业而言,需要比对手更廉价的巨额资本,因此宏观经济政策就要求宽货币/财政(贷款生成货币)、降息、加税(货币增信)、强货币(将人民币作为储备货币)、货币国际化(输出资本)、低关税(输出货币)、高杠杆、通货膨胀(有利于债务人)……所有这些政策的后果,都会提高虚拟财富在总财富中的估值。而对于劳动密集型产业而言,则要求货币中包含更多现金流,以免在与虚拟财富(未来收益贴现)的兑换中吃亏。由于现代货币是以信用为基础的,信用越高,货币中虚拟财富的比重就越高,劳动在交易时就越吃亏。而提高货币中现金流的含量,就需要紧货币/财政、加息、减税、弱货币(输出产品)、商品国际化(输出商品)、高关税(保护市场)、低杠杆、通货紧缩(有利于债权人)……所有这些政策的后果,都会提高真实财富(现金流)在总财富中的估值,如图6-5所示。

\todo[inline]{解决方案多是滑稽的或者邪恶的}

贴现倍数越高的政策,越有利于资本;贴现倍数越低的政策,越有利于劳动。加息有利
于实体(劳动),降息有利于虚拟(资本);加税有利于实体(虚拟),减税有利于虚
拟(资本);贬值有利于实体(劳动),升值有利于虚拟(资本)……

其中房地产市场创造的信用扮演了决定性角色。低息货币环境中孵化出大量新科技公司,它们的商业模式开始从以往的追随美国变为与之并驾齐驱。

过去20余年,中国房地产市场是世界上唯一脱离美元周期的大型资本市场。它为巨量人
民币提供了强大的信用,使中国成为唯一在低息资本(远低于银行利息)上能和美国一
拼的经济体。

正是依靠房地产市场,我国渡过了1997年和2008年两大金融险滩。

房地产市场同其他资本市场一样,是否崩盘取决于是否可以维持正的信用冗余。特朗普依靠减税为资本市场补充现金流,显著地增加了股票市场的信用冗余;而中国如果此时给房地产市场加税,将进一步减少其所剩无几的信用冗余。

债券市场主要是由中央政府信用创造的,房地产市场主要是由地方政府信用创造的,股票市场则主要是由企业信用创造的。

止血,就是要立即停止和减少不能马上带来现金流的新投资。现在有人一提到阻止经济下滑,就想到以前最有效的一招——固定资产投资。这些越多,维持其运转流失的现金流规模就越大,固定资产投资会通过折旧、付息等创口持续地给地方政府财政放血。
根据李嘉图等价定理,成功的融资背面就是令人痛苦的偿还要求。\textbf{硬约束和财经纪
  律}(好厉害,哈哈,狗头),要成为问责地方政府的头号优先依据。

造血,就是尽一切可能扩大、扶持现金流性收入。其中,最主要的就是企业税收。中国
政府绝对不能放弃对产业的支持尤其是在中国的股票市场没有超过发达国家的股票市场
之前。在私人资本无力进入的领域,国有资本必须带头进入。这不是因为国有资本更有
效率,而是因为中国资本市场的主体是土地,资本市场的性质决定了地方政府的市场参
与者角色。无为政府根本不可能把这些信用传递给市场。国有资本不是与民争利,而是
开疆拓土,打下市场后,再由民营经济跟进。\improve[inline]{是的,不是与民争利,
  而是给企业家,鱼肉百姓。}

对于人口增长减慢的城市,要迅速停止一切不能带来现金流的政绩型固定资产投资。扶贫、对口支援、边疆和民族政策等,都应将是否能增加受益对象的现金流作为衡量成败的标准;对于人口增长强劲的城市,则要放开约束(包括人口和土地限制),加大能带来增量现金流的固定资产投资;对于超级明星城市(比如深圳、苏州等),还可以考虑行政范围扩大、行政等级升级等手段,鼓励其全速增长。

输血,就是增大地方政府在现金流分配中的比例。。加强向地方政府的转移支付,甚至调整央地分税的比例,都要立即提上议事日程。当地方政府的资产负债表上升为中美贸易战的主战场,在关键时刻,甚至要不惜投入国家信用给地方政府背书。

这并不意味着应当给予地方政府无差别的支持,支持应当向那些能创造最多现金流的城市倾斜。所有城市都必须将创造现金流而不是GDP增长作为核心的经济指标。输血不是目的,造血才是目的。
地方政府为房地产市场构筑的第一道防火墙,就应当是迅速建立全覆盖的保障体系,全面接管资本市场现在还在承担的“住”的职能。

利率是货币的价格,反映的是货币“供不应求”的程度,是货币供需关系的晴雨表。

该文认为,“正是房地产抽干了货币,导致货币大量流向国企和地方融资平台,而实体
经济,主要是民营中小微企业,严重失血”。殊不知,由于流通中的住房价格同时也为
非流通住房定价,因此房地产不仅没有抽干货币,相反,它还为货币生成提供了巨大的
信用——相当一部分货币本身就是由房地产的信用生成的。没有房地产,那些货币根本
就不会存在,民营企业和中小微企业也就根本无血可失。\improve[inline]{对中小企
  业,利弊几何可能需要专业客观分析,笔者能力不足。}

按照“货币数量理论”,高贴现倍数创造出的货币可以将原来因收益低、风险高而无法投资的商业活动也卷入市场分工。相反,低贴现倍数意味着资本不足,也意味着开展高风险商业活动不合算。货币的真实贴现倍数k给所开展的商业活动提供了临界贴现倍数,货币泡沫越大,商业活动的临界贴现倍数也就越高,研发、创业这类高风险商业活动的信用冗余Δk值也就越大。

一个合理的推测就是,高贴现倍数资本市场虽然不利于已有的实体经济,却特别有利于研发、创新和创业这类高风险投资活动。

笔者从来没有说依靠高房价就可以多发行货币,而是认为房地产只有存在“\textbf{广泛的供不应求}”,才具有参与货币创造的资格。
\textbf{对于货币生成而言,高房价不是问题,高房价可能带来的流动性丧失才是最大的问题。}

只要中国地方政府的信用不断转移给企业,中国企业就可能在与美国企业的竞争中笑到最后。

如果我们把“泡沫”和“杠杆”等同于真实贴现倍数,就会知道,在现代经济中,这两个概念对资本来说是同义词。

资本市场管理有两个目标:一是降低融资成本,这通常意味着更低的利息(债市)、更
高的市盈率(股市)、更高的资产价格(房地产市场);二是降低市场风险,这通常意
味着更高的利息(债市)、更低的市盈率(股市)、更低的资产价格(房地产市场)。
如何管理这两个相反的目标?一个主要的手段就是在不同的资本市场之间重新配置信用,
这就涉及资本市场的信用结构分层假说。

根据“格雷欣效应”,在资本市场上“劣信用”会驱除“良信用”,具有高贴现倍数的
资产B会通过收购具有低贴现倍数的资产A的方法来套利。最终的结果是,具有低贴现倍
数的资产A被迫转变为具有高贴现倍数的资产B[1],整体经济随之“脱实向虚”。

自诞生之日起,“土地财政”便在咒骂声中顽强而丑陋地生长,褒贬不一,毁誉参半,
在帮助中国迅速崛起为世界级的创新体和高技术玩家的同时,也在不断掏空实体经济;
在帮助更多的人卷入货币分工的同时,也造成了财富分布的巨大落差;在帮助货币减轻
对美元信用依赖的同时,也使中美在经济上开始分离、在政治上开始对立;高信用带来
的高贴现倍数也导致中国面对和美国类似的问题——整个经济不断“脱实向虚”[3],出
现“泡沫”和“高杠杆”等症状。资本市场存在的系统性风险,像达摩克利斯之剑一样,
高悬在高速增长的中国经济之上。

这是最好的时代,也是最坏的时代。当以千年为尺度的时候,我们很难意识到历史的拐
点。洪水来袭,避险财富最后会流向能够提供最大临界贴现倍数的货币。要想在危机中
幸存,要做的就是比对手跑得更快。

在信用货币制度下,加息、降息会产生和实物货币制度下完全不同的财富分配效应:若
加息(去杠杆),财富会从虚拟部门向实体部门转移,实现资本支持实体经济发展的政
策效果;降息(加杠杆)则会导致财富从实体部门向虚拟部门转移,出现资本剥削实体
经济的现象。市场利率低于影子利率的幅度越大,资本越便宜,实体经济就会越萎缩,
虚拟经济就会越膨胀。在全球化分工的格局下,通过不断降低利率,资本大国就可以实
现财富从劳动大国(实体经济)净转移。而负利率更是意味着哪怕社会总财富缩水(资
本不创造新增财富),虚拟经济也可以通过从实体经济“吸血”获得更多的财富。

在中国,股市同房市、矿产开发和征地拆迁一起,构成了社会财富向少数人转移的四大渠道,急需加以改进。




彭波

https://www.guancha.cn/pengbo/2023_02_25_681392_1.shtml

改革开放以来,中国还没有遭遇过一次真正的完整意义上的经济周期,整个社会存在较强的投机倾向,很多地方政府与企业似乎早就忘记经济是有周期的,在繁荣过后是会出现危机的。此次如果再次出现饮鸩止渴的结果,未来要调控房地产及金融市场就将更加困难。而房价涨得越高越快,未来的危机也就越严重越猛烈。


所有资本主义经济制度的国家都必须是负债增长,而其本质上是国家和普通民众向资本负债来维持运转。在同样的资源约束下,产生了同样的经济成果,但所有权和收益权因为债务关系而发生了变化。因为资本控制了社会的核心资源与生产要素。
举例说明。
比如现在的房地产。我们在现有的资源条件下,修建了这么多房子,这些房子最终归结为各种资本出资修建。也就是说这个社会本质上具有修建这么多房子的经济资源条件,但是社会民众必须负债来消费。
比如,中国各级政府负债修建了大量的基础设施。这些基础设施是现有的经济资源条件下修建出来的,但是中国各级政府本身并不对这些经济资源拥有所有权,因此只能负债修建。
这就是私有化条件下市场配置资源的必然结果。国家和社会的绝大部分经济资源配置给了资本,因此其所有权和收益权归资本,但是使用权可以用债务的形式配置给国家和社会民众。


%%% Local Variables:
%%% mode: latex
%%% TeX-master: "../main"
%%% End:
