\section{土地金融}
\url{https://www.thepaper.cn/newsDetail_forward_11744994}


非常重要!!!!!!!!!!!!!
!!\textbf{https://www.aisixiang.com/data/138457.html}
\textbf{王小鲁:土地财政的昨天、今天和明天}

https://pdf.dfcfw.com/pdf/H301_AP202305101586440193_1.pdf
https://www.gov.cn/gzdt/2008-12/19/content_1182391.htm
刺破的泡沫—海南房地产往事  沈奇 杨政 https://www.sohu.com/a/761245832_482521
从沸腾到癫狂:泡沫背后的中国房地产真相
26年了,五届全国金融工作会议重大金融改革回顾(上)
https://new.qq.com/rain/a/20231029A080O900

\section{海南房地产泡沫危机}
\label{sec:hainan}

1988年8月23日,海南脱离广东省独立建省,并在海南设立经济特区。海南成为当时中国
最新、最大且是唯一的省级经济特区,成为全国各地淘金客们的狂欢
地。1989年到1993年间,商品房平均价格从1400元/平方米到7500元/平方米;商品房年
销售面积从1000万平方米到9000多万平方米;房地产投资总额从3.2亿元到93亿元;土地
价格从几十万元/亩到600万元/亩,


\begin{quotation}
  据太平洋证券数据,高峰时期,当时总人数不过656万的海岛上有两万多家房地产公司,
  平均每300个人一家房地产公司。与此同时,大量金融机构进入,海南短时间内成立
  了20家信托投资公司,35个城市信用社,此外还有一批股份制、会员制的金融性公司;
  证券行业从1989年初建立海南省证券公司,到后来已发展为26个证券机构34个证券交
  易点。\footnote{沈奇 杨政《刺破的泡沫—海南房地产往事》}
\end{quotation}
海南房地产此时玩的已经是金融“击鼓传花”游戏,下海官员、批条、虚拟概念(方案和图纸)、
炒楼花、信贷,上一人传给下一人,死的是最后一个接盘侠……

\begin{quotation}
  1993 年 6 月 23 日,时任国务院副总理的朱镕基发表讲话,宣布终止房地产公司上
  市、全面控制银行资金进入房地产业。24 日,国务院发布《关于当前经济情况和加强
  宏观调控意见》,\textbf{16 条}强力调控措施包括严格控制信贷总规模、提高存贷利率和
  国债利率、限期收回违章拆借资金、削减基建投资、清理所有在建项目等。银根全面
  紧缩,一路高歌猛进的海南房地产热顿时被釜底抽薪。这场调控的遗产,是给占全
  国 0.6%总人口的海南省留下了占全国 10%的积压商品房。全省“烂尾
  楼”高达 600 多栋、1600 多万平方米,闲置土地 18834 公顷,积压资金 800 亿元,
  仅四大国有商业银行的坏账就高达 300 亿元。
\end{quotation}

同期不止海南,北海、惠州、上海等地也出现了房地产热。朱镕基在1999年1月8日
在省部级领导干部金融研究班上对这时期的房地产热总结到:
\begin{quotation}
  从计划经济体制向社会主义市场经济体制过渡中,许多长期积累的深层次矛盾集中反
  映在金融领域。仅从房地产看,1992年、1993年的房地产热,就造成银行不良贷款几
  千亿元。目前,全国闲置的商品房多达7000多万平方米,价值在4000亿元左右;光是
  海南省就闲置1600万平方米,占压银行资金460亿元。原来占用的那些贷款,连本带息,
  现在要翻一番。同时,建起来的房子质量差、价格高,地理位置也不好,很难处理掉。
  这些绝大部分已成为银行的呆账、坏账。
\end{quotation}

\begin{quotation}
  海南烂尾房消化到2007年才接近结束。据泽平宏观数据,截至2006年10月,全省累计
  处置闲置建设用地23353.87公顷,占闲置总量的98.17%,处置积压商品房444.82万平
  方米,占积压总量的97.6%。而海南省的房价到2010年才再一次回到1993年时的水平,
  海南经济的萧条周期近20年。
\end{quotation}


\section{土地金融历史}
\label{sec:tudijnrong}


1990年5月19日国务院发布《城镇国有土地使用权出让和转让暂行条例》和《外商投资开
发经营成片土地暂行管理办法》,明确规定土地使用权可以采用\textbf{协议、招标和拍卖}三
种方式。“这标志着中国的土地市场走上了有法可依的轨道,从而使土地使用制度改革
在全国推开。”


1992 年 11 月,国务院发布《关于发展房地产业若干问题的通知》,提出进一步深化土
地所有制改革,继续深化城镇 居民住房制度改革。房地产的发展在 1992、1993 年迎来
了第一个高潮。1992、 1993 年商品房销售总额同比分别增
长 79.35\% 和 102.47\% ;1993 年固定投资完成额 累计同比始终保持在 60\%左右。

1993 年 6 月,国务院发布《国务院关于当前经济情况和加强宏观调控的意见》,提出
了“国 16 条”,即整顿金融秩序、加强宏观调控的 16 条政策措施。这是第一次国家
宏观调控过热房地产。其中包含要严控涉房信贷,坚决制止炒房地产获取暴利的行为,
房地产开发投资必须纳入固定资产投资计划等。海南房地产击鼓传花游戏走到尽头,泡
沫破灭。

经历初步探索之后,住房制度改革的市场化特征显现,逐渐形成以权力下放、培育市场
主体为特征的改革。1994年,国务院发布《关于深化城镇住房制度改革的决定》,提出
按照国家、企业和个人合理负担的原则进行住房体制改革,将公房实物分配改为货币工
资分配,建立面对中低收入的保障性住房和面对高收入家庭的商品房,建立住房公积金
制度。至此,开始全面市场化的改革。\cite{CJZK201802012}

1994年这份《决定》并没有有效刺激房地产市场,当时东北国企已开始出现实质性大下
岗,全国主流仍是福利分房,毕竟机关事业和国企工作人员如果可以福利分房的话谁会
去市场买房呢?房地产市场尚不具备较大投资和金融价值。

据谢家谨《房地产这10年》回忆,
\begin{quotation}
  1996年7月11日,朱镕基在听取国务院房改领导小组汇报时指出:目前的宏观经济形势
  很好,是1993年以来最好的时期。但问题是国有企业亏损增加,经济效益下降;原因是结
  构不合理,企业生产的产品不适销对路;解决的途径是加大国有企业改革力度,调整产品
  结构,减少库存积压。关键是打开市场,搞活流通,培植\textbf{新的消费热点和经济增长点}。
  当前最有可能形成消费热点的是\textbf{房地产}……一要推进\textbf{住房商品化},二要有个好
  的规划。
\end{quotation}

1997年初中央决定继续实行适度从紧的财政货币政策。同年12月6日,国务院发布《关于
深化金融改革,整顿金融秩序,防范金融风险的通知》,提到“前些年出现的房地产热、
开发区热,造成大量不良信贷资产,其中大部分已成为呆账、坏账,无法收回”。

但97年开始的亚洲金融危机已开始影响我国,形势严峻,财政紧缩尚未实施多少时日,
便转为宽松,\textbf{此后财政政策在一两年、甚至两三个月内紧松反复成为常态。}


1998年5月,《个人住房贷款管理办法》发布。7 月,国务院发布《关于进一步深化城镇
住房制度改革加快住房建设的通知》,正式宣布\textbf{1998年下半年开始停止住房实物分配
  (取消福利分房),逐步实行住房分配货币化} ;
建立和完善\textbf{以经济适用住房为主}的多层次城镇住房供应体系;\textbf{发展住房金融},培育和规
范住房交易市场。


\begin{quotation}
  我国金融领域风险究竟是怎么造成的?其中原因很多,主要有以下四个方面。

  首先是历史遗留的问题。从计划经济体制向社会主义市场经济体制过渡中,许多长期
  积累的深层次矛盾集中反映在金融领域。仅从房地产看,1992年、1993年的房地产热,
  就造成银行不良贷款几千亿元。目前,全国闲置的商品房多达7000多万平方米,价值
  在4000亿元左右;光是海南省就闲置1600万平方米,占压银行资金460亿元。原来占用
  的那些贷款,连本带息,现在要翻一番。同时,建起来的房子质量差、价格高,地理
  位置也不好,很难处理掉。这些绝大部分已成为银行的呆账、坏账。目前政策性银行
  的风险也很大。粮棉收购贷款长期被严重挤占挪用,大量的亏损在银行挂账。国家开
  发银行一个报告反映,煤炭建设项目所形成的不良贷款,不但使煤炭行业陷入困境,
  也对开发银行的生存和发展构成严重威胁。

  二是重复建设。多年来盲目上项目,搞了大量的重复建设,又主要是靠银行贷款,许
  多企业借银行钱的时候就根本没有想到要还。前些年银行固定资产投资贷款利率
  在10\%以上,什么项目有这么高的回报率?不少项目建成之日就是亏损之时。例
  如,90年代初期,全国一下子上了十几套小乙烯项目(年产11.5万吨)。小乙烯成本高,
  产品单一,没有销路。广州小乙烯投资花了80多亿元,投产后一年就得亏损几亿元。
  重复建设不仅形成了巨额银行不良贷款,也造成大量的重复生产,而靠银行流动资金
  贷款生产出来的产品没有市场,积压在仓库里,企业亏损又都挂在银行账上。

  三是行政干预。一些地方和部门的领导干部,缺乏金融知识,不懂金融法律、法规,
  随意干预银行和其他金融机构的正常经营活动,把金融机构贷款当做财政的钱来花。
  前些年,许多地方搞“现场办公会”、“资金调度会”,往往都是指令银行或其他金
  融机构贷款。缺乏市场调查,缺乏科学决策,结果不少贷款都投向了没有市场、没有
  效益、没有还贷能力的项目,大都是有去无回;甚至还有相当部分资金被用于盖办公
  大楼、发奖金和政府行政开支,最后都形成了金融不良资产。广东省恩平市原领导班
  子肆意干预金融活动,造成80多亿元的金融资产损失,就是一个很典型的事例。最近
  一个时期,一些地方刮起出售国有小企业之风,搞所谓“一卖了之”。名义上卖,实
  际上是半卖半送,甚至是明卖暗送,许多银行的贷款都被冲掉了。工商银行去年搞了
  一次检查,发现在企业“改制”中悬空和逃废该行债务有1000多亿元。

  四是金融系统腐败现象严重。许多银行和其他金融机构严重违法违规经营,高息揽储,
  搞账外经营,谋取小团体甚至是个人私利,造成许多资金收不回来。一些金融从业人
  员素质不高,以权谋私,内外勾结,营私舞弊,腐化堕落,使大量金融资产流失。对
  金融系统中的腐败行为,我们虽然一直采取严厉的惩处措施,但这些年金融案件仍屡
  屡发生。

  当前我国金融领域的问题确实比较严重,防范和化解金融风险的任务也十分艰巨,但
  信心绝不可动摇。我们完全有条件、有办法、有能力解决所面临的问题。20年来的改
  革和发展奠定了良好的基础,经济实力比较雄厚,当前宏观环境比较宽松,重要产品
  和外汇储备充足,党中央有驾驭复杂局势的能力,我们也积累了一些化解金融风险的
  经验。正确认识问题所在,也是解决问题的必要条件。充分运用这些有利条件,我们
  就能够抵御任何风险,战胜任何困难。因此,我们在看到问题和困难的时候,更要看
  到光明,鼓起勇气,增强信心。当然,从根本上解决金融领域的问题,绝非轻而易举,
  需要全党、全国上下齐心协力,同舟共济。
\end{quotation}




\begin{quotation}
  基层的首创精神,在2001年国务院15号文件(即《关于加强土地资产管理的通知》)
  和2002年国土资源部11号令(即《招标拍卖挂牌出让国有土地使用权规定》)中受到
  了高度重视。国务院15号文件提出,从严格控制建设用地供应总量,严格实行国有土
  地有偿使用制度,大力推行招标拍卖,加强土地使用权转让管理,加强地价管理和规
  范土地审批行为。国土资源部11号令则对经营性土地协议出让“叫停”,明确四类经
  营性用地使用权出让必须采用招拍挂方式。这两个文件对土地市场建设的推进作用显
  而易见:2000年全国招标拍卖出让土地的收益为350亿元,2001年为492亿元,增长率
  为40\%。2002年,全国国有土地使用权招标拍卖挂牌出让的宗数、面积、价款分别是
  上年的108.55\%、273.8\%和197\%。



中国人民银行办公厅2003年6月6日印发《关于进一步加强房地产信贷业务管理的通知(银发
〔2003〕121号):
\begin{quotation}
  通知规定:

  商业银行严禁以房地产开发流动资金贷款及其他形式贷款科目发放房地产开发贷款,
  并要求\textbf{企业自有资金不低于开发项目总投资的30\%;对土地储备机构发放抵押贷款,
    贷款额度不得超过所收购土地评估价值的70\%,贷款期限最长不得超过2年};对未
  取得土地使用权证书、建设用地规划许可证、建设工程规划许可证和施工许可证的项
  目,\textbf{不得发放任何形式的贷款};商业银行发放的房地产贷款,\textbf{严禁跨地区使用};商业
  银行不得发放用于缴交土地出让金的贷款。

  \textbf{商业银行只能对购买主体结构已封顶住房的个人发放个人住房贷款};对借款人申请个
  人住房贷款购买第一套自住住房的,首付比例仍执行20\%的规定;“对购买第二套以
  上(含第二套)住房的,应适当提高首付款比例”;“个人商业用房贷款的抵借比不
  得超过60\%,\textbf{贷款期限最长不得超过10年,并且所购商业用房为竣工验收的房屋}”。
  (人民银行提供)
\end{quotation}
时任金融监管机构——中国人民银行货币政策司司长的戴根有被认为是起草121号文件的
关键人物。据他自述2001年就注意到房地产市场一些局部问题,其2002年向人行领导汇
报“一是房地产投资增长明显高于销售的增长,第二是商品房空置面积明显增长,而且
增长幅度非常高。因而给出了“部分地区房地产出现过热”的判断意见。”2002年4季度
抽查商业银行房地产信贷情况发现,房地产信贷违规金额接近1/4。\footnote{21世纪新闻报
  道,\href{https://finance.sina.com.cn/x/20031101/1105501003.shtml}{121文件
    内幕 戴根有回应:谁说121与18矛盾?}}

121号文件受到很多大房地产商的声讨和行动抵
制\footnote{https://news.sina.com.cn/c/2003-09-16/14051753586.shtml}。两个多月后《国
务院关于促进房地产市场持续健康发展的通知》(国发〔2003〕18号)出台,提到“房
地产业关联度高,带动力强,已经成为\textbf{国民经济的支柱产业}”“\textbf{发展住房信
  贷}”“\textbf{建设部要会同有关部门……指导各地具体实施并负责对本通知贯彻落实情况
  的监督检查。}”。

地产商冯仑就此说到“\textbf{商人的声音首次大过了政府的声音}”,一些评论家认为这
是“\textbf{地产商阶层的胜利}”。10月份,戴根有调任中国人民银行征信管理局局长。

\begin{quotation}
  有媒体透露,1987年至1999年,深圳市利用拍卖和招标两种方式一共卖出了80多宗地,
  面积基本上都在1万平方米左右。而每年协议出让面积是100多万平方米。两者相差甚
  远。1995年、1996年还一度终止了土地拍卖。1997年连一次招标或拍卖都没有举
  行。1998年深圳市土地出让金达108亿元,但这一年仅有的两次招标和两次拍卖,一共
  只有3.3亿元。1999年之前,深圳90\%的土地实行的是非市场价格的协议出让。

  协议出让意味着存在很强的人为操纵的可能性。让行政权力从富有诱惑力的利益空间
  中退出,发挥市场机制在土地资源配置的基础性作用,需要很大的勇气,但能带来巨
  额的回报。1999年,浙江开始在全省范围内实行\textbf{经营性用地一律招标拍卖制
    度}。2000年,全国土地招标拍卖收益为350亿元,而浙江一个省就达195亿元。
\end{quotation}

1999年1月1日起,修订后的《中华人民共和国土地管理法》开始实施。其中明确规定,
农业用地要是想变为建设用地,需经过征地环节。只有变成国有土地后才能上市转让。
这也就确立了政府对于建设用地的垄断地位。

2003年6月25日,时任国家审计署审计长的李金华作2002年年度审计报告时提到:
\begin{quotation}
  (2001)年抽查建行广州地区8家支行的楼宇按揭贷款,发现有\textbf{10亿元是虚假按揭},
  有的不法分子甚至内外勾结,骗取银行资金。如广东省汕尾市公安局某副局
  长,1998至1999年,冒用他人名义,出具虚假证明,骗取建行广州市芳村支行按揭贷
  款3793万元,有3270万元已无法追回,其中转入该副局长等人个人账户的2576万元全
  部被提取现金,去向不明。目前,该副局长等9名犯罪嫌疑人已被公安机关拘捕。
\end{quotation}

2005年8月15日,央行金融市场司房地产金融分析小组在发布的《2004中国房地产金融报
告》中提出,“很多市场风险和交易问题都源于商品房新房的预售制度,目前经营良好
的房地产商已经积累了一定的实力,可以考虑\textbf{取消现行的房屋预售制度},改期房销售
为\textbf{现房销售}”。

房地产商们多以现房销售将引起房价上涨为理由反对现房销售。笔者认为,真实原因显
而易见,资本只有自我增殖这一本质目的,现房销售将引起房地产大幅降杠杆,资金流
也随之匮乏才是真正原因。房价上涨是这一真正原因的副作用而已。

袁一泓写到:
\begin{quotation}
  8月24日,\textbf{建设部}新闻发言人表示,商品房预售制度是《城市房地产管理法》确立的
  一项制度。从十多年的实践看,这一制度与我国国情是适应的,目前还不能取消。

  \textbf{地产商们又赢了。}\cite{2011feiteng}
\end{quotation}
笔者认为地产商、建设部乃至商业银行和地方政府,均属于这一阶层阵营。各方对金融
资本增殖的贪婪第二次战胜了人行的金融监管职能和防范金融风险的要求。


一些地方为招商引资,工业用地出让中长期存在着低地价乃至“零地价”行为,严重干
扰了土地市场秩序,为一些地方搞低水平重复建设和扩大固定资产投资提供了条件。其
获取土地的方式必须加以改革。国务院15号文件已经提出:除按现行规定必须实行招拍
挂的土地外,工业用地也要创造条件逐步实行招拍挂出让。这个规定是工业用地进入招
拍挂的一个信号。

  工业用地实行招拍挂出让呈不可阻挡之势。2004年出台的《国务院关于深化改革严格
  土地管理的决定》(即国务院28号文件),有针对性地指出:“必须严禁非法压低地
  价招商”,同时要求要加快探索和实践,加快工业用地进入市场化配置。2006年出台
  的《国务院关于加强土地调控有关问题的通知》(即国务院31号文件),则完全把工
  业用地纳入了市场竞争的范围,要求“工业用地必须采用招标拍卖挂牌方式出让,其
  出让价格不得低于公布的最低价标准” 。

  2006年12月27日,国土资源部发布《全国工业用地出让最低价标准》,并将
  从2007年1月1日起实施。此举标志着,我国工业用地必须采用招标拍卖挂牌方式出让,
  其出让底价和成交价格均不得低于所在地土地等别相对应的最低价标准。

  工业和经营性用地招拍挂出让,由国家政策上升为国家法律。2007年3月16日颁布的
  《物权法》明确规定:“工业、商业、旅游、娱乐和商品住宅等经营性用地以及同一
  土地有两个以上意向用地者的,应当采取招标、拍卖等公开竞价的方式出让。”

  要用地,找市场。土地进入市场的配套制度建设至关重要。随着市场机制配置土地资
  源的范围不断扩大,我国土地市场运行制度和组织建设取得了明显进展,市场服务体
  系逐步形成。一是土地公开交易、基准地价定期公布、土地市场交易信息发布和土地
  登记资料公开查询等制度基本建立,保障了土地市场的规范运行;二是完善土地市场
  组织形式,全国各地纷纷设立土地有形市场,土地使用权交易更加便利、安全、公开
  和公平;三是转变政府职能,减少和规范行政审批,培育和发展了土地市场中介服务
  组织,土地估价、咨询、交易代理、登记代理等中介服务机构独立、客观、公正执
  业。

  竞争机制的引入,使我国土地资源市场化配置程度显著提高。2007年,我国全面推行工
  业用地招拍挂出让和工业用地最低价标准,全国共出让土地22.65万公顷,其中招拍挂出
  让土地11.53万公顷,占出让总面积的50.9\%,比上年提高了20.4个百分点。2008年上半
  年,我国招拍挂出让土地74.2万亩,占出让总量的81.9\%,同比提高45.9个百分点,其
  中工矿仓储用地招拍挂出让比重达74.3\%,同比提高63.8个百分点。地价监测范围扩大
  到105个城市。

  由此可见,推进土地使用制度改革,不是简单地让土地进入市场,而是一个建设、培
  育、规范土地市场的复杂过程。
\end{quotation}

国家税务总局从2005年以来的5年内,曾经8次发文,对土地增值税的预征和清算提出要
求,但各地进展甚微。包括房地产上市公司在内,地产商一般是以1%的预征率来进行土
地增值税的计提。

北京科技大学中国经济研究组赵晓、陈金保、田艳春,2010年4月联名发表的文章指出:
目前,大部分地方土地增值税的预征率只有1%,远远低于实际清算的税率。因此,很多
房地产项目销售完成后,房地产企业想尽一切办法,故意久拖不清算,或者项目结束后
干脆玩“人间蒸发”,令土地增值税清算无从谈起。2007年1月,国家税务总局曾要求各
地税务机关严格土地增值税清算,此举曾令开发商大为紧张。然而,金融危机呼啸而来,
经济形势急转直下,这一号称“革房地产商命”的税收政策再一次不了了之。赵晓等吁
请严征土地增值税,以“100%预征、100%清算”来遏制新一轮楼市疯狂。

2006 年在国内房价高启、地产行业景气上升的大背景下,房地产上市公司的盈利能力
超过了 A股市场的平均水平。共实现主营业务收入 927.48 亿元,净利润 83.36 亿元;
同比增长接近 30\%和 60\%。(据水清木华研究中心《2006-2007 中国房地产业上市
公司研究报告》)


据21世纪新闻报道\footnote{\url{https://news.cctv.com/financial/20080326/100431.shtml}},
\begin{quotation}
  国家审计署2007年第86号《上海市社保基金运营及管理情况专项审计》报告所披露的
  社保基金整体状况远比此前所暴露的福禧个案更为触目惊心:"多年来,上海市社保局共
  计运营基本社会保险基金、企业年金、小城镇保险等8个险种的社保资金,金额共计
  达329.44亿,截至2006年7月17日,尚未收回的资金达255.41亿,占运营资金余
  额387.31亿元的66\%。"

  根据报告,这些违规运营的社保资金大量投向了和国家宏观政策导向不符的产业领
  域——比如房地产业。

  "截至2006年7月17日,\textbf{对44家房地产企业贷款余额201.25亿},社保资金甚至给不法商
  人和企业提供了大量的资金支持。"
\end{quotation}

养老金个人账户依照1997年《国务院关于建立统一的企业职工基本养老保险制度的决定》
要求,应是实收账户——账钱相符、账人相符、账账相符,但历年来均有被地方政府挪
用至养老金统筹账户或者金融投资、搞基建和房地产,具体数字不可考。其后二十年,国
务院数次要求坐实养老金个人账户,目前可能已直接转为名义账户——只记帐面数字,
不做实。

2007年全国共完成房地产开发投资额25280亿元,增长30\%,增速上升7.4个百分点,高
于其他行业的投资增速;占城镇总投资比重高达21.5\%……吸引了大量信贷资金进入,
房地产业利用国内贷款规模达到6961亿元,增长32.3\%。07年虽是房地产行业宏观调控
力度最大的一年,但外资并未放缓进入中国的步伐,全年房地产开发利用外资高达650亿
元,增长65\%,增速加快8.24个百分点。预收款大幅增长31\%。2007年,全国商品房销
售面积、销售金额分别为76193万平方米、29604亿元,分别增长25.7\%、44.3\%;其中,
住宅销售面积、销售金额分别为69104万平方米、25323亿元,分别增长27\%、48.6\%。
 一线城市房价涨幅尤为巨大,07年12月份,深圳住宅价格同比涨幅为51\%,北京45\%,
广州达到了30\%,津、渝、沪三地的同比涨幅也都超过了15\%。各地地王频出。一些大房
地产商盲目乐观,大举借贷。

美国次贷危机显现,加之国内房地产过热,销售不畅,信贷规模过大,央行数次加息后
于2007年9月27日联合银监会联合发布《关于加强商业性房地产信贷管理的通知》(银发
〔2007〕359号),被称为“927房贷新政”,严格房地产相关信贷管理。

 一、昨天:土地财政从何而来?

在旧的计划经济体制下,我国土地不是商品,需要的建设用地由政府从农民手里征用,并进行无偿划拨。

上世纪80年代,少数城市开始尝试土地有偿出让和住房商品化改革,1990和1992年国务
院分别发布了土地出让的暂行条例和暂行管理办法,1993—1994年开始推行住房商品化
改革,土地和住房才开始成为商品。

1998年,全国国有土地出让成交价款约500亿元,相当于当年地方财政预算收入的10\%。
此后土地出让收入以不可抑制的高速度膨胀——2001年1300亿元,接近地方财政收入的
17\%;2011年3.3万亿元,相当于地方财政收入的63\%;2021年8.7万亿元,相当于地方
财政收入的78\%(均不包括在地方财政预算内)。

从2001到2021这20年,现价GDP年均增长12.4\%,现价全国财政预算收入年均增长
13.4\%,而土地出让收入年均增长23.4\%,远超过了经济增长和财政收入增长(据财政
部、国土资源部、国家统计局公布数据计算)。

计算平均地价,2001年每公顷土地出让收入143万元,2011年为978万元,2021年为2393
万元。20年间地价上涨15.7倍。

同期消费价格指数(CPI)只上涨58\%,工业生产者价格指数(PPI)只上涨38\%(据财
政部、国土资源部、国家统计局、华经产业研究院数据计算)。

过高的地价大幅度推高了房价,并改变了国民收入分配格局。由于不包括地价和房价,
消费者价格指数(CPI)和生产者价格指数(PPI)已不足以反映真实的通货膨胀率。

而地价和房价飞速上涨,是几方面影响因素合成的结果。

从供给侧来说,一个正常因素是城市规模经济的溢出效应带来的城市用地和城市周边待
开发土地升值。

从需求侧而言,城市化快速发展造成的房地产需求上升,而土地资源相对有限,持续拉
动了房价和地价上涨。

第三个因素的作用可能更加突出,即长期以来货币增长大幅度快于经济增长,货币超发
造成的多余购买力大量流向房地产,造成越买越涨、越涨越买的趋势,成了长期不破的
资产泡沫。

第四个重要影响因素是地方政府的作用。对地方政府来说,土地出让收入来得容易,用
得方便,监管不严,透明度低,一方面可以用来搞各种地方政府想搞但没有其他资金来
源的投资项目,既包括城市基础设施和其他公共建设,也包括政府楼堂馆所等各种自我
服务设施,另一方面还可以增加包括“三公消费”在内的党政机关行政管理费用。

更严重的是,在政府垄断地源的情况下,大量的官商勾结腐败案件都与土地和房地产交
易有关,给腐败官员带来了巨额收入,造就了一大批隐性富豪——在种种利益驱动之下,
地方政府可以利用其独占地位,通过控制土地投放量,借“招拍挂”尽量抬高地价,以
获得最大收益。这也导致政府在资源配置中所占份额不断扩大,挤压了市场配置资源的
份额,形成与改革开放前面20年的放权让利截然相反的趋势。

※※※※※※※※※※※※※※※※※※※※※※※※※※


反面论述
https://www.aisixiang.com/data/115819.html
赵燕菁:从土地金融到土地财政:资本的胜利、有为的政府与城市的转型

※※※※※※※※※※※※※※※※※※※※※※※※※※
https://bfi.uchicago.edu/wp-content/uploads/2022/02/BFI_WP_2022-24.pdf

文献中的典型观点是,住宅用地出让主要是地方政府增加收入的一种方式,而工业用地
出卖主要是为了补贴产业、刺激经济增长、支持劳动力需求。

中国的土地市场具有相当大的工业折扣:工业区用地比住宅用地便宜一个数量级。与以
产业补贴或促进产业增长为中心的解释相反,我们强调了未来土地税收的重要性,并发
现地方公共财政激励措施可以在很大程度上合理化这种价格差距。在“土地融资”制度
下,土地出让是中国地方政府的重要收入来源。研究表明,在中国,地方政府作为垄断
性土地销售者,面临着住宅用地或工业用地供应之间的权衡,这取决于工业和住宅用地
销售收入的不同时间分布、地方政府的财政约束以及地方政府与其他各级政府分享税收
的程度。

公司税收收入和土地出让收入;2019年,这两个数字分别约为8.7万亿元人民币和7.3万
亿元人民币.2工业用地产生持续的未来税收流动,因为工业企业缴纳增值税和所得税,
以及各种费用。由于中国没有住宅物业税,住宅用地销售只会暂时增加房屋开发商缴纳
的税款.3这意味着地方政府面临着一个选择,即出售前期收入较大的住宅用地和出售工
业用地,因为工业用地支付的税收现金流比实际收入更持久。

这种动态的观点意味着,大量的前期工业用地折扣并不一定意味着政府正在通过廉价土
地系统地补贴工业。事实上,我们表明,在调整住宅开发商缴纳的税款后,来自工业用
地的税收流动可以定量补偿前期工业用地折扣。我们还提供了地方政府的融资需求影响
土地分区的因果证据,表明地方公共财政在通过土地分配渠道塑造中国经济增长路径方
面发挥着被低估的作用

我们从一个概念框架开始,分析推动工业用地而不是住宅用地供应均衡回报的力量。我
们考虑的是地方政府,其目标是最大化其财政收入的现值。除了全部属于地方政府的前
期土地销售收入外,住宅用地还产生了由房屋开发商支付的一次性税款,而工业用地则
产生了工业税的持续现金流,并与中央政府共享。在均衡状态下,由于未来的税收优惠,
当地政府愿意以较低的价格出售工业用地。该框架指出了两个简单且可衡量的汇总统计
数据。首先是工业折扣,即工业用地和住宅用地之间的价格差异。第二种是工业用地销
售的内部收益率(IRR),计算为贴现率,折现率等于工业和住宅用地销售的所有现金
流的现值

※※※※※※※※※※※※※※※※※※※※※※※※※※

就像许多其他国家一样,中国也有严格的分区限制。正如Chen等人(2018)所强调的那
样,划为住宅用途的土地的售价大约比划为工业用途的土地高出十倍。2019年,中国住
宅用地平均价格为3,619元/平方米,工业用地平均价格为304元/平方米。我们将住宅用
地和工业用地之间的这种价格差异称为工业用地折扣(或工业用地互换)。

※※※※※※※※※※※※※※※※※※※※※※※※※※

地方政府从土地中获取融资有两条直接渠道:一是土地财政,即通过出让国有土地(主
要是商服用地和住宅用地)50至70年的使用权来获取级差地租;二是土地金融,即将土
地注入地方融资平台来撬动资金为城市建设融资。

已有研究表明,土地财政和土地金融相结合的以地融资模式催生出一个高效的融资体系,
极大推动了近二十年来的中国经济增长,特别是与房地产和基建相关(包括钢铁、水泥)
的重工业部门得到了飞速发展,居住环境和基础设施的改善又进一步促进了地价和房价
的升值,为下一轮以地融资创造了有利条件。这样的正向反馈机制使得中国自1998年以
来一直处于投资驱动型的增长阶段。

如果说曾经的以地融资模式主要依赖土地财政,那么如今,随着部分地区土地资源的告
罄和征地补偿标准的提升,一些地方政府对土地财政的依赖开始减退,而对土地金融却
愈加青睐。由表1可知,越是土地资源稀缺的地区(东部地区),土地抵押贷款规模越
是高于土地出让规模,而对北上广这样的一线城市而言,土地抵押与土地出让的比值更
是高于东部地区的平均水平。

图1显示,2003至2010年间,出让成本占比基本维持在六成左右,但在2011至2018年间,无论是土地出让成本的规模还是其占土地出让毛收入的比重都呈加速上升趋势,甚至到了2018年,土地出让成本占土地出让毛收入比重已接近九成。

从征地规模的角度亦可佐证土地出让成本在2010年前后出现了飘升的现象。图2显示,农用地和总的土地征收规模在2011年之前处于上升区间,但在2011年后呈逐年加速下滑的态势,这恰好是土地出让成本开始飘升的年份(参见图1)。迅速缩小的征地规模表明,未来国有土地的出让规模将大幅缩减,利用土地财政进行创收的空间将被进一步压缩。

1994年的税收分享改革,该改革减少了地方政府在许多税收收入来源中的份额,同时并没有减少他们的支出需求。要回答“为什么土地金融在中国(到目前为止)有效运作”要复杂得多。在这方面,了解其促进土地金融体系盈利能力的制度基础是很重要的。首先,地方政府被指定为地方城市LUR的垄断提供者。因此,当一个城市提出购买农村土地并将其转换为城市土地时,它不会面临来自其他实体的竞争。此外,支付的补偿是基于当前(农业)用途的土地价值,而不是未来城市用途。地方政府回购、分配或出售的城市土地利用率并将其转换为新的土地利用率时,也适用类似的规定。这种安排使城市政府能够获利,直到最近,土地价格的暂时上升趋势加剧了这一点。值得注意的是,支撑土地金融体系的丰富制度框架既有法律、官方文件等显性规则,也有在实践中具有影响力的隐性规范。其中一些细节在现有文献中似乎没有得到适当的承认,并可能导致对该系统的误解。希望我们的分析能够更清楚地说明该系统在中国联邦体系中的运作方式

自1980年代以来,中央和地方政府在中国经济中扮演的角色已经很明确——中央政府负责规划和监管,而地方政府则负责在地方层面实施这些法规。

从1980年到2020年,地方政府在国家一般公共预算中的份额从45.7\%增加到85.7\%,而同期收入在一般公共预算中的份额从75.5\%下降到55.5\%。为了弥补收支缺口,地方政府依靠中央政府补贴、债务融资和土地融资。

1994年实行的税制改革在随后的几年中对土地融资产生了巨大的溢出效应。改革规定,中央政府将把所有与土地有关的税收分配给地方政府。这包括土地价值税、城市土地使用税、耕地占用税、土地增值税、契税(财产转让税)和一般财产税。更重要的是,土地出让的所有利润都转移到了地方政府。

这一新制度为地方政府增加了土地出让收入提供了强大的动力。从2000年到2020年,地方政府土地出让收入占总收入的比重从5.9\%提高到42\%。如果将其他相关的土地税也考虑在内,土地收入占地方政府总收入的一半以上(52\%)。

o“通过信用制度,未来的收益可以贴现到今天,使得资本的形成方式得以摆脱对过去积累依赖,转向预期收益。”“中国城市伟大成就背后的真正秘密,就是创造性地发展出一套将土地作为信用基础的制度——‘土地财政’。”


土地财政是指政府出让国有土地使用权收入,及与土地有关的税收如耕地占用税、土地增值税等,这都是有法律、制度依据的。土地金融是指政府拿土地做抵押,向银行贷款,属于政府负债。这样做,没有法律依据,是打了制度的“擦边球”,因为制度允许政府作为国有土地所有权的代表经营土地。


但是当市场经济更多要求个人为个人负责,并将此观念植入人心时,一些原住民的死缠
烂打、坐地起价也就避无可避,因此拆迁往往要大费周章。突破口似乎只有合法或非法
的“暴力”。

合法暴力,可以宣称土地权利不归属于抗拒拆迁的人员,如印度最高法院声称孟买等地
贫民窟的长期居民是土地的非法占有者,没有权利要求拆迁赔偿,\textbf{承认获赔权等于奖
励盗窃行为};美国滥用征地权,把合理建筑中的长期居民赶走,鼓励高层建筑;声称
土地为集体所有,集体才是土地权利所有者,而非小部分人。

非法暴力,如20世纪90年代韩国首尔开发商雇佣黑帮“相扑似的打手”暴力拆迁;印度
西孟加拉邦南迪格莱姆地区的冲突惨案等。


《从沸腾到癫狂》

北京2005~2009年政府公布的商品房住宅建设用地计划供给指标为7130公顷,非商品住
房建设用地指标为1320公顷,两者的比重为5.4∶1。而实际供给的情况则是,商品房住
宅用地(招拍挂)2394公顷,完成计划数的33.6%,供地差额为4736公顷;非商品住宅
供地1283公顷,完成97.20%,扣除商品房中的配建,非商品房完成率为103%。这些土
地为享受经济适用住房政策的用地,约为商品房用地的2倍,由特定单位使用了。商品住
宅用地与非商品住宅用地的面积相加,可以计算出商品住宅用地在全部住宅建设用地中
的比重仅为28.3%。



任志强认为,市场中计算出的商品住宅平均销售价格仅为这28.3%的住宅销售价格,而并非北京市的住宅实际价格。如果用地中的平均容积率相等,且大部分经济适用住房和享受经济适用住房政策的住房价格为4500元/平方米,则北京市的一手房销售价格约仅为6200元/平方米。

任志强说,虽然那些低价的定向住房没有向社会公开销售,比如分给了国务院事务管理局,再分配给中央或国务院的各机关、管理机构以解决公务员的住房、进京干部的住房、老干部的住房等,但这些住房也是解决北京的住房问题,所以,应该在统计时计算进来进行平均。

房真实房价平均仅为6200元/平方米。而根据北京市统计局、国家统计局北京调查总队公布的《2009年北京市房地产市场运行情况》,2009年第四季度,四环路以内商品住宅期房销售均价为25907元/平方米。

6200元/平方米与25907元/平方米,为什么两者的差距如此之大?



任志强如是写道:



北京2005~2009年政府公布的商品房住宅建设用地计划供给指标为7130公顷,非商品住房建设用地指标为1320公顷,两者的比重为5.4∶1。而实际供给的情况则是,商品房住宅用地(招拍挂)2394公顷,完成计划数的33.6%,供地差额为4736公顷;非商品住宅供地1283公顷,完成97.20%,扣除商品房中的配建,非商品房完成率为103%。这些土地为享受经济适用住房政策的用地,约为商品房用地的2倍,由特定单位使用了。商品住宅用地与非商品住宅用地的面积相加,可以计算出商品住宅用地在全部住宅建设用地中的比重仅为28.3%。



任志强认为,市场中计算出的商品住宅平均销售价格仅为这28.3%的住宅销售价格,而并非北京市的住宅实际价格。如果用地中的平均容积率相等,且大部分经济适用住房和享受经济适用住房政策的住房价格为4500元/平方米,则北京市的一手房销售价格约仅为6200元/平方米。

任志强说,虽然那些低价的定向住房没有向社会公开销售,比如分给了国务院事务管理局,再分配给中央或国务院的各机关、管理机构以解决公务员的住房、进京干部的住房、老干部的住房等,但这些住房也是解决北京的住房问题,所以,应该在统计时计算进来进行平均。

按照任志强的说法,商品住宅用地28.3%之外的用地,一部分被用于建设北京市和在京中央和国家机关的公务员住房,他们的名称可以是两限房、经济适用住房、享受经济适用住房政策的住房、合建房、自建房中的任何一个。当然,也不排除这些公务员在享受了经济适用房等保障房以后,仍然自掏腰包去购买商品房。



地产商什么时候成为一个阶层了?正统的社会学者肯定要嘲笑我的无知。他们会教育我:你可以称他们为一个群体,一个集团,就是不能称其为一个阶层。是啊,连天天挂在老百姓嘴边的“中产阶层”都未获正式承认,哪里来的地产商阶层?

为什么地方政府会产生对土地财政像网瘾一样的依赖?根子在于土地招拍挂制度。这是一个影响力仅次于1998年房改的制度,并将继续对房地产市场和地方财政产生决定性的导向作用。

招拍挂制度必然导致地价上涨,而地价上涨则必然推动房价上涨。只要招拍挂制度存在,地王的产生就不可避免。地产GDP主义是被一种强力制度牢牢地固定住的地方政府价值取向。

并于2010年6月后成功超过日本成为全球第二大经济体,其中房地产贡献良多。据国家统计局数据,2010年,全国房地产开发投资完成48267亿元,同比增长33.2%;全国商品房销售面积10.43亿平方米,同比增长10.1%;商品房销售额约为5.25万亿元,同比增长18.3%。

因此,即使是全面调控的2010年,房地产的地位也丝毫没有下降。自从1998年实行住房商品化以后,GDP对房地产业的依赖就逐步形成,以至于到了离不开的地步,关键时刻总是要靠房地产“挺身救主”。实事求是地说,这方面没有一个产业能比得上房地产。

以上是从全局,从中央政府层面来说。对地方政府来说,房地产就更重要了。土地财政,这已成为大多数地方政府最重要的财政收入来源。城市形象、政绩工程也离不开地产商和房地产业,许多地方的道路等基础设施都是由地产商修建的。

2004年以来,地方政府体会到土地公开出让的无可替代的好处,招拍挂出让普遍推行。而招拍挂一般是净地出让,拆迁是政府的事。而土地收益当然要远大于拆迁补偿款,否则他们是没动力拆的。地方政府为了尽快把土地卖出去,就会做出强制拆迁、暴力拆迁的事情来。


—1991年实施和2001年修改的《城市房屋拆迁管理条例》。依此条例,地方政府贴一纸拆迁通知,在被拆迁的房屋上圈画一个大大的“拆”字就够了。居民不肯搬迁?先有房管部门行政裁决强制拆迁;还不服?接着可能就是法院裁决直接带着铲车来。许多地方连行政和法院裁决都懒得走,直接派消防队和铲车来。这才叫雷厉风行,超级效率。中国城市面貌的日新月异,城市建设的突飞猛进,该《条例》立了大功。

不少专家称这部条例是一部恶法,出台伊始就呼吁修改之。修改后的条例草案数易其稿,并分别于2010年1月29日和12月15日分两次向全社会公开征求意见,这在我国立法史上前所未有,也从一个侧面证明了拆迁所涉及的利益极为纷繁复杂。北大法学教授说,新的条例难以正式出台,主要就是遭遇到地方政府的强烈阻挠。千呼万唤之后,2011年1月21日,国务院正式出台《国有土地上房屋征收与补偿条例》。



2007年1月,国家税务总局曾要求各地税务机关严格土地增值税清算,此举曾令开发商大
为紧张。然而,金融危机呼啸而来,经济形势急转直下,这一号称“革房地产商命”的
税收政策再一次不了了之。赵晓等吁请严征土地增值税,以“100%预征、100%清
算”来遏制新一轮楼市疯狂。

到遏制投机的作用一样,对地产商和投资者个人尤其是地产商征收土地增值税,亦未能成为阻止涨价的动力。

事实上,土地增值税更像是为地方政府找到了一个新的税收来源,虽然这可能并非该税种设立的初衷。

严格征收土地增值税,能够为地方政府带来多大的税收?《21世纪经济报道》记者荆宝洁曾获得一份未公开的政协提案。一位不愿透露姓名的全国政协委员在这份提案中说,根据计算,2005年到2009年土地增值税共流失2.52万亿元,即平均每年流失5000亿元。而2005年以来的这5年,恰恰是全国房价和地价普遍快速上涨的时期,也就是说,这5年流失的土地增值税按规定应该是逐年递增的。

实事求是地说,2010年以来的保障房建设规模,令人惊讶。

很难想象,如果没有2008年10月推出的4万亿政府投资计划,保障房建设会有今天如此庞大的规模。我并不是说加快保障房建设是为了应对国际金融危机而采取的措施,而是说,保障房建设规模如此庞大,沾了4万亿“救市”资金的光。

2008年11月5日召开的国务院常务会议,确定了进一步扩大内需、促进经济增长的十项措施(“出手要快,出拳要重”的说法由此而来),第一条就是加快建设保障性安居工程。在4万亿投资里,坊间曾有保障房占8000亿元、9000亿元的说法,但并不准确。

根据国家发改委主任张平2009年3月8日在十一届全国人大二次会议新闻发布会的介绍,在4万亿政府投资计划中,保障性住房投资总规模为4000亿元左右。从张平的介绍来看,保障房包括廉租房、林区、垦区、煤矿棚户区改造,但没有提到限价房、经济适用房、公共租赁房等的投资,所以实际投资规模要远远大于4000亿元。

2008年受国际金融危机影响,除北京等极个别城市外,各地对保障房投入仍然不足。真正较大规模的保障房建设是从2009年开始的。

根据2010年3月5日全国“两会”上的《政府工作报告》,2009年中央财政安排保障性安居工程补助资金551亿元,比上年增长2倍。新建、改扩建各类保障性住房200万套,棚户区改造解决住房130万套。从中央财政投入看,增幅甚大。那么,2009年地方对保障房的投入有多少呢?

全国人大常委会专题调研组2009年10月月底披露的一份调查报告显示,截至当年8月月底,全国保障性住房建设完成投资394.9亿元,完成率仅为23.6%。以此推算,当年全国保障房投资计划大约为1600亿元。减去中央财政投入的551亿元,地方即使足额完成投入计划,亦不过1050亿元。

财政部2007年11月颁布的《廉租住房保障资金管理办法》规定,应从土地出让净收益中按照不低于10%的比例安排用于廉租住房保障资金。2009年全国土地收益实际到账14239亿元。也就是说,2009年各地仅廉租房就应投入1424亿元。加上公共租赁房、经济适用房和限价房,总投入至少应该倍增。而事实上,2009年各地保障房实际投入额尚不足财政部规定的廉租房投入额,遑论加上公共租赁房、经济适用房和限价房。

当然,从全国“两会”《政府工作报告》公布的数据看,2009年全国保障房建设任务是完成了的。330万套保障房的投资,应远远超过1600亿元。

2010年的保障房建设规模更加惊人。根据住房和城乡建设部的安排,2010年全国计划建设廉租住房、公共租赁住房等300万套,改造各类棚户区住房280万套,农村危房改造试点120万户。农村危房改造不计,人们将2010年的保障房任务简称为580万套。从330万套,到580万套,增长超过70%。

https://www.mof.gov.cn/zhuantihuigu/czjbqk2011/czsr2011/201208/t20120831_679821.html
全国土地出让收入管理及使用情况

2007年之前,土地出让收入先纳入预算外专户管理,再将扣除征地补偿和拆迁费用以及
土地开发支出等成本性支出后的余额缴入地方国库,纳入地方政府性基金预算管理。
从2007年开始,国家对土地出让收入管理制度进行了改革,将全部土地出让收入缴入地
方国库,纳入地方政府性基金预算管理,与公共财政预算分开核算,专款专用。土地出
让收支纳入政府性基金预决算编制范围,实行预决算管理制度。每年第三季度,国土资
源管理等有关部门按照相关规定编制下一年度土地出让收支预算。每年年度终了,国土
资源管理等有关部门按照规定编制上年土地出让收支决算。同时,按照规定程序向同级
人民政府报告,政府依法向同级人大报告。

%%% Local Variables:
%%% mode: latex
%%% TeX-master: "../main"
%%% End:
