\chapter{结合中国经济史谈一些经济概念}

\section{土地金融总结}


\section{通货膨胀税}

\textbf{通货膨胀税}是从宏观经济层次考虑的通货膨胀对货币持有者的影响,\textbf{指持有中央银
  行纸币者由于一般物价上涨(通货膨胀)所受到的损失。}

\textbf{引起通货膨胀税的原因很多,并非都是中央银行增加基础货币引起的。}例如,由于石油
价格上升等导致的成本推动型通货膨胀,由于支付技术进步导致的微观经济单位减少对
中央银行现金的需求而增加商业银行活期存款(或之前支付宝、微信钱包等金融存款方式)
导致的一般价格水平的上升。而恶性通货膨胀下,追求效用最大化的微观经济单位在恶
性通货膨胀条件下一般减少对中央银行纸币的实际需求,例如发生货币替代,对本国货
币需求减少……这样中央银行铸币税小于通货膨胀税。

\section{铸币税}

在现代经济理论中,\textbf{铸币税多指货币发行者由于在货币发行中具有一定程度的市场垄
  断权力(非完全竞争)而从货币发行中获得的利润,是一种隐形税收。}可以认为铸币
税是通货膨胀税的一个子集。

简单举一闭环例子来说,假设市场总价值变化不大,流通货币 $x$ 元。央行再发行货
币 $y$ 元,相应支出 $y$元购买商品。长期市场流通循环后,市场价值不变,却已
有 $x+y$ 元货币,通货膨胀率 $ \sfrac{y}{x}$ ,货币贬值 $\sfrac{y}{x+y}$ 。央
行通过征税、债券等方式回收货币,但货币整体已贬值,此时央行赚取的并不是超发货
币之初的 $y$ 元,而是回落到 $(1 - 货币贬值率)y$。


可以这样来简单形象理解:央行依仗垄断权利向全民强制发行\textbf{无息债券},其中的\textbf{本
  应支付却未支付的债券利息}\footnote{具体论述可见张怀清论文《人民银行铸币税的测算和
  运用 :1986--2008》}便是央行铸币暗税,\textbf{其实际收益与临期时长成正比,与周转次
  数成反比}。越先使用这张债券的部门,越受益;不参与其中投资周转的人,例如只是
储蓄或拿死工资的人蒙受全部应付未付利息的损失。

铸币税是偏向于顶富阶层的财富再分配,促使财富分化现象更加严重。在经济攀升期中
产阶级或许能收益;但即使在这种乐观时期,下层阶级也不可避免要承担最高隐形税赋!
从这种意义上来说,\textbf{铸币税是劫贫济富,这是人类从原始社会至今固有滥觞。}

以下为张怀清几篇论文的摘选:
\begin{quotation}
  不仅金属铸币、中央银行货币的发行产生铸币税,商业银行由于在存款市场具有垄断
  而获得的利润也可看作是铸币税。不仅如此,还有诸如政府债券铸币税(bond
  seigniorage)等形式的铸币税。

  随着电子通信技术的发展、金融市场的完善和金融产品的丰富,微观经济单位对中央
  银行纸币的需求呈现相对减少趋势,不仅商业银行类金融机构正在发行可在一定程度
  上替代中央银行纸币的货币,其他金融机构发行的金融产品也在很大程度上替代中央
  银行纸币和商业银行负债。

  国际货币基金组织(IMF)曾于1998年分别考察了欧洲和北美等21个发达国家以及亚洲和
  拉丁美洲等79个发展中国家,得出结论: 1980–1995年期间,发展中国家铸币税收
  占GDP的比重平均在1.4\%–3\%之间,大大高于发达国家平均在0.64\%左右的水平。而
  中国在同一时期,铸币税占GDP的比重平均为6.52\%( Massonetal, 1998),除了少数
  几个发生过超级通货膨胀的国家之外,已是世界上最高的国家之一。

  不同的学者利用不同的方法得到的估计有所差异,谢平(1994)按照基础货币增量的算
  法,得出我国1986--1993年之间,政府每年得到的货币发行收入占国内生产总值的比
  重平均为5.4\%;易纲(1996)得出1978--1992年真实铸币收入平均占GNP的3\%左右;周
  立(2003)认为1984--1996年期间的大部分年份的真实铸币收入
  占GDP的5\%--7\%,1993年和1996年达到8.5\%。但无论是哪一种算法得出的结果都远
  远高于发达国家的平均水平0.64\%,也在很大程度上超出发展中国家的平均水
  平1.4\%--3\%。
\end{quotation}

另外本国货币如能作为国际货币使用,自然也可向其他国家收取铸币税。人民币国际
化还任重道远,较多采用货币互换方式,能收取的国际铸币税较少。根据人行《2023 年
人民币国际化报告》
\begin{quotation}
  2023 年一季度末,人民币国际化综合指数为 3.26,同比上升 10.2%。
  同期,美元、欧元、英镑、日元等主要国际货币国际化指数分别
  为 57.68、22.27、7.66 和 5.48。

  2022 年,人民币跨境收付金额合计为 42.1 万亿元,同比增长15.1\%。其中,实
  收 20.5 万亿元,同比增长 10.9\%;实付 21.6 万亿元,同比增长 19.5\%,\textbf{收付
    比为 1:1}。
\end{quotation}

\improve[inline]{读者如有意,可加入商业银行存贷铸币税、互联网金融公司铸币税
  等论述。}

\section{超发货币和居民消费价格指数CPI}

根据《黄金时代:应对超现实风险的真实解决方
案》\cite{piepenburg2022gold},自1983年起,以购房是投资而非消费为由,不再将\textbf{房
  价}计入消费者物价指数CPI,取而代之的是增长速度远低于房价的% 主要居所租
% 金(rent of primary residence)和
\textbf{业主等价租金}(owners’ equivalent rent, OER,简单理解就是如果房主将房子出
租所能获得的租金),CPI被严重低估。以中国为例,中国重点50城租售比自2019年后始
终在1:600以下,且租售偏离程度持续扩大,也就是租房50余年才可收回房价。


此外《黄金时代》还写到:
\begin{quotation}
  美联储多年来一直公开撒谎,淡化真正的通货膨胀……\textbf{CPI表是一个公开的骗局},
  这允许美国劳工统计局(BLS),因此也允许美联储,以他们认为合适的方
  式 "报告 "通货膨胀——至少目前是这样。

  如果使用20世纪80年代美联储CPI通胀加权方法衡量今日,那么美国在2021年的CPI通
  胀率将接近15\%,而不是报告的6\%多。

  美联储简单调整了其衡量通货膨胀的CPI尺度,有效淡化了住房、医疗保健和教育方面
  的成本上升,以衡量消费者价格通货膨胀……简言之,美联储不喜欢用旧的CPI尺度来
  衡量通货膨胀,所以他们简单地用\textbf{2+2=2}的CPI来代替它。

  同样,美联储为了保持其\textbf{由欠条驱动(即债务驱动)的“复苏”假象},别无选择,
  只能\textbf{发明一个可控的(即较低的)CPI通货膨胀率},以便使美国国债在与通货膨胀
  相比较时,看起来对其他债务驱动买家更有适度的吸引力。


  \textbf{保持债券和债务市场活力的唯一方法是通过“过度印刷”来摧毁国家的基础货币。简
  而言之,畸形膨胀的市场可能在利率抑制的刀锋下生存,但货币,嗯......他们死在
  同一把剑下。}

  当然,\textbf{这是在吃你的蛋糕,但不是在吃它}……

  为了进一步欺骗人民,这个“魔术”背后的所谓专家想出了MMT(即现代货币理论,主
  张\textbf{财政赤字货币化}。)这个舒缓的概念,以使这种腐败的东西看起来更符合逻辑,
  更正式,甚至更聪明。对政策魔术师来说,这种语义上的技巧并不新鲜。当他们需要
  欺骗人民时,他们会巧妙地从字母表中抽出令人平静的字母,发明一些听起来很有学
  术性、有效和健全的政策名称,比如MMT或QE(\textbf{量化宽松})。但是,这两个现在常
  见的政策名称都不过是指\textbf{凭空伪造货币,这是一种公开的荒谬}。

  \textbf{量化宽松只是制造并扩大了有记录以来最大的风险资产泡沫和贫富差距。}

  \textbf{今天全球的特点是不顾一切地扩大广义货币供应,以解决不可持续且史无前例的债
    务水平。}这种扭曲而一贯的政策错误,导致了历史上最大的风险资产泡沫……正如
  历史所提醒的,所有的泡沫都会破灭。


  当股票\textbf{在投机性政策的支持下不合逻辑地上涨时,尽管其政策制定者有信誉的“逻
    辑”,但事实上没有逻辑,随之而来的极端纸面财富获得了永久甚至稳定的幻觉。}但
  是,正如我们和赫斯曼当时所警告的那样,今天更是如此,投资者很快就会集体陷入
  一种错觉,认为他们今天投资组合中的数万亿美元代表着明天的持久购买力。换句话
  说,“合乎逻辑”的投资者总是忽略了一个历史上被证实的事实,即\textbf{一旦上升的东
    西崩溃,大部分的财富最终会蒸发掉。}简而言之,风险资产从未变得更有风险。例
  如,截至目前,全球金融资产的价值(股票、债券和房地产)是520万亿美元,是全
  球84万亿美元GDP的\textbf{6.2倍}。

  考虑到持续增长和怪诞的债务水平,决策者实际上别无选择,只能膨胀他们的债务。
  作为聪明的小狐狸,公共政策制定者当然会尽一切努力,故意允许\textbf{通货膨胀(以偿
    还债务)},同时控制收益率曲线来\textbf{人为抑制利率},\textbf{从而抑制债务成本}。同样,
  这种绝望的利率压制对于爬进2020年代的“破产”主权国家来说是必须的。再说一遍:
  他们别无选择……\textbf{摆脱债务的唯一方法是让通胀率高于利率——差距越大,摆脱债
    务的速度就越快。}

  考虑到当前大于300万亿美元的全球债务水平,如果允许\textbf{利率自然上升}(即在真正
  的资本主义中),政府债务的利息支出将在几秒钟内大幅超过GDP的50\%,全球债务方
  和所谓的“经济复苏”将立即以戏剧性的方式结束。

  如果 CPI 通胀率被准确报告,那么公开虚假的美国国债实际收益率将为负值,美联储
  通过给这种收益率涂上口红,可以继续依靠更多债务、更“有吸引力”的欠条和更多
  的欺骗为生。这种隐蔽的通胀欺诈行为让美国能够有效地延长和掩盖历史上史无前例
  的债务狂潮,而美国信贷市场就像一个名副其实的科学怪人一样前进——它已经死了,
  但仍然在所谓“无通胀”但永久的货币创造的氧气下前进。但即便是弗兰肯斯坦最终
  也会死去。
\end{quotation}

就今日世界大国强国实践,尤其是美国而言,超发货币、寅吃卯粮已是长期传统,\textbf{报
  表上的CPI指数意图掩盖的是真实通货膨胀、货币发行机构所收取的铸币税;超发货币
  实则是金融虚拟资本(包括房地产,房地产真实属性是金融)对其他赛道人民的盘剥。
  收入不平等加剧、财富极度分化就这样静悄悄地来到了现代社会。}

其实现代国家往往都习以为常施用各种“真实数据烟雾弹”……数据可以是真实的,但却是在
各种别有用心统计方法下的真实数据。


\section{GDP、GNI和分配正义}

据国家统计局,

\begin{quotation}
  国内生产总值(Gross Domestic Product,GDP),是指一个国家或地区所有常住单位在
  一定时期内生产活动的全部最终成果,等于所有常住单位创造的增加值之和。可
  见,GDP强调国内生产,体现的是增加值的生产创造。


  国民总收入(Gross National Income,GNI),是指一个国家或地区所有常住单位在一
  定时期内收入初次分配的最终结果,等于所有常住单位的初次分配收入之和。

  GNI,即在GDP的基础上,扣除外国在本国的资本和劳务收入,加上本国从国外获得的
  资本和劳务收入。
\end{quotation}

不管GDP还是GNI,都是以加总方式来衡量国家经济。当代所有国际几乎都热衷于用这两
种方式来描述国内经济状况。这两个指标都无法说明、甚至不想说明分配不均、贫富差
距、社会福利这类问题。资本主义发端之初,新古典(自由主义)经济学家们便惯于使
用国民经济这一加总说法。

如亚当·斯密《国富论》:
\begin{quotation}
  劳动获得宽裕的报酬,不仅是一国财富\textbf{不断增加}的必然结果,同时也是一国财
  富\textbf{不断增加}的自然症候。另一方面,贫穷的劳动阶级生活捉襟见肘,是一国财
  富\textbf{停滞}的自然症候,而该阶级人民濒临饿死,是一国财富\textbf{迅速萎缩}的自然症候。
\end{quotation}


国民经济的发展(GDP或GNI等) \neq 大部分国民的经济发展 \neq 国民的幸福。

恩格斯和马克思分别写到:
\begin{quotation}
  这一事实无非是表明:劳动国民财富这个用语是由于自由主义经济学家努力进行概括
  才产生的。只要私有制存在一天,这个用语便没有任何意义。英国人的“国民财
  富”很多,他们却是世界上最穷的民族……在这种科学看来,社会关系只是为了私有
  制而存在。\pagescite[][60]{maenwen1}


  我们且从当前的国民经济的事实出发。工人生产的财富越多,他的生产的影响和规模
  越大,他就越贫穷。工人创造的商品越多,他就越变成廉价的商品。\textbf{物的世界的增
    值同人的世界的贬值成正比。}劳动生产的不仅是商品,它还生产作为商品的劳动自
  身和工人,而且是按它一般生产商品的比例生产的。\pagescite[][156]{maenwen1}
\end{quotation}

张文喜探讨了《所有制与所有权正义:马克思与“亚当·斯密问题”》\cite{ZXYJ201404002}:
\footnote{\url{https://www.dswxyjy.org.cn/n1/2019/0617/c427160-31162202.html}}:
\begin{quotation}
  在对斯密等人的评述中,马克思明确指出存在私有财产和工人需要的悖论,认为市民
  式的自私自利能够保证与公共利益先天和谐,这是一种幻象。斯密的自然自由原理与
  利益和谐的自由市场学说注定满足不了任何一个阶级的要求:\textbf{工人和资本家同样苦
  恼,没有财产的工人是为他的生存而苦恼,“资本家则是为他的死钱财的赢利而苦
    恼”}(《马克思恩格斯全集》第3卷,第227页);因而,“既然按照斯密的意见,大
  多数人遭受痛苦的社会是不幸福的,\textbf{社会的最富裕状态会造成大多数人遭受这种痛
    苦},而且国民经济学(总之,私人利益的社会)是要导致这种最富裕状态,那么国民
  经济学的目的也就是社会的不幸”。(同上,第230页)

  \textbf{生产资料私有制给特定的阶级带来的具体结果,在本质上绝不是像斯密所讲的共同富
  裕,而是两极分化。}
\end{quotation}


1993年,中国正式采用GDP作为经济表现的指标,从此开始了追求GDP增速的历史。
回到亚当·斯密,中国持续发展可以“先富带动后富”,并且确实使国人生活水平普遍
大幅提高。但中国如何应对“财富停滞”时“贫困的劳动阶级生活捉襟见肘”,又如何应对
经济危机、财富迅速萎缩时的“濒临饿死”呢?

四十多年来,中国刺激经济持续增长多是以\textbf{大规模超发货币}为主要手段,用基建、房地产等
作为通货膨胀蓄水池,甚至在某种意义上使基建、房地产本身成为超发的货币。这一政策核
心至少自上世纪70年代末的“洋跃进”便已出现,只是那时候还没有房地产的加入。

凯恩斯主义是短期应用经济学,且要求针对经济萧条和上升期适时调整;它对于长期无
力,长期应用必然持续集聚毒性。每一次的负债经济增长都必然使将来市场出清、结算
埋单时期更加残酷惨痛。

为缓解危机,除非我们能将危机转嫁至他国或者参加战争。问题是,太多发达国家也是
这样想的——一方面担忧已在世界资本市场长袖善舞的中国经济危机导致世界经济危机,
引发种种问题;另一方面希望啃蚀世界最大市场之中国的血肉,将自身危机转嫁至中
国。我们太愿意以债养债、寅吃卯粮了,到了一种非理性的程度。

另外我们的贫富分化已在狂热追求GDP的过程中持续加大.

\begin{quotation}
  北京大学以全国25个省市160个区县的14960个家庭为基线样本所得的《中国民生发展
  报告\textbf{2015}》显示,\textbf{最富有的1\%的家庭占有近1/3的全国财产,而底端25\%的家庭
    拥有的财产总量只占1\%左右。}\cite{dajueqi}
\end{quotation}

2023年贝恩公司与招商银行联合发布《2023中国私人财富报告》,提到
\begin{quotation}
  2022 年,中国个人可投资资产总规模达 278 万亿人民币,2020-2022 年年均复合增
  速为 7\%;到 2024 年底,可投资资产总规模预计将突破 300 万亿关口。

  2022 年,可投资资产在 1,000 万人民币以上的中国高净值人群数量达 316 万人,人
  均持有可投资资产约3,183 万人民币,共持有可投资资产 101 万亿人民
  币,2020-2022 年年均复合增速为 10\%;预计未来两年,中国高净值人群数量和持有
  的可投资资产规模将以约 11\% 和 12\% 的复合增速继续增长。
\end{quotation}
也就是说在中国,\textbf{0.22\%的人口(高净值人群)占据了总可投资资产的31.96\%,}高
净值人群可投资资产增速倍数于GDP增速,也意味着财富分化趋势更一步加大。

相比贝恩和招行报告,瑞士信贷和瑞士银行发布的《2023年世界财富报告》相对乐观
些,但只是相对。中国的百万美元富翁数量已占世界11\%,仅次于美国的38\%。
\begin{quotation}
  自2000年以来,\textbf{中国的财富不平等现象大幅上升}。2000年\textbf{财富基尼系数为59.5},稳
  步上升,\textbf{2016年达到71.7}。2000年,前1\%人群的财富份额
  为20.7\%,2021年为30.5\%,2022年上升至31.4\%。

  迄今为止美国\textbf{百万美元}富翁人数最多,为2270万,占世界总数的\textbf{38.2\%}。这遥
  遥领先于排名第二的中国,中国占全球百万富翁总数的\textbf{10.5\%}。在本世纪初,日本
  的百万富翁数量与美国竞争,之后日本的地位一直在稳步下降,于2014年被中国超
  越,2022年仅占百万富翁的4.6\%,首次排在第四位,仅次于法国(4.8\%),并受到
  德国(4.4\%)和英国(4.3\%)的挑战。
\end{quotation}

\begin{table}[hbt!]
\centering
\resizebox{\textwidth}{!}{%
\begin{tabular}{@{}llllllllllllllll@{}}
\toprule
\multicolumn{1}{c}{} & \multicolumn{7}{c}{\textbf{基尼系数}}              &  & \multicolumn{7}{c}{\textbf{1\%最富有的人财富占比}}      \\
                     & 2000 & 2005 & 2010 & 2015 & 2020 & 2021 & 2022 &  & 2000 & 2005 & 2010 & 2015 & 2020 & 2021 & 2022 \\ \midrule
巴西                   & 84.5 & 82.7 & 82.1 & 88.7 & 88.9 & 89.2 & 88.4 &  & 44.2 & 45   & 40.2 & 48.7 & 49.5 & 49.3 & 48.4 \\
美国                   & 80.6 & 81.1 & 84.1 & 84.9 & 85   & 85   & 83   &  & 32.9 & 32.8 & 33.4 & 34.8 & 35.3 & 35.1 & 34.2 \\
印度                   & 74.6 & 80.9 & 82.1 & 83.3 & 82.3 & 82.3 & 82.6 &  & 33.2 & 41.9 & 41.4 & 42.3 & 40.5 & 40.6 & 41   \\
德国                   & 81.2 & 82.7 & 77.4 & 79.2 & 77.9 & 78.8 & 76.9 &  & 29.1 & 30.4 & 25.7 & 32.1 & 29.2 & 31.7 & 30   \\
加拿大                  & 74.9 & 73.3 & 71.7 & 71.8 & 71.8 & 72.6 & 72.3 &  & 29.1 & 25.9 & 22.4 & 23.3 & 23.6 & 25   & 24.3 \\
\textbf{中国大陆} &
  \textbf{59.5} &
  \textbf{63.8} &
  \textbf{70} &
  \textbf{71.2} &
  \textbf{70.5} &
  \textbf{70.1} &
  \textbf{70.7} &
  \textbf{} &
  \textbf{20.7} &
  \textbf{24.2} &
  \textbf{31.5} &
  \textbf{31.7} &
  \textbf{30.8} &
  \textbf{30.5} &
  \textbf{31.1} \\
台湾                   & 64.7 & 67.8 & 72.6 & 70.5 & 70.7 & 70.7 & 70.5 &  & 24.3 & 23.6 & 29.8 & 26.9 & 27.3 & 26.6 & 26.4 \\
法国                   & 69.7 & 67   & 69.8 & 69.9 & 70   & 70.2 & 70.3 &  & 25.5 & 21   & 21   & 22.3 & 21.9 & 22.3 & 21.2 \\
英国                   & 70.5 & 67.6 & 69.1 & 73   & 71.7 & 70.6 & 70.1 &  & 22.1 & 20.6 & 23.6 & 25   & 23.1 & 21.1 & 20.7 \\
西班牙                  & 65.5 & 62.2 & 61.4 & 69.5 & 69.1 & 69.1 & 68.3 &  & 24.1 & 18.7 & 18.5 & 24.1 & 22.7 & 23.1 & 22.4 \\
韩国                   & 69.7 & 70.1 & 74.7 & 72.4 & 67.7 & 68.2 & 67.9 &  & 21.3 & 21.8 & 26   & 26.9 & 23.4 & 24   & 23.1 \\
意大利                  & 60.4 & 59.4 & 63.1 & 66.9 & 66.4 & 67.2 & 67.8 &  & 22   & 18.2 & 17.4 & 22.6 & 21.9 & 23.3 & 23.1 \\
澳大利亚                 & 63.7 & 63.1 & 63   & 64.9 & 65.5 & 66.2 & 66.3 &  & 20.5 & 20   & 19.2 & 20.5 & 20.6 & 21.8 & 21.7 \\
日本                   & 64.5 & 63.1 & 62.5 & 63.6 & 64.4 & 64.7 & 64.8 &  & 20.4 & 18.8 & 16.7 & 18.2 & 18.1 & 18.7 & 18.8 \\ \bottomrule
\end{tabular}
}
\caption{财富不平等趋势}
\capsource{来源:瑞士信贷和瑞士银行《2023年世界财富报告》}
\label{tab:gini}
\end{table}

上一节已经提过,超发货币可以采用直接超发,量化宽松或赤字货币化等形式。这些形
式均是为大金融资本服务的财富持续分化武器!

无论是新官不理旧账的官僚考核制度,或是被债务绑架选择以债养债,都只像是表象而
非本质,没有触及核心本质。本质是金融精英和权力贵族的结合体?希望读者或者他人
能够进一步阐述。

\todo[inline]{我国热衷超发货币、追求GDP的深层动因、逻辑,或许是金融资本精英加
  权力贵族对于资本增殖的需求?考虑列入土地金融总结部分。}
