\section{计划经济时期的成就}

本节部分内容来自李慎明文章《正确评价改革开放前后两个历史时期》。

\begin{enumerate}

\item “实现和巩固了全国范围(除台湾等岛屿以外)的国家统一,根本改变了旧中国四分五
  裂的局面。实现和巩固了全国各族人民的大团结,形成和发展了五十多个民族平等互
  助的社会主义民族关系。战胜了帝国主义、霸权主义的侵略、破坏和武装挑衅,维护
  了国家的安全和独立,胜利地进行了保卫祖国边疆的斗争。”

\item 排除种种干扰重返联合国。由于毛泽东关于“三个世界”划分理论的正确指导,我国
  与美国、欧洲诸国和日本等主要国家的外交关系取得突破性进展,成功打破外部霸权
  主义和强权政治对我国的严酷封锁,真正跨入了大国的行列,并迎来和平与发展的时
  代主题。

\item 1949年,中国人口5.4亿,人均预期寿命不足35岁。1976年,人口9.3亿,人均预期寿
  命64.6岁,人口数量增长近4亿,预期寿命增长近30岁。当然,这里也有之前战乱动荡,
  底子过于薄弱的加成。

  新中国成立后,中央将扫盲列为成人教育的首位,并在1952年,成立了“中央扫除文
  盲工作委员会”,积极推行“速成识字法”。1956年3月29日,中共中央和国务院发布
  《扫除文盲的决定》,将扫盲提高到了空前的高度,第一次把扫盲作为国家发展大
  计。1960年5月11日,中共中央发布了《关于推广注音识字的指示》。

  解放初期文盲率80\%,1964年15岁以上人口文盲率56.8\%,1982年34.5\%。\footnote{参考黄
    晨熹《1964\~2005年我国人口受教育状况的变动》}

\item 在经济落后的条件下,以严重城乡二元对立为代价,保证了高积累和优先快速发展重
  工业,建立了比较完整的独立的工业体系和基础设施。

  根据中共十一届六中全会《关于建国以来党的若干历史问题的决议》:
  \begin{quotation}
    一九八〇年同完成经济恢复的一九五二年相比,全国工业固定资产按原价计算,增长
    二十六倍多,达到四千一百多亿元;棉纱产量增长三点五倍,达到二百九十三万吨;
    原煤产量增长八点四倍,达到六亿二千万吨;发电量增长四十倍,达到三千多亿度;
    原油产量达到一亿零五百多万吨;钢产量达到三千七百多万吨;机械工业产值增长五
    十三倍,达到一千二百七十多亿元。在辽阔的内地和少数民族地区,兴建了一批新的
    工业基地。国防工业从无到有地逐步建设起来。资源勘探工作成绩很大。铁路、公路、
    水运、空运和邮电事业,都有很大的发展。

    ……

    我们现在(80年代初)赖以进行现代化建设的物质技术基础,很大一部分是这个期间
    建设起来的;全国经济文化建设等方面的骨干力量和他们的工作经验,大部分也是在
    这个期间培养和积累起来的。这是这个期间党的工作的主导方面。
  \end{quotation}
  在毛泽东时期,我国从一个落后的农业国跻身为世界第六大工业国。

\item 独立自主、自力更生研发出“两弹一星一潜艇”。1964年10月,我国第一颗原
  子弹爆炸成功。1966年12月,我国第一颗氢弹原理试验爆炸成功。1970年4月,我国
  第一颗人造卫星发射成功。1971年9月,我国第一艘核潜艇下水,并于1974年8月,
  正式加入人民海军战斗序列。在成熟的核潜艇的基础上,1981年4月,我国第一艘战
  略核潜艇下水,使陆海空全都具备了第二次核反击能力。

  邓小平1988年明确指出,“如果六十年代以来中国没有原子弹、氢弹,没有发射卫
  星,中国就不能叫有重要影响的大国,就没有现在这样的国际地位。”



\item 政治层面上前所未有的管控能力。计划经济时期通过“政社合一”人民公社体制实现
  了对农村前所未有、甚至可能后不见来者的资源管控能力。一切资源,包
  括“国有”和“集体”土地、企业、,事实上均被政府所垄断,为之后的市场经济时
  期提供了多方面可以运作的巨额红利空间。

  因城市经济不可承受的压力,20世纪60年代初和三次知青下乡等以千万人为单位的城
  市向农民的劳动力逆向转移,没有这样管控能力和“信仰”加成其实是办不到的,只
  是“信仰”在现实面前逐次衰减。
\end{enumerate}
